%%%% ijcai17.tex

\typeout{IJCAI-17 Instructions for Authors}

% These are the instructions for authors for IJCAI-17.
% They are the same as the ones for IJCAI-11 with superficical wording
%   changes only.

\documentclass{article}
% The file ijcai17.sty is the style file for IJCAI-17 (same as ijcai07.sty).
\usepackage{ijcai17}

% Use the postscript times font!
\usepackage{times}

\usepackage{macros}

\usepackage{latexsym} 


\title{Thresholding Bandits with Augmented UCB}
%\author{Author names withheld}

\author{Subhojyoti Mukherjee${}^1$, K. P. Naveen${}^2$, Nandan
Sudarsanam${}^3$, Balaraman Ravindran${}^1$\\
${}^1$Department of Computer Science \& Engineering,\\ ${}^2$Department of Electrical Engineering,
${}^3$Department of Management Studies,\\ Indian Institute of
Technology Madras}


\begin{document}
\maketitle


\vspace*{2mm}
\begin{abstract}
In this paper we propose the Augmented-UCB (AugUCB) algorithm for a fixed-budget version of the thresholding bandit problem (TBP), where the objective is to identify a set of arms whose quality is above a threshold. A key feature of AugUCB is that it uses both mean and variance estimates to eliminate arms that have been sufficiently explored; to the best of our knowledge this is the first algorithm to employ such an approach for the considered TBP.  Theoretically, we obtain an upper bound on the loss (probability of mis-classification) incurred by AugUCB. Although UCBEV in literature provides a better guarantee, it is important to emphasize that UCBEV has access to problem complexity (whose computation requires arms' mean and variances), and hence is not realistic in practice; this is in contrast to AugUCB whose implementation does not require any such complexity inputs. We conduct extensive simulation experiments to validate the performance of AugUCB. Through our simulation work, we establish that AugUCB, owing to its utilization of variance estimates, performs significantly better than the state-of-the-art APT, CSAR and other non variance-based algorithms.
\end{abstract}

%\begin{keywords}
%Multi-Armed Bandit, Regret, Exploration-exploitation, UCB
%\end{keywords}

\section{Introduction}
\label{intro}
In today's world artificial intelligence has proved to be a game-changer in designing agents that interact with an evolving environment and make decisions on the fly. The main goal of artificial intelligence is to design artificial agents that make dynamic decisions in an evolving environment. In pursuit of these the agent can be thought of as making a series of  sequential decisions by interacting with the dynamic environment which provides it with some sort of feedback after every decision which the agent incorporates into its decision-making strategy to formulate the next decision to be made. A large number of problems in science and engineering, robotics and game playing, resource management, financial portfolio management, medical treatment design, ad placement, website optimization and packet routing can be modeled as sequential decision-making under uncertainty. Many of these real-world interesting
sequential decision-making problems can be formulated as reinforcement learning (RL) problems (\citep{bertsekas1996neuro}, \citep{sutton1998reinforcement}). In an RL problem, an agent interacts with a dynamic, stochastic, and unknown environment, with the goal of finding an action-selection strategy or policy that optimizes some long-term performance measure. Every time when the agent interacts with the environment it receives a signal/reward from the environment based on which it modifies its policy. The agent learns to optimize the choice of actions over several time steps which is learned from the sequences of data that it receives from the environment. This is the crux of online sequential learning. An illustrative image depicting the reinforcement learning scenario is shown in Figure \ref{fig:rl}.

	This is in contrast to supervised learning methods that deal with labeled data which are independently and identically distributed (i.i.d.) samples from the domain and train some classifier on the entire training dataset to learn the pattern of this distribution to predict future samples (test dataset) with the assumption that it is sampled from the same domain, whereas the RL agent learns from the samples that are collected from the trajectories generated by its sequential interaction with the system. For an RL agent the trajectory consists of a series of sequential interactions whereby it transitions from one state to another following some dynamics intrinsic to the environment while collecting the reward till some stopping condition is reached. This is known as an episode. For a single-step interaction, i.e., when the episode terminates after a single transition, the problem is captured by the multi-armed bandit (MAB) model. Our work will focus on this idea of MAB model.

\begin{figure}[!th]
\includegraphics[scale=0.7]{Chapter1/img/RL.png}
\caption{Reinforcement Learning}
\label{fig:rl}
\end{figure}

%To express an RL problem more formally, we have to define the idea of Markov Decision Process (MDP) which consists of states, actions, transition probabilities and rewards which in turn helps in deciding the strategy to be followed by the agent. 

%An MDP consists of states

%\subsection{Motivation}
%\label{motivation}
The above TBP formulation has several applications, for instance, from areas ranging from anomaly detection and classification (see  \citet{locatelli2016optimal}) to industrial application. Particularly in industrial applications, a learners objective is to choose (i.e., keep in operation) all machines whose productivity is above a threshold. The TBP also finds applications in mobile communications (see \citet{audibert2010best})  where the users are to be allocated only those channels whose quality is above an acceptable threshold. Again, some of these problems have been already discussed in chapter \ref{chap:intro}, section \ref{motivation} and an interested reader can refer to it. In some cases the TBP problem is more relevant than the variants of $p$-best problem (identifying the best $p$ arms from $K$ given arms). As explained in \citet{locatelli2016optimal}, the $p$-best problem is a "contest" whereas the TBP is an  "exam" and in many domains, one requires the idea of "efficiency" or "correctness" threshold above which one wants to utilize an option rather than simply selecting the $p$-best options.

%where the learner has to keep all those workers active whose productivity is above a particular threshold $\tau$, or allocating channels whose quality is above a threshold for Mobile Communications 
% or in crowd-sourcing while hiring workers the TBP problem 

%
%	1. \emph{Product Selection:} A company wants to introduce a new product in market and there is a clear separation of the test phase from the commercialization phase. In this case the company tries to minimize the loss it might incur in the commercialization phase by testing as much as possible in the test phase. So from the several variants of the product that are in the test phase the learning agent must suggest the product variant(s) that are above a particular threshold $\tau$ at the end of the test phase that have the highest probability of minimizing loss in the commercialization phase. A similar problem has been discussed for single best product variant identification without threshold in \cite{bubeck2011pure}. 
%
%	2. \emph{Mobile Phone Channel Allocation:} Another similar problem as above concerns channel allocation for mobile phone communications (\cite{audibert2009exploration}). Here there is a clear separation between the allocation phase and communication phase whereby in the allocation phase a learning algorithm has to explore as many channels as possible to suggest the best possible set of channel(s) that are above a particular threshold $\tau$. The threshold depends on the subscription level of the customer. With higher subscription the customer is allowed better channel(s) with the $\tau$ set high. Each evaluation of a channel is noisy and the learning algorithm must come up with the best possible suggestion within a very small  number of attempts.
%
%	3. \emph{Anomaly Detection and Classification:} Thresholding bandit can also be used for anomaly detection and classification where we define a cutoff level $\tau$ and for any samples above this cutoff gets classified as an anomaly. For further reading we point the reader to section 3 of \cite{locatelli2016optimal}.
%
%
\subsection{Related Work}
\label{prevRes}
Significant amount of literature is available on the stochastic MAB setting with respect to minimizing the cumulative regret. Chapter \ref{chap:SMAB} and \ref{chap:EUCBV} deals with that. In this work we are particularly interested in \emph{pure-exploration MABs},  where the focus in primarily on simple regret rather than the cumulative regret. The relationship between cumulative regret and simple regret is proved in \citet{bubeck2011pure} where the authors prove that minimizing the simple regret necessarily results in maximizing the cumulative regret.
The pure exploration problem has been explored  mainly under the following two settings:
	
\subsection{Fixed Budget setting} 

Here the learning algorithm has to suggest the best arm(s) within a fixed time-horizon $T$, that is usually given as an input. The objective is to maximize the probability of returning the best arm(s).  This is the scenario we consider in this chapter. Some of the important algorithms used in pure exploration setting are discussed in the next part.

\subsubsection{UCB-Exploration Algorithm}

\begin{algorithm}[!h]
\caption{UCBE}
\label{alg:ucbe}
\begin{algorithmic}[1]
\State \textbf{Input: } The budget $T$
\State Pull each arm once
\For{$t=K+1,..., T$}
\State Pull the arm such that $\argmax_{i\in \A}\bigg\lbrace\hat{r}_{i} + \sqrt{\dfrac{a}{n_i}}\bigg\rbrace$, where $a = \dfrac{25(T-K)}{36 H_1}$ and $H_1 = \sum_{i=1}^{K}\dfrac{1}{\Delta_i^2}$.
\State $t:=t+1 $
\EndFor
\end{algorithmic}
\end{algorithm}

In \citet{audibert2010best} the authors propose the  UCBE and the Successive Reject (SR) algorithm, and prove simple-regret guarantees for the problem of identifying the single best arm.  In the combinatorial fixed budget setup \citet{gabillon2011multi} propose the GapE and GapE-V algorithms that suggest, with high probability, the best $m$ arms at the end of the time budget. 


\subsubsection{Successive Reject Algorithm}


\begin{algorithm}[h!]
\caption{Successive Reject(SR)}
\label{alg:sr}
\begin{algorithmic}[1]
\State \textbf{Input: } The budget $T$
\State \textbf{Initialization: } $n_0 = 0$
\State \textbf{Definition: } $\bar{\log K} = \dfrac{1}{2} + \sum_{i=2}^{K}\dfrac{1}{i}$, $n_k = \dfrac{1}{\bar{\log K}}\dfrac{T-K}{K + 1 - m}$
\For{For each phase $m=1,..., K-1$}
\State For each $i \in B_{m}$, select arm $i$ for $n_k - n_{k-1}$ rounds.
\State Let $B_{m+1} = B_m\setminus \argmin_{i\in B_m} \hat{r}_i$
(remove one element from $B_m$ , if there
is a tie, select randomly the arm to dismiss among the worst arms).
\State $m:=m+1 $
\EndFor
\State Output the single remaining $i\in B_{m}$.
\end{algorithmic}
\end{algorithm}


Similarly, \citet{bubeck2013multiple} introduce the  Successive Accept Reject (SAR) algorithm, which is an extension of the SR algorithm; SAR is a round based algorithm whereby at the end of each round an arm is either accepted or rejected (based on certain confidence conditions) until the top $m$ arms are suggested at the end of the budget with high probability. A similar combinatorial setup was explored in \citet{chen2014combinatorial} where the authors propose the Combinatorial Successive Accept Reject (CSAR) algorithm, which is similar in concept to SAR but with a more general setup. 

\subsection{Fixed Confidence setting} 

In this setting the learning algorithm has to suggest the best arm(s) with a fixed confidence (given as input) with as fewer number of attempts as possible. The single best arm identification has been studied in \citet{even2006action}, while for the combinatorial setup \citet{kalyanakrishnan2012pac} have proposed the LUCB algorithm which, on termination, returns  $m$ arms which are at least $\epsilon$ close to the true top-$m$ arms with probability at least $1-\delta$. For a detail survey of this setup we refer the reader to \citet{jamieson2014best}. 

\subsection{Unified Setting}
Apart from these two settings some unified approaches has also been suggested in \citet{gabillon2012best} which proposes the algorithms UGapEb and UGapEc which can work in both the above two settings. The thresholding bandit problem is a specific instance of the pure-exploration setup of \citet{chen2014combinatorial}. 



	
	


\subsection{Our Contribution}
\label{contribution}
The main contributions of the thesis are as follows:-
\begin{enumerate}
\item We propose the MLTimer alogrithm that uses gate-sizing for reducing the leakage power consumption of a digital design. We propose a smart one-pass tool that can leverage the right optimization technique at the appropriate stage of the flow thereby improving design productivity. A key observation reported in MLTimer is that there exists significant correlation between the timing slacks of gates in the current iteration to the gate replacements in successive iterations. MLTimer leverages this observation to reduce the number of STA runs thereby reducing the overall time taken for optimization.

\item We propose the Karna algorithm which uses gate-sizing for reducing the information leakage via the power side-channel of a digital design. We show that each region in a given design leaks information differently. Thus, it is sufficient to optimize gates in the highly sensitive regions to reduce information leakage. Karna leverages this observation and optimizes gates in these sensitive regions to reduce the power side-channel vulnerability. 

%\item We proposed a general framework of bandit algorithms that combines change-point detection algorithm with aggregation of expert strategies in order to define efficient pulling strategies in the context of the piecewise stochastic distributions. The algorithms that we proposed for the piecewise stochastic setting are actively adaptive algorithms which perform very similarly to the oracle algorithm which has access to the changepoints and suffers no additional delay in adapting to the changing environment. 
\end{enumerate}
 
%
%\section{Notation Used and Assumptions}
%\label{notation}
%\textbf{Notation and assumptions:} $\mathcal{A}$ denotes the set of arms, and $|\mathcal{A}|=K$ is the number of arms in $\mathcal{A}$. 
%Arms generic arm is indexed by $i,j\in\mathcal{A}$. 
For arm $i\in\mathcal{A}$, we use $r_{i}$ to denote the true mean of the distribution from which the rewards are sampled, while $\hat{r}_{i}(t)$ denotes the estimated mean at time $t$. Formally, using $n_i(t)$ to denote the number of times arm $i$ has been pulled until time $t$, we have $\hat{r}_{i}(t)=\frac{1}{n_{i}(t)}\sum_{z=1}^{n_i(t)} X_{i,z}$, where $X_{i,z}$ is the reward sample received when arm $i$ is pulled for the $z$-th time. %
Similarly, we use $\sigma_{i}^{2}$ to denote the true variance of the reward distribution corresponding to arm $i$, while $\hat{v}_{i}(t)$ is the estimated variance, i.e., $\hat{v}_{i}(t)=\frac{1}{n_i(t)}\sum_{z=1}^{n_{i}(t)}(X_{i,z}-\hat{r}_{i})^{2}$. Whenever there is no ambiguity about the underlaying  time index $t$, for simplicity we neglect $t$ from the notations and simply use  $\hat{r}_i, \hat{v}_i,$ and $n_i, $ to denote the respective quantities.  Let  $\Delta_{i}=|\tau-r_{i}|$ denote the distance of the true mean from the threshold $\tau$. Also, the rewards are assumed to take values in $[0,1]$.

%%%%%%%%%%
%1-sub-gaussian assumption removed
%%%%%%%%%%
%Along the lines of \cite{locatelli2016optimal} we assume that all the reward distributions are $1$-sub-Gaussian (note that,  $1$-sub-Gaussian includes Gaussian distributions with variance less than $1$, distributions supported on an interval of length less than 2, etc).


%
%
\vspace*{-1em}
\section{Augmented-UCB Algorithm}
\label{algorithm}
\textbf{The Algorithm:} The Augmented-UCB (AugUCB) algorithm is presented in Algorithm~\ref{alg:augucb}.
AugUCB is essentially based on the arm elimination method of the UCB-Improved \cite{auer2010ucb}, but adapted to the thresholding bandit setting proposed in \cite{locatelli2016optimal}. However, unlike the UCB improved (which is based on mean estimation) our algorithm employs \emph{variance estimates} (as in \cite{audibert2009exploration}) for arm elimination; to the best of our knowledge this is the first variance-aware  algorithm for the thresholding bandit problem. Further, we allow for arm-elimination at each time-step, which is in contrast to the earlier work (e.g., \cite{auer2010ucb,chen2014combinatorial}) where the arm elimination task is deferred to the end of the respective exploration rounds. The details are presented below.

% In algorithm \ref{alg:augucb}, hence referred to as AugUCB, we have two exploration parameters, $\rho_{\mu}$ and $\rho_v$ which are the arm elimination parameters. $\psi_{m}$ is the exploration regulatory factor. 
%The main approach is based on the UCB-Improved algorithm with modifications suited for the thresholding bandit problem. 
The active set $B_{0}$ is initialized with all the arms from $\mathcal{A}$. We divide the entire budget $T$ into rounds/phases like in UCB-Improved, CCB, SAR and CSAR. At every time-step AugUCB checks for arm elimination conditions, while updating parameters at the end of each round. As suggested by \cite{liu2016modification} to make AugUCB to overcome too much early exploration, we no longer pull all the arms equal number of times in each round. Instead, we choose an arm in the active set $B_m$ that minimizes $(|\hat{r}_{i} - \tau |-2s_i)$ where 
%$\min_{i\in B_{m}}\big\lbrace |\hat{r}_{i} - \tau | - 2\sqrt{\frac{\rho_v\psi_m \hat{V}_{i} \log ( T \epsilon_{m})}{4 n_{i}} + \frac{\rho_v\psi_m \log{( T\epsilon_{m})}}{4 n_{i}}} \big\rbrace $
\begin{small}
\begin{align*}
s_i & = \sqrt{\frac{\rho\psi_m (\hat{v}_{i}+1) \log ( T \epsilon_{m})}{4 n_{i}}} %+ \frac{\rho\psi_m \log{( T\epsilon_{m})}}{4 n_{i}}}.
\end{align*}
\end{small} 
with $\rho$ being the arm elimination parameter and $\psi_{m}$ being the exploration regulatory factor.
%  in the active set $B_{m}$. 
The above condition ensures that an arm closer to the threshold $\tau$ is pulled; 
%and with suitable choice of $\rho_{\mu}$ and $\rho_v$ we can fine tune the exploration. 
parameter $\rho$ can be used to fine tune the elimination interval.
The choice of exploration factor, $\psi_m$,
% $\psi_m=\frac{T\epsilon_m}{(\log(\frac{3}{16} K\log K))^{2}}$ 
comes directly from \cite{audibert2010best} and \cite{bubeck2011pure} where it is  stated that in pure exploration setup, the exploring factor must be linear in $T$ (so that an exponentially small probability of error is achieved) rather than being logarithmic in $T$ (which is more suited for minimizing cumulative regret).

\begin{algorithm}[t!]
\caption{AugUCB}
\label{alg:augucb}
\begin{algorithmic}
\State {\bf Input:} Time budget $T$; parameter $\rho$; 
% $\rho_{\mu}$, $\rho_v$ 
  threshold $\tau$
\State {\bf Initialization:} $B_{0}=\mathcal{A}$; $m=0$; $\epsilon_{0}=1$;
\begin{small}
\begin{align*}
M&=\left\lfloor \frac{1}{2}\log_{2} \frac{T}{e}\right\rfloor; 
\hspace{2mm}\psi_{0}=\frac{T\epsilon_{0}}{128\Big(\log(\frac{3}{16}K\log K)\Big)^2}; \\
\ell_{0}&=\left\lceil \frac{2\psi_0\log( T\epsilon_{0})}{\epsilon_{0}} \right\rceil;
\hspace{2mm}N_{0}=K\ell_{0}
\end{align*}
\end{small}
%$M=\left\lfloor \frac{1}{2}\log_{2} \frac{T}{e}\right\rfloor $,  
%$\psi_{0}=\frac{T\epsilon_{0}}{(\log(\frac{3}{16}K\log K)^2}$,
% $\ell_{0}=\left\lceil \frac{2\psi\log( T\epsilon_{0})}{\epsilon_{0}} \right\rceil$ and 
% $N_{0}=K\ell_{0} $. Pull each arm once.
\State Pull each arm once
\vspace{-2mm}
\State \For{$t=K+1,..,T$}
\State Pull arm $j\in\argmin_{i\in B_{m}}\Big\lbrace |\hat{r}_{i} - \tau | - 2s_{i}\Big\rbrace$
% \State where $s_j=\sqrt{\frac{\rho\psi_{m}\hat{v}_{j}\log{( T\epsilon_{m})}}{4 n_{j}} + \frac{\rho\psi_{m} \log{(T\epsilon_{m})}}{4 n_{j}}}$
\State $t\leftarrow t+1$ 
\vspace{-4mm}
%\ArmElim
%\State For each arm $i \in B_{m}$, remove arm ${i}$ from $B_{m}$ if
%\begin{align*}
%\hat{r}_{i} + c_i  < \tau - c_i \mbox{ or } \hat{r}_{i} - c_i  > \tau + c_i \\
%\text{where $c_i=\sqrt{\frac{\rho_{\mu}\psi_{m}\log{( T\epsilon_{m})}}{2 n_{i}}}$}
%\end{align*}
%\EndArmElim
%\ArmElimV
%\State \For{$i\in B_m$}
%\State For each arm $i \in B_{m}$, remove arm ${i}$ from $B_{m}$ if
\State \For{$i\in B_m$}
\vspace{-4mm}
\State \If{$(\hat{r}_{i} + s_i  < \tau - s_i)$ or $(\hat{r}_{i} - s_i > \tau + s_i)$}
\State $B_m\leftarrow B_m\backslash\{i\}$\hspace{4mm} (Arm deletion)
\EndIf
\EndFor
%\begin{align*}
%\hat{r}_{i} + s_i  < \tau - s_i,\hspace{1mm} \mbox{ or } \hspace{1mm}\hat{r}_{i} - s_i  > \tau + s_i \\
%% \text{where $s_i=\sqrt{\frac{\rho\psi_{m}\hat{v}_{i}\log{( T\epsilon_{m})}}{4 n_{i}} + \frac{\rho\psi_{m} \log{(T\epsilon_{m})}}{4 n_{i}}}$}
%\end{align*}
%\EndFor
%\EndArmElimV
\vspace{-2mm}
\State \If{$t\geq N_{m}$ and $m \leq M$}
%\ResetParam
\State \textbf{Reset Parameters}
\State $\epsilon_{m+1}\leftarrow\frac{\epsilon_{m}}{2}$
\State $B_{m+1} \leftarrow B_{m}$
\State $\psi_{m+1}\leftarrow \frac{T\epsilon_{m+1}}{128(\log(\frac{3}{16}K\log K))^{2}}$
\State $\ell_{m+1}\leftarrow\left\lceil \frac{2\psi_{m+1}\log( T\epsilon_{m+1})}{\epsilon_{m+1}} \right\rceil$
\State $N_{m+1} \leftarrow t + |B_{m+1}|\ell_{m+1}$
\State $m \leftarrow m+1$
%\EndResetParam
\EndIf
\EndFor
\State \textbf{Output:} $\hat{S}_{\tau}=\lbrace i: \hat{r}_{i}\geq \tau \rbrace$.
\end{algorithmic}
\end{algorithm}


%Also because of the said condition, like \cite{liu2016modification} we also claim that AugUCB is an anytime algorithm.

%
\vspace{-2mm}
\section{Theoretical Results}
\label{results}
\section{Results}
\label{sec:results}
% \begin{table}[!ht]
% \centering
% \caption{My caption}
% \label{my-label}
% \begin{tabular}{|p{1.2cm}|l|l|l|l|l|l|}
% \hline
% \multirow{2}{*}{\begin{tabular}[c]{@{}c@{}}Feature \\ Name\end{tabular}} & \multicolumn{2}{c}{Vt Classifier}                                       & \multicolumn{4}{|c|}{Size classifier}                                                                                                   \\ \cline{2-7} 
%                                                                          & \multicolumn{1}{c|}{1} & \multicolumn{1}{c|}{2} & \multicolumn{1}{c|}{ 1} & \multicolumn{1}{c|}{ 2} & \multicolumn{1}{c|}{3} & \multicolumn{1}{c|}{4} \\ \hline
% Sub-circuit                                                              & \multicolumn{1}{l|}{}        &                               &                               &                               &                              &                              \\ \hline
% Gate Type                                                                & \multicolumn{1}{l|}{}        &                               &                               &                               &                              &                              \\ \hline
% LNS                                                  & \multicolumn{1}{l|}{}        &                               &                               &                               &                              &                              \\ \hline
% \#Fanins                                                                 & \multicolumn{1}{l|}{}        &                               &                               &                               &                              &                              \\ \hline
% \#Fanouts                                                                & \multicolumn{1}{l|}{}        &                               &                               &                               &                              &                              \\ \hline
% \begin{tabular}[c]{@{}l@{}}\#Negative \\ Slack Paths\end{tabular}                                          & \multicolumn{1}{l|}{}        &                               &                               &                               &                              &                              \\ \hline
% Slack                                                                    & \multicolumn{1}{l|}{}        &                               &                               &                               &                              &                              \\ \hline
% \end{tabular}
% \end{table}
% \begin{table*}[!ht]
% \centering
% \caption{result3}
% \label{results3}
% \begin{tabular}{|l|c|l|l|l|l|l|l|l|l|l|l|}
% \hline
% \multicolumn{1}{|c|}{\begin{tabular}[c]{@{}c@{}}Benchmark \\  Name\end{tabular}} & \begin{tabular}[c]{@{}c@{}}Number \\ Of gates\end{tabular} & \multicolumn{1}{c|}{\begin{tabular}[c]{@{}c@{}}Target \\ Delay\end{tabular}} & \begin{tabular}[c]{@{}l@{}}Inital \\ Delay\end{tabular} & \multicolumn{2}{c|}{SVM}                                         & \multicolumn{2}{c|}{Final}                                      & \multicolumn{2}{c|}{Igor Markov}                               & \multicolumn{2}{c|}{Flach}                                     \\ \hline
% \multicolumn{1}{|c|}{}                                                           &                                                            & \multicolumn{1}{c|}{}                                                        & \multicolumn{1}{c|}{}                                   & \multicolumn{1}{c|}{Delay (ns)} & \multicolumn{1}{c|}{Power (W)} & \multicolumn{1}{c|}{Delay (ns)} & \multicolumn{1}{c|}{Power(W)} & \multicolumn{1}{c|}{Delay(ns)} & \multicolumn{1}{c|}{Power(W)} & \multicolumn{1}{c|}{Delay(ns)} & \multicolumn{1}{c|}{Power(W)} \\ \hline
% DMA\_fast                                                                        & 25.3K                                                      &                                                                              &                                                         &                                 &                                &                                 &                               &                                &                                                0.299    &       &                               \\ \hline
% DMA\_slow                                                                        & 25.3K                                                      &                                                                              &                                                         &                                 &                                &                                 &                               &                                &                                                       0.145   &     &                               \\ \hline
% pci\_fast                                                              & 33.2K                                                      &                                                                              &                                                         &                                 &                                &                                 &                               &                                &                                                     0.183     &     &                               \\ \hline
% pci\_slow                                                              & 33.2K                                                      &                                                                              &                                                         &                                 &                                &                                 &                               &                                &                                                  0.111         &    &                               \\ \hline
% des\_perf\_fast                                                                  & 111K                                                       &                                                                              &                                                         &                                 &                                &                                 &                               &                                &                                                         1.842   &   &                               \\ \hline
% des\_perf\_slow                                                                  & 111K                                                       &                                                                              &                                                         &                                 &                                &                                 &                               &                                &                                                          0.614   &  &                               \\ \hline
% vga\_lcd\_fast                                                                   & 165K                                                       &                                                                              &                                                         &                                 &                                &                                 &                               &                                &                                                            0.471  & &                               \\ \hline
% vga\_lcd\_slow                                                                   & 165K                                                       &                                                                              &                                                         &                                 &                                &                                 &                               &                                &                                                            0.351  & &                               \\ \hline
% b19\_fast                                                                        & 219K                                                       &                                                                              &                                                         &                                 &                                &                                 &                               &                                &                                                           0.771   & &                               \\ \hline
% b19\_slow                                                                        & 219K                                                       &                                                                              &                                                         &                                 &                                &                                 &                               &                                &                                                             0.583 & &                               \\ \hline
% leon3mp\_fast                                                                    & 649K                                                       &                                                                              &                                                         &                                 &                                &                                 &                               &                                &                                                             1.487 & &                               \\ \hline
% leon3mp\_slow                                                                    & 649K                                                       &                                                                              &                                                         &                                 &                                &                                 &                               &                                &                                                            1.341  & &                               \\ \hline
% netcard\_fast                                                                    & 959K                                                       &                                                                              &                                                         &                                 &                                &                                 &                               &                                &                                                            1.861  & &                               \\ \hline
% netcard\_slow                                                                    & 959K                                                       &                                                                              &                                                         &                                 &                                &                                 &                               &                                &                                                           1.770  &  &                               \\ \hline
% \end{tabular}
% \end{table*}


% Please add the following required packages to your document preamble:
% \usepackage{multirow}
% Please add the following required packages to your document preamble:
% \usepackage{multirow}
% \begin{table}[]
% \centering
% \caption{My caption}
% \label{my-label}
% \begin{tabular}{|l|l|l|l|l|l|}
% \hline
% \multirow{3}{*}{Benchmark} & \multicolumn{5}{c|}{Runtime}                                                            \\ \cline{2-6} 
%                            & \multirow{2}{*}{Igor Markov} & \multirow{2}{*}{Flach} & \multicolumn{3}{c|}{\textit{MLTimer}} \\ \cline{4-6} 
%                            &                              &                        & SVM  & Delay Recovery  & Total  \\ \hline
% DMA\_fast                  &                              &                        &      &                 &        \\ \hline
% DMA\_slow                  &                              &                        &      &                 &        \\ \hline
% pci\_fast        &                              &                        &      &                 &        \\ \hline
% pci\_brdige32\_slow        &                              &                        &      &                 &        \\ \hline
% vga\_lcd\_fast             &                              &                        &      &                 &        \\ \hline
% vga\_lcd\_slow             &                              &                        &      &                 &        \\ \hline
% des\_perf\_fast            &                              &                        &      &                 &        \\ \hline
% des\_perf\_slow            &                              &                        &      &                 &        \\ \hline
% b19\_fast                  &                              &                        &      &                 &        \\ \hline
% b19\_slow                  &                              &                        &      &                 &        \\ \hline
% leon3mp\_fast              &                              &                        &      &                 &        \\ \hline
% leon3mp\_slow              &                              &                        &      &                 &        \\ \hline
% netcard\_fast              &                              &                        &      &                 &        \\ \hline
% netcard\_slow              &                              &                        &      &                 &        \\ \hline
% \end{tabular}
% \end{table}


% Please add the following required packages to your document preamble:
% \usepackage{multirow}
% Please add the following required packages to your document preamble:
% \usepackage{multirow}
% Please add the following required packages to your document preamble:
% \usepackage{multirow}
\begin{table*}[!t]
\caption{Leakage power and Runtime comparisons between the baseline greedy algorithm and the \textit{MLTimer} algorithm on the ISPD 2012 benchmarks. Implementation 1 is the baseline implementation(non-SVM,non-adaptive timing analysis), Implementation 2 is with SVM and non-adaptive timing analysis, Implementation 3 is with non-SVM and adaptive timing analysis and Implementation 4 is with SVM and adaptive timing analysis. It can be seen that using just SVM improves the solution quality greatly, while using just the adaptive timing analysis improves the runtime. A combination of both improves the runtime and solution qualtiy.}
\label{tab:tab5}

\begin{tabular}{|l|l|l|l|l|l|l|l|l|l|}
\hline
\multirow{2}{*}{Benchmarks} & \multirow{2}{*}{\#Gates} & \multicolumn{2}{l|}{Implementation 1}                                                                                                         & \multicolumn{2}{l|}{Implementation 2}                                                                                                           & \multicolumn{2}{l|}{Implementation 3}                                                                                                        & \multicolumn{2}{l|}{Implementation 4}                                                                                                        \\ \cline{3-10} 
                            &                          & \begin{tabular}[c]{@{}l@{}}Run-\\ time \\ (mins)\end{tabular} & \begin{tabular}[c]{@{}l@{}}Leakage \\ Power\\ (W)\end{tabular} & \begin{tabular}[c]{@{}l@{}}Run-\\ time\\ (mins)\end{tabular} & \begin{tabular}[c]{@{}l@{}}Leakage \\ Power\\ \\ (W)\end{tabular} & \begin{tabular}[c]{@{}l@{}}Run-\\ time\\ (mins)\end{tabular} & \begin{tabular}[c]{@{}l@{}}Leakage\\  Power\\ (W)\end{tabular} & \begin{tabular}[c]{@{}l@{}}Run-\\ time\\ (mins)\end{tabular} & \begin{tabular}[c]{@{}l@{}}Leakage \\ Power\\ (W)\end{tabular} \\ \hline
DMA\_fast                   & 23,000                   & 16                                                            & 0.79                                                           & 14.00                                                        & 0.30                                                              & 14.00                                                        & 0.79                                                           & 13.00                                                        & 0.30                                                           \\ \hline
pci\_bridge32\_fast         & 30,000                   & 37                                                            & 0.25                                                           & 17.00                                                        & 0.14                                                              & 17.00                                                        & 0.24                                                           & 17.00                                                        & 0.14                                                           \\ \hline
des\_perf\_fast             & 102,000                  & 219                                                           & 1.73                                                           & 164.00                                                       & 1.80                                                              & 190.00                                                       & 1.73                                                           & 130.00                                                       & 1.80                                                           \\ \hline
vga\_lcd\_fast              & 148,000                  & 384                                                           & 2.80                                                           & 139.00                                                       & 0.47                                                              & 207.00                                                       & 2.72                                                           & 77.00                                                        & 0.47                                                           \\ \hline
b19\_fast                   & 213,000                  & 547                                                           & 2.13                                                           & 239.00                                                       & 0.75                                                              & 366.00                                                       & 2.13                                                           & 174.00                                                       & 0.75                                                           \\ \hline
leon3mp\_fast               & 540,000                  & 2,046                                                         & 4.00                                                           & 875.00                                                       & 1.49                                                              & 716.00                                                       & 4.00                                                           & 639.00                                                       & 1.49                                                           \\ \hline
netcard\_fast               & 861,000                  & 1,033                                                         & 2.09                                                           & 519.00                                                       & 1.77                                                              & 609.00                                                       & 2.07                                                           & 306.00                                                       & 1.77                                                           \\ \hline
\end{tabular}
\end{table*}

\begin{table*}[!ht]
%\centering
\caption{Leakage power comparisons with ISPD 2012 contest winners and other state of the art works. We use geometric mean to calculate the efficiency of our proposed solution. We exclude the infeasible solutions in our mean calculation. All the solutions reported below have no timing violations.}
\label{tab:tab6}

    \begin{tabular}{|l|l|l|p{1.2cm}|p{1.6cm}|p{1.6cm}|p{1cm}|l|p{1.2cm}|}
\hline
\multirow{2}{*}{Benchmark} & \multirow{2}{*}{\begin{tabular}[c]{@{}l@{}}Number \\ of gates\end{tabular}} & \multicolumn{5}{c|}{Leakage Power (W)} & \multicolumn{2}{c|}{Runtime (mins)}\\ \cline{3-9} 
    &  & \cite{hu:12}  & NTUgs & UFRGSgs & Powervalve & \textbf{Ours} & \cite{hu:12} & \textbf{Ours}\\ \hline
    \texttt{DMA\_fast} & 23,000 & 0.30  & 0.51 & 0.32 & 0.31 & 0.30 & 13.90 & 13.30\\ \hline
    \texttt{DMA\_slow} & 23,000  & 0.15  & 0.21 & 0.16 & 0.15 & 0.14 & 9.90 & 7.51 \\ \hline
    \texttt{pci\_fast} & 30,000 & 0.18  & 0.51 & 0.17 & 0.23 & 0.14 & 13.00 & 17.10
     \\ \hline
    \texttt{pci\_slow} & 30,000 & 0.11   & 0.20 & 0.12 & 0.12 & 0.09 & 10.20 & 9.32 \\ \hline
    \texttt{des\_perf\_fast} & 102,000 & 1.84 & 2.39 & 3.52 & 2.32 & 1.80  & 82.70 & 130.40 \\ \hline
    \texttt{des\_perf\_slow} & 102,000 & 0.61 & 0.67 & 0.88 & 0.70 & 0.64 & 70.10 & 43.50 \\ \hline
    \texttt{vga\_lcd\_fast} & 148,000 & 0.47 & 0.76 & 0.58 & 0.77 & 0.47 & 45.60 & 77.32\\ \hline
    \texttt{vga\_lcd\_slow} & 148,000 & 0.35 & 0.42 & 0.38 & 0.39 & 0.37 & 87.50 & 50.40 \\ \hline
    \texttt{b19\_fast} & 213,000 & 0.77 & 2.71 & - & 4.49 & 0.75 & 206.50 & 174.11 \\ \hline
    \texttt{b19\_slow} & 213,000 & 0.58 & 0.63 & 0.61 & 0.74 & 0.61 & 213.90 & 102.20\\ \hline
    \texttt{leon3mp\_fast} & 540,000 & 1.49 & -&  - & 4.94 & 1.49 & 1,323.20 & 639.40\\ \hline
    \texttt{leon3mp\_slow} & 540,000 & 1.34 & 1.42 & 1.79 & 2.96 & 1.30 & 1,274.20 & 325.13  \\ \hline
    \texttt{netcard\_fast} & 861,000 & 1.86 & 2.01 & 2.30 & 2.97 & 1.86 & 1,096.90 & 306.57\\ \hline
    \texttt{netcard\_slow} & 861,000 & 1.77 & 1.77 & 1.97 & 1.94 & 1.77 & 299.90 & 164.14\\ \hline
Geometric mean &  & $1.03\times$ & $1.52\times$ & $1.13\times$ & $1.57\times$ &   & $1.44\times$ & \\ \hline
\end{tabular}
\end{table*}

% Please add the following required packages to your document preamble:
% \usepackage{multirow}
\begin{table*}[!ht]
\centering
\caption{Leakage power comparisons with \cite{hu:13} on the  ISPD 2013 contest benchmark. All the solutions reported below are violation free. It can be observed that \texttt{MLTimer} outperforms \cite{hu:13} both with respect to leakage power and runtime on the larger benchmarks. The detailed results for other benchmarks were not reported in \cite{hu:13}.}
\label{tab:tab34}
\begin{tabular}{|l|l|l|l|l|l|}
\hline
\multirow{2}{*}{Benchmark} & \multirow{2}{*}{Gates} & \multicolumn{2}{l|}{\texttt{MLTimer}}                                                                                                  & \multicolumn{2}{l|}{\cite{hu:13}}                                                                                                  \\ \cline{3-6} 
                           &                        & \begin{tabular}[c]{@{}l@{}}Run-\\ time\\ (mins)\end{tabular} & \begin{tabular}[c]{@{}l@{}}Leakage\\ Power\\ (mW)\end{tabular} & \begin{tabular}[c]{@{}l@{}}Run-\\ time\\ (mins)\end{tabular} & \begin{tabular}[c]{@{}l@{}}Leakage \\ Power\\ (mW)\end{tabular} \\ \hline
usb\_phy\_fast             & 510                    & 0.48                                                         & 2.03                                                           & \textbf{0.21}                                                & \textbf{1.56}                                                   \\ \hline
usb\_phy\_slow             & 510                    & \textbf{0.11}                                                & 1.13                                                           & 0.17                                                         & \textbf{1.07}                                                   \\ \hline
pci\_bridge32\_fast        & 28,000                 & 20.83                                                        & 116.87                                                         & \textbf{12.00}                                               & \textbf{101.90}                                                 \\ \hline
pci\_bridge32\_slow        & 28,000                 & 6.78                                                         & 58.91                                                          & \textbf{5.39}                                                & \textbf{58.83}                                                  \\ \hline
fft\_fast                  & 31,000                 & 40.00                                                        & 320.37                                                         & \textbf{32.58}                                               & \textbf{305.29}                                                 \\ \hline
fft\_slow                  & 31,000                 & 25.00                                                        & 96.69                                                          & \textbf{17.40}                                               & \textbf{93.10}                                                  \\ \hline
cordic\_slow               & 42,000                 & \textbf{94.40}                                               & \textbf{397.81}                                                & 98.39                                                        & 511.91                                                          \\ \hline
des\_perf\_slow            & 104,000                & 88.18                                                        & 386.41                                                         & \textbf{62.30}                                               & \textbf{375.80}                                                 \\ \hline
edit\_dist\_fast           & 121,000                & \textbf{163.10}                                              & \textbf{572.12}                                                & 170.60                                                       & 619.30                                                          \\ \hline
edit\_dist\_slow           & 121,000                & \textbf{56.34}                                               & \textbf{423.50}                                                & 107.20                                                       & 465.60                                                          \\ \hline
matrix\_mult\_slow         & 153,000                & \textbf{139.80}                                              & \textbf{482.23}                                                & 212.60                                                       & 499.90                                                          \\ \hline
netcard\_fast              & 884,000                & \textbf{372.70}                                              & \textbf{5,157.93}                                              & 716.80                                                       & 5271.80                                                         \\ \hline
netcard\_slow              & 884,000.               & \textbf{297.12}                                              & \textbf{5,102.25}                                              & 439.60                                                       & 5183.89                                                         \\ \hline
GEOMETRIC MEAN             &                &                                               &                                               & 1.005                                                       & 1.005                                                         \\ \hline

\end{tabular}
\end{table*}

% \begin{table*}[!ht]
% \begin{center}
% \label{results3}
% \begin{tabular}{|p{2.1cm}|p{2cm}|p{2cm}|p{2cm}|p{2cm}|p{1.5cm}|}
% \hline
% Benchmark & $\#$ gates & %\multicolumn{2}{|c|}
% {\textit{MLTimer}} &%\multicolumn{2}{|c|} 
% {Igor Markov} ~\cite{hu:12} & Improvement \\
% \hline
%   &  & Leakage Power (W)      %& Running Time    
%   & Leakage Power (W) & \\% & Running Time \\ 
 
% \hline
% %USB\_PHY &536 &$<$1s &$<$1s & - \\
% \hline
% DMA\_fast & 25.3K & 0.08W %& 17m
% & 0.299W & 73\% \\ %& 13m\\
% \hline
% %DMA\_slow & 25.3 & 0.134W %& 1m44s
% %& 0.145W & 7\%  \\% & 9.9m\\
% %\hline
% pci\_bridge	&	33.2K	&	0.1331W %&	1m29s
% &	0.183W & 27\%\\%	&	13m \\
% \hline
% %pci\_bridge\_slow	&	33.2K	&	0.07W	%&	8m
% %&	0.111W & 36\% \\%	&	11m \\
% %\hline 
% b19	&	219K  &		.58W	%&	9h
% &	0.771W & 24\% \\%	&	206m \\
% \hline
% %b19\_slow 	&	219K &		0.486W%	&	9h	
% %&	0.583W & 19.21\% \\	%&	213m \\
% %\hline
% Des\_perf & 165K & .546W %& 1331m       
% & .471W & \-15.3\% \\%    & 45m \\
% \hline
% netcard & 959k & 1.8W %& 2046m 
% & 1.861W & 6.01 \% \\% & 1096m \\
% \hline
% leon3mp & 649K & 2W %&  2816m 
% & 1.487W & \-34.5\% \\ %& 1323.2 \\
% \hline
% Average & & & & 21.37\% \\ \hline
% %leon3mp\_slow & 649K & 2W &  2816m & 1.487W & 1323.2 \\
% %\hline
% \end{tabular}
% \caption{Leakage and Running Time Comparisons for ISPD benchmarks between \textit{MLTimer} and Igor Markov. In the table, {\bf h}, {\bf m} and {\bf s} stand for hours, minutes and seconds respectively.}


% \end{center}
% \end{table*}
\subsection{Comparisons with state-of-the-art}

The performance of our proposed algorithm is shown in Table~\ref{tab:tab5}. A simple greedy algorithm, implemented for obtaining the final $V_t$ and $size$ values, serves as the baseline algorithm. It can be seen that our \textit{MLTimer} implementation outperforms the baseline algorithm by 46\% in terms of solution quality. It can also be seen that the SVM module improves the solution quality and the adaptive timing analysis module improves the runtime. 

We compare the performance of our algorithm with ~\cite{hu:12} which is the best performing heuristic based algorithm reported so far in the literature. We use the ISPD 2012 benchmark set and SHAKTIC to quantify the performance our algorithm. In comparing with the state-of-the-art techniques we make the following observations:
\begin{itemize}
\item Our solution outperforms the top 3 submissions of the ISPD 2012 contest NTUgs, UFRGSgs and Powervalve by 52\%,13\% and 57\% respectively.
\item Our solution outperforms \cite{hu:12} both in terms of average runtime and solution quality by 44\% and 3\% respectively. Table~\ref{tab:tab6} highlights the performance of \textit{MLTimer} algorithm in terms of runtime and solution quality. This is because as most of the circuits share a large  number of repeating sub-circuits whose value is accurately predicted by the SVM engine and hence these gates do not undergo delay and power recovery algorithm leading to savings in runtime. 
\item It can be seen from Table~\ref{tab:tab34} that our tool outperforms \cite{hu:13} which is an extension of \cite{hu:12}. It can be observed that while \textit{MLTimer} underperforms for the smaller benchmarks, it significantly outperforms \cite{hu:13} on the larger benchmarks. Although the overall improvement in solution quality is around 0.004\%, the improvement in the larger benchmarks is around 53\% for the runtime and 10\% for solution quality.
\item In Table~\ref{tab:tab9} we compare our implementation with a commercial synthesis tool and our implementation of \cite{hu:12}. It can be observed our proposed solution performs significantly better than the commercial tool in terms of leakage power. 
\end{itemize}

\begin{table}[!t]
    \caption{The Table comparing the performance of \textit{MLTimer} versus a commercial synthesis tool on SHAKTIC. We see that the solution quality is 57\% better than that of the tool.}
    \label{tab:tab9}

    \centering
    \begin{tabular}{|l|l|l|l|l|l|l|}
        \hline
        \textbf{Metric}           & \multicolumn{2}{c|}{Commercial Tool}                                                                                     &        &            & \multicolumn{2}{c|}{Percentage Improvement} \\ \hline
                         & \begin{tabular}[c]{@{}l@{}}$LV_t$\\  synthesis\end{tabular} & \begin{tabular}[c]{@{}l@{}}Mixed $V_t$ \\ synthesis\end{tabular} & \cite{hu:12} & \textit{MLTimer} & Tool              & \cite{hu:12}       \\ \hline
                    %         \textbf{Runtime (mins)}    & 38                                                       & 14                                                            & 87     & 64         & -78.12           &  26.44       \\ \hline
                             \textbf{Leakage power (W)} & 5                                                        & 1                                                             & 0.59  & 0.43       & 57        & 27        \\ \hline
    \end{tabular}

\end{table}


\subsection{Analysis of the Learning Module}


\begin{table}[!t]
\caption{Table showing the weights assigned to each feature at each stage of the $V_t$ and $size$ classifiers. An extremely low magnitude implies that the corresponding feature does not contribute significantly to the output and can thus be discarded. However it can be seen that none of the features chosen fall into that category.}
\label{tab:tab7}

    \centering

\begin{tabular}{|l|l|l|l|l|l|l|l|}
\hline
    \multirow{2}{*}{\textbf{Feature}}       & \multicolumn{2}{c|}{$\mathbf{V_t}$} & \multicolumn{5}{c|}{\textbf{size}}             \\ \cline{2-8} 
                               & 1          & 2          & 1     & 2     & 3     & 4     & 5     \\ \hline
    \textbf{Sub-circuit}                    & -0.88      & -0.43      & -0.58 & 1.32  & -0.08 & 0.44  & 0.16  \\ \hline
    \textbf{Gate type}                      & -0.19      & -0.43      & -0.96 & -0.81 & 0.42  & -0.40 & 0.29  \\ \hline
    \textbf{LNS}                            & 1.09       & 0.33       & -0.07 & -0.11 & 0.76  & 0.76  & 0.08  \\ \hline
    \textbf{Number of Fanins}               & 2.64       & 0.04       & 3.27  & -2.38 & -1.10 & -0.34 & -1.26 \\ \hline
    \textbf{Number of Fanouts}              & -2.92      & -0.29      & -4.08 & -0.53 & 1.82  & 0.50  & -1.07 \\ \hline
    \textbf{Number of Negative Slack Paths} & 0.33       & -0.16      & 0.50  & 0.40  & -0.22 & 0.21  & -0.68 \\ \hline
    \textbf{Slack}                          & 1.89       & 0.35       & 1.22  & -3.20 & -1.64 & -1.41 & 0.37  \\ \hline
\end{tabular}

\end{table}


The learning module forms a critical component of our framework as it serves to reduce the runtime by using a simple SVM model that uses seven features.  A complex ML model with large number of redundant features might cause runtime overheads due to i) complex training procedure ii) complicated inference procedure, and; iii) reduced interpretability of the ML model. Hence there is a need to eliminate the redundant features in order to simplify the learning module. Logistic regression was performed to estimate the importance of the chosen features. The Logistic Regression model was initially trained on the set of chosen features and the importance of each feature,  obtained via the coefficient assigned by the model,  is quantified in Table~\ref{tab:tab7}.  It can be observed that none of the feature weights have extremely low value and hence cannot be eliminated.


%As mentioned earlier, an improperly trained learning engine could initialize the netlist to a sub-optimal configuration leading to more delay and power recovery cycles than necessary thereby increasing the runtime overhead. The thresholding function plays an important role in predicting the final choice ($V_t$/$size$) for a given cell. We use the b19\_fast benchmark to show the impact of varying the thresholding function on the solution quality of the SVM engine. We show the impact of the thresholding function in table ~\ref{results4}. We see that as the thresholding function increases the runtime goes up. This is because the number of gates that are marked unsure increases causing more delay and power optimizations. We us class probability to determine the class label ($V_t$/$size$). We use a thresholding value of $0.75$ for both the $V_t$ classifiers while we use a thresholding value of x and y for the $size$ classifiers. 





% \begin{table*}[!t]
% \parbox{.3\linewidth}{
% \begin{center}

% \begin{tabular}{|p{3cm}|p{1.3cm}|}
% \hline
% Metric & Number \\ \hline
% Total gates & 333\\
% \hline
% Combinational gate types &  11 \\ \hline
% Sequential gates & 1 \\ \hline
% $V_t$ choices & 3 \\ \hline
% $size$ choices & 10 \\ \hline
% $V_{cc}$ and $gnd$ cells & 2 \\ \hline
% %leon3mp\_slow & 649K & -6401 &  -6479 & 1W & 47s \\
% %\hline
% \end{tabular}
% \caption{Library statistics}
% \label{tab:lib}
% \end{center}
% %\end{table*}
% }
% \hfill
% \parbox{.6\linewidth}{
% %\begin{table*}[!t]
% \begin{center}

% \begin{tabular}{|p{2cm}|p{1cm}|p{1cm}|p{1.6cm}|p{1.6cm}|p{1.6cm}|}
% \hline
% Benchmark & \#Input & \#Output & \#Comb cell & \#Seq cell & \#Total cell \\
% \hline
% DMA &  683 & 276 & 23109 & 2192 & 25301\\ \hline
% pci & 160 & 201& 29844& 3359& 33203\\ \hline
% des\_perf &  234 & 140 & 102427 & 8802& 111229\\ \hline
% vga\_lcd &  85 & 99 &  147812 & 17079 & 164891\\ \hline
% b19 &  22 & 25 &  212674 & 6594 & 219268\\ \hline
% leon3mp & 254 & 79 & 540352 &  108839 & 649191\\ \hline
% netcard &  1836 & 10 & 860949 & 97831 & 958780\\ \hline

% %leon3mp\_slow & 649K & -6401 &  -6479 & 1W & 47s \\
% %\hline
% \end{tabular}
% \caption{Benchmark statistics}
% \label{tab:benchmark}
% \end{center}
% }
% \end{table*}


% Please add the following required packages to your document preamble:
% \usepackage{booktabs}
% \usepackage{multirow}
% Please add the following required packages to your document preamble:
% \usepackage{multirow}
% Please add the following required packages to your document preamble:
% \usepackage{multirow}
% Please add the following required packages to your document preamble:
% \usepackage{multirow}

% \begin{table*}[!t]
% \parbox{.5\linewidth}{
% \begin{center}

% \label{tab:log}
% \begin{tabular}{|p{2.5cm}|p{3cm}|}
% \hline
% Feature & Weight \\
% \hline
% Gate footprint &-0.8832794232852276 \\ \hline
% Gateid & -0.1914242984939835 \\ \hline
% Local negative slack & 1.090781658200261 \\ \hline
% \#Fanins & 2.642338600894165  \\ \hline
% \#Fanouts & -2.92081747517804  \\ \hline
% \#Negative slack paths & 0.3349856667382776  \\ \hline
% Slack & 1.894864001133556 \\ \hline

% %leon3mp\_slow & 649K & -6401 &  -6479 & 1W & 47s \\
% %\hline
% \end{tabular}
% \caption{Feature Weights for the first $V_t$ classifier }
% \end{center}
% %\end{table*}
% }
% \hfill
% %\begin{table*}[!h]
% \parbox{.5\linewidth}{
% \begin{center}

% \label{tab:log2}
% \begin{tabular}{|p{2.5cm}|p{3cm}|}
% \hline
% Feature & Weight \\
% \hline
% Gate footprint & 0.03119793516364029
% \\ \hline
% Gateid &  0.346427255804566\\ \hline
% Local negative slack & -0.01019723367101055\\ \hline
% \#Fanins & -0.007490175140922838 \\ \hline
% \#Fanouts & -0.2180352793079041 \\ \hline
% \#Negative slack paths & -0.06062286295401311  \\ \hline
% Slack & 0.663389798807628 \\ \hline

% %leon3mp\_slow & 649K & -6401 &  -6479 & 1W & 47s \\
% %\hline
% \end{tabular}
% \caption{Feature Weights for the second stage $V_t$ classifier }
% \end{center}
% }
% \end{table*}

The efficiency of the SVM engine is analyzed in Table~\ref{tab:tab8}. We see that on an average the SVM engine is able to recover a significant amount of power in a short amount of time. However, It can be observed that the solution provided by the SVM engine is not optimal hence  the delay and leakage power recovery steps are used to further optimize the solution provided by the learning step.

 \begin{table*}[!t]
  \caption{Leakage and Running Time Comparisons for ISPD benchmarks and ShaktiC with just SVM. In the table, {\bf h}, {\bf m} and {\bf s} stand for hours, minutes and seconds respectively. It can be seen that with the exception of leon3mp our SVM implementation is able to recover significant delay and power.} \
\label{tab:tab8}

     \begin{center}
 \begin{tabular}{|p{4.2cm}|p{2cm}|p{2.2cm}|p{2cm}|p{2cm}|p{2cm}|}
 \hline
    \textbf{Benchmark} & \textbf{Gate count} & \textbf{Initial Worst Negative Slack (WNS)} & \multicolumn{3}{|c|}{ \textit{MLTimer}}  \\
 \hline
   &   & & WNS (ps) &  Leakage Power (W)      & Running Time          \\
 \hline
     \texttt{ DMA\_fast} & 25,300&  -1485 & -774 &0.09  & 3s \\
 \hline
     \texttt{pci\_bridge32\_fast}	& 33,200& -1881 & -2284	&	0.18   &	3s	 \\
 \hline
     \texttt{des\_perf\_fast} &  102,000 & -669 & -1029 &.316 & 1m        \\
 \hline
     \texttt{vga\_lcd\_fast} & 148,000 & -1254 & -2964 &.29 & 1m          \\
\hline

     \texttt{b19\_fast}	&	219,000 & -2835 & -1738	&	1.6 	&	15s	\\ \hline
     \texttt{leon3mp\_fast} & 649,000 & -6401 & -3913 & 21 &  47s \\
 \hline

     \texttt{netcard\_fast} & 959,000 & -4102 &-3268 & 8 & 1m  \\ \hline
     \texttt{ShaktiC} & 174,756 & -5199 & -1067 & 0.67 & 1m \\ 
 \hline
 \end{tabular}
 \end{center}

 \end{table*}
\section{Conclusion}
\label{sec:conclusion}
Leakage optimization  techniques have been studied extensively for more than a decade.  However, the lack of a robust algorithm that is optimal in terms of both execution time and solution quality motivates research in this area. It is seen that varying window size adaptively according to the status of the timing updates produces faster solutions than for a fixed window size. The proposed \textit{MLTimer} algorithm improves the running-time considerably while still retaining the solution quality of a greedy heuristic. It is observed that for large circuits \textit{MLTimer} with initial configuration provided by SVM performs significantly better than when used with power optimal configuration as initial solution.  Extending the concepts involved in the construction of \textit{MLTimer} to other steps of EDA including placement and routing is an interesting direction for future work.
% * <sristisravan@gmail.com> 2017-06-28T10:13:29.636Z:
% 
% Check "... the lack of a robust heuristic that optimal in terms of both.... "
% 
% ^.

%\begin{table*}[t]
% \begin{center}
% \caption{Leakage and Running Time Comparisons for ISPD benchmarks with just SVM and delay recovery. In the table, {\bf h}, {\bf m} and {\bf s} stand for hours, minutes and seconds respectively.}
% \label{results2}
% \begin{tabular}{|p{1.7cm}|p{2cm}|p{2cm}|p{2cm}|p{2cm}|}
% \hline
% Benchmark & $\#$ gates & \multicolumn{3}{|c|}{\textit{MLTimer}}  \\
% \hline
%   &  & Delay(ps) &  Leakage Power (W)      & Running Time          \\
% \hline
% %USB\_PHY &536 &$<$1s &$<$1s & - \\
% \hline
% DMA_fast & 25.3K & &0.08W & 17m \\
% \hline
% DMA_slow & 25.3 & &0.134W & 1m44s \\
% \hline
% pci_bridge_fast	& 33.2K&	&	0.1331W &	1m29s	 \\
% \hline
% pci_bridge_slow	& 	33.2K&	&	0.07W	&	8m	\\
% \hline 
% b19_fast	&	219K  &	&	.58W	&	9h	\\
% \hline
% b19_slow 	&	219K & &		0.486W	&	9h	 \\
% \hline
% Des_perf_fast &  165K & &.546W & 1331m        \\
% \hline
% Des_perf_slow &  165K & & .546W & 1331m         \\
% \hline
% vga_lcd_slow & 165K & & .546W & 1331m          \\
% \hline
% vga_lcd_slow & 165K &  &.546W & 1331m          \\
% \hline
% netcard_fast & 959k & & 1.8W & 2046m  \\
% \hline
% netcard_slow & 959k & & 1.8W & 2046m \\
% \hline
% leon3mp_fast & 649K & & 2W &  --- \\
% \hline
% leon3mp_slow & 649K & & 2W &  --- \\
% \hline
% \end{tabular}
% \end{center}
%\end{table*}





%
%
%\vspace{-1em}
\vspace{-5mm}
\section{Numerical Experiments}
\label{expt}
In this section we present two experiments in two different environments.


\begin{figure}[!th]
    \centering
    \begin{tabular}{cc}
    %\setlength{\tabcolsep}{0.1pt}
    \subfigure[\Large\textwidth][\large Expt-$1$: $3$ Bernoulli-distributed arms (From Dr. Odalric's Draft).]
    %with $r_{i_{{i}\neq {*}}}=0.07$ and $r^{*}=0.1$
    {
    		\pgfplotsset{
		tick label style={font=\normalsize},
		label style={font=\normalsize},
		legend style={font=\normalsize},
		ylabel style={yshift=12pt},
		%legend style={legendshift=32pt},
		}
        \begin{tikzpicture}[scale=0.7]
      	\begin{axis}[
		xlabel={timestep},
		ylabel={Cumulative Regret},
		grid=major,
        %clip mode=individual,grid,grid style={gray!30},
        clip=true,
        %clip mode=individual,grid,grid style={gray!30},
  		legend style={at={(0.5,1.4)},anchor=north, legend columns=3} ]
      	% UCB
		
		\addplot table{Chapter6/results/NewExpt/Expt5/comp_subsampled_DUCB01.txt};
		\addplot table{Chapter6/results/NewExpt/Expt5/comp_subsampled_ETS01.txt};
		\addplot table{Chapter6/results/NewExpt/Expt5/comp_subsampled_ETS02.txt};
		\addplot table{Chapter6/results/NewExpt/Expt5/comp_subsampled_ETS03.txt};
		\addplot table{Chapter6/results/NewExpt/Expt5/comp_subsampled_ETS1E01.txt};
		\addplot table{Chapter6/results/NewExpt/Expt5/comp_subsampled_ETS1E02.txt};
		\addplot table{Chapter6/results/NewExpt/Expt5/comp_subsampled_TS01.txt};
		\addplot table{Chapter6/results/NewExpt/Expt5/comp_subsampled_OTS01.txt};
		\addplot table{Chapter6/results/NewExpt/Expt5/comp_subsampled_DTS01.txt};
      	
      	\legend{DUCB($\gamma=1-\frac{1}{4\sqrt{T}}$),ETSDAE1,ETSDAE2,ETSDAE3,ETSD1E1,ETSD1E2,TS,OTS,DTS($\gamma=1-\frac{1}{4\sqrt{T}})$}    
      	\end{axis}
      	\end{tikzpicture}
  		\label{psbandit:fig:1}
    }
    &
    \subfigure[\Large\textwidth][\large Expt-$1$: $3$ Bernoulli-distributed arms (From Dr. Odalric's Draft).]
    %with $r_{i_{{i}\neq {*}}}=0.07$ and $r^{*}=0.1$
    {
    		\pgfplotsset{
		tick label style={font=\normalsize},
		label style={font=\normalsize},
		legend style={font=\normalsize},
		ylabel style={yshift=12pt},
		%legend style={legendshift=32pt},
		}
        \begin{tikzpicture}[scale=0.7]
      	\begin{axis}[
		xlabel={timestep},
		ylabel={Cumulative Regret},
		grid=major,
        %clip mode=individual,grid,grid style={gray!30},
        clip=true,
        %clip mode=individual,grid,grid style={gray!30},
  		legend style={at={(0.5,1.4)},anchor=north, legend columns=3} ]
      	% UCB
		
		\addplot table{Chapter6/results/NewExpt/Expt5/comp_subsampled_DUCB01.txt};
		\addplot table{Chapter6/results/NewExpt/Expt5/comp_subsampled_ETS04.txt};
		\addplot table{Chapter6/results/NewExpt/Expt5/comp_subsampled_ETS1E03.txt};
		\addplot table{Chapter6/results/NewExpt/Expt5/comp_subsampled_ETS06.txt};
		\addplot table{Chapter6/results/NewExpt/Expt5/comp_subsampled_ETS1E04.txt};
		\addplot table{Chapter6/results/NewExpt/Expt5/comp_subsampled_TS01.txt};
		\addplot table{Chapter6/results/NewExpt/Expt5/comp_subsampled_OTS01.txt};
		\addplot table{Chapter6/results/NewExpt/Expt5/comp_subsampled_DTS01.txt};
		\addplot table{Chapter6/results/NewExpt/Expt5/comp_subsampled_ETS1E06.txt};
      	
      	\legend{DUCB($\gamma=1-\frac{1}{4\sqrt{T}}$),EAggrCPD1,CPD1-E1,EAggrCPD2,CPD2-E1,TS,OTS,DTS($\gamma=1-\frac{1}{4\sqrt{T}})$,CPD2-E1-N}    
      	\end{axis}
      	\end{tikzpicture}
  		\label{psbandit:fig:2}
    }
    \end{tabular}
    \caption{Cumulative regret for various bandit algorithms on a piecewise stochastic 3-armed bandit environment. }
    \label{fig:karmed1}
\end{figure}


%\begin{figure}[!th]
%    \centering
%    \begin{tabular}{c}
%    %\setlength{\tabcolsep}{0.1pt}
%    \subfigure[\Large\textwidth][\large Expt-$1$: $3$ Bernoulli-distributed arms (From Dr. Odalric's Draft).]
%    %with $r_{i_{{i}\neq {*}}}=0.07$ and $r^{*}=0.1$
%    {
%    		\pgfplotsset{
%		tick label style={font=\normalsize},
%		label style={font=\normalsize},
%		legend style={font=\normalsize},
%		ylabel style={yshift=12pt},
%		%legend style={legendshift=32pt},
%		}
%        \begin{tikzpicture}[scale=0.7]
%      	\begin{axis}[
%		xlabel={timestep},
%		ylabel={Cumulative Regret},
%		grid=major,
%        %clip mode=individual,grid,grid style={gray!30},
%        clip=true,
%        %clip mode=individual,grid,grid style={gray!30},
%  		legend style={at={(0.5,1.4)},anchor=north, legend columns=3} ]
%      	% UCB
%		
%		\addplot table{results/NewExpt/Expt5/comp_subsampled_DUCB01.txt};
%		\addplot table{results/NewExpt/Expt5/comp_subsampled_ETS04.txt};
%		\addplot table{results/NewExpt/Expt5/comp_subsampled_ETS1E03.txt};
%		\addplot table{results/NewExpt/Expt5/comp_subsampled_TS01.txt};
%		\addplot table{results/NewExpt/Expt5/comp_subsampled_OTS01.txt};
%		\addplot table{results/NewExpt/Expt5/comp_subsampled_DTS01.txt};
%      	
%      	\legend{DUCB($\gamma=1-\frac{1}{4\sqrt{T}}$),ETSDAE4,ETSD1E3,TS,OTS,DTS($\gamma=1-\frac{1}{4\sqrt{T}})$}    
%      	\end{axis}
%      	\end{tikzpicture}
%  		\label{fig:2}
%    }
%    \end{tabular}
%    \caption{Cumulative regret for various bandit algorithms on a piecewise stochastic 3-armed bandit environment. }
%    \label{fig:karmed2}
%\end{figure}


%
%\vspace{-1.2em}
\vspace{-1mm}
\section{Conclusion}
\label{conclusion}
In this chapter, we looked at the stochastic multi-armed bandit (SMAB) setting and discussed how it is important in the general reinforcement learning setup. We also looked at the various state-of-the-art algorithms in the literature for the SMAB setting and discussed the advantages and disadvantages of them. The regret bounds that have been proven for the said algorithms have also been discussed at length and their confidence intervals have also been compared against each other. In the next chapter, we provide our solution to this SMAB setting which achieves an almost order-optimal regret bound.

% Acknowledgments---Will not appear in anonymized version
%\acks{We thank a bunch of people.}

%\clearpage
%\newpage
%\bibliographystyle{named}
%\bibliography{ijcai17}

%\clearpage
%\newpage
%\section{Appendix}
%\label{appendix}
%\appendix
\begin{align*}
& H_{1}^{\sigma}=\sum_{i=1}^{K}\frac{\sigma_{i}+\sqrt{\sigma_{i}^{2}+(16/3)\Delta_{i}}}{\Delta_{i}^{2}}\\
& H_{2}^{\sigma}=\min_{i\in \mathcal{A}} i\tilde{\Delta}_{(i)}^{-2} \text{, where } \tilde{\Delta}_{i}^{-2}=\frac{\sigma_{i}+\sqrt{\sigma_{i}^{2}+(16/3)\Delta_{i}}}{\Delta_{i}^{2}}
%& H_{2}^{\sigma}=\min_{i\in \mathcal{A}} i\frac{12\sigma_{(i)}^{2} + \Delta_{(i)}}{12\Delta_{(i)}^{2}}
\end{align*}

We know that $\sigma_{i}\in [0,1], \forall i\in \mathcal{A}$ and $\Delta_{i}\in [0,1], \forall i\in \mathcal{A}$ and so $\sigma_{i}^{2} \leq \sigma_{i}$ and $\sqrt{\Delta_{i}} \geq \Delta_{i}$.

\begin{align*}
(3\Delta_{i}^{2}).\left(\frac{4\sigma_{i}^{2}+\Delta_{i}+4}{12\sigma_{i}^{2}+\Delta_{i}}\right) & >  \left(\frac{12\Delta_{i}^{2}}{12\sigma_{i}^{2}+\Delta_{i}}\right)\\
& > \left(\frac{12\Delta_{i}^{2}}{12\sigma_{i}^{2}+12\Delta_{i}}\right)\\
& > \left(\frac{\Delta_{i}^{2}}{\sigma_{i}+\Delta_{i}}\right)\\
%& > \left(\frac{\Delta_{i}^{2}}{\sigma_{i}+(\sigma_{i}^{2} + (16/3)\Delta_{i})}\right)\\
& > \left(\frac{\Delta_{i}^{2}}{\sigma_{i}+\sqrt{\sigma_{i}^{2} + (16/3)\Delta_{i}}}\right)\\
& > \left(\frac{1}{\min_{i}i\tilde{\Delta}_{i}^{2}}\right)\\
\end{align*}

Now, from \cite{audibert2010best} we know that,
\begin{align*}
\sum_{i=1}^{K}\tilde{\Delta}_{i}^{-2} = \tilde{\Delta}_{(2)}^{-2} + \sum_{i=2}^{K}\frac{1}{i}i\tilde{\Delta}_{(i)}^{-2} &\leq \bar{\log K}\min_{i}i\tilde{\Delta}_{(i)}^{-2}\\
& \leq \log(2K) H_{2}^{\sigma}, \text{ as $\bar{\log K} \leq \log(2K)$}
\end{align*}

So, $H_{2}^{\sigma} \leq H_{1}^{\sigma} \leq \log(2K) H_{2}^{\sigma}$

\textbf{Regarding union bound}\\
\begin{align*}
\Pb\lbrace\xi_1 \cup \xi_2\rbrace &= \Pb\lbrace\xi_1\rbrace + \Pb\lbrace\xi_2\rbrace - \Pb\lbrace\xi_1 \cap \xi_2\rbrace\\
 &\leq \Pb\lbrace\xi_1\rbrace + \Pb\lbrace\xi_2\rbrace
\end{align*}
So,
\begin{align*}
1-\Pb\lbrace\xi_1 \cup \xi_2\rbrace \geq 1-\Pb\lbrace\xi_1\rbrace + \Pb\lbrace\xi_2\rbrace \geq \E[\Ls(T)] \end{align*}

\section{Cumulative Regret Bound}

\begin{theorem}
\label{proofTheorem:Prop:1}
The regret $R_T$ for AugUCB satisfies
\begin{align*}
&\E [R_{T}]\\
& \leq \sum\limits_{i\in A:\Delta_{i} > b}\bigg\lbrace \dfrac{T\Delta_{i}}{( \frac{3}{2} T\Delta_i^{2})^{\psi_{m_i}}}
  + \left( \Delta_i +\dfrac{22\Delta_i\psi_{m_i}\log( \frac{3}{2} T\Delta_i^2)}{ \Delta_i}\right)\\
  & \bigg(\dfrac{4T^{1-\psi_{m_i}}2^{\psi_{m_i}-\frac{1}{2}}}{\Delta_{i}^{\psi_{m_i}-\frac{1}{2}}} \bigg)\bigg \rbrace +\sum_{i\in A: 0 < \Delta_{i} \leq b}\bigg(\dfrac{4T^{1-\psi_{m_i}}2^{\psi_{m_i}-\frac{1}{2}}}{b^{\psi_{m_i}-\frac{1}{2}}} \bigg) \\
  & + \max_{\substack{i\in A: \\ \Delta_{i}\leq b}}\Delta_{i}T, \text{  for all $b\geq\sqrt{\frac{e}{T}}$}. 
\end{align*} 
\end{theorem}


\begin{proof}
\textbf{\textit{Case a: A sub-optimal arm i is not eliminated on or before round $m_{i}$ with $ * \in B_{m_i}$}}
\newline
%%%%%%%%%%%%%
For any arm $i\in A$, if it is eliminated from active set $B_{m_{i}}$ then the below two events have to come true,
\begin{align}
\hat{r}_{i} + s_{i} < \tau - s_{i}, \label{eq:armelim-var-casea1}\\
\hat{r}_{i} - s_{i} > \tau + s_{i}, \label{eq:armelim-var-caseb1}
\end{align}

From Theorem \ref{Result:Theorem:1}, we know that a sub-optimal arm $i\in\mathcal{A}$, the probability that it is not correctly eliminated in the $m_i$-th round (or before) is also bounded by $4\exp\left(- \dfrac{3\rho \psi_{m_i}}{2} \left(\dfrac{\sigma_{i}^{2}+\sqrt{\rho\epsilon_{m_{i}}}+1}{3\sigma_{i}^{2}+\sqrt{\rho \epsilon_{m_{i}}}}\right) \log( T\epsilon_{m_{i}}) \right)$.

Summing over all arms in $\mathcal{A}^{'}$ and trivially bounding the regret for each arm $i\in \mathcal{A}^{'}$,

\begin{align*}
&\sum_{i\in A^{'}}T\Delta_{i}\exp\left(- \dfrac{3\rho \psi_{m_i}}{2} \left(\dfrac{\sigma_{i}^{2}+\sqrt{\rho\epsilon_{m_{i}}}+1}{3\sigma_{i}^{2}+\sqrt{\rho \epsilon_{m_{i}}}}\right) \log( T\epsilon_{m_{i}}) \right)\\
&\overset{(a)}{\le}  \sum_{i\in A^{'}}T\Delta_{i} \exp\left(- \dfrac{3\rho \psi_{m_i}}{2} \log( T\epsilon_{m_{i}}) \right)\\
&\overset{(b)}{\le}  \sum_{i\in A^{'}}T\Delta_{i} \exp\left(-\psi_{m_i} \log( T\epsilon_{m_{i}}) \right)\\
&\overset{(c)}{\le}  \sum_{i\in A^{'}}\dfrac{T\Delta_{i}}{( T\dfrac{\Delta_i^{2}}{\rho})^{\psi_{m_i}}} \leq \sum_{i\in A^{'}}\dfrac{T\Delta_{i}}{( \frac{3}{2} T\Delta_i^{2})^{\psi_{m_i}}}
\end{align*}

In the above formulation, $(a)$ happens because  $\left(\dfrac{\sigma_{i}^{2}+\sqrt{\rho\epsilon_{m_{i}}}+1}{3\sigma_{i}^{2}+\sqrt{\rho \epsilon_{m_{i}}}}\right) \geq 1$. For $(b)$ we put $\rho=\dfrac{2}{3}$ and for $(c)$ we take the help of the inequality $\sqrt{\rho \epsilon_{m_i}} < \Delta_i$ in the $m_i$-th round.

\textbf{\textit{Case b: A sub-optimal arm i is either eliminated on or before round $m_{i}$ or there is no $ * \in B_{m_i}$}}


\textbf{\textit{Case b1: A sub-optimal arm i is in $B_{m_i}$}}


A sub-optimal arm is in $B_{m_i}$ and hence pulled no more than,
\begin{align*}
\left\lceil \dfrac{2\psi_{m_i}\log( T\epsilon_{m_i})}{\epsilon_{m_i}} \right\rceil\leq \left\lceil \dfrac{32\psi_{m_i}\log( \frac{3}{2} T\Delta_i^2)}{\frac{3}{2} \Delta_i^2}\right\rceil
\end{align*}

Hence, the total contribution to the expected regret is,
\begin{align*}
&\sum_{i\in A^{'}} \left\lceil \dfrac{32\Delta_i\psi_{m_i}\log( \frac{3}{2} T\Delta_i^2)}{\frac{3}{2} \Delta_i^2}\right\rceil\\
&\leq \sum_{i\in A^{'}} \left( \Delta_i +\dfrac{22\Delta_i\psi_{m_i}\log( \frac{3}{2} T\Delta_i^2)}{ \Delta_i}\right)
\end{align*}

\textbf{\textit{Case b2: Optimal arm ${*}$ is eliminated by a sub-optimal arm}}

	Firstly, if conditions of Case $a$ holds then the optimal arm ${*}$ will not be eliminated in round $m=m_{*}$ or it will lead to the contradiction that $r_{i}>r^{*}$. In any round $m_{*}$, if the optimal arm ${*}$ gets eliminated then for any round from $1$ to $m_{j}$ all arms ${j}$ such that $m_{j}< m_{*}$ were eliminated according to assumption in Case $a$. Let the arms surviving till $m_{*}$ round be denoted by $A^{'}$. This leaves any arm $a_{b}$ such that $m_{b}\geq m_{*}$ to still survive and eliminate arm ${*}$ in round $m_{*}$. Let such arms that survive ${*}$ belong to $A^{''}$. Also maximal regret per step after eliminating ${*}$ is the maximal $\Delta_{j}$ among the remaining arms ${j}$ with $m_{j}\geq m_{*}$.  Let $m_{b}=\min\lbrace m|\sqrt{\rho\epsilon_{m}}<\dfrac{\Delta_{b}}{2}\rbrace$. Hence, the maximal regret after eliminating the arm ${*}$ is upper bounded by, 
\begin{align*}
&\sum_{m_{*}=0}^{max_{j\in A^{'}}m_{j}}\sum_{i\in A^{''}:m_{i}>m_{*}}\bigg(\dfrac{1}{(  T\epsilon_{m_{*}})^{\psi_{m_i}}} \bigg).T\max_{j\in A^{''}:m_{j}\geq m_{*}}{\Delta}_{j}\\
&\leq\sum_{m_{*}=0}^{max_{j\in A^{'}}m_{j}}\sum_{i\in A^{''}:m_{i}>m_{*}}\bigg(\dfrac{1}{(  T\epsilon_{m_{*}})^{\psi_{m_i}}} \bigg).T.2\sqrt{\rho\epsilon_{m_{*}}}\\
&\leq\sum_{m_{*}=0}^{max_{j\in A^{'}}m_{j}}\sum_{i\in A^{''}:m_{i}>m_{*}}\bigg(\dfrac{4T^{1-\psi_{m_i}}}{\epsilon_{m_{*}}^{\psi_{m_i}-\frac{1}{2}}} \bigg)\\
&\leq\sum_{i\in A^{''}:m_{i}>m_{*}}\sum_{m_{*}=0}^{\min{\lbrace m_{i},m_{b}\rbrace}}\bigg(\dfrac{4T^{1-\psi_{m_i}}}{2^{-(\psi_{m_i}-\frac{1}{2})m_{*}}} \bigg)\\
&\leq\sum_{i\in A^{'}}\bigg(\dfrac{4T^{1-\psi_{m_i}}}{2^{-(\psi_{m_i}-\frac{1}{2})m_{*}}} \bigg)+\sum_{i\in A^{''}\setminus A^{'}}\bigg(\dfrac{4T^{1-\psi_{m_i}}}{2^{-(\psi_{m_i}-\frac{1}{2})m_{b}}} \bigg)\\
&\leq\sum_{i\in A^{'}}\bigg(\dfrac{4T^{1-\psi_{m_i}}*2^{\psi_{m_i}-\frac{1}{2}}}{\Delta_{i}^{\psi_{m_i}-\frac{1}{2}}} \bigg)+\sum_{i\in A^{''}\setminus A^{'}}\bigg(\dfrac{4T^{1-\psi_{m_i}}*2^{\psi_{m_i}-\frac{1}{2}}}{b^{\psi_{m_i}-\frac{1}{2}}} \bigg)\\
&\leq\sum_{i\in A^{'}}\bigg(\dfrac{4T^{1-\psi_{m_i}}2^{\psi_{m_i}-\frac{1}{2}}}{\Delta_{i}^{\psi_{m_i}-\frac{1}{2}}} \bigg)+\sum_{i\in A^{''}\setminus A^{'}}\bigg(\dfrac{4T^{1-\psi_{m_i}}2^{\psi_{m_i}-\frac{1}{2}}}{b^{\psi_{m_i}-\frac{1}{2}}} \bigg)\\
%& = \sum_{i\in A^{'}}\bigg(\dfrac{ C_{2}(\rho_{a}) T^{1-\rho_{a}}}{\Delta_{i}^{4\rho_{a}-1}} \bigg)+\sum_{i\in A^{''}\setminus A^{'}}\bigg(\dfrac{C_{2(\rho_{a})}T^{1-\rho_{a}}}{b^{4\rho_{a}-1}} \bigg) \text{, where } C_2(x) = \frac{2^{2x+\frac{3}{2}}x^{2x}}{\psi^{x}}
\end{align*}

%\text{, since } \sqrt{\rho_{a}\epsilon_{m}}<\dfrac{\Delta_{i}}{2}

 
Summing up \textbf{Case a} and \textbf{Case b}, the total expected regret is given by,
\begin{align*}
 &\sum\limits_{i\in A:\Delta_{i} > b}\bigg\lbrace \dfrac{T\Delta_{i}}{( \frac{3}{2} T\Delta_i^{2})^{\psi_{m_i}}}
  + \left( \Delta_i +\dfrac{22\Delta_i\psi_{m_i}\log( \frac{3}{2} T\Delta_i^2)}{ \Delta_i}\right)\\
  & \bigg(\dfrac{4T^{1-\psi_{m_i}}2^{\psi_{m_i}-\frac{1}{2}}}{\Delta_{i}^{\psi_{m_i}-\frac{1}{2}}} \bigg)\bigg \rbrace +\sum_{i\in A: 0 < \Delta_{i} \leq b}\bigg(\dfrac{4T^{1-\psi_{m_i}}2^{\psi_{m_i}-\frac{1}{2}}}{b^{\psi_{m_i}-\frac{1}{2}}} \bigg)
\end{align*}

% R_{T} \leq &\sum\limits_{i\in A:\Delta_{i}\geq b}\bigg\lbrace\bigg(\dfrac{2^{1+4\rho_{a}}\rho_{a}^{2\rho_{a}}T^{1-\rho_{a}}}{\psi^{\rho_{a}}\Delta_{i}^{4\rho_{a}-1}}\bigg) + \bigg(\Delta_{i}+\dfrac{32\rho_{a}\log{(\psi  T\dfrac{\Delta_{i}^{4}}{16\rho_{a}^{2}})}}{\Delta_{i}}\bigg)\\
%&  +  \bigg(\dfrac{T^{1-\rho_{a}}\rho_{a}^{2\rho_{a}}2^{2\rho_{a}+\frac{3}{2}}}{\psi^{\rho_{a}}\Delta_{i}^{4\rho_{a} -1}} \bigg) \bigg \rbrace+\sum\limits_{i\in A:0\leq\Delta_{i}\leq b}\bigg(\dfrac{T^{1-\rho_{a}}\rho_{a}^{2\rho_{a}}2^{2\rho_{a}+\frac{3}{2}}}{\psi^{\rho_{a}}b^{4\rho_{a} -1}} \bigg) + max_{i:\Delta_{i}\leq b}\Delta_{i}T

\end{proof}

\end{document}


\begin{thebibliography}{1}
\bibitem{dennard:74}
R. H. Dennard, F. H. Gaensslen, V. L. Rideout, E. Bassous and A. R. LeBlanc, "Design of ion-implanted MOSFET's with very small physical dimensions", IEEE JSSC ,vol.9 ,no.5 ,1974, pp.256-268.

\bibitem{borkar:99}
Borkar, S., "Design challenges of technology scaling," Micro, IEEE , vol.19, no.4, pp.23,29, Jul-Aug 1999.

\bibitem{kim:03}
N. S. Kim, T. Austin, D. Blaauw, T. Mudge, K. Flautner, J. S. Hu, M. J. Irwin, M. Kandemir and V. Narayanan, V, "Leakage current: Moore's law meets static power", IEEE Computer , vol.36, no.12, 2003, pp.68-75.

%\bibitem{roy:03}
%K. Roy, S. Mukhopadhyay and H. Mahmoodi-Meimand, "Leakage current mechanisms and leakage reduction techniques in deep-submicrometer CMOS circuits", Proceedings of the IEEE , Vol.91, No.2, 2003, pp.305-327.

\bibitem{virat}
V. Gandhi, V.R. Devanathan, V. Visvanathan, M. Patnaik, and V. Kamakoti. "Supply and Body-Bias Voltage Assignment Based Technique for Power and Temperature Control on a Chip at Iso-Performance Conditions". Journal of Low Power Electronics. 9. 207-228. 10.1166/jolpe.2013.1254. 

\bibitem{feng:09}
Y. Feng and S. Hu,"The epsilon-approximation to discrete VT assignment for leakage power minimization", International Conference on Computer Aided Design, IEEE/ACM, 2009, pp.281-287.

\bibitem{itrs:2011}
ITRS 2011 Design Chapter. http://www.itrs2.net/2011-itrs.html

\bibitem{kahngtalk}
A. B. Kahng, “PPAC Scaling at 7nm and Below”, Cadence Distinguished Speaker Series talk, San Jose, CA, April 7, 2016.

\bibitem{hu:12}
J. Hu, A. B. Kahng, S. Kang, M.C. Kim and I. L. Markov, "Sensitivity-guided metaheuristics for accurate discrete gate sizing", International Conference on Computer Aided Design, IEEE/ACM, 2012, pp.233-239.

\bibitem{hu:13}
 Andrew B. Kahng, Seokhyeong Kang, Hyein Lee, Igor L. Markov, and Pankit Thapar. 2013. High-performance gate sizing with a signoff timer. In Proceedings of the International Conference on Computer-Aided Design (ICCAD '13). IEEE Press, Piscataway, NJ, USA, 450-457. 

\bibitem{mok:12}
S. Mok, J. Lee and P. Gupta, "Discrete sizing for leakage power optimization in physical design: A comparative study", ACM Transactions on Design Automation of Electronic Systems, Vol. 18, No. 1, 15 (2012).

\bibitem{reiman:13}
T. Reimann, G. Posser, G. Flach, M. Johann, and R. Reis, "Simultaneous gate sizing and Vt assignment using fanin/fanout ratio and simulated annealing." in Proc. ISCAS, 2013, pp.2549-2552.

\bibitem{riscv}
N. Gala, A. Menon, R. Bodduna, G. S. Madhusudan and V. Kamakoti, "SHAKTI Processors: An Open-Source Hardware Initiative," 29th International Conference on VLSI Design and 2016 15th International Conference on Embedded Systems (VLSID), Kolkata, 2016, pp. 7-8.


\bibitem{taucontest}
https://sites.google.com/view/taucontest2018/home

\bibitem{li:12}
L. Li, P. Kang, Y. Lu, and H. Zhou, “An efficient algorithm for library-
based cell-type selection in high-performance low-power designs,” in
Proc. ICCAD , Nov. 2012, pp.226-232.

 


\bibitem{ozdal:12}
M. Ozdal, S. Burns, and J. Hu, “Gate sizing and device technology selection algorithms for high-performance industrial designs,” in Proc. ICCAD , Nov.2012.




 
 
 \bibitem{rahman:12}
M. Rahman and C. Sechen, "Post-synthesis leakage power minimization," 2012 Design, Automation \& Test in Europe Conference \& Exhibition (DATE), Dresden, 2012, pp. 99-104. 









%\bibitem{wu:08}
%T. H. Wu, L. Xie and A. Davoodi, "A Parallel and Randomized Algorithm for Large-Scale Discrete Dual-Vt Assignment and Continuous Gate Sizing", ASP Journal of Low Power Electronics, Vol. 4, No. 2, pp. 191-201.
%
%\bibitem{boyd:08}
%S. Joshi and S. Boyd, "An Efficient method for large scale gate sizing", IEEE Transactions on Circuits and Systems-I, Vol.55, No.9, 2008, pp.2760-2773.


\bibitem{ispd:12}
M. M. Ozdal, C. Amin, A. Ayupov, S. Burns, G. Wilke, C. Zhuo, "The ISPD -2012 Discrete Cell Sizing Contest and Benchmark Suite", Proc. ACM International Symposium on Physical Design, pp. 161-164, 2012.

\bibitem{ispd:13}
Muhammet Mustafa Ozdal, Chirayu Amin, Andrey Ayupov, Steven M. Burns, Gustavo R. Wilke, and Cheng Zhuo. 2013. An improved benchmark suite for the ISPD-2013 discrete cell sizing contest. In Proceedings of the 2013 ACM International symposium on Physical Design (ISPD '13). ACM, New York, NY, USA, 168-170. DOI: https://doi.org/10.1145/2451916.2451959





\bibitem{Ketkar:09}
S. Hu, M. Ketkar, and J. Hu, “Gate sizing for cell-library-based designs,”
IEEE Transactions on Computer Aided Design
, vol. 28, no. 6, pp. 818-825, 2009

\bibitem{Liu:09}
Y. Liu and J. Hu, “A new algorithm for simultaneous gate sizing and
threshold voltage assignment,” in
Proc.  International  Symposium  on
Physical Design
, 2009, pp. 27-34.

\bibitem{livramento:14}
 Vinicius S. Livramento, Chrystian Guth, José Luís Güntzel, and Marcelo O. Johann. 2014. "A Hybrid Technique for Discrete Gate Sizing Based on Lagrangian Relaxation." ACM Trans. Des. Autom. Electron. Syst. 19, 4, Article 40 (August 2014),pp 40:1-40:25.  

\bibitem{ren:08}
H. Ren and S. Dutt, “A Network-Flow Based Cell Sizing Algorithm”,Proc. IWLS, 2008, pp. 7–14.

\bibitem{flach:13}
Flach Guilherme,  Reimann Tiago, Posser Gracieli, Johann Marcelo and Reis Ricardo. "Simultaneous Gate Sizing and Vth Assignment using Lagrangian Relaxation and Delay Sensitivities." Proceedings of IEEE Computer Society Annual Symposium on VLSI, ISVLSI 2013

\bibitem{livramento:13}
V. S. Livramento, C. Guth, J. L. Guntzel, and M. O. Johann, “Fast
and efficient Lagrangian relaxation-based discrete gate sizing,” in
Proc. DATE , 2013, pp. 1855–1860.

 

\bibitem{sharma:15}
 Ankur Sharma, David Chinnery, Sarvesh Bhardwaj, and Chris Chu. 2015." Fast Lagrangian Relaxation Based Gate Sizing using Multi-Threading." In Proceedings of the IEEE/ACM International Conference on Computer-Aided Design (ICCAD '15). IEEE Press, Piscataway, NJ, USA, 426-433. 

\bibitem{reis:16}
 T. J. Reimann, C. C. N. Sze, and R. Reis, “Cell Selection for High-Performance Designs in an Industrial Design Flow,” Ispd, pp. 65–72, 2016.

\bibitem{papa:10}
D. A. Papa, M. D. Moffitt, C. J. Alpert and I. L. Markov, "Speeding Up Physical Synthesis with Transactional Timing Analysis", IEEE Design and Test of Computers, vol.27, no.5, 2010, pp.14-25.



\bibitem{lee:95}
J. Lee and D.T. Tang, "An Algorithm for Incremental Timing Analysis", Design Automation Conference, IEEE/ACM, 1995, pp.696-701.

\bibitem{abato:96}
R. P. Abato, A. D. Drumm, D. J. Hathaway, and L. P. P. P. van Ginnekenl., "Incremental Timing Analysis", US patent, 5,508,937, to IBM Corp., Patent and Trademark Office, 1996.

\bibitem{sapatnekar:96}
S.S. Sapatnekar, "Efficient Calculation of All-Pairs Input-to-Output Delays in Synchronous Sequential Circuits", International Symposium on Circuits and Systems, IEEE, 1996, pp.724-727.

\bibitem{mondal:04}
A. Mondal and C.A. Mandal, "A New Approach to Timing Analysis Using Event Propagation and Temporal Logic",  Design, Automation and Test in Europe ,IEEE, 2004, pp.1198-1203.

\bibitem{das:06}
Debasish Das, Ahmed Shebaita, Hai Zhou, Yehea Ismail, and Kip Killpack, "FA-STAC: A Framework for Fast and Accurate Static Timing Analysis with Coupling", International Conference on Computer Design, IEEE, 2006, pp.43-49.

\bibitem{coudert:97}
O. Coudert, "Gate sizing for constrained delay/power/area optimization", IEEE Transactions on Very Large Scale Integration Systems, Vol.5, No.4, 1997, pp.465-472.


\bibitem{prowatch}
 Milan Patnaik, Chidhambaranathan R, Chirag Garg, Arnab Roy, V. R. Devanathan, Shankar Balachandran, and V. Kamakoti. 2015. "ProWATCh: A Proactive Cross-Layer Workload-Aware Temperature Management Framework for Low-Power Chip Multi-Processors." J. Emerg. Technol. Comput. Syst. 12, 3, Article 22 (September 2015), 25 pages.


\bibitem{kahng:2}
 Seung-Soo Han, Andrew B. Kahng, Siddhartha Nath, and Ashok S. Vydyanathan. 2014. A deep learning methodology to proliferate golden signoff timing. In Proceedings of the conference on Design, Automation \& Test in Europe (DATE '14). European Design and Automation Association, 3001 Leuven, Belgium, Belgium, , Article 260 , 6 pages. 
 \bibitem{kahng:3}
 Song Bian, Michihiro Shintani, Masayuki Hiromoto, and Takashi Sato. 2017. LSTA: Learning-Based Static Timing Analysis for High-Dimensional Correlated On-Chip Variations. In Proceedings of the 54th Annual Design Automation Conference 2017 (DAC '17). ACM, New York, NY, USA, Article 66, 6 pages. DOI: https://doi.org/10.1145/3061639.3062280 

\bibitem{kahng:16}
A. B. Kahng, H. Lee, and J. Li, “Measuring progress and value of IC implementation technology,” Proc. 35th Int. Conf. Comput. Des.  - ICCAD ’16, pp. 1–8, 2016.

\bibitem{kahng:14}
 A. B. Kahng, H. Lee, and J. Li, “Horizontal benchmark extension for improved assessment of physical CAD research,” Proc. 24th Ed. Gt. lakes Symp. VLSI - GLSVLSI ’14, pp. 27–32, 2014.

\bibitem{abramovci}
 Abramovici M., Breuer M., Friedman A.: “Digital Systems Testing and Testable Design”, Computer Science. Press, 1990.


\bibitem{SVM}
http://www.cs.columbia.edu/~kathy/cs4701/documents/$jason\_svm\_tutorial.pdf$

\bibitem{libsvm}
Chih-Chung Chang and Chih-Jen Lin, LIBSVM : a library for support vector machines. ACM Transactions on Intelligent Systems and Technology, 2:27:1--27:27, 2011. Software available at http://www.csie.ntu.edu.tw/~cjlin/libsvm

\bibitem{liblinear}
Lee, Mu-Chu, Wei-Lin Chiang, and Chih-Jen Lin. "Fast matrix-vector multiplications for large-scale logistic regression on shared-memory systems." Data Mining (ICDM), 2015 IEEE International Conference on. IEEE, 2015.

\bibitem{OpenTimer}
T. W. Huang and M. D. F. Wong, "OpenTimer: A high-performance timing analysis tool," 2015 IEEE/ACM International Conference on Computer-Aided Design (ICCAD), Austin, TX, 2015, pp. 895-902.
doi: 10.1109/ICCAD.2015.7372666



%\bibitem{osfa:16}
%S. Roy, D. Liu, J. Singh, J. Um and D. Z. Pan, "OSFA: A New Paradigm of Aging Aware Gate-Sizing for Power/Performance Optimizations Under Multiple Operating Conditions," in IEEE Transactions on Computer-Aided Design of Integrated Circuits and Systems, vol. 35, no. 10, pp. 1618-1629, Oct. 2016. doi: 10.1109/TCAD.2016.2523439
%
%
%
%
%
%
%
%
%
%
%\bibitem{chinnery:05}
% D. G. Chinnery and K. Keutzer, "Linear programming for sizing, $V_{th}$ and $V_{dd}$ assignment", International Symposium on Low power electronics and design, ACM, 2005, pp.149-154.
%
%\bibitem{ispd:13}
%M. M. Ozdal, C. Amin, A. Ayupov, S. Burns, G. Wilke, C. Zhuo, "An Improved Benchmark Suite for the ISPD-2013 Discrete Cell Sizing Contest", Proc. of ACM International Symposium on Physical Design, pp. 168-170, 2013.
%
%
%
%
%
%%\bibitem{SENSEOPT}
%%http://vlsicad.ucsd.edu/SIZING/optimizer.html
%%\bibitem{craft}
%%CRAFT Program Aims for Affordable Designer Circuits that Do More with Less Power. http://www.darpa.mil/news-events/2015-08-17
%
%
%
%
%
%
%
%
%
%\bibitem{sundarajan:99}
%V. Sundararajan and K. K. Parhi, “Low power synthesis of dual threshold voltage CMOS circuits", International Symposium on Low Power Electronics and Design, IEEE, 1999, pp.139–144. 
%
%
%
%
%
%\bibitem{cormen:01}
%T. H. Cormen, C. Stein, R. L. Rivest and C. E. Leiserson, "Introduction to Algorithms", $2^{nd}$ Edition, 2001, McGraw-Hill Higher Education.
%
%\bibitem{liu:10}
%Y. Liu and J. Hu, "A New Algorithm for Simultaneous Gate Sizing and Threshold Voltage Assignment", IEEE Transactions on Computer-Aided Design of Integrated Circuits and Systems, Vol.29, No.2, pp.223-234, Feb. 2010. 


\end{thebibliography}

\documentclass[PhD]{iitmdiss}
%\documentclass[MS]{iitmdiss}
%\documentclass[MTech]{iitmdiss}
%\documentclass[BTech]{iitmdiss}
\usepackage{times}
 \usepackage{t1enc}

\usepackage{graphicx}
\usepackage{epstopdf}
\usepackage{hyperref} % hyperlinks for references.
\usepackage{amsmath} % easier math formulae, align, subequations \ldots


%%%%%%%%%%%%%%%%%%%%%%%%%%%%%%%
%\usepackage{ijcai17}


\usepackage{macros}

\usepackage{latexsym} 


\begin{document}

%%%%%%%%%%%%%%%%%%%%%%%%%%%%%%%%%%%%%%%%%%%%%%%%%%%%%%%%%%%%%%%%%%%%%%
% Title page

\title{\LaTeX\ CLASS FOR DISSERTATIONS SUBMITTED TO IITM}

\author{Name}

\date{December 2017}
\department{Computer Science and Engineering}

%\nocite{*}
\maketitle

%%%%%%%%%%%%%%%%%%%%%%%%%%%%%%%%%%%%%%%%%%%%%%%%%%%%%%%%%%%%%%%%%%%%%%
% Certificate
\certificate

\vspace*{0.5in}

\noindent This is to certify that the thesis titled {\bf \LaTeX\ CLASS  
  FOR DISSERTATIONS SUBMITTED TO IIT-M}, submitted by {\bf Subhojyoti Mukherjee}, 
  to the Indian Institute of Technology, Madras, for
the award of the degree of {\bf Master of Science (Research)}, is a bona fide
record of the research work done by him under our supervision.  The
contents of this thesis, in full or in parts, have not been submitted
to any other Institute or University for the award of any degree or
diploma.

\vspace*{1.5in}

\begin{singlespacing}
\hspace*{-0.25in}
\parbox{2.5in}{
\noindent {\bf Prof.~1} \\
\noindent Research Guide \\ 
\noindent Professor \\
\noindent Dept. of Physics\\
\noindent IIT-Madras, 600 036 \\
} 
\hspace*{1.0in} 
%\parbox{2.5in}{
%\noindent {\bf Prof.~S.~C.~Rajan} \\
%\noindent Research Guide \\ 
%\noindent Assistant Professor \\
%\noindent Dept.  of  Aerospace Engineering\\
%\noindent IIT-Madras, 600 036 \\
%}  
\end{singlespacing}
\vspace*{0.25in}
\noindent Place: Chennai\\
Date: 22nd December 2017 


%%%%%%%%%%%%%%%%%%%%%%%%%%%%%%%%%%%%%%%%%%%%%%%%%%%%%%%%%%%%%%%%%%%%%%
% Acknowledgements
\acknowledgements

Thanks to all those who made \TeX\ and \LaTeX\ what it is today.

%%%%%%%%%%%%%%%%%%%%%%%%%%%%%%%%%%%%%%%%%%%%%%%%%%%%%%%%%%%%%%%%%%%%%%
% Abstract

\abstract

\noindent KEYWORDS: \hspace*{0.5em} \parbox[t]{4.4in}{\LaTeX ; Thesis;
  Style files; Format.}

\vspace*{24pt}

\noindent A \LaTeX\ class along with a simple template thesis are
provided here.  These can be used to easily write a thesis suitable
for submission at IIT-Madras.  The class provides options to format
PhD, MS, M.Tech.\ and B.Tech.\ thesis.  It also allows one to write a
synopsis using the same class file.  Also provided is a BIB\TeX\ style
file that formats all bibliography entries as per the IITM format.

The formatting is as (as far as the author is aware) per the current
institute guidelines.

\pagebreak

%%%%%%%%%%%%%%%%%%%%%%%%%%%%%%%%%%%%%%%%%%%%%%%%%%%%%%%%%%%%%%%%%
% Table of contents etc.

\begin{singlespace}
\tableofcontents
\thispagestyle{empty}

\listoftables
\addcontentsline{toc}{chapter}{LIST OF TABLES}
\listoffigures
\addcontentsline{toc}{chapter}{LIST OF FIGURES}
\end{singlespace}


%%%%%%%%%%%%%%%%%%%%%%%%%%%%%%%%%%%%%%%%%%%%%%%%%%%%%%%%%%%%%%%%%%%%%%
% Abbreviations
\abbreviations

\noindent 
\begin{tabbing}
xxxxxxxxxxx \= xxxxxxxxxxxxxxxxxxxxxxxxxxxxxxxxxxxxxxxxxxxxxxxx \kill
\textbf{IITM}   \> Indian Institute of Technology, Madras \\
\textbf{RTFM} \> Read the Fine Manual \\
\end{tabbing}

\pagebreak

%%%%%%%%%%%%%%%%%%%%%%%%%%%%%%%%%%%%%%%%%%%%%%%%%%%%%%%%%%%%%%%%%%%%%%
% Notation

\chapter*{\centerline{NOTATION}}
\addcontentsline{toc}{chapter}{NOTATION}

\begin{singlespace}
\begin{tabbing}
xxxxxxxxxxx \= xxxxxxxxxxxxxxxxxxxxxxxxxxxxxxxxxxxxxxxxxxxxxxxx \kill
\textbf{$r$}  \> Radius, $m$ \\
\textbf{$\alpha$}  \> Angle of thesis in degrees \\
\textbf{$\beta$}   \> Flight path in degrees \\
\end{tabbing}
\end{singlespace}

\pagebreak
\clearpage

% The main text will follow from this point so set the page numbering
% to arabic from here on.
\pagenumbering{arabic}


%%%%%%%%%%%%%%%%%%%%%%%%%%%%%%%%%%%%%%%%%%%%%%%%%%
% Introduction.
\chapter{Introduction}
\label{chap:intro}
\input{ThesisIntro}

%%%%%%%%%%%%%%%%%%%%%%%%%%%%%%%%%%%%%%%%%%%%%%%%%%%%%%%%%%%%

%%%%%%%%%%%%%%%%%%%%%%%%%%%%%%%%%%%%%%%%%%%%%%%%%%%%%%%%%%%%
\chapter{Thresholding Bandits}
\label{chap:tbandit}
\section{Introduction}
\label{tbandit:intro}
Stochastic multi-armed bandit (MAB) problems are instances of the classic sequential decision-making scenario; specifically an MAB problem comprises of a learner and a collection of actions (or arms), denoted $\mathcal{A}$. In each trial the learner plays (or pulls) an arm $i\in\mathcal{A}$ which yields independent and identically distributed (i.i.d.) reward samples from a distribution (corresponding to arm $i$), whose expectation is denoted by $r_i$. 
%whose rewards are  samples from the distribution specific to the arm $i\in A$ and whose expected reward is denoted by $r_{i},\forall i\in A$. 
The learner's objective is to identify an arm corresponding to the maximum expected reward, denoted $r^{*}$. Thus, at each time-step the learner 
%selects an arm $i$ and hence
is faced with the \emph{exploration vs.\ exploitation dilemma}, where it can pull an arm which has yielded the highest mean reward (denoted $\hat{r}_{i}$) thus far (\emph{exploitation}) or continue to explore other arms with the prospect of finding a better arm 
%superior performance 
whose performance has not been observed sufficiently (\emph{exploration}).

%In the stochastic multi-armed bandit setting a learning agent is required to choose from a set of decisions or arms at every round. The agent is then presented with a reward for that round, which is an independent draw from a stationary distribution specific to the arm selected. The agent, however, does not know the mean of the distributions associated with each arm, denoted by $r_{i}$, including the optimal arm which will give it the best reward, denoted by $r^{*}$. The agent attempts to make arm choices that will maximize some performance measure by keeping track of the reward that has been gathered from previous selections of the arm, for each arm. This is called the estimated mean reward of an arm denoted by $\hat{r}_{i}$. The bandit problem can be conceptualized as a sequential decision making process where the agent is at each round presented with an \emph{exploration-exploitation dilemma}. The agent could pull the arm which has the highest observed mean reward till now (exploitation) or to explore other arms, with the prospect of finding superior performance which was previously unobserved (exploration). 

%
%	Formally, let $r_i$, $i=1,\ldots,K$ denote the mean rewards of the $K$ arms and $r^* = \max_i r_i$ the optimal mean reward. The objective in some of the stochastic bandit problem is to minimize the cumulative regret, which is defined as follows:
%\begin{align*}
%R_{T}=r^{*}T - \sum_{i\in A} r_{i}N_{i}(T),
%\end{align*}
%where $T$ is the number of rounds, $N_{i}(T)=\sum_{m=1}^T I(I_m=i)$ is the number of times the algorithm chose arm $i$ up to round $T$.
%The expected regret of an algorithm after $T$ rounds can be written as,
%
%\begin{align*}
%\E[R_{T}]= \sum_{i=1}^K \E[N_i(T)] \Delta_i,
%\end{align*}
%where $\Delta_{i}=r^{*}-r_{i}$ denotes the gap between the means of the optimal arm and of the $i$-th arm. 


Pure-exploration MAB problems are unlike their traditional (exploration vs.\ exploitation)  counterparts where the  objective is to minimize the cumulative regret (which is the total loss incurred by the learner for not playing the optimal arm throughout the time horizon $T$). Instead, in pure-exploration problems a learning algorithm, until time $T$, can invest entirely on exploring the arms without being concerned about the loss incurred while exploring; the objective is to minimize the probability that the arm recommended at time $T$ is not the best arm.  In this paper, we further consider a combinatorial version of the pure-exploration MAB, called the thresholding bandit problem (TBP).  Here, the learning algorithm is provided with a threshold $\tau$, and the objective, after exploring for $T$ rounds, is to  output all arms $i$ whose $r_{i}$ is above $\tau$. 
It is important to emphasize that the \emph{thresholding} bandit problem is different from the \emph{threshold} bandit setup studied in \cite{abernethy2016threshold}, where the learner receives an unit reward whenever the value of an observation is above a threshold. 

%%%%%%%%%%%%%%%
% Old para
%%%%%%%%%%%%%%%
%Pure-exploration problems are unlike their traditional (exploration vs.\ exploitation) counterparts where the  objective is to minimize the cumulative regret, which is the total loss incurred by the learner for not playing the optimal arm throughout the time horizon $T$. Instead, in the pure exploration setup the learning algorithm is provided with a threshold $\tau$, and the objective, after exploring for $T$ rounds, is to  output all arms $i$ whose $r_{i}$ is above $\tau$. Thus, the learning algorithm, until  time $T$, can invest entirely on exploring the arms  without being concerned about the loss incurred while exploring. It is important to emphasize that the \emph{thresholding bandit} problem is different from the \emph{threshold bandit} setup studied in \cite{abernethy2016threshold}, where the learner receives an unit reward whenever the value of an observation is above a threshold. 
%<<<<<<< Updated upstream
%Pure-exploration problems are unlike their traditional (exploration vs.\ exploitation) counterparts where the  objective is to minimize the cumulative regret, which is the total loss incurred by the learner for not playing the optimal arm throughout the time horizon $T$. In this paper we study the fixed-budget setting of a specific combinatorial pure-exploration problem, called the thresholding bandit problem (TBP), in the context of (MAB) setting. In this pure-exploration setup the learning algorithm is provided with a threshold $\tau$, and the objective, after exploring for $T$ rounds, is to  output all arms $i$ whose $r_{i}$ is above $\tau$. Thus, the learning algorithm, until time $T$, can invest entirely on exploring the arms  without being concerned about the loss incurred while exploring. It is important to emphasize that the \emph{thresholding} bandit problem is different from the \emph{threshold} bandit setup studied in \cite{abernethy2016threshold}, where the learner receives an unit reward whenever the value of an observation is above a threshold. 
%=======

%Pure-exploration problems are unlike their traditional (exploration vs.\ exploitation) counterparts where the  objective is to minimize the cumulative regret, which is the total loss incurred by the learner for not playing the optimal arm throughout the time horizon $T$. Instead, in the pure exploration setup the learning algorithm is provided with a threshold $\tau$, and the objective, after exploring for $T$ rounds, is to  output all arms $i$ whose $r_{i}$ is above $\tau$. Thus, the learning algorithm, until  time $T$, can invest entirely on exploring the arms  without being concerned about the loss incurred while exploring. It is important to emphasize that the \emph{thresholding bandit} problem is different from the \emph{threshold bandit} setup studied in \cite{abernethy2016threshold}, where the learner receives an unit reward whenever the value of an observation is above a threshold. 
%>>>>>>> Stashed changes

% reward on each time-step depends on a threshold value and the learner receives the reward only if the reward is above the threshold.


%This is a specific instance of combinatorial pure exploration where the learning algorithm can explore as much as possible given a fixed horizon $T$ and not be concerned with the usual exploration-exploitation dilemma. 

Formally, the problem we consider is the following. First, we define the set $S_{\tau}=\lbrace i\in \mathcal{A}: r_{i}\geq \tau \rbrace$. Note that, $S_\tau$ is the set of all arms whose reward mean is greater than $\tau$. Let 
$S_\tau^c$ % =\mathcal{A}\backslash S_\tau$
 denote the complement of $S_\tau$, i.e.,  $S_{\tau}^{c}=\lbrace i\in \mathcal{A}: r_{i} < \tau \rbrace$. Next, let $\hat{S}_{\tau}=\hat{S}_{\tau}(T)\subseteq \mathcal{A}$ denote the recommendation of a learning algorithm (under consideration) after $T$ time units of exploration, while $\hat{S}_{\tau}^c$ denotes its complement.
%  Also we define $\hat{S}_{\tau}=\hat{S}_{\tau}(T)\subset \mathcal{A}$ and its complementary set $\hat{S}_{\tau}^{C}$ as the recommendation of the learning algorithm after $T$ rounds. 
% Given such sets exists, 
The performance of the learning agent is measured by the accuracy with which it can classify the arms into $S_{\tau}$ and $S_{\tau}^{c}$ after time horizon $T$. Equivalently, using $\mathbb{I}(E)$ to denote the indicator of an event $E$, the \emph{loss} $\mathcal{L}(T)$ is defined as
\begin{align*}
\Ls (T) = \mathbb{I}\big(\lbrace S_{\tau}\cap \hat{S}_{\tau}^{c}\neq \emptyset\rbrace    \cup    \lbrace\hat{S}_{\tau}\cap S_{\tau}^{c}\neq \emptyset\rbrace\big).
\end{align*}			
Finally, the goal of the learning agent is to minimize the expected loss:
% So, the expected loss after $T$ rounds is,
\begin{align*}
\E[\Ls(T)] = \Pb\big(\lbrace S_{\tau}\cap \hat{S}_{\tau}^{c} \neq \emptyset \rbrace  \cup   \lbrace \hat{S}_{\tau}\cap S_{\tau}^{c} \neq \emptyset\rbrace\big).
\end{align*}
Note that the expected loss is simply the \emph{probability of mis-classification} (i.e., error), that occurs either if a good arm is rejected or a bad arm is accepted as a good one.
% (represented by the events $\lbrace S_{\tau}\cap \hat{S}_{\tau}^{c} \neq \emptyset \rbrace$ and $\lbrace \hat{S}_{\tau}\cap S_{\tau}^{c} \neq \emptyset\rbrace$, respectively).

%which we can say is the probability of making mistake, that is whether the learning agent at the end of round $T$ rejects arms from $S_{\tau}$ or accepts arms from $S_{\tau}^{C}$ in its final recommendation. 

%Also, we are looking at an anytime algorithm, so the knowledge of $T$ may not be known to the learner.





The above TBP formulation has several applications, for instance, from areas ranging from anomaly detection and classification (see  \citet{locatelli2016optimal}) to industrial application. Particularly in industrial applications, a learners objective is to choose (i.e., keep in operation) all machines whose productivity is above a threshold. The TBP also finds applications in mobile communications (see \citet{audibert2010best})  where the users are to be allocated only those channels whose quality is above an acceptable threshold. Again, some of these problems have been already discussed in chapter \ref{chap:intro}, section \ref{motivation} and an interested reader can refer to it. In some cases the TBP problem is more relevant than the variants of $p$-best problem (identifying the best $p$ arms from $K$ given arms). As explained in \citet{locatelli2016optimal}, the $p$-best problem is a "contest" whereas the TBP is an  "exam" and in many domains, one requires the idea of "efficiency" or "correctness" threshold above which one wants to utilize an option rather than simply selecting the $p$-best options.

%where the learner has to keep all those workers active whose productivity is above a particular threshold $\tau$, or allocating channels whose quality is above a threshold for Mobile Communications 
% or in crowd-sourcing while hiring workers the TBP problem 

%
%	1. \emph{Product Selection:} A company wants to introduce a new product in market and there is a clear separation of the test phase from the commercialization phase. In this case the company tries to minimize the loss it might incur in the commercialization phase by testing as much as possible in the test phase. So from the several variants of the product that are in the test phase the learning agent must suggest the product variant(s) that are above a particular threshold $\tau$ at the end of the test phase that have the highest probability of minimizing loss in the commercialization phase. A similar problem has been discussed for single best product variant identification without threshold in \cite{bubeck2011pure}. 
%
%	2. \emph{Mobile Phone Channel Allocation:} Another similar problem as above concerns channel allocation for mobile phone communications (\cite{audibert2009exploration}). Here there is a clear separation between the allocation phase and communication phase whereby in the allocation phase a learning algorithm has to explore as many channels as possible to suggest the best possible set of channel(s) that are above a particular threshold $\tau$. The threshold depends on the subscription level of the customer. With higher subscription the customer is allowed better channel(s) with the $\tau$ set high. Each evaluation of a channel is noisy and the learning algorithm must come up with the best possible suggestion within a very small  number of attempts.
%
%	3. \emph{Anomaly Detection and Classification:} Thresholding bandit can also be used for anomaly detection and classification where we define a cutoff level $\tau$ and for any samples above this cutoff gets classified as an anomaly. For further reading we point the reader to section 3 of \cite{locatelli2016optimal}.
%
%

\subsection{Related Work}
\label{tbandit:prevRes}
Significant amount of literature is available on the stochastic MAB setting with respect to minimizing the cumulative regret. Chapter \ref{chap:SMAB} and \ref{chap:EUCBV} deals with that. In this work we are particularly interested in \emph{pure-exploration MABs},  where the focus in primarily on simple regret rather than the cumulative regret. The relationship between cumulative regret and simple regret is proved in \citet{bubeck2011pure} where the authors prove that minimizing the simple regret necessarily results in maximizing the cumulative regret.
The pure exploration problem has been explored  mainly under the following two settings:
	
\subsection{Fixed Budget setting} 

Here the learning algorithm has to suggest the best arm(s) within a fixed time-horizon $T$, that is usually given as an input. The objective is to maximize the probability of returning the best arm(s).  This is the scenario we consider in this chapter. Some of the important algorithms used in pure exploration setting are discussed in the next part.

\subsubsection{UCB-Exploration Algorithm}

\begin{algorithm}[!h]
\caption{UCBE}
\label{alg:ucbe}
\begin{algorithmic}[1]
\State \textbf{Input: } The budget $T$
\State Pull each arm once
\For{$t=K+1,..., T$}
\State Pull the arm such that $\argmax_{i\in \A}\bigg\lbrace\hat{r}_{i} + \sqrt{\dfrac{a}{n_i}}\bigg\rbrace$, where $a = \dfrac{25(T-K)}{36 H_1}$ and $H_1 = \sum_{i=1}^{K}\dfrac{1}{\Delta_i^2}$.
\State $t:=t+1 $
\EndFor
\end{algorithmic}
\end{algorithm}

In \citet{audibert2010best} the authors propose the  UCBE and the Successive Reject (SR) algorithm, and prove simple-regret guarantees for the problem of identifying the single best arm.  In the combinatorial fixed budget setup \citet{gabillon2011multi} propose the GapE and GapE-V algorithms that suggest, with high probability, the best $m$ arms at the end of the time budget. 


\subsubsection{Successive Reject Algorithm}


\begin{algorithm}[h!]
\caption{Successive Reject(SR)}
\label{alg:sr}
\begin{algorithmic}[1]
\State \textbf{Input: } The budget $T$
\State \textbf{Initialization: } $n_0 = 0$
\State \textbf{Definition: } $\bar{\log K} = \dfrac{1}{2} + \sum_{i=2}^{K}\dfrac{1}{i}$, $n_k = \dfrac{1}{\bar{\log K}}\dfrac{T-K}{K + 1 - m}$
\For{For each phase $m=1,..., K-1$}
\State For each $i \in B_{m}$, select arm $i$ for $n_k - n_{k-1}$ rounds.
\State Let $B_{m+1} = B_m\setminus \argmin_{i\in B_m} \hat{r}_i$
(remove one element from $B_m$ , if there
is a tie, select randomly the arm to dismiss among the worst arms).
\State $m:=m+1 $
\EndFor
\State Output the single remaining $i\in B_{m}$.
\end{algorithmic}
\end{algorithm}


Similarly, \citet{bubeck2013multiple} introduce the  Successive Accept Reject (SAR) algorithm, which is an extension of the SR algorithm; SAR is a round based algorithm whereby at the end of each round an arm is either accepted or rejected (based on certain confidence conditions) until the top $m$ arms are suggested at the end of the budget with high probability. A similar combinatorial setup was explored in \citet{chen2014combinatorial} where the authors propose the Combinatorial Successive Accept Reject (CSAR) algorithm, which is similar in concept to SAR but with a more general setup. 

\subsection{Fixed Confidence setting} 

In this setting the learning algorithm has to suggest the best arm(s) with a fixed confidence (given as input) with as fewer number of attempts as possible. The single best arm identification has been studied in \citet{even2006action}, while for the combinatorial setup \citet{kalyanakrishnan2012pac} have proposed the LUCB algorithm which, on termination, returns  $m$ arms which are at least $\epsilon$ close to the true top-$m$ arms with probability at least $1-\delta$. For a detail survey of this setup we refer the reader to \citet{jamieson2014best}. 

\subsection{Unified Setting}
Apart from these two settings some unified approaches has also been suggested in \citet{gabillon2012best} which proposes the algorithms UGapEb and UGapEc which can work in both the above two settings. The thresholding bandit problem is a specific instance of the pure-exploration setup of \citet{chen2014combinatorial}. 



	
	


\subsection{Our Contribution}
\label{tbandit:contribution}
The main contributions of the thesis are as follows:-
\begin{enumerate}
\item We propose the MLTimer alogrithm that uses gate-sizing for reducing the leakage power consumption of a digital design. We propose a smart one-pass tool that can leverage the right optimization technique at the appropriate stage of the flow thereby improving design productivity. A key observation reported in MLTimer is that there exists significant correlation between the timing slacks of gates in the current iteration to the gate replacements in successive iterations. MLTimer leverages this observation to reduce the number of STA runs thereby reducing the overall time taken for optimization.

\item We propose the Karna algorithm which uses gate-sizing for reducing the information leakage via the power side-channel of a digital design. We show that each region in a given design leaks information differently. Thus, it is sufficient to optimize gates in the highly sensitive regions to reduce information leakage. Karna leverages this observation and optimizes gates in these sensitive regions to reduce the power side-channel vulnerability. 

%\item We proposed a general framework of bandit algorithms that combines change-point detection algorithm with aggregation of expert strategies in order to define efficient pulling strategies in the context of the piecewise stochastic distributions. The algorithms that we proposed for the piecewise stochastic setting are actively adaptive algorithms which perform very similarly to the oracle algorithm which has access to the changepoints and suffers no additional delay in adapting to the changing environment. 
\end{enumerate}
 


\section{Augmented-UCB Algorithm}
\label{tbandit:algorithm}
\textbf{The Algorithm:} The Augmented-UCB (AugUCB) algorithm is presented in Algorithm~\ref{alg:augucb}.
AugUCB is essentially based on the arm elimination method of the UCB-Improved \cite{auer2010ucb}, but adapted to the thresholding bandit setting proposed in \cite{locatelli2016optimal}. However, unlike the UCB improved (which is based on mean estimation) our algorithm employs \emph{variance estimates} (as in \cite{audibert2009exploration}) for arm elimination; to the best of our knowledge this is the first variance-aware  algorithm for the thresholding bandit problem. Further, we allow for arm-elimination at each time-step, which is in contrast to the earlier work (e.g., \cite{auer2010ucb,chen2014combinatorial}) where the arm elimination task is deferred to the end of the respective exploration rounds. The details are presented below.

% In algorithm \ref{alg:augucb}, hence referred to as AugUCB, we have two exploration parameters, $\rho_{\mu}$ and $\rho_v$ which are the arm elimination parameters. $\psi_{m}$ is the exploration regulatory factor. 
%The main approach is based on the UCB-Improved algorithm with modifications suited for the thresholding bandit problem. 
The active set $B_{0}$ is initialized with all the arms from $\mathcal{A}$. We divide the entire budget $T$ into rounds/phases like in UCB-Improved, CCB, SAR and CSAR. At every time-step AugUCB checks for arm elimination conditions, while updating parameters at the end of each round. As suggested by \cite{liu2016modification} to make AugUCB to overcome too much early exploration, we no longer pull all the arms equal number of times in each round. Instead, we choose an arm in the active set $B_m$ that minimizes $(|\hat{r}_{i} - \tau |-2s_i)$ where 
%$\min_{i\in B_{m}}\big\lbrace |\hat{r}_{i} - \tau | - 2\sqrt{\frac{\rho_v\psi_m \hat{V}_{i} \log ( T \epsilon_{m})}{4 n_{i}} + \frac{\rho_v\psi_m \log{( T\epsilon_{m})}}{4 n_{i}}} \big\rbrace $
\begin{small}
\begin{align*}
s_i & = \sqrt{\frac{\rho\psi_m (\hat{v}_{i}+1) \log ( T \epsilon_{m})}{4 n_{i}}} %+ \frac{\rho\psi_m \log{( T\epsilon_{m})}}{4 n_{i}}}.
\end{align*}
\end{small} 
with $\rho$ being the arm elimination parameter and $\psi_{m}$ being the exploration regulatory factor.
%  in the active set $B_{m}$. 
The above condition ensures that an arm closer to the threshold $\tau$ is pulled; 
%and with suitable choice of $\rho_{\mu}$ and $\rho_v$ we can fine tune the exploration. 
parameter $\rho$ can be used to fine tune the elimination interval.
The choice of exploration factor, $\psi_m$,
% $\psi_m=\frac{T\epsilon_m}{(\log(\frac{3}{16} K\log K))^{2}}$ 
comes directly from \cite{audibert2010best} and \cite{bubeck2011pure} where it is  stated that in pure exploration setup, the exploring factor must be linear in $T$ (so that an exponentially small probability of error is achieved) rather than being logarithmic in $T$ (which is more suited for minimizing cumulative regret).

\begin{algorithm}[t!]
\caption{AugUCB}
\label{alg:augucb}
\begin{algorithmic}
\State {\bf Input:} Time budget $T$; parameter $\rho$; 
% $\rho_{\mu}$, $\rho_v$ 
  threshold $\tau$
\State {\bf Initialization:} $B_{0}=\mathcal{A}$; $m=0$; $\epsilon_{0}=1$;
\begin{small}
\begin{align*}
M&=\left\lfloor \frac{1}{2}\log_{2} \frac{T}{e}\right\rfloor; 
\hspace{2mm}\psi_{0}=\frac{T\epsilon_{0}}{128\Big(\log(\frac{3}{16}K\log K)\Big)^2}; \\
\ell_{0}&=\left\lceil \frac{2\psi_0\log( T\epsilon_{0})}{\epsilon_{0}} \right\rceil;
\hspace{2mm}N_{0}=K\ell_{0}
\end{align*}
\end{small}
%$M=\left\lfloor \frac{1}{2}\log_{2} \frac{T}{e}\right\rfloor $,  
%$\psi_{0}=\frac{T\epsilon_{0}}{(\log(\frac{3}{16}K\log K)^2}$,
% $\ell_{0}=\left\lceil \frac{2\psi\log( T\epsilon_{0})}{\epsilon_{0}} \right\rceil$ and 
% $N_{0}=K\ell_{0} $. Pull each arm once.
\State Pull each arm once
\vspace{-2mm}
\State \For{$t=K+1,..,T$}
\State Pull arm $j\in\argmin_{i\in B_{m}}\Big\lbrace |\hat{r}_{i} - \tau | - 2s_{i}\Big\rbrace$
% \State where $s_j=\sqrt{\frac{\rho\psi_{m}\hat{v}_{j}\log{( T\epsilon_{m})}}{4 n_{j}} + \frac{\rho\psi_{m} \log{(T\epsilon_{m})}}{4 n_{j}}}$
\State $t\leftarrow t+1$ 
\vspace{-4mm}
%\ArmElim
%\State For each arm $i \in B_{m}$, remove arm ${i}$ from $B_{m}$ if
%\begin{align*}
%\hat{r}_{i} + c_i  < \tau - c_i \mbox{ or } \hat{r}_{i} - c_i  > \tau + c_i \\
%\text{where $c_i=\sqrt{\frac{\rho_{\mu}\psi_{m}\log{( T\epsilon_{m})}}{2 n_{i}}}$}
%\end{align*}
%\EndArmElim
%\ArmElimV
%\State \For{$i\in B_m$}
%\State For each arm $i \in B_{m}$, remove arm ${i}$ from $B_{m}$ if
\State \For{$i\in B_m$}
\vspace{-4mm}
\State \If{$(\hat{r}_{i} + s_i  < \tau - s_i)$ or $(\hat{r}_{i} - s_i > \tau + s_i)$}
\State $B_m\leftarrow B_m\backslash\{i\}$\hspace{4mm} (Arm deletion)
\EndIf
\EndFor
%\begin{align*}
%\hat{r}_{i} + s_i  < \tau - s_i,\hspace{1mm} \mbox{ or } \hspace{1mm}\hat{r}_{i} - s_i  > \tau + s_i \\
%% \text{where $s_i=\sqrt{\frac{\rho\psi_{m}\hat{v}_{i}\log{( T\epsilon_{m})}}{4 n_{i}} + \frac{\rho\psi_{m} \log{(T\epsilon_{m})}}{4 n_{i}}}$}
%\end{align*}
%\EndFor
%\EndArmElimV
\vspace{-2mm}
\State \If{$t\geq N_{m}$ and $m \leq M$}
%\ResetParam
\State \textbf{Reset Parameters}
\State $\epsilon_{m+1}\leftarrow\frac{\epsilon_{m}}{2}$
\State $B_{m+1} \leftarrow B_{m}$
\State $\psi_{m+1}\leftarrow \frac{T\epsilon_{m+1}}{128(\log(\frac{3}{16}K\log K))^{2}}$
\State $\ell_{m+1}\leftarrow\left\lceil \frac{2\psi_{m+1}\log( T\epsilon_{m+1})}{\epsilon_{m+1}} \right\rceil$
\State $N_{m+1} \leftarrow t + |B_{m+1}|\ell_{m+1}$
\State $m \leftarrow m+1$
%\EndResetParam
\EndIf
\EndFor
\State \textbf{Output:} $\hat{S}_{\tau}=\lbrace i: \hat{r}_{i}\geq \tau \rbrace$.
\end{algorithmic}
\end{algorithm}


%Also because of the said condition, like \cite{liu2016modification} we also claim that AugUCB is an anytime algorithm.



\section{Theoretical Results}
\label{tbandit:results}
\section{Results}
\label{sec:results}
% \begin{table}[!ht]
% \centering
% \caption{My caption}
% \label{my-label}
% \begin{tabular}{|p{1.2cm}|l|l|l|l|l|l|}
% \hline
% \multirow{2}{*}{\begin{tabular}[c]{@{}c@{}}Feature \\ Name\end{tabular}} & \multicolumn{2}{c}{Vt Classifier}                                       & \multicolumn{4}{|c|}{Size classifier}                                                                                                   \\ \cline{2-7} 
%                                                                          & \multicolumn{1}{c|}{1} & \multicolumn{1}{c|}{2} & \multicolumn{1}{c|}{ 1} & \multicolumn{1}{c|}{ 2} & \multicolumn{1}{c|}{3} & \multicolumn{1}{c|}{4} \\ \hline
% Sub-circuit                                                              & \multicolumn{1}{l|}{}        &                               &                               &                               &                              &                              \\ \hline
% Gate Type                                                                & \multicolumn{1}{l|}{}        &                               &                               &                               &                              &                              \\ \hline
% LNS                                                  & \multicolumn{1}{l|}{}        &                               &                               &                               &                              &                              \\ \hline
% \#Fanins                                                                 & \multicolumn{1}{l|}{}        &                               &                               &                               &                              &                              \\ \hline
% \#Fanouts                                                                & \multicolumn{1}{l|}{}        &                               &                               &                               &                              &                              \\ \hline
% \begin{tabular}[c]{@{}l@{}}\#Negative \\ Slack Paths\end{tabular}                                          & \multicolumn{1}{l|}{}        &                               &                               &                               &                              &                              \\ \hline
% Slack                                                                    & \multicolumn{1}{l|}{}        &                               &                               &                               &                              &                              \\ \hline
% \end{tabular}
% \end{table}
% \begin{table*}[!ht]
% \centering
% \caption{result3}
% \label{results3}
% \begin{tabular}{|l|c|l|l|l|l|l|l|l|l|l|l|}
% \hline
% \multicolumn{1}{|c|}{\begin{tabular}[c]{@{}c@{}}Benchmark \\  Name\end{tabular}} & \begin{tabular}[c]{@{}c@{}}Number \\ Of gates\end{tabular} & \multicolumn{1}{c|}{\begin{tabular}[c]{@{}c@{}}Target \\ Delay\end{tabular}} & \begin{tabular}[c]{@{}l@{}}Inital \\ Delay\end{tabular} & \multicolumn{2}{c|}{SVM}                                         & \multicolumn{2}{c|}{Final}                                      & \multicolumn{2}{c|}{Igor Markov}                               & \multicolumn{2}{c|}{Flach}                                     \\ \hline
% \multicolumn{1}{|c|}{}                                                           &                                                            & \multicolumn{1}{c|}{}                                                        & \multicolumn{1}{c|}{}                                   & \multicolumn{1}{c|}{Delay (ns)} & \multicolumn{1}{c|}{Power (W)} & \multicolumn{1}{c|}{Delay (ns)} & \multicolumn{1}{c|}{Power(W)} & \multicolumn{1}{c|}{Delay(ns)} & \multicolumn{1}{c|}{Power(W)} & \multicolumn{1}{c|}{Delay(ns)} & \multicolumn{1}{c|}{Power(W)} \\ \hline
% DMA\_fast                                                                        & 25.3K                                                      &                                                                              &                                                         &                                 &                                &                                 &                               &                                &                                                0.299    &       &                               \\ \hline
% DMA\_slow                                                                        & 25.3K                                                      &                                                                              &                                                         &                                 &                                &                                 &                               &                                &                                                       0.145   &     &                               \\ \hline
% pci\_fast                                                              & 33.2K                                                      &                                                                              &                                                         &                                 &                                &                                 &                               &                                &                                                     0.183     &     &                               \\ \hline
% pci\_slow                                                              & 33.2K                                                      &                                                                              &                                                         &                                 &                                &                                 &                               &                                &                                                  0.111         &    &                               \\ \hline
% des\_perf\_fast                                                                  & 111K                                                       &                                                                              &                                                         &                                 &                                &                                 &                               &                                &                                                         1.842   &   &                               \\ \hline
% des\_perf\_slow                                                                  & 111K                                                       &                                                                              &                                                         &                                 &                                &                                 &                               &                                &                                                          0.614   &  &                               \\ \hline
% vga\_lcd\_fast                                                                   & 165K                                                       &                                                                              &                                                         &                                 &                                &                                 &                               &                                &                                                            0.471  & &                               \\ \hline
% vga\_lcd\_slow                                                                   & 165K                                                       &                                                                              &                                                         &                                 &                                &                                 &                               &                                &                                                            0.351  & &                               \\ \hline
% b19\_fast                                                                        & 219K                                                       &                                                                              &                                                         &                                 &                                &                                 &                               &                                &                                                           0.771   & &                               \\ \hline
% b19\_slow                                                                        & 219K                                                       &                                                                              &                                                         &                                 &                                &                                 &                               &                                &                                                             0.583 & &                               \\ \hline
% leon3mp\_fast                                                                    & 649K                                                       &                                                                              &                                                         &                                 &                                &                                 &                               &                                &                                                             1.487 & &                               \\ \hline
% leon3mp\_slow                                                                    & 649K                                                       &                                                                              &                                                         &                                 &                                &                                 &                               &                                &                                                            1.341  & &                               \\ \hline
% netcard\_fast                                                                    & 959K                                                       &                                                                              &                                                         &                                 &                                &                                 &                               &                                &                                                            1.861  & &                               \\ \hline
% netcard\_slow                                                                    & 959K                                                       &                                                                              &                                                         &                                 &                                &                                 &                               &                                &                                                           1.770  &  &                               \\ \hline
% \end{tabular}
% \end{table*}


% Please add the following required packages to your document preamble:
% \usepackage{multirow}
% Please add the following required packages to your document preamble:
% \usepackage{multirow}
% \begin{table}[]
% \centering
% \caption{My caption}
% \label{my-label}
% \begin{tabular}{|l|l|l|l|l|l|}
% \hline
% \multirow{3}{*}{Benchmark} & \multicolumn{5}{c|}{Runtime}                                                            \\ \cline{2-6} 
%                            & \multirow{2}{*}{Igor Markov} & \multirow{2}{*}{Flach} & \multicolumn{3}{c|}{\textit{MLTimer}} \\ \cline{4-6} 
%                            &                              &                        & SVM  & Delay Recovery  & Total  \\ \hline
% DMA\_fast                  &                              &                        &      &                 &        \\ \hline
% DMA\_slow                  &                              &                        &      &                 &        \\ \hline
% pci\_fast        &                              &                        &      &                 &        \\ \hline
% pci\_brdige32\_slow        &                              &                        &      &                 &        \\ \hline
% vga\_lcd\_fast             &                              &                        &      &                 &        \\ \hline
% vga\_lcd\_slow             &                              &                        &      &                 &        \\ \hline
% des\_perf\_fast            &                              &                        &      &                 &        \\ \hline
% des\_perf\_slow            &                              &                        &      &                 &        \\ \hline
% b19\_fast                  &                              &                        &      &                 &        \\ \hline
% b19\_slow                  &                              &                        &      &                 &        \\ \hline
% leon3mp\_fast              &                              &                        &      &                 &        \\ \hline
% leon3mp\_slow              &                              &                        &      &                 &        \\ \hline
% netcard\_fast              &                              &                        &      &                 &        \\ \hline
% netcard\_slow              &                              &                        &      &                 &        \\ \hline
% \end{tabular}
% \end{table}


% Please add the following required packages to your document preamble:
% \usepackage{multirow}
% Please add the following required packages to your document preamble:
% \usepackage{multirow}
% Please add the following required packages to your document preamble:
% \usepackage{multirow}
\begin{table*}[!t]
\caption{Leakage power and Runtime comparisons between the baseline greedy algorithm and the \textit{MLTimer} algorithm on the ISPD 2012 benchmarks. Implementation 1 is the baseline implementation(non-SVM,non-adaptive timing analysis), Implementation 2 is with SVM and non-adaptive timing analysis, Implementation 3 is with non-SVM and adaptive timing analysis and Implementation 4 is with SVM and adaptive timing analysis. It can be seen that using just SVM improves the solution quality greatly, while using just the adaptive timing analysis improves the runtime. A combination of both improves the runtime and solution qualtiy.}
\label{tab:tab5}

\begin{tabular}{|l|l|l|l|l|l|l|l|l|l|}
\hline
\multirow{2}{*}{Benchmarks} & \multirow{2}{*}{\#Gates} & \multicolumn{2}{l|}{Implementation 1}                                                                                                         & \multicolumn{2}{l|}{Implementation 2}                                                                                                           & \multicolumn{2}{l|}{Implementation 3}                                                                                                        & \multicolumn{2}{l|}{Implementation 4}                                                                                                        \\ \cline{3-10} 
                            &                          & \begin{tabular}[c]{@{}l@{}}Run-\\ time \\ (mins)\end{tabular} & \begin{tabular}[c]{@{}l@{}}Leakage \\ Power\\ (W)\end{tabular} & \begin{tabular}[c]{@{}l@{}}Run-\\ time\\ (mins)\end{tabular} & \begin{tabular}[c]{@{}l@{}}Leakage \\ Power\\ \\ (W)\end{tabular} & \begin{tabular}[c]{@{}l@{}}Run-\\ time\\ (mins)\end{tabular} & \begin{tabular}[c]{@{}l@{}}Leakage\\  Power\\ (W)\end{tabular} & \begin{tabular}[c]{@{}l@{}}Run-\\ time\\ (mins)\end{tabular} & \begin{tabular}[c]{@{}l@{}}Leakage \\ Power\\ (W)\end{tabular} \\ \hline
DMA\_fast                   & 23,000                   & 16                                                            & 0.79                                                           & 14.00                                                        & 0.30                                                              & 14.00                                                        & 0.79                                                           & 13.00                                                        & 0.30                                                           \\ \hline
pci\_bridge32\_fast         & 30,000                   & 37                                                            & 0.25                                                           & 17.00                                                        & 0.14                                                              & 17.00                                                        & 0.24                                                           & 17.00                                                        & 0.14                                                           \\ \hline
des\_perf\_fast             & 102,000                  & 219                                                           & 1.73                                                           & 164.00                                                       & 1.80                                                              & 190.00                                                       & 1.73                                                           & 130.00                                                       & 1.80                                                           \\ \hline
vga\_lcd\_fast              & 148,000                  & 384                                                           & 2.80                                                           & 139.00                                                       & 0.47                                                              & 207.00                                                       & 2.72                                                           & 77.00                                                        & 0.47                                                           \\ \hline
b19\_fast                   & 213,000                  & 547                                                           & 2.13                                                           & 239.00                                                       & 0.75                                                              & 366.00                                                       & 2.13                                                           & 174.00                                                       & 0.75                                                           \\ \hline
leon3mp\_fast               & 540,000                  & 2,046                                                         & 4.00                                                           & 875.00                                                       & 1.49                                                              & 716.00                                                       & 4.00                                                           & 639.00                                                       & 1.49                                                           \\ \hline
netcard\_fast               & 861,000                  & 1,033                                                         & 2.09                                                           & 519.00                                                       & 1.77                                                              & 609.00                                                       & 2.07                                                           & 306.00                                                       & 1.77                                                           \\ \hline
\end{tabular}
\end{table*}

\begin{table*}[!ht]
%\centering
\caption{Leakage power comparisons with ISPD 2012 contest winners and other state of the art works. We use geometric mean to calculate the efficiency of our proposed solution. We exclude the infeasible solutions in our mean calculation. All the solutions reported below have no timing violations.}
\label{tab:tab6}

    \begin{tabular}{|l|l|l|p{1.2cm}|p{1.6cm}|p{1.6cm}|p{1cm}|l|p{1.2cm}|}
\hline
\multirow{2}{*}{Benchmark} & \multirow{2}{*}{\begin{tabular}[c]{@{}l@{}}Number \\ of gates\end{tabular}} & \multicolumn{5}{c|}{Leakage Power (W)} & \multicolumn{2}{c|}{Runtime (mins)}\\ \cline{3-9} 
    &  & \cite{hu:12}  & NTUgs & UFRGSgs & Powervalve & \textbf{Ours} & \cite{hu:12} & \textbf{Ours}\\ \hline
    \texttt{DMA\_fast} & 23,000 & 0.30  & 0.51 & 0.32 & 0.31 & 0.30 & 13.90 & 13.30\\ \hline
    \texttt{DMA\_slow} & 23,000  & 0.15  & 0.21 & 0.16 & 0.15 & 0.14 & 9.90 & 7.51 \\ \hline
    \texttt{pci\_fast} & 30,000 & 0.18  & 0.51 & 0.17 & 0.23 & 0.14 & 13.00 & 17.10
     \\ \hline
    \texttt{pci\_slow} & 30,000 & 0.11   & 0.20 & 0.12 & 0.12 & 0.09 & 10.20 & 9.32 \\ \hline
    \texttt{des\_perf\_fast} & 102,000 & 1.84 & 2.39 & 3.52 & 2.32 & 1.80  & 82.70 & 130.40 \\ \hline
    \texttt{des\_perf\_slow} & 102,000 & 0.61 & 0.67 & 0.88 & 0.70 & 0.64 & 70.10 & 43.50 \\ \hline
    \texttt{vga\_lcd\_fast} & 148,000 & 0.47 & 0.76 & 0.58 & 0.77 & 0.47 & 45.60 & 77.32\\ \hline
    \texttt{vga\_lcd\_slow} & 148,000 & 0.35 & 0.42 & 0.38 & 0.39 & 0.37 & 87.50 & 50.40 \\ \hline
    \texttt{b19\_fast} & 213,000 & 0.77 & 2.71 & - & 4.49 & 0.75 & 206.50 & 174.11 \\ \hline
    \texttt{b19\_slow} & 213,000 & 0.58 & 0.63 & 0.61 & 0.74 & 0.61 & 213.90 & 102.20\\ \hline
    \texttt{leon3mp\_fast} & 540,000 & 1.49 & -&  - & 4.94 & 1.49 & 1,323.20 & 639.40\\ \hline
    \texttt{leon3mp\_slow} & 540,000 & 1.34 & 1.42 & 1.79 & 2.96 & 1.30 & 1,274.20 & 325.13  \\ \hline
    \texttt{netcard\_fast} & 861,000 & 1.86 & 2.01 & 2.30 & 2.97 & 1.86 & 1,096.90 & 306.57\\ \hline
    \texttt{netcard\_slow} & 861,000 & 1.77 & 1.77 & 1.97 & 1.94 & 1.77 & 299.90 & 164.14\\ \hline
Geometric mean &  & $1.03\times$ & $1.52\times$ & $1.13\times$ & $1.57\times$ &   & $1.44\times$ & \\ \hline
\end{tabular}
\end{table*}

% Please add the following required packages to your document preamble:
% \usepackage{multirow}
\begin{table*}[!ht]
\centering
\caption{Leakage power comparisons with \cite{hu:13} on the  ISPD 2013 contest benchmark. All the solutions reported below are violation free. It can be observed that \texttt{MLTimer} outperforms \cite{hu:13} both with respect to leakage power and runtime on the larger benchmarks. The detailed results for other benchmarks were not reported in \cite{hu:13}.}
\label{tab:tab34}
\begin{tabular}{|l|l|l|l|l|l|}
\hline
\multirow{2}{*}{Benchmark} & \multirow{2}{*}{Gates} & \multicolumn{2}{l|}{\texttt{MLTimer}}                                                                                                  & \multicolumn{2}{l|}{\cite{hu:13}}                                                                                                  \\ \cline{3-6} 
                           &                        & \begin{tabular}[c]{@{}l@{}}Run-\\ time\\ (mins)\end{tabular} & \begin{tabular}[c]{@{}l@{}}Leakage\\ Power\\ (mW)\end{tabular} & \begin{tabular}[c]{@{}l@{}}Run-\\ time\\ (mins)\end{tabular} & \begin{tabular}[c]{@{}l@{}}Leakage \\ Power\\ (mW)\end{tabular} \\ \hline
usb\_phy\_fast             & 510                    & 0.48                                                         & 2.03                                                           & \textbf{0.21}                                                & \textbf{1.56}                                                   \\ \hline
usb\_phy\_slow             & 510                    & \textbf{0.11}                                                & 1.13                                                           & 0.17                                                         & \textbf{1.07}                                                   \\ \hline
pci\_bridge32\_fast        & 28,000                 & 20.83                                                        & 116.87                                                         & \textbf{12.00}                                               & \textbf{101.90}                                                 \\ \hline
pci\_bridge32\_slow        & 28,000                 & 6.78                                                         & 58.91                                                          & \textbf{5.39}                                                & \textbf{58.83}                                                  \\ \hline
fft\_fast                  & 31,000                 & 40.00                                                        & 320.37                                                         & \textbf{32.58}                                               & \textbf{305.29}                                                 \\ \hline
fft\_slow                  & 31,000                 & 25.00                                                        & 96.69                                                          & \textbf{17.40}                                               & \textbf{93.10}                                                  \\ \hline
cordic\_slow               & 42,000                 & \textbf{94.40}                                               & \textbf{397.81}                                                & 98.39                                                        & 511.91                                                          \\ \hline
des\_perf\_slow            & 104,000                & 88.18                                                        & 386.41                                                         & \textbf{62.30}                                               & \textbf{375.80}                                                 \\ \hline
edit\_dist\_fast           & 121,000                & \textbf{163.10}                                              & \textbf{572.12}                                                & 170.60                                                       & 619.30                                                          \\ \hline
edit\_dist\_slow           & 121,000                & \textbf{56.34}                                               & \textbf{423.50}                                                & 107.20                                                       & 465.60                                                          \\ \hline
matrix\_mult\_slow         & 153,000                & \textbf{139.80}                                              & \textbf{482.23}                                                & 212.60                                                       & 499.90                                                          \\ \hline
netcard\_fast              & 884,000                & \textbf{372.70}                                              & \textbf{5,157.93}                                              & 716.80                                                       & 5271.80                                                         \\ \hline
netcard\_slow              & 884,000.               & \textbf{297.12}                                              & \textbf{5,102.25}                                              & 439.60                                                       & 5183.89                                                         \\ \hline
GEOMETRIC MEAN             &                &                                               &                                               & 1.005                                                       & 1.005                                                         \\ \hline

\end{tabular}
\end{table*}

% \begin{table*}[!ht]
% \begin{center}
% \label{results3}
% \begin{tabular}{|p{2.1cm}|p{2cm}|p{2cm}|p{2cm}|p{2cm}|p{1.5cm}|}
% \hline
% Benchmark & $\#$ gates & %\multicolumn{2}{|c|}
% {\textit{MLTimer}} &%\multicolumn{2}{|c|} 
% {Igor Markov} ~\cite{hu:12} & Improvement \\
% \hline
%   &  & Leakage Power (W)      %& Running Time    
%   & Leakage Power (W) & \\% & Running Time \\ 
 
% \hline
% %USB\_PHY &536 &$<$1s &$<$1s & - \\
% \hline
% DMA\_fast & 25.3K & 0.08W %& 17m
% & 0.299W & 73\% \\ %& 13m\\
% \hline
% %DMA\_slow & 25.3 & 0.134W %& 1m44s
% %& 0.145W & 7\%  \\% & 9.9m\\
% %\hline
% pci\_bridge	&	33.2K	&	0.1331W %&	1m29s
% &	0.183W & 27\%\\%	&	13m \\
% \hline
% %pci\_bridge\_slow	&	33.2K	&	0.07W	%&	8m
% %&	0.111W & 36\% \\%	&	11m \\
% %\hline 
% b19	&	219K  &		.58W	%&	9h
% &	0.771W & 24\% \\%	&	206m \\
% \hline
% %b19\_slow 	&	219K &		0.486W%	&	9h	
% %&	0.583W & 19.21\% \\	%&	213m \\
% %\hline
% Des\_perf & 165K & .546W %& 1331m       
% & .471W & \-15.3\% \\%    & 45m \\
% \hline
% netcard & 959k & 1.8W %& 2046m 
% & 1.861W & 6.01 \% \\% & 1096m \\
% \hline
% leon3mp & 649K & 2W %&  2816m 
% & 1.487W & \-34.5\% \\ %& 1323.2 \\
% \hline
% Average & & & & 21.37\% \\ \hline
% %leon3mp\_slow & 649K & 2W &  2816m & 1.487W & 1323.2 \\
% %\hline
% \end{tabular}
% \caption{Leakage and Running Time Comparisons for ISPD benchmarks between \textit{MLTimer} and Igor Markov. In the table, {\bf h}, {\bf m} and {\bf s} stand for hours, minutes and seconds respectively.}


% \end{center}
% \end{table*}
\subsection{Comparisons with state-of-the-art}

The performance of our proposed algorithm is shown in Table~\ref{tab:tab5}. A simple greedy algorithm, implemented for obtaining the final $V_t$ and $size$ values, serves as the baseline algorithm. It can be seen that our \textit{MLTimer} implementation outperforms the baseline algorithm by 46\% in terms of solution quality. It can also be seen that the SVM module improves the solution quality and the adaptive timing analysis module improves the runtime. 

We compare the performance of our algorithm with ~\cite{hu:12} which is the best performing heuristic based algorithm reported so far in the literature. We use the ISPD 2012 benchmark set and SHAKTIC to quantify the performance our algorithm. In comparing with the state-of-the-art techniques we make the following observations:
\begin{itemize}
\item Our solution outperforms the top 3 submissions of the ISPD 2012 contest NTUgs, UFRGSgs and Powervalve by 52\%,13\% and 57\% respectively.
\item Our solution outperforms \cite{hu:12} both in terms of average runtime and solution quality by 44\% and 3\% respectively. Table~\ref{tab:tab6} highlights the performance of \textit{MLTimer} algorithm in terms of runtime and solution quality. This is because as most of the circuits share a large  number of repeating sub-circuits whose value is accurately predicted by the SVM engine and hence these gates do not undergo delay and power recovery algorithm leading to savings in runtime. 
\item It can be seen from Table~\ref{tab:tab34} that our tool outperforms \cite{hu:13} which is an extension of \cite{hu:12}. It can be observed that while \textit{MLTimer} underperforms for the smaller benchmarks, it significantly outperforms \cite{hu:13} on the larger benchmarks. Although the overall improvement in solution quality is around 0.004\%, the improvement in the larger benchmarks is around 53\% for the runtime and 10\% for solution quality.
\item In Table~\ref{tab:tab9} we compare our implementation with a commercial synthesis tool and our implementation of \cite{hu:12}. It can be observed our proposed solution performs significantly better than the commercial tool in terms of leakage power. 
\end{itemize}

\begin{table}[!t]
    \caption{The Table comparing the performance of \textit{MLTimer} versus a commercial synthesis tool on SHAKTIC. We see that the solution quality is 57\% better than that of the tool.}
    \label{tab:tab9}

    \centering
    \begin{tabular}{|l|l|l|l|l|l|l|}
        \hline
        \textbf{Metric}           & \multicolumn{2}{c|}{Commercial Tool}                                                                                     &        &            & \multicolumn{2}{c|}{Percentage Improvement} \\ \hline
                         & \begin{tabular}[c]{@{}l@{}}$LV_t$\\  synthesis\end{tabular} & \begin{tabular}[c]{@{}l@{}}Mixed $V_t$ \\ synthesis\end{tabular} & \cite{hu:12} & \textit{MLTimer} & Tool              & \cite{hu:12}       \\ \hline
                    %         \textbf{Runtime (mins)}    & 38                                                       & 14                                                            & 87     & 64         & -78.12           &  26.44       \\ \hline
                             \textbf{Leakage power (W)} & 5                                                        & 1                                                             & 0.59  & 0.43       & 57        & 27        \\ \hline
    \end{tabular}

\end{table}


\subsection{Analysis of the Learning Module}


\begin{table}[!t]
\caption{Table showing the weights assigned to each feature at each stage of the $V_t$ and $size$ classifiers. An extremely low magnitude implies that the corresponding feature does not contribute significantly to the output and can thus be discarded. However it can be seen that none of the features chosen fall into that category.}
\label{tab:tab7}

    \centering

\begin{tabular}{|l|l|l|l|l|l|l|l|}
\hline
    \multirow{2}{*}{\textbf{Feature}}       & \multicolumn{2}{c|}{$\mathbf{V_t}$} & \multicolumn{5}{c|}{\textbf{size}}             \\ \cline{2-8} 
                               & 1          & 2          & 1     & 2     & 3     & 4     & 5     \\ \hline
    \textbf{Sub-circuit}                    & -0.88      & -0.43      & -0.58 & 1.32  & -0.08 & 0.44  & 0.16  \\ \hline
    \textbf{Gate type}                      & -0.19      & -0.43      & -0.96 & -0.81 & 0.42  & -0.40 & 0.29  \\ \hline
    \textbf{LNS}                            & 1.09       & 0.33       & -0.07 & -0.11 & 0.76  & 0.76  & 0.08  \\ \hline
    \textbf{Number of Fanins}               & 2.64       & 0.04       & 3.27  & -2.38 & -1.10 & -0.34 & -1.26 \\ \hline
    \textbf{Number of Fanouts}              & -2.92      & -0.29      & -4.08 & -0.53 & 1.82  & 0.50  & -1.07 \\ \hline
    \textbf{Number of Negative Slack Paths} & 0.33       & -0.16      & 0.50  & 0.40  & -0.22 & 0.21  & -0.68 \\ \hline
    \textbf{Slack}                          & 1.89       & 0.35       & 1.22  & -3.20 & -1.64 & -1.41 & 0.37  \\ \hline
\end{tabular}

\end{table}


The learning module forms a critical component of our framework as it serves to reduce the runtime by using a simple SVM model that uses seven features.  A complex ML model with large number of redundant features might cause runtime overheads due to i) complex training procedure ii) complicated inference procedure, and; iii) reduced interpretability of the ML model. Hence there is a need to eliminate the redundant features in order to simplify the learning module. Logistic regression was performed to estimate the importance of the chosen features. The Logistic Regression model was initially trained on the set of chosen features and the importance of each feature,  obtained via the coefficient assigned by the model,  is quantified in Table~\ref{tab:tab7}.  It can be observed that none of the feature weights have extremely low value and hence cannot be eliminated.


%As mentioned earlier, an improperly trained learning engine could initialize the netlist to a sub-optimal configuration leading to more delay and power recovery cycles than necessary thereby increasing the runtime overhead. The thresholding function plays an important role in predicting the final choice ($V_t$/$size$) for a given cell. We use the b19\_fast benchmark to show the impact of varying the thresholding function on the solution quality of the SVM engine. We show the impact of the thresholding function in table ~\ref{results4}. We see that as the thresholding function increases the runtime goes up. This is because the number of gates that are marked unsure increases causing more delay and power optimizations. We us class probability to determine the class label ($V_t$/$size$). We use a thresholding value of $0.75$ for both the $V_t$ classifiers while we use a thresholding value of x and y for the $size$ classifiers. 





% \begin{table*}[!t]
% \parbox{.3\linewidth}{
% \begin{center}

% \begin{tabular}{|p{3cm}|p{1.3cm}|}
% \hline
% Metric & Number \\ \hline
% Total gates & 333\\
% \hline
% Combinational gate types &  11 \\ \hline
% Sequential gates & 1 \\ \hline
% $V_t$ choices & 3 \\ \hline
% $size$ choices & 10 \\ \hline
% $V_{cc}$ and $gnd$ cells & 2 \\ \hline
% %leon3mp\_slow & 649K & -6401 &  -6479 & 1W & 47s \\
% %\hline
% \end{tabular}
% \caption{Library statistics}
% \label{tab:lib}
% \end{center}
% %\end{table*}
% }
% \hfill
% \parbox{.6\linewidth}{
% %\begin{table*}[!t]
% \begin{center}

% \begin{tabular}{|p{2cm}|p{1cm}|p{1cm}|p{1.6cm}|p{1.6cm}|p{1.6cm}|}
% \hline
% Benchmark & \#Input & \#Output & \#Comb cell & \#Seq cell & \#Total cell \\
% \hline
% DMA &  683 & 276 & 23109 & 2192 & 25301\\ \hline
% pci & 160 & 201& 29844& 3359& 33203\\ \hline
% des\_perf &  234 & 140 & 102427 & 8802& 111229\\ \hline
% vga\_lcd &  85 & 99 &  147812 & 17079 & 164891\\ \hline
% b19 &  22 & 25 &  212674 & 6594 & 219268\\ \hline
% leon3mp & 254 & 79 & 540352 &  108839 & 649191\\ \hline
% netcard &  1836 & 10 & 860949 & 97831 & 958780\\ \hline

% %leon3mp\_slow & 649K & -6401 &  -6479 & 1W & 47s \\
% %\hline
% \end{tabular}
% \caption{Benchmark statistics}
% \label{tab:benchmark}
% \end{center}
% }
% \end{table*}


% Please add the following required packages to your document preamble:
% \usepackage{booktabs}
% \usepackage{multirow}
% Please add the following required packages to your document preamble:
% \usepackage{multirow}
% Please add the following required packages to your document preamble:
% \usepackage{multirow}
% Please add the following required packages to your document preamble:
% \usepackage{multirow}

% \begin{table*}[!t]
% \parbox{.5\linewidth}{
% \begin{center}

% \label{tab:log}
% \begin{tabular}{|p{2.5cm}|p{3cm}|}
% \hline
% Feature & Weight \\
% \hline
% Gate footprint &-0.8832794232852276 \\ \hline
% Gateid & -0.1914242984939835 \\ \hline
% Local negative slack & 1.090781658200261 \\ \hline
% \#Fanins & 2.642338600894165  \\ \hline
% \#Fanouts & -2.92081747517804  \\ \hline
% \#Negative slack paths & 0.3349856667382776  \\ \hline
% Slack & 1.894864001133556 \\ \hline

% %leon3mp\_slow & 649K & -6401 &  -6479 & 1W & 47s \\
% %\hline
% \end{tabular}
% \caption{Feature Weights for the first $V_t$ classifier }
% \end{center}
% %\end{table*}
% }
% \hfill
% %\begin{table*}[!h]
% \parbox{.5\linewidth}{
% \begin{center}

% \label{tab:log2}
% \begin{tabular}{|p{2.5cm}|p{3cm}|}
% \hline
% Feature & Weight \\
% \hline
% Gate footprint & 0.03119793516364029
% \\ \hline
% Gateid &  0.346427255804566\\ \hline
% Local negative slack & -0.01019723367101055\\ \hline
% \#Fanins & -0.007490175140922838 \\ \hline
% \#Fanouts & -0.2180352793079041 \\ \hline
% \#Negative slack paths & -0.06062286295401311  \\ \hline
% Slack & 0.663389798807628 \\ \hline

% %leon3mp\_slow & 649K & -6401 &  -6479 & 1W & 47s \\
% %\hline
% \end{tabular}
% \caption{Feature Weights for the second stage $V_t$ classifier }
% \end{center}
% }
% \end{table*}

The efficiency of the SVM engine is analyzed in Table~\ref{tab:tab8}. We see that on an average the SVM engine is able to recover a significant amount of power in a short amount of time. However, It can be observed that the solution provided by the SVM engine is not optimal hence  the delay and leakage power recovery steps are used to further optimize the solution provided by the learning step.

 \begin{table*}[!t]
  \caption{Leakage and Running Time Comparisons for ISPD benchmarks and ShaktiC with just SVM. In the table, {\bf h}, {\bf m} and {\bf s} stand for hours, minutes and seconds respectively. It can be seen that with the exception of leon3mp our SVM implementation is able to recover significant delay and power.} \
\label{tab:tab8}

     \begin{center}
 \begin{tabular}{|p{4.2cm}|p{2cm}|p{2.2cm}|p{2cm}|p{2cm}|p{2cm}|}
 \hline
    \textbf{Benchmark} & \textbf{Gate count} & \textbf{Initial Worst Negative Slack (WNS)} & \multicolumn{3}{|c|}{ \textit{MLTimer}}  \\
 \hline
   &   & & WNS (ps) &  Leakage Power (W)      & Running Time          \\
 \hline
     \texttt{ DMA\_fast} & 25,300&  -1485 & -774 &0.09  & 3s \\
 \hline
     \texttt{pci\_bridge32\_fast}	& 33,200& -1881 & -2284	&	0.18   &	3s	 \\
 \hline
     \texttt{des\_perf\_fast} &  102,000 & -669 & -1029 &.316 & 1m        \\
 \hline
     \texttt{vga\_lcd\_fast} & 148,000 & -1254 & -2964 &.29 & 1m          \\
\hline

     \texttt{b19\_fast}	&	219,000 & -2835 & -1738	&	1.6 	&	15s	\\ \hline
     \texttt{leon3mp\_fast} & 649,000 & -6401 & -3913 & 21 &  47s \\
 \hline

     \texttt{netcard\_fast} & 959,000 & -4102 &-3268 & 8 & 1m  \\ \hline
     \texttt{ShaktiC} & 174,756 & -5199 & -1067 & 0.67 & 1m \\ 
 \hline
 \end{tabular}
 \end{center}

 \end{table*}
\section{Conclusion}
\label{sec:conclusion}
Leakage optimization  techniques have been studied extensively for more than a decade.  However, the lack of a robust algorithm that is optimal in terms of both execution time and solution quality motivates research in this area. It is seen that varying window size adaptively according to the status of the timing updates produces faster solutions than for a fixed window size. The proposed \textit{MLTimer} algorithm improves the running-time considerably while still retaining the solution quality of a greedy heuristic. It is observed that for large circuits \textit{MLTimer} with initial configuration provided by SVM performs significantly better than when used with power optimal configuration as initial solution.  Extending the concepts involved in the construction of \textit{MLTimer} to other steps of EDA including placement and routing is an interesting direction for future work.
% * <sristisravan@gmail.com> 2017-06-28T10:13:29.636Z:
% 
% Check "... the lack of a robust heuristic that optimal in terms of both.... "
% 
% ^.

%\begin{table*}[t]
% \begin{center}
% \caption{Leakage and Running Time Comparisons for ISPD benchmarks with just SVM and delay recovery. In the table, {\bf h}, {\bf m} and {\bf s} stand for hours, minutes and seconds respectively.}
% \label{results2}
% \begin{tabular}{|p{1.7cm}|p{2cm}|p{2cm}|p{2cm}|p{2cm}|}
% \hline
% Benchmark & $\#$ gates & \multicolumn{3}{|c|}{\textit{MLTimer}}  \\
% \hline
%   &  & Delay(ps) &  Leakage Power (W)      & Running Time          \\
% \hline
% %USB\_PHY &536 &$<$1s &$<$1s & - \\
% \hline
% DMA_fast & 25.3K & &0.08W & 17m \\
% \hline
% DMA_slow & 25.3 & &0.134W & 1m44s \\
% \hline
% pci_bridge_fast	& 33.2K&	&	0.1331W &	1m29s	 \\
% \hline
% pci_bridge_slow	& 	33.2K&	&	0.07W	&	8m	\\
% \hline 
% b19_fast	&	219K  &	&	.58W	&	9h	\\
% \hline
% b19_slow 	&	219K & &		0.486W	&	9h	 \\
% \hline
% Des_perf_fast &  165K & &.546W & 1331m        \\
% \hline
% Des_perf_slow &  165K & & .546W & 1331m         \\
% \hline
% vga_lcd_slow & 165K & & .546W & 1331m          \\
% \hline
% vga_lcd_slow & 165K &  &.546W & 1331m          \\
% \hline
% netcard_fast & 959k & & 1.8W & 2046m  \\
% \hline
% netcard_slow & 959k & & 1.8W & 2046m \\
% \hline
% leon3mp_fast & 649K & & 2W &  --- \\
% \hline
% leon3mp_slow & 649K & & 2W &  --- \\
% \hline
% \end{tabular}
% \end{center}
%\end{table*}







\section{Numerical Experiments}
\label{tbandit:expt}
In this section we present two experiments in two different environments.


\begin{figure}[!th]
    \centering
    \begin{tabular}{cc}
    %\setlength{\tabcolsep}{0.1pt}
    \subfigure[\Large\textwidth][\large Expt-$1$: $3$ Bernoulli-distributed arms (From Dr. Odalric's Draft).]
    %with $r_{i_{{i}\neq {*}}}=0.07$ and $r^{*}=0.1$
    {
    		\pgfplotsset{
		tick label style={font=\normalsize},
		label style={font=\normalsize},
		legend style={font=\normalsize},
		ylabel style={yshift=12pt},
		%legend style={legendshift=32pt},
		}
        \begin{tikzpicture}[scale=0.7]
      	\begin{axis}[
		xlabel={timestep},
		ylabel={Cumulative Regret},
		grid=major,
        %clip mode=individual,grid,grid style={gray!30},
        clip=true,
        %clip mode=individual,grid,grid style={gray!30},
  		legend style={at={(0.5,1.4)},anchor=north, legend columns=3} ]
      	% UCB
		
		\addplot table{Chapter6/results/NewExpt/Expt5/comp_subsampled_DUCB01.txt};
		\addplot table{Chapter6/results/NewExpt/Expt5/comp_subsampled_ETS01.txt};
		\addplot table{Chapter6/results/NewExpt/Expt5/comp_subsampled_ETS02.txt};
		\addplot table{Chapter6/results/NewExpt/Expt5/comp_subsampled_ETS03.txt};
		\addplot table{Chapter6/results/NewExpt/Expt5/comp_subsampled_ETS1E01.txt};
		\addplot table{Chapter6/results/NewExpt/Expt5/comp_subsampled_ETS1E02.txt};
		\addplot table{Chapter6/results/NewExpt/Expt5/comp_subsampled_TS01.txt};
		\addplot table{Chapter6/results/NewExpt/Expt5/comp_subsampled_OTS01.txt};
		\addplot table{Chapter6/results/NewExpt/Expt5/comp_subsampled_DTS01.txt};
      	
      	\legend{DUCB($\gamma=1-\frac{1}{4\sqrt{T}}$),ETSDAE1,ETSDAE2,ETSDAE3,ETSD1E1,ETSD1E2,TS,OTS,DTS($\gamma=1-\frac{1}{4\sqrt{T}})$}    
      	\end{axis}
      	\end{tikzpicture}
  		\label{psbandit:fig:1}
    }
    &
    \subfigure[\Large\textwidth][\large Expt-$1$: $3$ Bernoulli-distributed arms (From Dr. Odalric's Draft).]
    %with $r_{i_{{i}\neq {*}}}=0.07$ and $r^{*}=0.1$
    {
    		\pgfplotsset{
		tick label style={font=\normalsize},
		label style={font=\normalsize},
		legend style={font=\normalsize},
		ylabel style={yshift=12pt},
		%legend style={legendshift=32pt},
		}
        \begin{tikzpicture}[scale=0.7]
      	\begin{axis}[
		xlabel={timestep},
		ylabel={Cumulative Regret},
		grid=major,
        %clip mode=individual,grid,grid style={gray!30},
        clip=true,
        %clip mode=individual,grid,grid style={gray!30},
  		legend style={at={(0.5,1.4)},anchor=north, legend columns=3} ]
      	% UCB
		
		\addplot table{Chapter6/results/NewExpt/Expt5/comp_subsampled_DUCB01.txt};
		\addplot table{Chapter6/results/NewExpt/Expt5/comp_subsampled_ETS04.txt};
		\addplot table{Chapter6/results/NewExpt/Expt5/comp_subsampled_ETS1E03.txt};
		\addplot table{Chapter6/results/NewExpt/Expt5/comp_subsampled_ETS06.txt};
		\addplot table{Chapter6/results/NewExpt/Expt5/comp_subsampled_ETS1E04.txt};
		\addplot table{Chapter6/results/NewExpt/Expt5/comp_subsampled_TS01.txt};
		\addplot table{Chapter6/results/NewExpt/Expt5/comp_subsampled_OTS01.txt};
		\addplot table{Chapter6/results/NewExpt/Expt5/comp_subsampled_DTS01.txt};
		\addplot table{Chapter6/results/NewExpt/Expt5/comp_subsampled_ETS1E06.txt};
      	
      	\legend{DUCB($\gamma=1-\frac{1}{4\sqrt{T}}$),EAggrCPD1,CPD1-E1,EAggrCPD2,CPD2-E1,TS,OTS,DTS($\gamma=1-\frac{1}{4\sqrt{T}})$,CPD2-E1-N}    
      	\end{axis}
      	\end{tikzpicture}
  		\label{psbandit:fig:2}
    }
    \end{tabular}
    \caption{Cumulative regret for various bandit algorithms on a piecewise stochastic 3-armed bandit environment. }
    \label{fig:karmed1}
\end{figure}


%\begin{figure}[!th]
%    \centering
%    \begin{tabular}{c}
%    %\setlength{\tabcolsep}{0.1pt}
%    \subfigure[\Large\textwidth][\large Expt-$1$: $3$ Bernoulli-distributed arms (From Dr. Odalric's Draft).]
%    %with $r_{i_{{i}\neq {*}}}=0.07$ and $r^{*}=0.1$
%    {
%    		\pgfplotsset{
%		tick label style={font=\normalsize},
%		label style={font=\normalsize},
%		legend style={font=\normalsize},
%		ylabel style={yshift=12pt},
%		%legend style={legendshift=32pt},
%		}
%        \begin{tikzpicture}[scale=0.7]
%      	\begin{axis}[
%		xlabel={timestep},
%		ylabel={Cumulative Regret},
%		grid=major,
%        %clip mode=individual,grid,grid style={gray!30},
%        clip=true,
%        %clip mode=individual,grid,grid style={gray!30},
%  		legend style={at={(0.5,1.4)},anchor=north, legend columns=3} ]
%      	% UCB
%		
%		\addplot table{results/NewExpt/Expt5/comp_subsampled_DUCB01.txt};
%		\addplot table{results/NewExpt/Expt5/comp_subsampled_ETS04.txt};
%		\addplot table{results/NewExpt/Expt5/comp_subsampled_ETS1E03.txt};
%		\addplot table{results/NewExpt/Expt5/comp_subsampled_TS01.txt};
%		\addplot table{results/NewExpt/Expt5/comp_subsampled_OTS01.txt};
%		\addplot table{results/NewExpt/Expt5/comp_subsampled_DTS01.txt};
%      	
%      	\legend{DUCB($\gamma=1-\frac{1}{4\sqrt{T}}$),ETSDAE4,ETSD1E3,TS,OTS,DTS($\gamma=1-\frac{1}{4\sqrt{T}})$}    
%      	\end{axis}
%      	\end{tikzpicture}
%  		\label{fig:2}
%    }
%    \end{tabular}
%    \caption{Cumulative regret for various bandit algorithms on a piecewise stochastic 3-armed bandit environment. }
%    \label{fig:karmed2}
%\end{figure}




\section{Conclusion}
\label{tbandit:conclusion}
In this chapter, we looked at the stochastic multi-armed bandit (SMAB) setting and discussed how it is important in the general reinforcement learning setup. We also looked at the various state-of-the-art algorithms in the literature for the SMAB setting and discussed the advantages and disadvantages of them. The regret bounds that have been proven for the said algorithms have also been discussed at length and their confidence intervals have also been compared against each other. In the next chapter, we provide our solution to this SMAB setting which achieves an almost order-optimal regret bound.


\section{Summary}
\label{tbandit:Summary}
In this chapter we looked at the Augmented-UCB (AugUCB) algorithm for a fixed-budget version of the thresholding bandit problem (TBP), where the objective is to identify a set of arms whose expected mean is above a threshold. A key feature of AugUCB is that it uses both mean and variance estimates to eliminate arms that have been sufficiently explored; to the best of our knowledge this is the first algorithm to employ such an approach for the considered TBP.  Theoretically, we obtain an upper bound on the loss (probability of mis-classification) incurred by AugUCB. Although UCBEV in literature provides a better guarantee, it is important to emphasize that UCBEV has access to problem complexity (whose computation requires arms' mean and variances), and hence is not realistic in practice; this is in contrast to AugUCB whose implementation does not require any such complexity inputs. We conduct extensive simulation experiments to validate the performance of AugUCB. Through our simulation work, we establish that AugUCB, owing to its utilization of variance estimates, performs significantly better than the state-of-the-art APT, CSAR and other non variance-based algorithms.


%%%%%%%%%%%%%%%%%%%%%%%%%%%%%%%%%%%%%%%%%%%%%%%%%%%%%%%%%%%%


%%%%%%%%%%%%%%%%%%%%%%%%%%%%%%%%%%%%%%%%%%%%%%%%%%%%%%%%%%%%
\chapter{Efficient UCB Variance}
\label{chap:EUCBV}
\section{Introduction}
\label{sec:intro}
In this paper, we deal with the stochastic multi-armed bandit (MAB) setting. In its classical form, stochastic MABs represent a sequential learning problem where a learner is exposed to a finite set of actions (or arms) and needs to choose one of the actions at each timestep. After choosing (or pulling) an arm the learner  receives a reward, which is conceptualized as an independent random draw from stationary distribution associated with the selected arm. 
%Each of these rewards is random and drawn independently from the distribution associated with each arm. 
The mean of the reward distribution associated with an arm $i$ is denoted by $r_i$ whereas the mean of the reward distribution of the optimal arm $*$ is denoted by $r^*$ such that $r_i < r^*, \forall i\in \A$, where $\A$ is the set of arms such that $|\A|=K$. With this formulation the learner faces the task of balancing exploitation and exploration. In other words, should the learner pull the arm which currently has the best known estimates or explore arms more thoroughly to ensure that a correct decision is being made. The objective in the stochastic bandit problem is to minimize the cumulative regret, which is defined as follows:
\begin{align*}
R_{T}=r^{*}T - \sum_{i\in \A} r_{i}z_{i}(T),
\end{align*}
where $T$ is the number of timesteps, and  $z_{i}(T)$ is the number of times the algorithm has chosen arm $i$ up to timestep $T$.
The expected regret of an algorithm after $T$ timesteps can be written as,
\begin{align*}
\E[R_{T}]= \sum_{i=1}^{K} \E[z_i (T)] \Delta_i,
\end{align*}
where $\Delta_{i}=r^{*}-r_{i}$ is the gap between the means of the optimal arm and the $i$-th arm.

% One of the fundamental assumptions in stochastic MAB is that the distribution associated with each arm does not change over the entire time horizon $T$.

	In recent years the MAB setting has garnered extensive popularity because of its simple learning  model and its practical applications in a wide-range of industries, including, but not limited to, mobile channel allocations, online advertising and computer simulation games. 
	
	%industry defined problems

\subsection{Related Work}
\label{sec:related}
%There has been a significant amount of research in the area of stochastic MABs. One of the earliest work can be traced to \cite{thompson1933likelihood}, which deals with  the problem of choosing between two treatments to administer on patients who come in sequentially. Other seminal works include that of  \cite{robbins1952some} and then that of \cite{lai1985asymptotically} which established an asymptotic lower bound for the cumulative regret. It showed that for any consistent allocation strategy, we can have
%$\liminf_{T \to \infty}\frac{\E[R_{T}]}{\log T}\geq\sum_{\{i:r_{i}<r^{*}\}}\frac{(r^{*}-r_{i})}{D(Q_{i}||Q^{*})},$
%where $D(Q_{i}||Q^{*})$ is the Kullback-Leibler divergence between the reward densities $Q_{i}$ and $Q^{*}$, corresponding to arms with mean $r_{i}$ and $r^{*}$, respectively.

	Bandit problems have been extensively studied in several earlier works such as \citet{thompson1933likelihood}, \citet{robbins1952some} and \citet{lai1985asymptotically}. Lai and Robbins in  \citet{lai1985asymptotically} established an asymptotic lower bound for the cumulative regret. Over the years stochastic MABs have seen several algorithms with strong regret guarantees. For further reference an interested reader can look into \citet{bubeck2012regret}. The upper confidence bound algorithms balance the exploration-exploitation dilemma by linking the uncertainty in estimate of an arm with the number of times an arm is pulled, and therefore ensuring sufficient exploration. One of the earliest among these algorithms is UCB1 \citep{auer2002finite}, which has a gap-dependent regret upper bound of  $O\left(\frac{K\log T}{\Delta}\right)$, where $\Delta = \min_{i:\Delta_i>0} \Delta_i$. This result is asymptotically order-optimal for the class of distributions considered. But, the worst case gap-independent regret bound of UCB1 is found to be  $O \left(\sqrt{KT\log T}\right)$. In the later work of \citet{audibert2009minimax}, the authors propose the MOSS algorithm and showed that the worst case gap-independent regret bound of MOSS is $O\left( \sqrt{KT} \right)$ which improves upon UCB1 by a factor of order $\sqrt{\log T}$. However, the gap-dependent regret of MOSS is $O\left( \frac{K^{2}\log\left(T\Delta^{2}/K\right)}{\Delta}\right)$ and in certain regimes, this can be worse than even UCB1 (see \citet{audibert2009minimax,lattimore2015optimally}).
	
	 The UCB-Improved algorithm, proposed in \citet{auer2010ucb}, is a round-based\footnote{An algorithm is \textit{round-based} if it pulls all the arms equal number of times in each round and then eliminates one or more arms that it deems  to be sub-optimal.} variant of UCB1, that 
incurs a gap-dependent regret bound of $O\left(\frac{K\log (T\Delta^{2})}{\Delta}\right)$, which is better than that of UCB1. On the other hand, the worst case gap-independent regret bound of UCB-Improved is $O\left(\sqrt{KT\log K}\right)$. Recently in \citet{lattimore2015optimally}, the authors showed that  the algorithm OCUCB achieves order-optimal gap-dependent regret bound of $O\left(\sum_{i=2}^{K}\frac{\log\left(T/H_i\right)}{\Delta_i}\right)$ where $H_i=\sum_{j=1}^{K}\min\left\lbrace \frac{1}{\Delta_i^2},\frac{1}{\Delta_j^2}\right\rbrace$, and a gap-independent regret bound of $O\left( \sqrt{KT}\right)$. This is the best known gap-dependent and gap-independent regret bounds in the stochastic MAB framework. However, unlike our proposed EUCBV algorithm, OCUCB does not take into account the variance of the arms; as a result, empirically  we find  that our algorithm outperforms OCUCB in all the environments considered. 

	In contrast to the above work, the UCBV \citep{audibert2009exploration} algorithm utilizes variance estimates to compute the confidence intervals for each arm. UCBV has a gap-dependent regret bound of $O\left(\frac{K\sigma_{\max}^{2}\log T}{\Delta}\right)$, where $\sigma_{\max}^{2}$ denotes the maximum variance among all the arms $i\in \A$. Its gap-independent regret bound can be inferred to be same as that of UCB1 i.e $O \left(\sqrt{KT\log T}\right)$. Empirically, \citet{audibert2009exploration} showed that UCBV outperforms UCB1 in several scenarios. 
	
	Another notable design principle which has recently gained a lot of popularity is the Thompson Sampling (TS) algorithm (\citep{thompson1933likelihood}, \citep{agrawal2011analysis})  and  Bayes-UCB (BU) algorithm \citep{kaufmann2012bayesian}. % which employs the Bayesian approach in solving the MAB problem.
The TS algorithm maintains a posterior reward distribution for each arm; at each round, the algorithm samples values from these distribution and the arm corresponding to the highest sample value is chosen. Although TS is found to perform extremely well when the reward distributions are Bernoulli, it is established that with Gaussian priors the worst case regret can be as bad as $\Omega \left( \sqrt{KT\log T}\right)$ \citep{lattimore2015optimally}. The BU algorithm is an extension of the TS algorithm that takes quartile deviations into consideration while choosing arms.
	
	The final design principle we state is the information theoretic approach of DMED  \citep{honda2010asymptotically} and KLUCB \citep{garivier2011kl} algorithms. The algorithm KLUCB uses Kullbeck-Leibler divergence to compute the upper confidence bound for the arms. KLUCB is stable for a short horizon and is known to reach the \citet{lai1985asymptotically} lower bound in the special case of Bernoulli distribution. However, \citet{garivier2011kl} showed that KLUCB, MOSS and UCB1 algorithms are  empirically outperformed by UCBV in the exponential distribution as they do not take the variance of the arms into consideration. 


\subsection{Our Contributions}
\label{sec:contri}
In this paper we propose the Efficient-UCB-Variance (henceforth referred to as EUCBV) algorithm for the stochastic MAB setting. EUCBV combines the approach of UCB-Improved, CCB \citep{liu2016modification} and UCBV algorithms. EUCBV, by virtue of taking into account the empirical variance of the arms, exploration parameters  and non-uniform arm selection (as opposed to UCB-Improved), performs significantly better than the existing algorithms in the stochastic MAB setting. EUCBV outperforms UCBV \citep{audibert2009exploration} which also takes into account empirical variance but is less powerful than EUCBV because of the usage of exploration regulatory factor by EUCBV. Also, we carefully design the confidence interval term with the variance estimates along with the pulls allocated to each arm to balance the risk of eliminating the optimal arm against excessive optimism. Theoretically we refine the analysis of \citet{auer2010ucb} and prove that for $T\geq K^{2.4}$ our algorithm is order optimal and achieves a worst case gap-independent regret bound of $O\left( \sqrt{KT} \right)$ which is same as that of MOSS and OCUCB but better than that of UCBV, UCB1 and UCB-Improved. Also, the gap-dependent regret bound of EUCBV is better than UCB1, UCB-Improved and MOSS but is poorer than OCUCB. However, EUCBV's gap-dependent bound matches OCUCB in the worst case scenario when all the gaps are equal. Through our theoretical analysis we establish the exact values of the exploration parameters for the best performance of EUCBV. Our proof technique is highly generic and can be easily extended to other MAB settings. An illustrative table containing the bounds is provided in Table \ref{tab:comp-bds}. 


\begin{table}[t]
\caption{Regret upper bound of different algorithms}
\label{tab:comp-bds}
\begin{center}
\begin{tabular}{|p{5em}|p{12em}|p{7em}|}
\hline
Algorithm  &   \hspace*{1mm}Gap-Dependent & Gap-Independent \\
\hline
\hline
EUCBV		& $O\left( \dfrac{K\sigma_{\max}^{2}\log (\frac{T\Delta^2}{K})}{\Delta}\right)$ & $O\left(\sqrt{KT}\right)$\\
\hline
\hline
UCB1        & $O\left( \dfrac{K\log T}{\Delta} \right)$ & $O\left(\sqrt{KT\log T}\right)$ \\%\midrule
\hline
\hline
UCBV        & $O\left( \dfrac{K\sigma_{\max}^{2}\log T}{\Delta} \right)$ & $O\left(\sqrt{KT\log T}\right)$ \\
\hline
\hline
UCB-Imp 		& $O\left( \dfrac{K\log (T\Delta^2)}{\Delta} \right)$ & $O\left(\sqrt{KT\log K}\right)$ \\%\midrule
\hline
\hline
MOSS	     	& $O\left( \dfrac{K^2\log (T\Delta^2 /K)}{\Delta}\right)$ & $O\left(\sqrt{KT}\right)$\\%\midrule
\hline
\hline
OCUCB     	& $O\left( \dfrac{K\log (T/ H_{i})}{\Delta}\right)$ & $O\left(\sqrt{KT}\right)$\\\midrule
\end{tabular}
\end{center}
%\vspace*{-2em}
\end{table}


Empirically, we show that EUCBV, owing to its estimating the variance of the arms, exploration parameters and non-uniform arm pull, performs significantly better than MOSS, OCUCB, UCB-Improved, UCB1, UCBV, TS, BU, DMED, KLUCB and Median Elimination algorithms. Note that except UCBV, TS, KLUCB and BU (the last three with Gaussian priors) all the aforementioned algorithms do not take into account the empirical variance estimates of the arms. Also, for the optimal performance of TS, KLUCB and BU one has to have the prior knowledge of the type of distribution, but EUCBV requires no such prior knowledge. EUCBV is the first arm-elimination algorithm that takes into account the variance estimates of the arm for minimizing cumulative regret and thereby answers an open question raised by \citet{auer2010ucb}, where the authors conjectured that an UCB-Improved like arm-elimination algorithm can greatly benefit by taking into consideration the variance of the arms. Also, it is the first algorithm that follows the same proof technique of UCB-Improved and achieves a gap-independent regret bound of $O\left( \sqrt{KT} \right)$ thereby, closing the gap of UCB-Improved which achieved a gap-independent regret bound of $O\left( \sqrt{KT\log K} \right)$. 
	
	The rest of the paper is organized as follows. In section~\ref{sec:eucbv} we present the  EUCBV algorithm. Our main theoretical results are stated in section~\ref{sec:results}, while the proofs are established in   section \ref{sec:proofTheorem}. Section~\ref{sec:expt} contains results and discussions from our numerical experiments. We draw our conclusions in section \ref{sec:conc} and section \ref{sec:app} is Appendix (supplementary material).
	
	%discuss about future works. 
	
	%The section \ref{sec:app} containing further proofs is given as supplementary.
	
	
	

\section{Algorithm: Efficient UCB Variance}
\label{sec:eucbv}

\begin{algorithm}[!h]
\caption{EUCBV}
\label{alg:eucbv}
\begin{algorithmic}
\State {\bf Input:} Time horizon $T$, exploration parameters $\rho$ and $\psi$.
\State {\bf Initialization:} Set $m:=0$, $B_{0}:=\mathcal{A}$, $\epsilon_{0}:=1$, $M=\big \lfloor \frac{1}{2}\log_{2} \frac{T}{e}\big\rfloor$, $n_{0}=\big\lceil\frac{\log{(\psi T\epsilon_{0}^{2})}}{2\epsilon_{0}}\big\rceil$ and  $N_{0}=Kn_{0}$.
\State Pull each arm once
\For{$t=K+1,..,T$}	
\State Pull arm $i\in \argmax_{j\in B_{m}}\bigg\lbrace \hat{r}_{j} + \sqrt{\frac{\rho(\hat{v}_{j}+2)\log{(\psi T\epsilon_{m})}}{4 z_{j}}} \bigg\rbrace$, where $z_j$ is the number of times arm $j$ has been pulled.
%\State $t:=t+1$
\ArmElim
\State For each arm $i \in B_{m}$, remove arm $i$ from $B_{m}$ if,
\begin{align*}
%%%%%%%%%%%%%%%%%%%%%%%
%& \hat{r}_{i} + \sqrt{\frac{\rho\hat{v}_{i}\log{(\psi T\epsilon_{m})}}{4 z_{i}} + \frac{\rho\log{(\psi T\epsilon_{m})}}{4 z_{i}}} < \max_{{j}\in B_{m}}\bigg\lbrace\hat{r}_{j} -\sqrt{\frac{\rho\hat{v}_{j}\log{(\psi T\epsilon_{m})}}{4 z_{j}} + \frac{\rho\log{(\psi T\epsilon_{m})}}{4 z_{j}}} \bigg\rbrace
%%%%%%%%%%%%%%%%%%%%%%%
 \hat{r}_{i} + & \sqrt{\frac{\rho(\hat{v}_{i}+2)\log{(\psi T\epsilon_{m})}}{4 z_{i}}}  
  < \max_{{j}\in B_{m}}\bigg\lbrace\hat{r}_{j} -\sqrt{\frac{\rho(\hat{v}_{j}+2)\log{(\psi T\epsilon_{m})}}{4 z_{j}}} \bigg\rbrace
\end{align*}
\EndArmElim

\If{$t\geq N_{m}$ and $m\leq M$}
\ResParam
\State $\epsilon_{m+1}:=\frac{\epsilon_{m}}{2}$\vspace{0.5ex}
\State $B_{m+1}:=B_{m}$
\State $n_{m+1}:=\bigg\lceil\frac{\log{(\psi T\epsilon_{m+1}^{2})}}{2\epsilon_{m+1}}\bigg\rceil$
\State $N_{m+1}:=t+|B_{m+1}| n_{m+1}$
\State $m:=m+1$
\EndResParam
\EndIf
\State Stop if $|B_{m}|=1$ and pull ${i}\in B_{m}$ till $T$ is reached.
\EndFor
\end{algorithmic}
%\vspace*{-0.42em}
\end{algorithm}
%\vspace*{-0.42em}

\textbf{The algorithm:} Earlier round-based arm elimination algorithms like Median Elimination \citep{even2006action} and UCB-Improved mainly suffered from two basic problems: \\
\begin{inparaenum}[\bfseries(i)]
\item \textit{Initial exploration:} Both of these algorithms pull each arm equal number of times in each round, and hence waste a significant number of pulls in initial explorations. \\
\item \textit{Conservative arm-elimination:} In UCB-Improved, arms are eliminated conservatively, i.e, only after $\epsilon_{m}<\frac{\Delta_{i}}{2}$, 
% the sub-optimal arm $i$ is discarded with high probability. 
where the quantity $\epsilon_{m}$ is initialized to $1$ and halved after every round. In the worst case scenario when $K$ is large, and the gaps are uniform  ($r_{1}=r_{2}=\cdots=r_{K-1}<r^{*}$) and small this results in very high regret.\\
\end{inparaenum}
%For any round $m$ UCB-Improved pulls all arms $n_{m}=\left\lceil \frac{ 2\log(T\epsilon_{m})}{\epsilon_{m}} \right\rceil$ number of times. The quantity $\epsilon_{m}$ is initialized to $1$ and halved after every round.
\\
	The EUCBV algorithm, which is mainly based on the arm elimination technique of the UCB-Improved algorithm,  remedies these by employing exploration regulatory factor $\psi$ and arm elimination parameter $\rho$ for aggressive elimination of sub-optimal arms. Along with these, similar to CCB \citep{liu2016modification} algorithm, EUCBV uses optimistic greedy sampling whereby at every timestep it only pulls the arm with the highest upper confidence bound rather than pulling all the arms equal number of times in each round. Also, unlike the UCB-Improved, UCB1, MOSS and OCUCB algorithms (which are based on mean estimation) EUCBV employs mean and variance estimates (as in \citet{audibert2009exploration}) for arm elimination. Further, we allow for arm-elimination at every time-step, which is in contrast to the earlier work (e.g., \citet{auer2010ucb}; \citet{even2006action}) where the arm elimination takes place only at the end of the respective exploration rounds. 






\section{Main Results} 
\label{sec:results}
\section{Results}
\label{sec:results}
% \begin{table}[!ht]
% \centering
% \caption{My caption}
% \label{my-label}
% \begin{tabular}{|p{1.2cm}|l|l|l|l|l|l|}
% \hline
% \multirow{2}{*}{\begin{tabular}[c]{@{}c@{}}Feature \\ Name\end{tabular}} & \multicolumn{2}{c}{Vt Classifier}                                       & \multicolumn{4}{|c|}{Size classifier}                                                                                                   \\ \cline{2-7} 
%                                                                          & \multicolumn{1}{c|}{1} & \multicolumn{1}{c|}{2} & \multicolumn{1}{c|}{ 1} & \multicolumn{1}{c|}{ 2} & \multicolumn{1}{c|}{3} & \multicolumn{1}{c|}{4} \\ \hline
% Sub-circuit                                                              & \multicolumn{1}{l|}{}        &                               &                               &                               &                              &                              \\ \hline
% Gate Type                                                                & \multicolumn{1}{l|}{}        &                               &                               &                               &                              &                              \\ \hline
% LNS                                                  & \multicolumn{1}{l|}{}        &                               &                               &                               &                              &                              \\ \hline
% \#Fanins                                                                 & \multicolumn{1}{l|}{}        &                               &                               &                               &                              &                              \\ \hline
% \#Fanouts                                                                & \multicolumn{1}{l|}{}        &                               &                               &                               &                              &                              \\ \hline
% \begin{tabular}[c]{@{}l@{}}\#Negative \\ Slack Paths\end{tabular}                                          & \multicolumn{1}{l|}{}        &                               &                               &                               &                              &                              \\ \hline
% Slack                                                                    & \multicolumn{1}{l|}{}        &                               &                               &                               &                              &                              \\ \hline
% \end{tabular}
% \end{table}
% \begin{table*}[!ht]
% \centering
% \caption{result3}
% \label{results3}
% \begin{tabular}{|l|c|l|l|l|l|l|l|l|l|l|l|}
% \hline
% \multicolumn{1}{|c|}{\begin{tabular}[c]{@{}c@{}}Benchmark \\  Name\end{tabular}} & \begin{tabular}[c]{@{}c@{}}Number \\ Of gates\end{tabular} & \multicolumn{1}{c|}{\begin{tabular}[c]{@{}c@{}}Target \\ Delay\end{tabular}} & \begin{tabular}[c]{@{}l@{}}Inital \\ Delay\end{tabular} & \multicolumn{2}{c|}{SVM}                                         & \multicolumn{2}{c|}{Final}                                      & \multicolumn{2}{c|}{Igor Markov}                               & \multicolumn{2}{c|}{Flach}                                     \\ \hline
% \multicolumn{1}{|c|}{}                                                           &                                                            & \multicolumn{1}{c|}{}                                                        & \multicolumn{1}{c|}{}                                   & \multicolumn{1}{c|}{Delay (ns)} & \multicolumn{1}{c|}{Power (W)} & \multicolumn{1}{c|}{Delay (ns)} & \multicolumn{1}{c|}{Power(W)} & \multicolumn{1}{c|}{Delay(ns)} & \multicolumn{1}{c|}{Power(W)} & \multicolumn{1}{c|}{Delay(ns)} & \multicolumn{1}{c|}{Power(W)} \\ \hline
% DMA\_fast                                                                        & 25.3K                                                      &                                                                              &                                                         &                                 &                                &                                 &                               &                                &                                                0.299    &       &                               \\ \hline
% DMA\_slow                                                                        & 25.3K                                                      &                                                                              &                                                         &                                 &                                &                                 &                               &                                &                                                       0.145   &     &                               \\ \hline
% pci\_fast                                                              & 33.2K                                                      &                                                                              &                                                         &                                 &                                &                                 &                               &                                &                                                     0.183     &     &                               \\ \hline
% pci\_slow                                                              & 33.2K                                                      &                                                                              &                                                         &                                 &                                &                                 &                               &                                &                                                  0.111         &    &                               \\ \hline
% des\_perf\_fast                                                                  & 111K                                                       &                                                                              &                                                         &                                 &                                &                                 &                               &                                &                                                         1.842   &   &                               \\ \hline
% des\_perf\_slow                                                                  & 111K                                                       &                                                                              &                                                         &                                 &                                &                                 &                               &                                &                                                          0.614   &  &                               \\ \hline
% vga\_lcd\_fast                                                                   & 165K                                                       &                                                                              &                                                         &                                 &                                &                                 &                               &                                &                                                            0.471  & &                               \\ \hline
% vga\_lcd\_slow                                                                   & 165K                                                       &                                                                              &                                                         &                                 &                                &                                 &                               &                                &                                                            0.351  & &                               \\ \hline
% b19\_fast                                                                        & 219K                                                       &                                                                              &                                                         &                                 &                                &                                 &                               &                                &                                                           0.771   & &                               \\ \hline
% b19\_slow                                                                        & 219K                                                       &                                                                              &                                                         &                                 &                                &                                 &                               &                                &                                                             0.583 & &                               \\ \hline
% leon3mp\_fast                                                                    & 649K                                                       &                                                                              &                                                         &                                 &                                &                                 &                               &                                &                                                             1.487 & &                               \\ \hline
% leon3mp\_slow                                                                    & 649K                                                       &                                                                              &                                                         &                                 &                                &                                 &                               &                                &                                                            1.341  & &                               \\ \hline
% netcard\_fast                                                                    & 959K                                                       &                                                                              &                                                         &                                 &                                &                                 &                               &                                &                                                            1.861  & &                               \\ \hline
% netcard\_slow                                                                    & 959K                                                       &                                                                              &                                                         &                                 &                                &                                 &                               &                                &                                                           1.770  &  &                               \\ \hline
% \end{tabular}
% \end{table*}


% Please add the following required packages to your document preamble:
% \usepackage{multirow}
% Please add the following required packages to your document preamble:
% \usepackage{multirow}
% \begin{table}[]
% \centering
% \caption{My caption}
% \label{my-label}
% \begin{tabular}{|l|l|l|l|l|l|}
% \hline
% \multirow{3}{*}{Benchmark} & \multicolumn{5}{c|}{Runtime}                                                            \\ \cline{2-6} 
%                            & \multirow{2}{*}{Igor Markov} & \multirow{2}{*}{Flach} & \multicolumn{3}{c|}{\textit{MLTimer}} \\ \cline{4-6} 
%                            &                              &                        & SVM  & Delay Recovery  & Total  \\ \hline
% DMA\_fast                  &                              &                        &      &                 &        \\ \hline
% DMA\_slow                  &                              &                        &      &                 &        \\ \hline
% pci\_fast        &                              &                        &      &                 &        \\ \hline
% pci\_brdige32\_slow        &                              &                        &      &                 &        \\ \hline
% vga\_lcd\_fast             &                              &                        &      &                 &        \\ \hline
% vga\_lcd\_slow             &                              &                        &      &                 &        \\ \hline
% des\_perf\_fast            &                              &                        &      &                 &        \\ \hline
% des\_perf\_slow            &                              &                        &      &                 &        \\ \hline
% b19\_fast                  &                              &                        &      &                 &        \\ \hline
% b19\_slow                  &                              &                        &      &                 &        \\ \hline
% leon3mp\_fast              &                              &                        &      &                 &        \\ \hline
% leon3mp\_slow              &                              &                        &      &                 &        \\ \hline
% netcard\_fast              &                              &                        &      &                 &        \\ \hline
% netcard\_slow              &                              &                        &      &                 &        \\ \hline
% \end{tabular}
% \end{table}


% Please add the following required packages to your document preamble:
% \usepackage{multirow}
% Please add the following required packages to your document preamble:
% \usepackage{multirow}
% Please add the following required packages to your document preamble:
% \usepackage{multirow}
\begin{table*}[!t]
\caption{Leakage power and Runtime comparisons between the baseline greedy algorithm and the \textit{MLTimer} algorithm on the ISPD 2012 benchmarks. Implementation 1 is the baseline implementation(non-SVM,non-adaptive timing analysis), Implementation 2 is with SVM and non-adaptive timing analysis, Implementation 3 is with non-SVM and adaptive timing analysis and Implementation 4 is with SVM and adaptive timing analysis. It can be seen that using just SVM improves the solution quality greatly, while using just the adaptive timing analysis improves the runtime. A combination of both improves the runtime and solution qualtiy.}
\label{tab:tab5}

\begin{tabular}{|l|l|l|l|l|l|l|l|l|l|}
\hline
\multirow{2}{*}{Benchmarks} & \multirow{2}{*}{\#Gates} & \multicolumn{2}{l|}{Implementation 1}                                                                                                         & \multicolumn{2}{l|}{Implementation 2}                                                                                                           & \multicolumn{2}{l|}{Implementation 3}                                                                                                        & \multicolumn{2}{l|}{Implementation 4}                                                                                                        \\ \cline{3-10} 
                            &                          & \begin{tabular}[c]{@{}l@{}}Run-\\ time \\ (mins)\end{tabular} & \begin{tabular}[c]{@{}l@{}}Leakage \\ Power\\ (W)\end{tabular} & \begin{tabular}[c]{@{}l@{}}Run-\\ time\\ (mins)\end{tabular} & \begin{tabular}[c]{@{}l@{}}Leakage \\ Power\\ \\ (W)\end{tabular} & \begin{tabular}[c]{@{}l@{}}Run-\\ time\\ (mins)\end{tabular} & \begin{tabular}[c]{@{}l@{}}Leakage\\  Power\\ (W)\end{tabular} & \begin{tabular}[c]{@{}l@{}}Run-\\ time\\ (mins)\end{tabular} & \begin{tabular}[c]{@{}l@{}}Leakage \\ Power\\ (W)\end{tabular} \\ \hline
DMA\_fast                   & 23,000                   & 16                                                            & 0.79                                                           & 14.00                                                        & 0.30                                                              & 14.00                                                        & 0.79                                                           & 13.00                                                        & 0.30                                                           \\ \hline
pci\_bridge32\_fast         & 30,000                   & 37                                                            & 0.25                                                           & 17.00                                                        & 0.14                                                              & 17.00                                                        & 0.24                                                           & 17.00                                                        & 0.14                                                           \\ \hline
des\_perf\_fast             & 102,000                  & 219                                                           & 1.73                                                           & 164.00                                                       & 1.80                                                              & 190.00                                                       & 1.73                                                           & 130.00                                                       & 1.80                                                           \\ \hline
vga\_lcd\_fast              & 148,000                  & 384                                                           & 2.80                                                           & 139.00                                                       & 0.47                                                              & 207.00                                                       & 2.72                                                           & 77.00                                                        & 0.47                                                           \\ \hline
b19\_fast                   & 213,000                  & 547                                                           & 2.13                                                           & 239.00                                                       & 0.75                                                              & 366.00                                                       & 2.13                                                           & 174.00                                                       & 0.75                                                           \\ \hline
leon3mp\_fast               & 540,000                  & 2,046                                                         & 4.00                                                           & 875.00                                                       & 1.49                                                              & 716.00                                                       & 4.00                                                           & 639.00                                                       & 1.49                                                           \\ \hline
netcard\_fast               & 861,000                  & 1,033                                                         & 2.09                                                           & 519.00                                                       & 1.77                                                              & 609.00                                                       & 2.07                                                           & 306.00                                                       & 1.77                                                           \\ \hline
\end{tabular}
\end{table*}

\begin{table*}[!ht]
%\centering
\caption{Leakage power comparisons with ISPD 2012 contest winners and other state of the art works. We use geometric mean to calculate the efficiency of our proposed solution. We exclude the infeasible solutions in our mean calculation. All the solutions reported below have no timing violations.}
\label{tab:tab6}

    \begin{tabular}{|l|l|l|p{1.2cm}|p{1.6cm}|p{1.6cm}|p{1cm}|l|p{1.2cm}|}
\hline
\multirow{2}{*}{Benchmark} & \multirow{2}{*}{\begin{tabular}[c]{@{}l@{}}Number \\ of gates\end{tabular}} & \multicolumn{5}{c|}{Leakage Power (W)} & \multicolumn{2}{c|}{Runtime (mins)}\\ \cline{3-9} 
    &  & \cite{hu:12}  & NTUgs & UFRGSgs & Powervalve & \textbf{Ours} & \cite{hu:12} & \textbf{Ours}\\ \hline
    \texttt{DMA\_fast} & 23,000 & 0.30  & 0.51 & 0.32 & 0.31 & 0.30 & 13.90 & 13.30\\ \hline
    \texttt{DMA\_slow} & 23,000  & 0.15  & 0.21 & 0.16 & 0.15 & 0.14 & 9.90 & 7.51 \\ \hline
    \texttt{pci\_fast} & 30,000 & 0.18  & 0.51 & 0.17 & 0.23 & 0.14 & 13.00 & 17.10
     \\ \hline
    \texttt{pci\_slow} & 30,000 & 0.11   & 0.20 & 0.12 & 0.12 & 0.09 & 10.20 & 9.32 \\ \hline
    \texttt{des\_perf\_fast} & 102,000 & 1.84 & 2.39 & 3.52 & 2.32 & 1.80  & 82.70 & 130.40 \\ \hline
    \texttt{des\_perf\_slow} & 102,000 & 0.61 & 0.67 & 0.88 & 0.70 & 0.64 & 70.10 & 43.50 \\ \hline
    \texttt{vga\_lcd\_fast} & 148,000 & 0.47 & 0.76 & 0.58 & 0.77 & 0.47 & 45.60 & 77.32\\ \hline
    \texttt{vga\_lcd\_slow} & 148,000 & 0.35 & 0.42 & 0.38 & 0.39 & 0.37 & 87.50 & 50.40 \\ \hline
    \texttt{b19\_fast} & 213,000 & 0.77 & 2.71 & - & 4.49 & 0.75 & 206.50 & 174.11 \\ \hline
    \texttt{b19\_slow} & 213,000 & 0.58 & 0.63 & 0.61 & 0.74 & 0.61 & 213.90 & 102.20\\ \hline
    \texttt{leon3mp\_fast} & 540,000 & 1.49 & -&  - & 4.94 & 1.49 & 1,323.20 & 639.40\\ \hline
    \texttt{leon3mp\_slow} & 540,000 & 1.34 & 1.42 & 1.79 & 2.96 & 1.30 & 1,274.20 & 325.13  \\ \hline
    \texttt{netcard\_fast} & 861,000 & 1.86 & 2.01 & 2.30 & 2.97 & 1.86 & 1,096.90 & 306.57\\ \hline
    \texttt{netcard\_slow} & 861,000 & 1.77 & 1.77 & 1.97 & 1.94 & 1.77 & 299.90 & 164.14\\ \hline
Geometric mean &  & $1.03\times$ & $1.52\times$ & $1.13\times$ & $1.57\times$ &   & $1.44\times$ & \\ \hline
\end{tabular}
\end{table*}

% Please add the following required packages to your document preamble:
% \usepackage{multirow}
\begin{table*}[!ht]
\centering
\caption{Leakage power comparisons with \cite{hu:13} on the  ISPD 2013 contest benchmark. All the solutions reported below are violation free. It can be observed that \texttt{MLTimer} outperforms \cite{hu:13} both with respect to leakage power and runtime on the larger benchmarks. The detailed results for other benchmarks were not reported in \cite{hu:13}.}
\label{tab:tab34}
\begin{tabular}{|l|l|l|l|l|l|}
\hline
\multirow{2}{*}{Benchmark} & \multirow{2}{*}{Gates} & \multicolumn{2}{l|}{\texttt{MLTimer}}                                                                                                  & \multicolumn{2}{l|}{\cite{hu:13}}                                                                                                  \\ \cline{3-6} 
                           &                        & \begin{tabular}[c]{@{}l@{}}Run-\\ time\\ (mins)\end{tabular} & \begin{tabular}[c]{@{}l@{}}Leakage\\ Power\\ (mW)\end{tabular} & \begin{tabular}[c]{@{}l@{}}Run-\\ time\\ (mins)\end{tabular} & \begin{tabular}[c]{@{}l@{}}Leakage \\ Power\\ (mW)\end{tabular} \\ \hline
usb\_phy\_fast             & 510                    & 0.48                                                         & 2.03                                                           & \textbf{0.21}                                                & \textbf{1.56}                                                   \\ \hline
usb\_phy\_slow             & 510                    & \textbf{0.11}                                                & 1.13                                                           & 0.17                                                         & \textbf{1.07}                                                   \\ \hline
pci\_bridge32\_fast        & 28,000                 & 20.83                                                        & 116.87                                                         & \textbf{12.00}                                               & \textbf{101.90}                                                 \\ \hline
pci\_bridge32\_slow        & 28,000                 & 6.78                                                         & 58.91                                                          & \textbf{5.39}                                                & \textbf{58.83}                                                  \\ \hline
fft\_fast                  & 31,000                 & 40.00                                                        & 320.37                                                         & \textbf{32.58}                                               & \textbf{305.29}                                                 \\ \hline
fft\_slow                  & 31,000                 & 25.00                                                        & 96.69                                                          & \textbf{17.40}                                               & \textbf{93.10}                                                  \\ \hline
cordic\_slow               & 42,000                 & \textbf{94.40}                                               & \textbf{397.81}                                                & 98.39                                                        & 511.91                                                          \\ \hline
des\_perf\_slow            & 104,000                & 88.18                                                        & 386.41                                                         & \textbf{62.30}                                               & \textbf{375.80}                                                 \\ \hline
edit\_dist\_fast           & 121,000                & \textbf{163.10}                                              & \textbf{572.12}                                                & 170.60                                                       & 619.30                                                          \\ \hline
edit\_dist\_slow           & 121,000                & \textbf{56.34}                                               & \textbf{423.50}                                                & 107.20                                                       & 465.60                                                          \\ \hline
matrix\_mult\_slow         & 153,000                & \textbf{139.80}                                              & \textbf{482.23}                                                & 212.60                                                       & 499.90                                                          \\ \hline
netcard\_fast              & 884,000                & \textbf{372.70}                                              & \textbf{5,157.93}                                              & 716.80                                                       & 5271.80                                                         \\ \hline
netcard\_slow              & 884,000.               & \textbf{297.12}                                              & \textbf{5,102.25}                                              & 439.60                                                       & 5183.89                                                         \\ \hline
GEOMETRIC MEAN             &                &                                               &                                               & 1.005                                                       & 1.005                                                         \\ \hline

\end{tabular}
\end{table*}

% \begin{table*}[!ht]
% \begin{center}
% \label{results3}
% \begin{tabular}{|p{2.1cm}|p{2cm}|p{2cm}|p{2cm}|p{2cm}|p{1.5cm}|}
% \hline
% Benchmark & $\#$ gates & %\multicolumn{2}{|c|}
% {\textit{MLTimer}} &%\multicolumn{2}{|c|} 
% {Igor Markov} ~\cite{hu:12} & Improvement \\
% \hline
%   &  & Leakage Power (W)      %& Running Time    
%   & Leakage Power (W) & \\% & Running Time \\ 
 
% \hline
% %USB\_PHY &536 &$<$1s &$<$1s & - \\
% \hline
% DMA\_fast & 25.3K & 0.08W %& 17m
% & 0.299W & 73\% \\ %& 13m\\
% \hline
% %DMA\_slow & 25.3 & 0.134W %& 1m44s
% %& 0.145W & 7\%  \\% & 9.9m\\
% %\hline
% pci\_bridge	&	33.2K	&	0.1331W %&	1m29s
% &	0.183W & 27\%\\%	&	13m \\
% \hline
% %pci\_bridge\_slow	&	33.2K	&	0.07W	%&	8m
% %&	0.111W & 36\% \\%	&	11m \\
% %\hline 
% b19	&	219K  &		.58W	%&	9h
% &	0.771W & 24\% \\%	&	206m \\
% \hline
% %b19\_slow 	&	219K &		0.486W%	&	9h	
% %&	0.583W & 19.21\% \\	%&	213m \\
% %\hline
% Des\_perf & 165K & .546W %& 1331m       
% & .471W & \-15.3\% \\%    & 45m \\
% \hline
% netcard & 959k & 1.8W %& 2046m 
% & 1.861W & 6.01 \% \\% & 1096m \\
% \hline
% leon3mp & 649K & 2W %&  2816m 
% & 1.487W & \-34.5\% \\ %& 1323.2 \\
% \hline
% Average & & & & 21.37\% \\ \hline
% %leon3mp\_slow & 649K & 2W &  2816m & 1.487W & 1323.2 \\
% %\hline
% \end{tabular}
% \caption{Leakage and Running Time Comparisons for ISPD benchmarks between \textit{MLTimer} and Igor Markov. In the table, {\bf h}, {\bf m} and {\bf s} stand for hours, minutes and seconds respectively.}


% \end{center}
% \end{table*}
\subsection{Comparisons with state-of-the-art}

The performance of our proposed algorithm is shown in Table~\ref{tab:tab5}. A simple greedy algorithm, implemented for obtaining the final $V_t$ and $size$ values, serves as the baseline algorithm. It can be seen that our \textit{MLTimer} implementation outperforms the baseline algorithm by 46\% in terms of solution quality. It can also be seen that the SVM module improves the solution quality and the adaptive timing analysis module improves the runtime. 

We compare the performance of our algorithm with ~\cite{hu:12} which is the best performing heuristic based algorithm reported so far in the literature. We use the ISPD 2012 benchmark set and SHAKTIC to quantify the performance our algorithm. In comparing with the state-of-the-art techniques we make the following observations:
\begin{itemize}
\item Our solution outperforms the top 3 submissions of the ISPD 2012 contest NTUgs, UFRGSgs and Powervalve by 52\%,13\% and 57\% respectively.
\item Our solution outperforms \cite{hu:12} both in terms of average runtime and solution quality by 44\% and 3\% respectively. Table~\ref{tab:tab6} highlights the performance of \textit{MLTimer} algorithm in terms of runtime and solution quality. This is because as most of the circuits share a large  number of repeating sub-circuits whose value is accurately predicted by the SVM engine and hence these gates do not undergo delay and power recovery algorithm leading to savings in runtime. 
\item It can be seen from Table~\ref{tab:tab34} that our tool outperforms \cite{hu:13} which is an extension of \cite{hu:12}. It can be observed that while \textit{MLTimer} underperforms for the smaller benchmarks, it significantly outperforms \cite{hu:13} on the larger benchmarks. Although the overall improvement in solution quality is around 0.004\%, the improvement in the larger benchmarks is around 53\% for the runtime and 10\% for solution quality.
\item In Table~\ref{tab:tab9} we compare our implementation with a commercial synthesis tool and our implementation of \cite{hu:12}. It can be observed our proposed solution performs significantly better than the commercial tool in terms of leakage power. 
\end{itemize}

\begin{table}[!t]
    \caption{The Table comparing the performance of \textit{MLTimer} versus a commercial synthesis tool on SHAKTIC. We see that the solution quality is 57\% better than that of the tool.}
    \label{tab:tab9}

    \centering
    \begin{tabular}{|l|l|l|l|l|l|l|}
        \hline
        \textbf{Metric}           & \multicolumn{2}{c|}{Commercial Tool}                                                                                     &        &            & \multicolumn{2}{c|}{Percentage Improvement} \\ \hline
                         & \begin{tabular}[c]{@{}l@{}}$LV_t$\\  synthesis\end{tabular} & \begin{tabular}[c]{@{}l@{}}Mixed $V_t$ \\ synthesis\end{tabular} & \cite{hu:12} & \textit{MLTimer} & Tool              & \cite{hu:12}       \\ \hline
                    %         \textbf{Runtime (mins)}    & 38                                                       & 14                                                            & 87     & 64         & -78.12           &  26.44       \\ \hline
                             \textbf{Leakage power (W)} & 5                                                        & 1                                                             & 0.59  & 0.43       & 57        & 27        \\ \hline
    \end{tabular}

\end{table}


\subsection{Analysis of the Learning Module}


\begin{table}[!t]
\caption{Table showing the weights assigned to each feature at each stage of the $V_t$ and $size$ classifiers. An extremely low magnitude implies that the corresponding feature does not contribute significantly to the output and can thus be discarded. However it can be seen that none of the features chosen fall into that category.}
\label{tab:tab7}

    \centering

\begin{tabular}{|l|l|l|l|l|l|l|l|}
\hline
    \multirow{2}{*}{\textbf{Feature}}       & \multicolumn{2}{c|}{$\mathbf{V_t}$} & \multicolumn{5}{c|}{\textbf{size}}             \\ \cline{2-8} 
                               & 1          & 2          & 1     & 2     & 3     & 4     & 5     \\ \hline
    \textbf{Sub-circuit}                    & -0.88      & -0.43      & -0.58 & 1.32  & -0.08 & 0.44  & 0.16  \\ \hline
    \textbf{Gate type}                      & -0.19      & -0.43      & -0.96 & -0.81 & 0.42  & -0.40 & 0.29  \\ \hline
    \textbf{LNS}                            & 1.09       & 0.33       & -0.07 & -0.11 & 0.76  & 0.76  & 0.08  \\ \hline
    \textbf{Number of Fanins}               & 2.64       & 0.04       & 3.27  & -2.38 & -1.10 & -0.34 & -1.26 \\ \hline
    \textbf{Number of Fanouts}              & -2.92      & -0.29      & -4.08 & -0.53 & 1.82  & 0.50  & -1.07 \\ \hline
    \textbf{Number of Negative Slack Paths} & 0.33       & -0.16      & 0.50  & 0.40  & -0.22 & 0.21  & -0.68 \\ \hline
    \textbf{Slack}                          & 1.89       & 0.35       & 1.22  & -3.20 & -1.64 & -1.41 & 0.37  \\ \hline
\end{tabular}

\end{table}


The learning module forms a critical component of our framework as it serves to reduce the runtime by using a simple SVM model that uses seven features.  A complex ML model with large number of redundant features might cause runtime overheads due to i) complex training procedure ii) complicated inference procedure, and; iii) reduced interpretability of the ML model. Hence there is a need to eliminate the redundant features in order to simplify the learning module. Logistic regression was performed to estimate the importance of the chosen features. The Logistic Regression model was initially trained on the set of chosen features and the importance of each feature,  obtained via the coefficient assigned by the model,  is quantified in Table~\ref{tab:tab7}.  It can be observed that none of the feature weights have extremely low value and hence cannot be eliminated.


%As mentioned earlier, an improperly trained learning engine could initialize the netlist to a sub-optimal configuration leading to more delay and power recovery cycles than necessary thereby increasing the runtime overhead. The thresholding function plays an important role in predicting the final choice ($V_t$/$size$) for a given cell. We use the b19\_fast benchmark to show the impact of varying the thresholding function on the solution quality of the SVM engine. We show the impact of the thresholding function in table ~\ref{results4}. We see that as the thresholding function increases the runtime goes up. This is because the number of gates that are marked unsure increases causing more delay and power optimizations. We us class probability to determine the class label ($V_t$/$size$). We use a thresholding value of $0.75$ for both the $V_t$ classifiers while we use a thresholding value of x and y for the $size$ classifiers. 





% \begin{table*}[!t]
% \parbox{.3\linewidth}{
% \begin{center}

% \begin{tabular}{|p{3cm}|p{1.3cm}|}
% \hline
% Metric & Number \\ \hline
% Total gates & 333\\
% \hline
% Combinational gate types &  11 \\ \hline
% Sequential gates & 1 \\ \hline
% $V_t$ choices & 3 \\ \hline
% $size$ choices & 10 \\ \hline
% $V_{cc}$ and $gnd$ cells & 2 \\ \hline
% %leon3mp\_slow & 649K & -6401 &  -6479 & 1W & 47s \\
% %\hline
% \end{tabular}
% \caption{Library statistics}
% \label{tab:lib}
% \end{center}
% %\end{table*}
% }
% \hfill
% \parbox{.6\linewidth}{
% %\begin{table*}[!t]
% \begin{center}

% \begin{tabular}{|p{2cm}|p{1cm}|p{1cm}|p{1.6cm}|p{1.6cm}|p{1.6cm}|}
% \hline
% Benchmark & \#Input & \#Output & \#Comb cell & \#Seq cell & \#Total cell \\
% \hline
% DMA &  683 & 276 & 23109 & 2192 & 25301\\ \hline
% pci & 160 & 201& 29844& 3359& 33203\\ \hline
% des\_perf &  234 & 140 & 102427 & 8802& 111229\\ \hline
% vga\_lcd &  85 & 99 &  147812 & 17079 & 164891\\ \hline
% b19 &  22 & 25 &  212674 & 6594 & 219268\\ \hline
% leon3mp & 254 & 79 & 540352 &  108839 & 649191\\ \hline
% netcard &  1836 & 10 & 860949 & 97831 & 958780\\ \hline

% %leon3mp\_slow & 649K & -6401 &  -6479 & 1W & 47s \\
% %\hline
% \end{tabular}
% \caption{Benchmark statistics}
% \label{tab:benchmark}
% \end{center}
% }
% \end{table*}


% Please add the following required packages to your document preamble:
% \usepackage{booktabs}
% \usepackage{multirow}
% Please add the following required packages to your document preamble:
% \usepackage{multirow}
% Please add the following required packages to your document preamble:
% \usepackage{multirow}
% Please add the following required packages to your document preamble:
% \usepackage{multirow}

% \begin{table*}[!t]
% \parbox{.5\linewidth}{
% \begin{center}

% \label{tab:log}
% \begin{tabular}{|p{2.5cm}|p{3cm}|}
% \hline
% Feature & Weight \\
% \hline
% Gate footprint &-0.8832794232852276 \\ \hline
% Gateid & -0.1914242984939835 \\ \hline
% Local negative slack & 1.090781658200261 \\ \hline
% \#Fanins & 2.642338600894165  \\ \hline
% \#Fanouts & -2.92081747517804  \\ \hline
% \#Negative slack paths & 0.3349856667382776  \\ \hline
% Slack & 1.894864001133556 \\ \hline

% %leon3mp\_slow & 649K & -6401 &  -6479 & 1W & 47s \\
% %\hline
% \end{tabular}
% \caption{Feature Weights for the first $V_t$ classifier }
% \end{center}
% %\end{table*}
% }
% \hfill
% %\begin{table*}[!h]
% \parbox{.5\linewidth}{
% \begin{center}

% \label{tab:log2}
% \begin{tabular}{|p{2.5cm}|p{3cm}|}
% \hline
% Feature & Weight \\
% \hline
% Gate footprint & 0.03119793516364029
% \\ \hline
% Gateid &  0.346427255804566\\ \hline
% Local negative slack & -0.01019723367101055\\ \hline
% \#Fanins & -0.007490175140922838 \\ \hline
% \#Fanouts & -0.2180352793079041 \\ \hline
% \#Negative slack paths & -0.06062286295401311  \\ \hline
% Slack & 0.663389798807628 \\ \hline

% %leon3mp\_slow & 649K & -6401 &  -6479 & 1W & 47s \\
% %\hline
% \end{tabular}
% \caption{Feature Weights for the second stage $V_t$ classifier }
% \end{center}
% }
% \end{table*}

The efficiency of the SVM engine is analyzed in Table~\ref{tab:tab8}. We see that on an average the SVM engine is able to recover a significant amount of power in a short amount of time. However, It can be observed that the solution provided by the SVM engine is not optimal hence  the delay and leakage power recovery steps are used to further optimize the solution provided by the learning step.

 \begin{table*}[!t]
  \caption{Leakage and Running Time Comparisons for ISPD benchmarks and ShaktiC with just SVM. In the table, {\bf h}, {\bf m} and {\bf s} stand for hours, minutes and seconds respectively. It can be seen that with the exception of leon3mp our SVM implementation is able to recover significant delay and power.} \
\label{tab:tab8}

     \begin{center}
 \begin{tabular}{|p{4.2cm}|p{2cm}|p{2.2cm}|p{2cm}|p{2cm}|p{2cm}|}
 \hline
    \textbf{Benchmark} & \textbf{Gate count} & \textbf{Initial Worst Negative Slack (WNS)} & \multicolumn{3}{|c|}{ \textit{MLTimer}}  \\
 \hline
   &   & & WNS (ps) &  Leakage Power (W)      & Running Time          \\
 \hline
     \texttt{ DMA\_fast} & 25,300&  -1485 & -774 &0.09  & 3s \\
 \hline
     \texttt{pci\_bridge32\_fast}	& 33,200& -1881 & -2284	&	0.18   &	3s	 \\
 \hline
     \texttt{des\_perf\_fast} &  102,000 & -669 & -1029 &.316 & 1m        \\
 \hline
     \texttt{vga\_lcd\_fast} & 148,000 & -1254 & -2964 &.29 & 1m          \\
\hline

     \texttt{b19\_fast}	&	219,000 & -2835 & -1738	&	1.6 	&	15s	\\ \hline
     \texttt{leon3mp\_fast} & 649,000 & -6401 & -3913 & 21 &  47s \\
 \hline

     \texttt{netcard\_fast} & 959,000 & -4102 &-3268 & 8 & 1m  \\ \hline
     \texttt{ShaktiC} & 174,756 & -5199 & -1067 & 0.67 & 1m \\ 
 \hline
 \end{tabular}
 \end{center}

 \end{table*}
\section{Conclusion}
\label{sec:conclusion}
Leakage optimization  techniques have been studied extensively for more than a decade.  However, the lack of a robust algorithm that is optimal in terms of both execution time and solution quality motivates research in this area. It is seen that varying window size adaptively according to the status of the timing updates produces faster solutions than for a fixed window size. The proposed \textit{MLTimer} algorithm improves the running-time considerably while still retaining the solution quality of a greedy heuristic. It is observed that for large circuits \textit{MLTimer} with initial configuration provided by SVM performs significantly better than when used with power optimal configuration as initial solution.  Extending the concepts involved in the construction of \textit{MLTimer} to other steps of EDA including placement and routing is an interesting direction for future work.
% * <sristisravan@gmail.com> 2017-06-28T10:13:29.636Z:
% 
% Check "... the lack of a robust heuristic that optimal in terms of both.... "
% 
% ^.

%\begin{table*}[t]
% \begin{center}
% \caption{Leakage and Running Time Comparisons for ISPD benchmarks with just SVM and delay recovery. In the table, {\bf h}, {\bf m} and {\bf s} stand for hours, minutes and seconds respectively.}
% \label{results2}
% \begin{tabular}{|p{1.7cm}|p{2cm}|p{2cm}|p{2cm}|p{2cm}|}
% \hline
% Benchmark & $\#$ gates & \multicolumn{3}{|c|}{\textit{MLTimer}}  \\
% \hline
%   &  & Delay(ps) &  Leakage Power (W)      & Running Time          \\
% \hline
% %USB\_PHY &536 &$<$1s &$<$1s & - \\
% \hline
% DMA_fast & 25.3K & &0.08W & 17m \\
% \hline
% DMA_slow & 25.3 & &0.134W & 1m44s \\
% \hline
% pci_bridge_fast	& 33.2K&	&	0.1331W &	1m29s	 \\
% \hline
% pci_bridge_slow	& 	33.2K&	&	0.07W	&	8m	\\
% \hline 
% b19_fast	&	219K  &	&	.58W	&	9h	\\
% \hline
% b19_slow 	&	219K & &		0.486W	&	9h	 \\
% \hline
% Des_perf_fast &  165K & &.546W & 1331m        \\
% \hline
% Des_perf_slow &  165K & & .546W & 1331m         \\
% \hline
% vga_lcd_slow & 165K & & .546W & 1331m          \\
% \hline
% vga_lcd_slow & 165K &  &.546W & 1331m          \\
% \hline
% netcard_fast & 959k & & 1.8W & 2046m  \\
% \hline
% netcard_slow & 959k & & 1.8W & 2046m \\
% \hline
% leon3mp_fast & 649K & & 2W &  --- \\
% \hline
% leon3mp_slow & 649K & & 2W &  --- \\
% \hline
% \end{tabular}
% \end{center}
%\end{table*}






\section{Proofs}
\label{sec:proofTheorem}
%\subsection{Lemma 1}
%\label{sec:proofTheorem:Lemma1}
We first present a few technical lemmas that is required  to prove the result in Theorem \ref{Result:Theorem:1}.

\begin{lemma}
\label{proofTheorem:Lemma:1}
If $T\geq K^{2.4}$, $\psi=\frac{T}{ K^2}$, $\rho=\frac{1}{2}$ and $m\leq \frac{1}{2} \log_2\left(\frac{T}{e}\right) $, then,
\begin{align*}
\dfrac{\rho m \log(2)}{\log(\psi T) - 2m\log( 2)} \leq \frac{3}{2}.
\end{align*}
\end{lemma}



\begin{lemma}
\label{proofTheorem:Lemma:2}
If $T\geq K^{2.4}$, $\psi=\frac{T}{ K^2}$, $\rho =\frac{1}{2}$, $m_i = min\lbrace m|\sqrt{4\epsilon_{m} } < \frac{\Delta_i}{4} \rbrace $ and $c_{i} =\sqrt{\frac{\rho (\hat{v}_i + 2)\log (\psi T\epsilon_{m_{i}})}{4 z_i}}$, then,
%\begin{align*}
\center $c_{i} < \frac{\Delta_i}{4}$.
%\end{align*}
\end{lemma}



\begin{lemma}
\label{proofTheorem:Lemma:3}
If $m_i = min\lbrace m|\sqrt{4\epsilon_{m} } < \frac{\Delta_i}{4} \rbrace $,  $c_{i} = \sqrt{\frac{\rho (\hat{v}_i + 2) \log (\psi T\epsilon_{m_{i}})}{4 z_{i}}}$ and $n_{m_i} = \frac{\log{(\psi T\epsilon_{m_{i}})}}{2\epsilon_{m_{i}}}$ then we can show that,
\begin{align*}
\mathbb{P}(\hat{r}_{i}> r_{i} + c_{i})\le \dfrac{2}{(\psi  T\epsilon_{m_{i}})^{\frac{3\rho}{2}}}.
\end{align*}
\end{lemma}



%\begin{lemma}
%\label{proofTheorem:Lemma:3}
%If $m_i = min\lbrace m|\sqrt{4\epsilon_{m} } < \frac{\Delta_i}{4} \rbrace $,  $\bar{c}_i=\sqrt{\frac{\rho (\sigma_{i}^{2}+\sqrt{\epsilon_{m_{i}}} + 2)\log(\psi T\epsilon_{m_{i}})}{4z_i}}$ and $n_{m_i} = \frac{\log{(\psi T\epsilon_{m_{i}})}}{2\epsilon_{m_{i}}}$ then we can show that,
%\begin{align*}
%\mathbb{P}\left( \hat{r}_{i} > r_{i}+ \bar{c}_i\right) 
%+ \mathbb{P}\left( \hat{v}_{i}\geq \sigma_{i}^{2}+\sqrt{\epsilon_{m_{i}}}\right) \leq \dfrac{2}{(\psi  T\epsilon_{m_{i}})^{\frac{3\rho}{2}}}.
%\end{align*}
%\end{lemma}



\begin{lemma}
\label{proofTheorem:Lemma:4}
If $m_i = min\lbrace m|\sqrt{4\epsilon_{m} } < \frac{\Delta_i}{4} \rbrace $, $\psi=\frac{T}{ K^2}$, $\rho=\frac{1}{2}$, $c_{i} =\sqrt{\frac{\rho(\hat{v}_i + 2)\log (\psi T\epsilon_{m_{i}})}{4 z_{i}}}$ and $n_{m_i}=\frac{\log{(\psi T\epsilon_{m_{i}}^{2})}}{2\epsilon_{m_{i}}}$ then in the $m_i$-th round, 
\begin{align*}
\Pb\lbrace c^{*} > c_i \rbrace  \leq \dfrac{182 K^4}{T^{\frac{5}{4}}\sqrt{\epsilon_{m_i}}}.
\end{align*}
\end{lemma}



\begin{lemma}
\label{proofTheorem:Lemma:5}
If $m_i = min\lbrace m|\sqrt{4\epsilon_{m} } < \frac{\Delta_i}{4} \rbrace $,$\psi=\frac{T}{ K^2}$, $\rho=\frac{1}{2}$, $c_{i} =\sqrt{\frac{\rho (\hat{v}_i + 2)\log (\psi T\epsilon_{m_{i}})}{4 z_i}}$ and $n_{m_i}=\frac{\log{(\psi T\epsilon_{m_{i}}^{2})}}{2\epsilon_{m_{i}}}$ then in the $m_i$-th round, 
\begin{align*}
\Pb\lbrace z_i < n_{m_i} \rbrace  \leq \dfrac{182 K^4}{T^{\frac{5}{4}}\sqrt{\epsilon_{m_i}}}.
\end{align*}
\end{lemma}



%\begin{lemma}
%\label{proofTheorem:Lemma:6}
%For $T\geq K^{2.4}$, $\epsilon_{m_i}\geq \sqrt{\frac{e}{T}}$, $\psi=\frac{T}{K^2}$ and $\rho=\frac{1}{2}$,  
%\begin{align*}
%\dfrac{6K}{(\psi T \epsilon_{m_i})^{\frac{3\rho}{2}}} > \dfrac{K\log T}{(\psi T)^{3\rho}}\sum_{m=0}^{m_i}\dfrac{1}{\epsilon_{m_i}^{3\rho + 1}}
%\end{align*}
%\end{lemma}



%\begin{lemma}
%\label{proofTheorem:Lemma:6}
%For all bounded rewards in $[0,1]$, $\frac{\Delta_i}{4} \geq \frac{\Delta_i}{4\sigma_i^2 + 4} $.
%\end{lemma}



\begin{lemma}
\label{proofTheorem:Lemma:6}
For two integer constants $c_1$ and $c_2$, if $20 c_1 \leq c_2$ then,
\begin{align*}
c_1 \frac{4\sigma_i^2 + 4}{\Delta_i}\log\bigg( \frac{T\Delta_i^2}{K}\bigg) \leq c_2 \frac{\sigma_i^2}{\Delta_i}\log\bigg( \frac{T\Delta_i^2}{K}\bigg).
\end{align*}
\end{lemma}


%\begin{lemma}
%\label{proofTheorem:Lemma:8}
%If $m_*$ be the first round that the optimal arm $*$ gets eliminated, then we can show that the regret is upper bounded by,
%
%\begin{align*}
%\sum_{m_{*}=0}^{max_{j\in \A^{'}}m_{j}}\sum_{i\in \A^{''}:m_{i}>m_{*}}\bigg(\dfrac{388 K}{(\psi  T\epsilon_{m_{*}})^{\frac{3\rho}{2}}} \bigg).T\max_{j\in \A^{''}:m_{j}\geq m_{*}}{\Delta}_{j} \\
%%%%%%%%%%%%%%%%%%%%%%%%%
% \leq\sum_{i\in \A^{'}}\dfrac{C_2^{'} K^{\frac{5}{2}}}{\sqrt{T\Delta_i}} +\sum_{i\in \A^{''}\setminus \A^{'}}\dfrac{C_2^{'} K^{\frac{5}{2}}}{\sqrt{T b}}
%\end{align*}
%
%\end{lemma}


The proofs of lemmas \ref{proofTheorem:Lemma:1} - \ref{proofTheorem:Lemma:6} can be found in Appendix ~\ref{App:Lemma:1}, ~\ref{App:Lemma:2}, ~\ref{App:Lemma:3}, ~\ref{App:Lemma:4}, ~\ref{App:Lemma:5} and
 ~\ref{App:Lemma:6} respectively.

%The proofs of all the Lemmas can be found in Appendix ~\ref{App:Lemma:1} - Appendix ~\ref{App:Lemma:9} respectively.

\subsection*{Proof of Theorem 1}
\label{sec:proofTheorem:Theorem1}
\begin{customproof}{1}
For each sub-optimal arm ${i}\in\mathcal{A}$, let $m_{i}=\min{\left\lbrace m|\sqrt{4\epsilon_{m_i}} < \frac{\Delta_{i}}{4}\right\rbrace}$. Also, let $\A^{'}=\lbrace i\in \A: \Delta_{i} > b \rbrace$ and $\A^{''}=\lbrace i\in \A: \Delta_{i} > 0 \rbrace$. Note that as all rewards are bounded in $[0,1]$, it implies that $0\leq \sigma_i^2 \leq \frac{1}{4},\forall i\in \A$. Now, as in \citet{auer2010ucb}, we bound the regret under the following two cases: 
\begin{itemize}
\item {Case $(a)$}: some sub-optimal arm ${i}$ is not eliminated in round $m_{i}$ or before and the optimal arm ${*}\in B_{m_{i}}$
\item {Case $(b)$}: an arm ${i}\in B_{m_i}$ is eliminated in round $m_{i}$ (or before), or there is no optimal arm $*\in B_{m_i}$
\end{itemize} 
The details of each case are contained in the following sub-sections.

%Note that in in round $m_i$ as $\sqrt{4\epsilon_{m_i}} < \dfrac{\Delta_{i}}{4}$ implies that $\sqrt{4\epsilon_{m_i}} < \dfrac{\Delta_{i}}{4\sigma_i^2}$, since $\sigma_i^2\in (0,1]$

\textbf{Case $(a)$:}
For simplicity, let $c_{i} := \sqrt{\frac{\rho (\hat{v}_i + 2) \log (\psi T\epsilon_{m_{i}})}{4 z_{i}}}$ denote the length of the confidence interval corresponding to arm $i$ in round $m_i$. Thus, in round $m_i$ (or before) whenever $z_i \geq n_{m_{i}}\ge\frac{\log{(\psi T\epsilon_{m_{i}}^{2})}}{2\epsilon_{m_{i}}}$, by applying Lemma \ref{proofTheorem:Lemma:2} we obtain $c_{i} < \frac{\Delta_{i}}{4}$.
%\begin{align*}
%	c_{i} < \dfrac{\Delta_{i}}{4} 
%\end{align*}
Now, the sufficient conditions for arm $i$ to get eliminated by an optimal arm in round $m_i$ is given by
	\begin{eqnarray}
	\hat{r}_{i} \leq r_{i} + c_{i} \text{, } 
 	\hat{r}^{*} \geq r^{*} - c^{*} \text{, } c_{i} \geq c^* \text{ and } z_i \geq n_{m_i} \label{eq:armelim-casea}.
	\end{eqnarray}

Indeed, in round $m_i$ suppose (\ref{eq:armelim-casea}) holds, then we have
%	 
  \begin{align*}
\hat{r}_{i} + c_{i}&\leq r_{i} + 2c_{i} 
= r_{i} + 4c_{i} - 2c_{i} \\
 &< r_{i} + \Delta_{i} - 2c_{i}
 \leq r^{*} -2c^{*} 
 \leq \hat{r}^{*} - c^{*}
  \end{align*}
  so that a sub-optimal arm ${i} \in \A^{'}$ gets eliminated.	
Thus, the probability of the complementary event of these four conditions in (\ref{eq:armelim-casea}) yields a bound on the probability that arm $i$ is not eliminated in round $m_i$. Following the proof of Lemma 1 of \citet{audibert2009exploration} we can show that a bound on the complementary of the first condition is given by,

\begin{align}
\mathbb{P}(\hat{r}_{i}> r_{i} + c_{i})
&\leq \mathbb{P}\left( \hat{r}_{i} > r_{i}+ \bar{c}_i\right) 
+ \mathbb{P}\left( \hat{v}_{i}\geq \sigma_{i}^{2}+\sqrt{\epsilon_{m_{i}}}\right)\label{eq:prob_eq2}
\end{align}
where 
\begin{align*}
\bar{c}_i=\sqrt{\dfrac{\rho (\sigma_{i}^{2}+\sqrt{\epsilon_{m_{i}}} + 2)\log(\psi T\epsilon_{m_{i}})}{4n_{m_i}}}.
\end{align*}

%%%%%%%%%%%%%%%%%%%%%%%%%%%%%%%%%%
% Shifted as Lemma
%%%%%%%%%%%%%%%%%%%%%%%%%%%%%%%%%%
%Note that, substituting $ n_{m_i} \geq \frac{\log{(\psi T\epsilon_{m_{i}})}}{2\epsilon_{m_{i}}}$, $\bar{c}_i$ can be simplified to obtain,
%\begin{align}
%\bar{c}_i
%\leq \sqrt{\dfrac{\rho\epsilon_{m_{i}}(\sigma_{i}^{2}+\sqrt{\epsilon_{m_{i}}} + 2)}{2}}\leq \sqrt{ \epsilon_{m_{i}}}.
%\label{si_bar_equn}
%\end{align}
%%
%The first term in the LHS of (\ref{eq:prob_eq2}) can be bounded using the Bernstein inequality as below:
%\begin{align}
%&\mathbb{P}\left( \hat{r}_{i} > r_{i}+ \bar{c}_i\right)\nonumber 
%\le \exp\left(- \dfrac{(\bar{c}_i)^2 z_{i}}{2\sigma_i^2 + \frac{2}{3}\bar{c}_i} \right)\nonumber 
%%%%%%%%%%%%%%%%
%\\
%& \overset{(a)}{\le} \exp\left(- \rho \left(\dfrac{3\sigma_{i}^{2}+3\sqrt{\epsilon_{m_{i}}} + 6}{6\sigma_i^2 + 2\sqrt{\epsilon_{m_i}}} \right)\log(\psi  T\epsilon_{m_{i}}\right)\nonumber \\
%%%%%%%%%%%%%%%%
%% &\le \exp\left(- \rho (\sigma_{i}^{2}+\sqrt{\epsilon_{m_{i}}} + 2)\log(\psi  T\epsilon_{m_{i}})\right)\nonumber \\
%%%%%%%%%%%%%%%%
%& \overset{(b)}{\leq} \exp\left(- \rho \log(\psi  T\epsilon_{m_{i}})\right) 
%%%%%%%%%%%%%%%%
%\le \dfrac{1}{(\psi  T\epsilon_{m_{i}})^{\rho}}
%\label{lhs1_equn}
%\end{align}
%where, $(a)$ is obtained by substituting equation \ref{si_bar_equn} and $(b)$ occurs because for all $\sigma_{i}^2 \in [0,1]$, $\left(\dfrac{3\sigma_{i}^{2}+3\sqrt{\epsilon_{m_{i}}} + 6}{6\sigma_i^2 + 2\sqrt{\epsilon_{m_i}}}\right) \geq 1$ .
%
% 
%The second term in the LHS of (\ref{eq:prob_eq2}) can be simplified as follows:
%\begin{align}
%&\mathbb{P}\bigg\lbrace \hat{v}_{i}\geq \sigma_{i}^{2}+\sqrt{\epsilon_{m_{i}}}\bigg\rbrace\nonumber\\
%%%%%%%%%%%%%%%%%%%
%&\leq \mathbb{P}\bigg\lbrace \dfrac{1}{n_{i}}\sum_{t=1}^{n_{i}}(X_{i,t}-r_{i})^{2}-(\hat{r}_{i}-r_{i})^{2}\geq \sigma_{i}^{2}+\sqrt{\epsilon_{m_{i}}}\bigg\rbrace\nonumber\\
%%%%%%%%%%%%%%%%%%%
%&\leq \mathbb{P}\bigg\lbrace \dfrac{\sum_{t=1}^{n_{i}}(X_{i,t}-r_{i})^{2}}{n_{i}}\geq \sigma_{i}^{2}+\sqrt{\epsilon_{m_{i}}} \bigg\rbrace\nonumber\\
%%%%%%%%%%%%%%%%%%%
%&\overset{(a)}{\leq} \mathbb{P}\bigg\lbrace \dfrac{\sum_{t=1}^{n_{i}}(X_{i,t}-r_{i})^{2}}{n_{i}}\geq \sigma_{i}^{2} + \bar{c}_i\bigg\rbrace \nonumber\\
%%%%%%%%%%%%%%%%%%%
%&\overset{(b)}{\leq} \exp\left(- \rho \left(\dfrac{3\sigma_{i}^{2}+3\sqrt{\epsilon_{m_{i}}} + 6}{6\sigma_i^2 + 2\sqrt{\epsilon_{m_i}}} \right)\log(\psi  T\epsilon_{m_{i}})\right)
%%%%%%%%%%%%%%%%%%
%\le \dfrac{1}{(\psi  T\epsilon_{m_{i}})^{\rho}}
%\label{lhs2_equn}
%\end{align}
%where inequality $(a)$ is obtained using (\ref{si_bar_equn}), while $(b)$ follows from the Bernstein inequality.
  
%Thus, using (\ref{lhs1_equn}) and (\ref{lhs2_equn}) in (\ref{eq:prob_eq2}) we obtain $\mathbb{P}(\hat{r}_{i}> r_{i} + c_{i})\le \dfrac{2}{(\psi  T\epsilon_{m_{i}})^{\rho}}$. 

From Lemma \ref{proofTheorem:Lemma:3} we can show that $\mathbb{P}(\hat{r}_{i}> r_{i} + c_{i})\leq\mathbb{P}\left( \hat{r}_{i} > r_{i}+ \bar{c}_i\right) + \mathbb{P}\left( \hat{v}_{i}\geq \sigma_{i}^{2}+\sqrt{\epsilon_{m_{i}}}\right) \leq \frac{2}{(\psi  T\epsilon_{m_{i}})^{\frac{3\rho}{2}}}$. Similarly, $\mathbb{P}\lbrace\hat{r}^{*} < r^{*} - c^{*}\rbrace \leq \frac{2}{(\psi  T\epsilon_{m_{i}})^{\frac{3\rho}{2}}}$. Summing the above two contributions, the probability that a sub-optimal arm ${i}$ is not eliminated on or before $m_{i}$-th round by the first two conditions in  (\ref{eq:armelim-casea}) is,  
\begin{eqnarray}
\bigg(\dfrac{4}{(\psi T\epsilon_{m_{i}})^{\frac{3\rho}{2}}} \bigg). \label{eq:arm:elim:c1}
\end{eqnarray}
 

Again, from Lemma \ref{proofTheorem:Lemma:4} and Lemma \ref{proofTheorem:Lemma:5} we can bound the probability of the  complementary of the event $c_{i} \geq c^* $ and $ z_i \geq n_{m_i}$ by,

\begin{eqnarray}
\dfrac{182 K^4}{T^{\frac{5}{4}}\sqrt{\epsilon_{m_i}}} + \dfrac{182 K^4}{T^{\frac{5}{4}}\sqrt{\epsilon_{m_i}}}\leq \dfrac{364 K^4}{T^{\frac{5}{4}}\sqrt{\epsilon_{m_i}}}. \label{eq:arm:elim:c2}
\end{eqnarray}

Also, for eq. $(\ref{eq:arm:elim:c1})$ we can show that for any $\epsilon_{m_i}\in[\sqrt{\frac{e}{T}},1]$
\begin{eqnarray}
\bigg(\dfrac{4}{(\psi T\epsilon_{m_{i}})^{\frac{3\rho}{2}}} \bigg) &\overset{(a)}{\leq} \bigg(\dfrac{4}{(\frac{T^2}{K^2}\epsilon_{m_{i}})^{\frac{3}{4}}} \bigg)\leq \bigg(\dfrac{4 K^{\frac{3}{2}}}{(T^\frac{3}{2} \epsilon_{m_i}^{\frac{1}{4}}\sqrt{\epsilon_{m_{i}}})}\bigg) \nonumber \\
%%%%%%%%%%%%%%%%%%%%%%%
&\overset{(b)}{\leq} \bigg(\dfrac{4 K^{\frac{3}{2}}}{(T^{\frac{3}{2}-\frac{1}{8}}\sqrt{\epsilon_{m_{i}}})}  \bigg)
\leq \dfrac{4 K^4}{T^{\frac{5}{4}}\sqrt{\epsilon_{m_i}}}. \label{eq:arm:elim:c3}
\end{eqnarray}

Here, in $(a)$ we substitute the values of $\psi$ and $\rho$ and $(b)$ follows from the identity $\epsilon_{m_i}^{\frac{1}{4}}\geq (\frac{e}{T})^{\frac{1}{8}} $ as $\epsilon_{m_i}\geq \sqrt{\frac{e}{T}}$.

Summing up over all arms in $\A^{'}$ and bounding the regret for all the \textit{four} arm elimination conditions in (\ref{eq:armelim-casea}) by $(\ref{eq:arm:elim:c2}) + (\ref{eq:arm:elim:c3})$ for each arm $i\in \A^{'}$ trivially by $T\Delta_{i}$, we obtain
	\begin{align*}
&\sum_{i\in \A^{'}}\bigg(\dfrac{4 K^4 T\Delta_i}{T^{\frac{5}{4}}\sqrt{\epsilon_{m_i}}}\bigg) + \sum_{i\in \A^{'}}\bigg(\dfrac{364 K^4 T\Delta_i}{T^{\frac{5}{4}}\sqrt{\epsilon_{m_i}}}\bigg)\\
%%%%%%%%%%%%%%%%%%%%%%%%%%%%%
&\overset{(a)}{\leq}\sum_{i\in \A^{'}}\bigg(\dfrac{368 K^4 T\Delta_{i}}{T^{\frac{5}{4}}\left(\frac{\Delta_{i}^{2}}{4.16}\right)^{\frac{1}{2}}}\bigg)
%%%%%%%%%%%%%%%%%%%%%%%%%%%%%%%
\overset{(b)}{\leq} \sum_{i\in \A^{'}}\bigg(\dfrac{C_1 K^4}{(T)^{\frac{1}{4}}}\bigg).\\  
%%%%%%%%%%%%%%%%%%%%%%%%%%%%%%%
	\end{align*}

%   \begin{align*}
%&\sum_{i\in \A^{'}}\bigg(\dfrac{388 K T\Delta_{i}}{(\psi T\epsilon_{m_{i}})^{\frac{3\rho}{2}}}\bigg)
%\leq\sum_{i\in \A^{'}}\bigg(\dfrac{388 K T\Delta_{i}}{(\psi T\dfrac{\Delta_{i}^{2}}{4.16})^{\frac{3\rho}{2}}}\bigg)\\
%%%%%%%%%%%%%%%%%%%%%%%%%%%%%%%
%&\leq \sum_{i\in \A^{'}}\bigg(\dfrac{388.2^{2+2\frac{3\rho}{2}}.16^{\frac{3\rho}{2}} K T^{1-\frac{3\rho}{2}}}{\psi^{\frac{3\rho}{2}}\Delta_{i}^{2\frac{3\rho}{2} -1}}\bigg)\\  
%%%%%%%%%%%%%%%%%%%%%%%%%%%%%%%
%& \overset{(a)}{\leq} \sum_{i\in \A^{'}}\bigg(\dfrac{388.2^{2+\frac{3}{2}}.16^{\frac{3}{4}} K T^{1-\frac{3}{4}}}{(\frac{T}{K^2})^{\frac{3}{4}}\Delta_{i}^{2.\frac{3}{4} -1}}\bigg)\leq \sum_{i\in \A^{'}}\dfrac{C_1 K^{\frac{5}{2}}}{\sqrt{T\Delta_i}}  
%   \end{align*}
%Here in $(a)$ we substitute the values of $\rho$ and $\psi$ and $C_1$ denotes a constant integer value.\\
Here, $(a)$ happens because $\sqrt{4\epsilon_{m_i}} < \frac{\Delta_i}{4}$, and in $(b)$, $C_1$ denotes a constant integer value.\\


%%%%%%%%%%%%%%%%%%%%%%%%%%%%%%%%%%%%%
% Case (b)
%%%%%%%%%%%%%%%%%%%%%%%%%%%%%%%%%%%%%
\textbf{Case $(b)$:} Here, there are two sub-cases to be considered.
% \subsection*{Case $b$: \textit{An arm ${i}\in B_{m_i}$ is eliminated in round $m_{i}$ or before or there is no $*\in B_{m_i}$}}

\noindent
\textbf{Case $(b1)$ (\textit{${*}\in B_{m_{i}}$ and each ${i}\in \A^{'}$ is  eliminated on or before $m_{i}$ }): } Since we are eliminating a sub-optimal arm ${i}$ on or before round $m_{i}$, it is pulled no longer than, 
 \begin{align*}
 z_{i} < \bigg\lceil\dfrac{\log{(\psi T\epsilon_{m_{i}}^{2})}}{2\epsilon_{m_{i}}}\bigg\rceil
 \end{align*}
%\hspace*{4em}
%%$, since $\sqrt{\rho_{a}\epsilon_{m_{i}}}\leq\dfrac{\Delta_{i}}{2}
So, the total contribution of ${i}$  till round $m_{i}$ is given by, 
\begin{align*}
&\Delta_{i}\bigg\lceil\dfrac{\log{(\psi T\epsilon_{m_{i}}^{2})}}{2\epsilon_{m_{i}}}\bigg\rceil
\overset{(a)}{\leq}    \Delta_{i}\bigg\lceil\dfrac{\log{(\psi T(\dfrac{\Delta_{i}}{16 \times 256})^{4})}}{2(\dfrac{\Delta_{i}}{4\sqrt{4}})^{2}}\bigg\rceil \\
%%%%%%%%%%%%%%%%%%%%%%%%%%%%%%
&\leq   \Delta_{i}\bigg(1+\dfrac{32\log{(\psi T(\dfrac{\Delta_{i}^{4}}{16384})}}{\Delta_{i}^{2}}\bigg)
\leq \Delta_{i}\bigg(1+\dfrac{32\log{(\psi T\Delta_{i}^{4})}}{\Delta_{i}^{2}}\bigg) .
\end{align*} 

Here, $(a)$ happens because $\sqrt{4\epsilon_{m_{i}}} < \frac{\Delta_{i}}{4}$. Summing over all arms in $\A^{'}$ the total regret is given by, 
\begin{align*}
&\sum_{i\in \A^{'}}\Delta_{i}\bigg(1+\dfrac{32\log{(\psi T\Delta_{i}^{4}})}{\Delta_{i}^{2}}\bigg) = \sum_{i\in \A^{'}}\bigg(\Delta_{i} +\dfrac{32\log{(\psi T\Delta_{i}^{4}})}{\Delta_{i}}\bigg) \\
%%%%%%%%%%%%%%%%%%%%%%%%%%%
&\overset{(a)}{\leq} \sum_{i\in \A^{'}} \left(\Delta_{i}+\dfrac{64\log{( \frac{T\Delta_{i}^{2}}{K})}}{\Delta_{i}}\right)\\
%%%%%%%%%%%%%%%%%%%%%%%%%%%
&\overset{(b)}{\leq} \sum_{i\in \A^{'}} \left(\Delta_{i} +\dfrac{16(4\sigma_i^2 + 4)\log{( \frac{T\Delta_{i}^{2}}{K})}}{\Delta_{i}}\right)\\
&%%%%%%%%%%%%%%%%%%%%%%%%%%%
\overset{(c)}{\leq} \sum_{i\in \A^{'}} \left(\Delta_{i} +\dfrac{320\sigma_i^2\log{( \frac{T\Delta_{i}^{2}}{K})}}{\Delta_{i}}\right).\\
\end{align*}

We obtain $(a)$ by substituting the value of $\psi$, $(b)$ from $0\leq\sigma_i^2 \leq\frac{1}{4},\forall i\in \A$ and $(c)$ from Lemma \ref{proofTheorem:Lemma:6}.\\

\noindent
\textbf{Case $(b2)$ (\textit{Optimal arm ${*}$ is eliminated by a sub-optimal arm):  }} Firstly, if conditions of Case $a$ holds then the optimal arm ${*}$ will not be eliminated in round $m=m_{*}$ or it will lead to the contradiction that $r_{i}>r^{*}$. In any round $m_{*}$, if the optimal arm ${*}$ gets eliminated then for any round from $1$ to $m_{j}$ all arms ${j}$ such that $m_{j}< m_{*}$ were eliminated according to assumption in Case $a$. Let the arms surviving till $m_{*}$ round be denoted by $\A^{'}$. This leaves any arm $a_{b}$ such that $m_{b}\geq m_{*}$ to still survive and eliminate arm ${*}$ in round $m_{*}$. Let such arms that survive ${*}$ belong to $\A^{''}$. Also maximal regret per step after eliminating ${*}$ is the maximal $\Delta_{j}$ among the remaining arms ${j}$ with $m_{j}\geq m_{*}$.  Let $m_{b}=\min\left\lbrace m|\sqrt{4\epsilon_{m}}<\frac{\Delta_{b}}{4}\right\rbrace$. Hence, the maximal regret after eliminating the arm ${*}$ is upper bounded by, 

\begin{align*}
&\sum_{m_{*}=0}^{max_{j\in \A^{'}}m_{j}}\sum_{i\in \A^{''}:m_{i}>m_{*}}\bigg(\dfrac{368 K^4}{(T^{\frac{5}{4}}\sqrt{\epsilon_{m_{*}}})} \bigg).T\max_{j\in \A^{''}:m_{j}\geq m_{*}}{\Delta}_{j}\\
%%%%%%%%%%%%%%%%%%%%%%%%%%%%
&\leq\sum_{m_{*}=0}^{max_{j\in \A^{'}}m_{j}}\sum_{i\in \A^{''}:m_{i}>m_{*}}\bigg(\dfrac{368 K^4 \sqrt{4}}{(T^{\frac{5}{4}}\sqrt{\epsilon_{m_{*}}})} \bigg).T.4\sqrt{\epsilon_{m_{*}}}\\
%%%%%%%%%%%%%%%%%%%%%%%%%%%%
&\overset{(a)}{\leq}\sum_{m_{*}=0}^{max_{j\in \A^{'}}m_{j}}\sum_{i\in \A^{''}:m_{i}>m_{*}}\bigg(\dfrac{C_2 K^4}{T^{\frac{1}{4}}\epsilon_{m_{*}}^{\frac{1}{2}-\frac{1}{2}}} \bigg)\\
%%%%%%%%%%%%%%%%%%%%%%%%%%%%
&\leq\sum_{i\in \A^{''}:m_{i}>m_{*}}\sum_{m_{*}=0}^{\min{\lbrace m_{i},m_{b}\rbrace}}\bigg(\dfrac{C_2 K^4}{T^{\frac{1}{4}}} \bigg)\\
%%%%%%%%%%%%%%%%%%%%%%%%%%%%
&\leq\sum_{i\in \A^{'}}\bigg(\dfrac{C_2 K^4}{T^{\frac{1}{4}}} \bigg)+\sum_{i\in \A^{''}\setminus \A^{'}}\bigg(\dfrac{C_2 K^4}{T^{\frac{1}{4}}} \bigg).\\
\end{align*}
Here at $(a)$, $C_2$ denotes an integer constant.



%\begin{align*}
%\sum_{m_{*}=0}^{max_{j\in \A^{'}}m_{j}}\sum_{i\in \A^{''}:m_{i}>m_{*}}\bigg(\dfrac{388 K}{(\psi  T\epsilon_{m_{*}})^{\frac{3\rho}{2}}} \bigg).T\max_{j\in \A^{''}:m_{j}\geq m_{*}}{\Delta}_{j}
%\end{align*}
%
%Again applying Lemma \ref{proofTheorem:Lemma:8} we can show that the above expression is upper bounded by 
%\begin{align*}
%\sum_{i\in \A^{'}}\dfrac{C_2^{'} K^{\frac{5}{2}}}{\sqrt{T\Delta_i}} +\sum_{i\in \A^{''}\setminus \A^{'}}\dfrac{C_2^{'} K^{\frac{5}{2}}}{\sqrt{T b}}
%\end{align*}

%%%%%%%%%%%%%%%%%%%%%%%%%%%%%%%%
%Moved to Appendix as Lemma 9
%%%%%%%%%%%%%%%%%%%%%%%%%%%%%%%%

%\begin{align*}
%&\sum_{m_{*}=0}^{max_{j\in \A^{'}}m_{j}}\sum_{i\in \A^{''}:m_{i}>m_{*}}\bigg(\dfrac{388 K}{(\psi  T\epsilon_{m_{*}})^{\frac{3\rho}{2}}} \bigg).T\max_{j\in \A^{''}:m_{j}\geq m_{*}}{\Delta}_{j}\\
%%%%%%%%%%%%%%%%%%%%%%%%%%%%%
%&\leq\sum_{m_{*}=0}^{max_{j\in \A^{'}}m_{j}}\sum_{i\in \A^{''}:m_{i}>m_{*}}\bigg(\dfrac{388 K\sqrt{4}}{(\psi  T\epsilon_{m_{*}})^{\frac{3\rho}{2}}} \bigg).T.4\sqrt{\epsilon_{m_{*}}}\\
%%%%%%%%%%%%%%%%%%%%%%%%%%%%%
%&\leq\sum_{m_{*}=0}^{max_{j\in \A^{'}}m_{j}}\sum_{i\in \A^{''}:m_{i}>m_{*}}C_2 K\bigg(\dfrac{T^{1-\frac{3\rho}{2}}}{\psi^{\frac{3\rho}{2}}\epsilon_{m_{*}}^{\frac{3\rho}{2}-\frac{1}{2}}} \bigg)\\
%%%%%%%%%%%%%%%%%%%%%%%%%%%%%
%&\leq\sum_{i\in \A^{''}:m_{i}>m_{*}}\sum_{m_{*}=0}^{\min{\lbrace m_{i},m_{b}\rbrace}}\bigg(\dfrac{C_2 K T^{1-\frac{3\rho}{2}}}{\psi^{\frac{3\rho}{2}}2^{-(\frac{3\rho}{2} -\frac{1}{2})m_{*}}} \bigg)\\
%%%%%%%%%%%%%%%%%%%%%%%%%%%%%
%&\leq\sum_{i\in \A^{'}}\bigg(\dfrac{C_2 K T^{1-\frac{3\rho}{2}}}{\psi^{\frac{3\rho}{2}}2^{-(\frac{3\rho}{2} -\frac{1}{2})m_{*}}} \bigg)+\sum_{i\in \A^{''}\setminus \A^{'}}\bigg(\dfrac{C_2 K T^{1-\frac{3\rho}{2} }}{\psi^{\frac{3\rho}{2}}2^{-(\frac{3\rho}{2} -\frac{1}{2})m_{b}}} \bigg)\\
%%%%%%%%%%%%%%%%%%%%%%%%%%%%%
%&\leq\sum_{i\in \A^{'}}\bigg(\dfrac{C_2 K T^{1-\frac{3\rho}{2}}.2^{\frac{\frac{3\rho}{2}}{2}-\frac{1}{4}}}{\psi^{\frac{3\rho}{2}}\Delta_{i}^{\frac{3\rho}{2} -\frac{1}{2}}} \bigg)+\sum_{i\in \A^{''}\setminus \A^{'}}\bigg(\dfrac{C_2 K T^{1-\frac{3\rho}{2}}.2^{\frac{\frac{3\rho}{2}}{2}-\frac{1}{4}}}{\psi^{\frac{3\rho}{2}}b^{\frac{3\rho}{2} -\frac{1}{2}}} \bigg)\\
%%%%%%%%%%%%%%%%%%%%%%%%%%%%%
%&\leq\sum_{i\in \A^{'}}\bigg(\dfrac{ C_2 K 2^{\frac{\frac{3\rho}{2}}{2}+\frac{19}{4}}.T^{1-\frac{3\rho}{2} } }{\psi^{\rho}\Delta_{i}^{2\frac{3\rho}{2} -1}} \bigg)+\sum_{i\in \A^{''}\setminus \A^{'}}\bigg(\dfrac{C_2 K 2^{\frac{\frac{3\rho}{2}}{2}+\frac{19}{4}}.T^{1-\frac{3\rho}{2}} }{\psi^{\frac{3\rho}{2} }b^{2\frac{3\rho}{2}-1}} \bigg)\\
%%%%%%%%%%%%%%%%%%%%%%%%%%%%%
%&\overset{(a)}{\leq}\sum_{i\in \A^{'}}\bigg(\dfrac{C_2^{'} K .T^{1-\frac{3}{4}}}{(\frac{T}{K^2})^{\frac{3}{4}}\Delta_{i}^{2.\frac{3}{4} -1}} \bigg)+\sum_{i\in \A^{''}\setminus \A^{'}}\bigg(\dfrac{C_2^{'} K T^{1-\frac{3}{4}}}{(\frac{T}{K^2})^{\frac{3}{4}}b^{2.\frac{3}{4}-1}} \bigg)\\
%%%%%%%%%%%%%%%%%%%%%%%%%%%%%
%&\leq\sum_{i\in \A^{'}}\dfrac{C_2^{'} K^{\frac{5}{2}}}{\sqrt{T\Delta_i}} +\sum_{i\in \A^{''}\setminus \A^{'}}\dfrac{C_2^{'} K^{\frac{5}{2}}}{\sqrt{T b}}
%%%%%%%%%%%%%%%%%%%%%%%%%%%%%
%\end{align*}
%In the above simplification, $(a)$ is obtained by substituting the values of $\psi$ and $\rho$.

Finally, summing up the regrets in \textbf{Case a} and \textbf{Case b}, the total regret is given by
\begin{align*}
\E [R_{T}] \leq &\sum\limits_{i\in \A :\Delta_{i} > b}\bigg\lbrace \dfrac{C_0 K^{4}}{T^{\frac{1}{4}}} + \bigg(\Delta_{i}+\dfrac{320\sigma_i^2\log{(\frac{T\Delta_{i}^{2}}{K})}}{\Delta_{i}}\bigg)\bigg \rbrace\\ 
  & +\sum\limits_{i\in \A :0 < \Delta_{i}\leq b} \dfrac{C_2 K^{4}}{T^{\frac{1}{4}}} + \max_{i\in \A :0 < \Delta_{i}\leq b}\Delta_{i}T
\end{align*}

where $C_0, C_1, C_2$ are integer constants s.t. $C_0 = C_1 + C_2$.
\end{customproof}




\section{Experiments}
\label{sec:expt}
In this section, we conduct extensive empirical evaluations of EUCBV against several other popular MAB  algorithms. We use expected cumulative regret as the metric of comparison. The comparison is conducted against the following algorithms: KLUCB+ \citep{garivier2011kl}, DMED \citep{honda2010asymptotically}, MOSS \citep{audibert2009minimax}, UCB1 \citep{auer2002finite}, UCB-Improved \citep{auer2010ucb}, Median Elimination \citep{even2006action}, Thompson Sampling (TS) \citep{agrawal2011analysis}, OCUCB \citep{lattimore2015optimally}, Bayes-UCB (BU) \citep{kaufmann2012bayesian} and UCB-V \citep{audibert2009exploration}\footnote{The implementation for KLUCB, Bayes-UCB and DMED were taken from \citet{CapGarKau12}}. The parameters of EUCBV algorithm for all the experiments are set as follows: $\psi=\frac{T}{K^2}$ and $\rho =0.5$ (as in Corollary \ref{Result:Corollary:1}). Note that KLUCB+ empirically outperforms KLUCB (as shown in \citet{garivier2011kl}).

\begin{figure}[!th]
    \centering
    \begin{tabular}{cc}
    \setlength{\tabcolsep}{0.1pt}
    \subfigure[0.25\textwidth][Expt-$1$: $20$ Bernoulli-distributed arms ]
    %with $r_{i_{{i}\neq {*}}}=0.07$ and $r^{*}=0.1$
    {
    		\pgfplotsset{
		tick label style={font=\Large},
		label style={font=\Large},
		legend style={font=\Large},
		ylabel style={yshift=5pt},
		%legend style={legendshift=32pt},
		}
        \begin{tikzpicture}[scale=0.7]
      	\begin{axis}[
		xlabel={timestep},
		ylabel={Cumulative Regret},
		grid=major,
        %clip mode=individual,grid,grid style={gray!30},
        clip=true,
        %clip mode=individual,grid,grid style={gray!30},
  		legend style={at={(0.5,1.5)},anchor=north, legend columns=3} ]
      	% UCB
		\addplot table{Chapter3/results/NewExpt/Expt1/UCBV01_comp_subsampled.txt};
		\addplot table{Chapter3/results/NewExpt/Expt1/EUCBV01_comp_subsampled.txt};
		\addplot table{Chapter3/results/NewExpt/Expt1/KLUCB01_comp_subsampled.txt};
		\addplot table{Chapter3/results/NewExpt/Expt1/MOSS01_comp_subsampled.txt};
		\addplot table{Chapter3/results/NewExpt/Expt1/DMED01_comp_subsampled.txt};
		\addplot table{Chapter3/results/NewExpt/Expt1/UCB01_comp_subsampled.txt};
		\addplot table{Chapter3/results/NewExpt/Expt1/TS01_comp_subsampled.txt};
		\addplot table{Chapter3/results/NewExpt/Expt1/OCUCB01_comp_subsampled.txt};
		\addplot table{Chapter3/results/NewExpt/Expt1/BU01_comp_subsampled.txt};
      	\legend{UCB-V,EUCBV,KLUCB+,MOSS,DMED,UCB1,TS,OCUCB,BU}      	
      	\end{axis}
      	\end{tikzpicture}
  		\label{fig:1}
    }
    &
    \subfigure[0.25\textwidth][Expt-$2$: $3$ Group Mean Setting ]
    %with $r_{i_{{i}\neq {*}:1-33}}=0.1$, $r_{i_{{i}\neq {*}:34-99}}=0.6$, $r^{*}_{i=100}=0.9$ and $\sigma_{i=1:100}^{2} = 0.3$
    {
    		\pgfplotsset{
		tick label style={font=\Large},
		label style={font=\Large},
		legend style={font=\Large},
		ylabel style={yshift=5pt},
		}
        \begin{tikzpicture}[scale=0.7]
        \begin{axis}[
		xlabel={timestep},
		ylabel={Cumulative Regret},
        %clip mode=individual,grid,grid style={gray!30},
       	grid=major,
       	clip=true,
  		legend style={at={(0.5,1.5)},anchor=north, legend columns=3} ]
      	% UCB
		\addplot table{Chapter3/results/NewExpt/Expt2/UCBV01_comp_subsampled.txt};
		\addplot table{Chapter3/results/NewExpt/Expt2/EUCBV01_comp_subsampled.txt};
		\addplot table{Chapter3/results/NewExpt/Expt2/KLUCB01_comp_subsampled.txt};
		\addplot table{Chapter3/results/NewExpt/Expt2/MOSS01_comp_subsampled.txt};
		\addplot table{Chapter3/results/NewExpt/Expt2/UCBR01_comp_subsampled.txt};
		\addplot table{Chapter3/results/NewExpt/Expt2/UCB01_comp_subsampled.txt};
		\addplot table{Chapter3/results/NewExpt/Expt2/TS01_comp_subsampled.txt};
		\addplot table{Chapter3/results/NewExpt/Expt2/OCUCB01_comp_subsampled.txt};
		\addplot table{Chapter3/results/NewExpt/Expt2/BU01_comp_subsampled.txt};      	
      	\legend{UCB-V,EUCBV,KLUCB-G+,MOSS,UCB-Imp,UCB1,TS-G,OCUCB,BU-G}
      	\end{axis}
      	\end{tikzpicture}
   		\label{fig:2}
    }
    \end{tabular}
    \caption{A comparison of the cumulative regret incurred by the various bandit algorithms. }
    \label{fig:karmed}
    \vspace*{-1em}
\end{figure}
% For the purpose of performance comparison


\textbf{Experiment-1 (Bernoulli with uniform gaps):} This experiment is conducted to observe the performance of EUCBV over a short horizon. The horizon $T$ is set to $60000$. The testbed comprises of $20$ Bernoulli distributed arms with expected rewards of the arms as $r_{1:19}=0.07$ and $r^{*}_{20}=0.1$ and these type of cases are frequently encountered in web-advertising domain (see \cite{garivier2011kl}). The regret is averaged over $100$ independent runs and is shown in Figure \ref{fig:1}. EUCBV, MOSS, OCUCB, UCB1, UCB-V, KLUCB+, TS, BU and DMED are run in this experimental setup. Not only do we observe that EUCBV performs better than all the non-variance based algorithms such as MOSS, OCUCB, UCB-Improved and UCB1, but it also outperforms UCBV because of the choice of the exploration parameters. Because of the small gaps and short horizon $T$, we do not compare with UCB-Improved and Median Elimination for this test-case. 

\textbf{Experiment-2 (Gaussian $3$ Group Mean Setting):} This experiment is conducted to observe the performance of EUCBV over a large horizon in Gaussian distribution testbed. This setting comprises of a large horizon of $T = 3\times 10^{5}$ timesteps and a large set of arms. This testbed comprises of $100$ arms involving Gaussian reward distributions with expected rewards of the arms in $3$ groups, $r_{1:66}=0.07$, $r_{67:99}=0.01$ and $r^{*}_{100}=0.09$ with variance set as $\sigma_{1:66}^{2} = 0.01,\sigma_{67:99}^{2} = 0.25$ and $\sigma^{2}_{100}=0.25$. The regret is averaged over $100$ independent runs and is shown in Figure \ref{fig:2}. From the results in Figure \ref{fig:2}, we observe that since the gaps are small and the variances of the optimal arm and the arms farthest from the optimal arm are the highest, EUCBV, which allocates pulls proportional to the variances of the arms, outperforms all the non-variance based algorithms MOSS, OCUCB, UCB1, UCB-Improved and Median-Elimination ($\epsilon=0.1,\delta=0.1$). The performance of Median-Elimination is extremely weak in comparison with the other algorithms and its plot is not shown in Figure \ref{fig:2}. We omit its plot in order to more clearly show the difference between EUCBV, MOSS and OCUCB. Also note that the order of magnitude in the y-axis (cumulative regret) of Figure \ref{fig:2} is $10^4$. KLUCB-Gauss+ (denoted by KLUCB-G+), TS-G and BU-G are initialized with Gaussian priors. Both KLUCB-G+ and UCBV which is a variance-aware algorithm perform much worse than TS-G and EUCBV. The performance of DMED is similar to KLUCB-G+ in this setup and its plot is omitted. 


\begin{figure}[!h]
    \centering
    \begin{tabular}{cc}
    \subfigure[0.25\textwidth][Expt-$3$: Failure of TS]
    {
    		\pgfplotsset{
		tick label style={font=\Large},
		label style={font=\Large},
		legend style={font=\Large},
		ylabel style={yshift=5pt},
		}
        \begin{tikzpicture}[scale=0.7]
      	\begin{axis}[
		ylabel={Cumulative Regret},
		xlabel={timestep},
		grid=major,
        %clip mode=individual,grid,grid style={gray!30},
        clip=true,
        %clip mode=individual,grid,grid style={gray!30},
  		legend style={at={(0.5,1.3)},anchor=north, legend columns=3} ]
      	% UCB
		\addplot table{Chapter3/results/NewExpt/Expt3/UCBV01_comp_subsampled.txt};
		\addplot table{Chapter3/results/NewExpt/Expt3/EUCBV01_comp_subsampled.txt};
		\addplot table{Chapter3/results/NewExpt/Expt3/MOSS01_comp_subsampled.txt};
		\addplot table{Chapter3/results/NewExpt/Expt3/TS01_comp_subsampled.txt};
		\addplot table{Chapter3/results/NewExpt/Expt3/OCUCB01_comp_subsampled.txt};
		\addplot table{Chapter3/results/NewExpt/Expt3/BU01_comp_subsampled.txt};
      	\legend{UCBV,EUCBV,MOSS,TS-G,OCUCB,BU-G} 
      	\end{axis}
      	\end{tikzpicture}
  		\label{fig:3}
    }
    &
    \subfigure[0.25\textwidth][Expt-$4$: $3$ Group Variance Setting]
    %with $r_{i_{{i}\neq {*}}}=0.05$ and $r^{*}=0.1$
    {
    	\pgfplotsset{
		tick label style={font=\Large},
		label style={font=\Large},
		legend style={font=\Large},
		ylabel style={yshift=5pt},
		}
        \begin{tikzpicture}[scale=0.7]
        \begin{axis}[
		xlabel={timestep},
		ylabel={Cumulative Regret},
        %clip mode=individual,grid,grid style={gray!30},
		grid=major,
		clip=true,
  		legend style={at={(0.5,1.3)},anchor=north, legend columns=3} ]
        % UCB
		\addplot table{Chapter3/results/NewExpt/Expt41/UCBV01_comp_subsampled.txt};
		\addplot table{Chapter3/results/NewExpt/Expt41/EUCBV01_comp_subsampled.txt};
		\addplot table{Chapter3/results/NewExpt/Expt41/MOSS01_comp_subsampled.txt};
		\addplot table{Chapter3/results/NewExpt/Expt41/TS01_comp_subsampled.txt};
		\addplot table{Chapter3/results/NewExpt/Expt41/OCUCB01_comp_subsampled.txt};
		\addplot table{Chapter3/results/NewExpt/Expt41/BU01_comp_subsampled.txt};
      	\legend{UCBV,EUCBV,MOSS,TS-G,OCUCB,BU-G} 
      	\end{axis}
        \end{tikzpicture}
        \label{fig:4}
    }
	\end{tabular}
	\label{fig:furtherExpt1}
    \caption{Further Experiments with EUCBV}
    \vspace*{-1em}
\end{figure}
%\vspace*{-0.5em}



\textbf{Experiment-3 (Failure of TS):} This experiment is conducted to demonstrate that in certain environments when the horizon is large, gaps are small and the variance of the optimal arm is high, the Bayesian algorithms (like TS) do not perform well but EUCBV performs exceptionally well. This experiment is conducted on $100$ Gaussian distributed arms such that expected rewards of the arms $r_{1:10}=0.045$, $r_{11:99}=0.04$, $r^{*}_{100}=0.05$ and the variance is set as $\sigma_{1:10}^{2}=0.01$,   $\sigma_{100}^{2}=0.25$ and $T=4\times 10^5$. The variance of the arms $i=11:99$ are chosen uniform randomly between $[0.2,0.24]$. TS and BU with Gaussian priors fail because here the chosen variance values are such that only variance-aware algorithms with appropriate exploration factors will perform  well or otherwise it will get bogged down in costly exploration. The algorithms that are not variance-aware will spend a significant amount of pulls trying to find the optimal arm. The result is shown in Figure \ref{fig:3}. Predictably EUCBV, which allocates pulls proportional to the variance of the arms, outperforms its closest competitors TS-G, BU-G, UCBV, MOSS and OCUCB. The plots for KLUCB-G+, DMED, UCB1, UCB-Improved and Median Elimination are omitted from the figure as their performance is extremely weak in comparison with other algorithms. We omit their plots to clearly show how EUCBV outperforms its nearest competitors. Note that EUCBV by virtue of its aggressive exploration parameters outperforms UCBV in all the experiments even though UCBV is a variance-based algorithm. The performance of TS-G is also weak and this is in line with the observation in \citet{lattimore2015optimally} that the worst case regret of TS when Gaussian prior is used is $\Omega\left( \sqrt{KT\log T}\right)$.



\textbf{Experiment-4 (Gaussian $3$ Group Variance setting):} This experiment is conducted to show that when the gaps are uniform and variance of the arms are the only discriminative factor then the EUCBV performs extremely well over a very large horizon and over a large number of arms. This testbed comprises of $100$ arms with Gaussian reward distributions, where the expected rewards of the arms are $r_{1:99}=0.09$ and $r^{*}_{100}=0.1$. The variances of the arms are divided into $3$ groups. The group $1$ consist of arms $i=1:49$ where the variances are chosen uniform randomly between $[0.0,0.05]$, group $2$ consist of arms $i=50:99$ where the variances are chosen uniform randomly between $[0.19,0.24]$ and for the optimal arm $i=100$ (group $3$) the variance is set as $\sigma_{*}^{2}=0.25$. We report the cumulative regret averaged over $100$ independent runs. The horizon is set at $T=4\times 10^{5}$ timesteps. We report the performance of MOSS,BU-G, UCBV, TS-G and OCUCB who are the closest competitors of EUCBV over this uniform gap setup. From the results in Figure \ref{fig:4}, it is evident that the growth of regret for EUCBV  is much lower than that of TS-G, MOSS, BU-G, OCUCB and UCBV. Because of the poor performance of KLUCB-G+ in the last two experiments we do not implement it in this setup. Also, note that for optimal performance BU-G, TS-G and KLUCB-G+ require the knowledge of the type of distribution to set their priors . Also, in all the experiments with Gaussian distributions EUCBV significantly outperforms all the Bayesian algorithms initialized with Gaussian priors.




\section{Conclusion and Future Works}
\label{sec:conc}
In this paper, we studied the EUCBV algorithm which takes into account the empirical variance of the arms and  employs aggressive exploration parameters in conjunction with non-uniform arm selection (as opposed to UCB-Improved) to eliminate sub-optimal arms. Our theoretical analysis conclusively established that EUCBV exhibits an order-optimal gap-independent regret bound of $O\left(\sqrt{KT}\right)$. Empirically, we show that EUCBV performs superbly across diverse experimental settings and outperforms most of the bandit algorithms in a  stochastic MAB setup. Our experiments show that EUCBV is extremely stable for larger horizons and performs consistently well across different types of distributions. One avenue for future work is to remove the constraint of $T\geq K^{2.4}$ required for EUCBV to reach the order optimal regret bound. Another future direction is to come up with an anytime version of EUCBV. An anytime algorithm does not need the horizon $T$ as an input parameter.


%\section*{Acknowledgement} This work was supported by a funding from IIT Madras under project CSE/14-15/831/ RFTP/BRAV.

\section{Summary}
\label{tbandit:Summary}
In this chapter we looked at a novel variant of the UCB algorithm (referred to as Efficient-UCB-Variance (EUCBV)) for minimizing cumulative regret in the stochastic multi-armed bandit (MAB) setting. EUCBV incorporates the arm elimination strategy proposed in UCB-Improved \citep{auer2010ucb}, while taking into account the variance estimates to compute the arms' confidence bounds, similar to UCBV \citep{audibert2009exploration}. Through a theoretical analysis we establish that EUCBV incurs a \emph{gap-dependent} regret bound of {\scriptsize $O\left( \dfrac{K\sigma^2_{\max} \log (T\Delta^2 /K)}{\Delta}\right)$} after $T$ trials, where $\Delta$ is the minimal gap between optimal and sub-optimal arms; the above bound is an improvement over that of existing state-of-the-art UCB algorithms (such as UCB1, UCB-Improved, UCBV,  MOSS). Further, EUCBV incurs a \emph{gap-independent} regret bound of {\scriptsize $O\left(\sqrt{KT}\right)$}  which is an improvement over that of UCB1, UCBV and UCB-Improved, while being comparable with that of MOSS and OCUCB. Through an extensive numerical study we show that EUCBV significantly outperforms the popular UCB variants (like MOSS, OCUCB, etc.) as well as Thompson sampling and Bayes-UCB algorithms. 


%%%%%%%%%%%%%%%%%%%%%%%%%%%%%%%%%%%%%%%%%%%%%%%%%%%%%%%%%%%%

%%%%%%%%%%%%%%%%%%%%%%%%%%%%%%%%%%%%%%%%%%%%%%%%%%%%%%%%%%%%
% Appendices.

\appendix

\chapter{APPENDIX}
\section{Appendix for EUCBV}
\label{sec:app}
\input{EUCBV/Appendix}

%%%%%%%%%%%%%%%%%%%%%%%%%%%%%%%%%%%%%%%%%%%%%%%%%%%%%%%%%%%%
% Bibliography.

\begin{singlespace}
\bibliographystyle{iitm}
\bibliography{refs}
\end{singlespace}


%%%%%%%%%%%%%%%%%%%%%%%%%%%%%%%%%%%%%%%%%%%%%%%%%%%%%%%%%%%%
% List of papers

\listofpapers

\begin{enumerate}  
\item Authors....  \newblock
 Title...
  \newblock {\em Journal}, Volume,
  Page, (year).
\end{enumerate}  

\end{document}

%\documentclass[PhD]{iitmdiss}
\documentclass[MS]{iitmdiss}
%\documentclass[MTech]{iitmdiss}
%\documentclass[BTech]{iitmdiss}
\usepackage{times}
 \usepackage{t1enc}

\usepackage{graphicx}
\usepackage{epstopdf}
\usepackage{hyperref} % hyperlinks for references.
\usepackage{amsmath} % easier math formulae, align, subequations \ldots


%%%%%%%%%%%%%%%%%%%%%%%%%%%%%%%
%\usepackage{ijcai17}


\usepackage{macros}

\usepackage{latexsym} 


\begin{document}

%%%%%%%%%%%%%%%%%%%%%%%%%%%%%%%%%%%%%%%%%%%%%%%%%%%%%%%%%%%%%%%%%%%%%%
% Title page

\title{A study on online sequential learning using Bandits}

\author{Subhojyoti Mukherjee}

\date{December 2017}
\department{COMPUTER SCIENCE AND ENGINEERING}

%\nocite{*}
\maketitle

%%%%%%%%%%%%%%%%%%%%%%%%%%%%%%%%%%%%%%%%%%%%%%%%%%%%%%%%%%%%%%%%%%%%%%
% Certificate
\certificate

\vspace*{0.5in}

\noindent This is to certify that the thesis titled {\bf A study on online sequential learning using Bandits}, submitted by {\bf Subhojyoti Mukherjee}, 
  to the Indian Institute of Technology, Madras, for
the award of the degree of {\bf Master of Science (Research)}, is a bona fide
record of the research work done by him under our supervision.  The
contents of this thesis, in full or in parts, have not been submitted
to any other Institute or University for the award of any degree or
diploma.

\vspace*{1.5in}

\begin{singlespacing}
\hspace*{-0.25in}
\parbox{2.5in}{
\noindent {\bf Dr. Balaraman Ravindran} \\
\noindent Research Guide \\ 
\noindent Associate Professor \\
\noindent Dept. of Computer Science\\
\noindent IIT-Madras, 600 036 \\
} 
\hspace*{1.0in} 
\parbox{2.5in}{
\noindent {\bf Dr. Nandan Sudarsanam} \\
\noindent Research Co-Guide \\ 
\noindent Assistant Professor \\
\noindent Dept.  of  Management Studies\\
\noindent IIT-Madras, 600 036 \\
}  
\end{singlespacing}
\vspace*{0.25in}
\noindent Place: Chennai\\
Date: 22nd December 2017 


%%%%%%%%%%%%%%%%%%%%%%%%%%%%%%%%%%%%%%%%%%%%%%%%%%%%%%%%%%%%%%%%%%%%%%
% Acknowledgements
\acknowledgements

Thanks to all those who made \TeX\ and \LaTeX\ what it is today.

%%%%%%%%%%%%%%%%%%%%%%%%%%%%%%%%%%%%%%%%%%%%%%%%%%%%%%%%%%%%%%%%%%%%%%
% Abstract

\abstract

\noindent KEYWORDS: \hspace*{0.5em} \parbox[t]{4.4in}{Reinforcement Learning, Bandits, UCB, EUCBV, Thresholding Bandits,  AugUCB, changepoint detection, SECPD.}

\vspace*{24pt}

\noindent The thesis studies the following topics in the area of Reinforcement Learning: Multi-armed Bandits, Multi-armed bandits in stationary distribution with the goal of cumulative regret minimization, Thresholding bandits in pure exploration setting, and analysis of bandit theory in piece-wise stationary distributions. The common underlying theme is the study of bandit theory and its application in various types of environments. We start with a general introduction to Multi-armed bandits, its connection to the wider reinforcement learning theory and then we discuss the various types of bandits available in the literature. Then we delve deep into the classic multi-armed bandit problem in stationary distribution, one of the first setting studied by the bandit community and which successively gave rise to several new directions in bandit theory. We propose a novel algorithm in this setting and compare both theoretically and empirically its performance against the available algorithms in this setting. Subsequently, we study a very specific type of bandit setup called the thresholding bandit problem and discuss extensively on its usage, available state-of-the-art algorithms on this setting and our solution to this problem. We give theoretical guarantees on the expected loss of our algorithm and also analyze its performance against state-of-the-art algorithms in numerical simulations in multiple synthetic environments. Finally, we study the notion of piece-wise stationary distribution and how available bandit algorithms can be modified to perform well in this setting. We propose a set of algorithms for this setting with the goal of minimizing cumulative regret which uses various techniques ranging from changepoint detection mechanism to aggregation of experts.

\pagebreak

%%%%%%%%%%%%%%%%%%%%%%%%%%%%%%%%%%%%%%%%%%%%%%%%%%%%%%%%%%%%%%%%%
% Table of contents etc.

\begin{singlespace}
\tableofcontents
\thispagestyle{empty}

\listoftables
\addcontentsline{toc}{chapter}{LIST OF TABLES}
\listoffigures
\addcontentsline{toc}{chapter}{LIST OF FIGURES}
\end{singlespace}


%%%%%%%%%%%%%%%%%%%%%%%%%%%%%%%%%%%%%%%%%%%%%%%%%%%%%%%%%%%%%%%%%%%%%%
% Abbreviations
\abbreviations

\noindent 
\begin{tabbing}
xxxxxxxxxxx \= xxxxxxxxxxxxxxxxxxxxxxxxxxxxxxxxxxxxxxxxxxxxxxxx \kill
\textbf{IITM}   \> Indian Institute of Technology, Madras \\
\textbf{RTFM} \> Read the Fine Manual \\
\end{tabbing}

\pagebreak

%%%%%%%%%%%%%%%%%%%%%%%%%%%%%%%%%%%%%%%%%%%%%%%%%%%%%%%%%%%%%%%%%%%%%%
% Notation

\chapter*{\centerline{NOTATION}}
\addcontentsline{toc}{chapter}{NOTATION}

\begin{singlespace}
\begin{tabbing}
xxxxxxxxxxx \= xxxxxxxxxxxxxxxxxxxxxxxxxxxxxxxxxxxxxxxxxxxxxxxx \kill
\textbf{$r$}  \> Radius, $m$ \\
\textbf{$\alpha$}  \> Angle of thesis in degrees \\
\textbf{$\beta$}   \> Flight path in degrees \\
\end{tabbing}
\end{singlespace}

\pagebreak
\clearpage

% The main text will follow from this point so set the page numbering
% to arabic from here on.
\pagenumbering{arabic}


%%%%%%%%%%%%%%%%%%%%%%%%%%%%%%%%%%%%%%%%%%%%%%%%%%
% Introduction.
\chapter{Introduction}
\label{chap:intro}

\section{Introduction}
\label{intro}
In today's world artificial intelligence has proved to be a game-changer in designing agents that interact with an evolving environment and make decisions on the fly. The main goal of artificial intelligence is to design artificial agents that make dynamic decisions in an evolving environment. In pursuit of these the agent can be thought of as making a series of  sequential decisions by interacting with the dynamic environment which provides it with some sort of feedback after every decision which the agent incorporates into its decision-making strategy to formulate the next decision to be made. A large number of problems in science and engineering, robotics and game playing, resource management, financial portfolio management, medical treatment design, ad placement, website optimization and packet routing can be modeled as sequential decision-making under uncertainty. Many of these real-world interesting
sequential decision-making problems can be formulated as reinforcement learning (RL) problems (\citep{bertsekas1996neuro}, \citep{sutton1998reinforcement}). In an RL problem, an agent interacts with a dynamic, stochastic, and unknown environment, with the goal of finding an action-selection strategy or policy that optimizes some long-term performance measure. Every time when the agent interacts with the environment it receives a signal/reward from the environment based on which it modifies its policy. The agent learns to optimize the choice of actions over several time steps which is learned from the sequences of data that it receives from the environment. This is the crux of online sequential learning. An illustrative image depicting the reinforcement learning scenario is shown in Figure \ref{fig:rl}.

	This is in contrast to supervised learning methods that deal with labeled data which are independently and identically distributed (i.i.d.) samples from the domain and train some classifier on the entire training dataset to learn the pattern of this distribution to predict future samples (test dataset) with the assumption that it is sampled from the same domain, whereas the RL agent learns from the samples that are collected from the trajectories generated by its sequential interaction with the system. For an RL agent the trajectory consists of a series of sequential interactions whereby it transitions from one state to another following some dynamics intrinsic to the environment while collecting the reward till some stopping condition is reached. This is known as an episode. For a single-step interaction, i.e., when the episode terminates after a single transition, the problem is captured by the multi-armed bandit (MAB) model. Our work will focus on this idea of MAB model.

\begin{figure}[!th]
\includegraphics[scale=0.7]{Chapter1/img/RL.png}
\caption{Reinforcement Learning}
\label{fig:rl}
\end{figure}

%To express an RL problem more formally, we have to define the idea of Markov Decision Process (MDP) which consists of states, actions, transition probabilities and rewards which in turn helps in deciding the strategy to be followed by the agent. 

%An MDP consists of states


\section{Motivation}
\label{motivation}

The MAB model fits very well in various real-world scenarios that can be modeled as sequential decision-making problems. Some of which are mentioned as follows:-
\begin{enumerate}
\item \emph{Online Shop Domain (\cite{ghavamzadeh2015bayesian}):} In the online shop domain, a retailer aims to maximize profit by sequentially suggesting products to online shopping customers. In this scenario, at every timestep,  the retailer displays an item to a customer from a pool of items which has the highest probability of being selected by the customer. The episode ends when the customer selects or does not select a product (which will be considered as a loss to the retailer) and the process is again repeated till a pre-specified number of times with the retailer gathering valuable information regarding the customer from this behaviour and modifying its policy to display the next item.
\item \emph{Medical Treatment Design (\cite{thompson1933likelihood}):} Here at every timestep, the agent chooses to administer one out of several treatments sequentially on a patient. Here, the episode ends when the patient responds well or does not respond well to the treatment whereby the agent modifies its policy for the next suggestion.
\item \emph{Financial Portfolio Management:} In financial portfolio management MAB model can be used. Here, the agent is faced with the choice of selecting the most profitable stock option out of several stock options. The simplest strategy where we can employ a bandit model is this; at the start of every trading session the agent suggests a stock to purchase worth Re $1$, while at the closing of the trading session it sells off the stock to witness its value after a day's trading. The  profit recorded is treated as the reward revealed by the environment and the agent modifies its policy for the next day.
%\item \emph{Product Selection:} A company wants to introduce a new product in market and there is a clear separation of the test phase from the commercialization phase. In this case the company tries to minimize the loss it might incur in the commercialization phase by testing as much as possible in the test phase. So from the several variants of the product that are in the test phase the learning agent must suggest a product variant at the end of the test phase that has the highest probability of minimizing loss in the commercialization phase(see \cite{bubeck2011pure}). 
%\item \emph{Mobile Phone Channel Allocation:} Another similar problem as above concerns channel allocation for mobile phone communications (\cite{audibert2009exploration}). Here there is a clear separation between the allocation phase and communication phase whereby in the allocation phase a learning algorithm has to explore as many channels as possible to suggest the best possible channel. Each evaluation of a channel is noisy and the learning algorithm must come up with the best possible suggestion within a very small  number of attempts. 
\end{enumerate}

	The thresholding bandit problem is a special case of combinatorial MAB problem where the learner has to suggest the best set of arms above a real valued threshold. This has several relevant industrial applications. The variants of TopM problem (identifying the best $M$ arms from $K$ given arms) can be readily used in the thresholding problem.

\begin{enumerate}
\item \emph{Product Selection:} A company wants to introduce a new product in market and there is a clear separation of the test phase from the commercialization phase. In this case the company tries to minimize the loss it might incur in the commercialization phase by testing as much as possible in the test phase. So from the several variants of the product that are in the test phase the learning agent must suggest the product variant(s) that are above a particular threshold $\tau$ at the end of the test phase that have the highest probability of minimizing loss in the commercialization phase. A similar problem has been discussed for single best product variant identification without threshold in \cite{bubeck2011pure}. 
\item \emph{Mobile Phone Channel Allocation:} Another similar problem as above concerns channel allocation for mobile phone communications (\cite{audibert2009exploration}). Here there is a clear separation between the allocation phase and communication phase whereby in the allocation phase a learning algorithm has to explore as many channels as possible to suggest the best possible set of channel(s) that are above a particular threshold $\tau$. The threshold depends on the subscription level of the customer. With higher subscription the customer is allowed better channel(s) with the $\tau$ set high. Each evaluation of a channel is noisy and the learning algorithm must come up with the best possible suggestion within a very small  number of attempts.
\item \emph{Anomaly Detection and Classification:} Thresholding bandit can also be used for anomaly detection and classification where we define a cutoff level $\tau$ and for any samples above this cutoff gets classified as an anomaly. For further reading we point the reader to section 3 of \cite{locatelli2016optimal}.
\end{enumerate}


	In all the above examples the MAB model performs well mainly because all of them suffer from \textit{exploration-exploitation dilemma}. This is characterized by action-selection choice faced by the agent where it must decide whether to stay with the action yielding highest reward till now or to explore newer actions which might be more profitable in the long run. MAB's are suited for such scenarios because 
\begin{enumerate}
\item They are easy to implement.
\item The switch between exploration and exploitation is more well defined theoretically.
\item They perform well empirically.
\end{enumerate}



\section{Types of Information Feedback}
\label{feed}
In an online sequential setting, the feedback that the learner receives from the environment can be characterized into three broad categories, full information feedback, partial information feedback and bandit feedback. 


	To illustrate the different types of feedback we will take help of the following example. Let a learner be given a set of actions $i\in\A$ such that $|A|=K$. Let, the environment be such that each action has a probability distribution $D_i$ attached to it which is fixed throughout the time horizon $T$. The learning proceeds as follows:-

\begin{algorithm}[!th]
\caption{An online sequential game}
\label{alg:OSeqGame}
\begin{algorithmic}
\State {\bf Input:} Time horizon $T$, $K$ number of arms with unknown parameters of reward distribution
\State \For{ each timestep $t=1,2,\ldots, T$}
\State The environment chooses a reward $r_{i,t},\forall i\in\A$.
\State The learner chooses $m$ actions such that $m < K$, where $A$ is the set of arms and $|A|=K$.
\State The learner observes the reward $R_{m,t}=F\left( r_{i,t}\right)$.
\State \EndFor
\end{algorithmic}
\end{algorithm}

%Let $G(V,E)$ denote a graph where $V$ denotes the set of nodes in the graph and $E$ denotes the set of edges of the graph. Let there be a single starting node, denoted by $\mathcal{s}\in V$ from where the learner must start and try to reach the destination node denoted by $\mathcal{d}\in V$. Each edge has a delay associated with it which is unknown to the learner. This delay is an $i.i.d$ random variable from the distribution $D_{ij}$ associated with the edge $e_{ij}$ between the vertices $v_i$ and $v_j$. Whenever an edge is chosen the environment reveals to the learner  At every timestep the learner chooses a set of edges and receives some form of feedback from the environment. The goal of the learner is to find the path from $s$ to $d$ which has the minimum delay associated with it. 


\subsection{Full information feedback}
In full information feedback, when a learner selects $m$ actions then the environment reveals the rewards of all the actions $i\in \A$. Hence, in this form of feedback  the learner observes $R_{m,t} = \lbrace r_{i,t},\forall i\in\A\rbrace$.


\subsection{Partial information feedback}
In partial information feedback, when a learner selects $m$ actions then the environment reveals the rewards of only those $m$ actions for $m\in \A$. Hence, in this form of feedback  the learner observes $R_{m,t} = \lbrace r_{m,t},\forall m\in\A\rbrace$. This is also sometimes called the semi-bandit feedback.


\subsection{Bandit feedback}
In bandit feedback, when a learner selects $m$ actions then the environment reveals a cumulative reward of those $m$ actions for $m\in \A$. Hence, in this form of feedback  the learner observes $R_{m,t} = \sum_{q=1}^{m} r_{q,t}$. Note, that when $m=1$, then the learner observes the reward of only that action that it has chosen out of $K$ actions.





\section{Different types of Bandits}
\label{types}
In this section we discuss on the various types of bandits that are available in literature. 


\subsection{Types of Bandits based on Environment}


\subsubsection{Stochastic Bandits}
In stochastic bandits, the distribution associated with each of the arms remains fixed throughout the time horizon $T$. Some of the notable papers associated with this type of setup are \citet{robbins1952some}, \citet{lai1985asymptotically},  \citet{agrawal1995sample}, \citet{auer2002finite}, \citet{auer2010ucb}, \citet{audibert2009minimax}, \citet{lattimore2015optimally}, etc. Chapter \ref{chap:SMAB} and Chapter \ref{chap:EUCBV} is based on this setup where we discuss extensively on the latest state-of-the-art algorithms.



\subsubsection{Non-stochastic Bandits}

In non-stochastic setting the distribution associated with each arm varies over the duration of the play. Two natoable examples of this are:-

\begin{itemize}
\item \textbf{Adversarial bandits: } In adversarial bandits, an adversary decides the payoff for each arm before the learner selects an arm. This adversary may or may not be oblivious to the learning algorithm employed by the learner. In each of these cases a different guarantee on the performance of the learner can be arrived at. Some of the important papers in this setting are \citet{auer2002nonstochastic}, \citet{auer2002using}, \citet{kocak2014efficient}.

\item \textbf{Piece-wise stationary:} Another setup under this setting can be the piece-wise stationary setting. In this setting, the distribution associated with each arm is not fixed throughout the time horizon and changes either arbitrarily at particular changepoints, or changes at a fixed period. The distribution associated with each arm then  remains fixed till the next chnagepoint is encountered. Several recent works have focussed on this such as  \citep{garivier2011upper}, \citep{mellor2013thompson}, and \citep{allesiardo2017non}. In our thesis, chapter \ref{chap:psbandit} is based on this setting. 
\end{itemize}


\subsubsection{Contextual Bandits}

The idea of clustering has been extensively studied in the contextual bandit setup, an extension of the MAB where side information or features are attached to each arm. The clustering is done over the features representing the arms to capture the complexity of the problem better when a large-number of arms are involved. Typical examples of this setting are in web-advertising domain, news article selection, etc. Some notable papers available for this setting are   \citet{auer2002using}, \citet{langford2008epoch}, \citet{li2010contextual}, \citet{beygelzimer2011contextual}, \citet{slivkins2014contextual},etc. 


%

%Clustering has been extensively studied in the area of contextual MAB. In contextual MAB, there are side-information or features attached to each arm (see  \citet{auer2002using,langford2008epoch,li2010contextual,beygelzimer2011contextual, slivkins2014contextual}).   \cite{bui2012clustered,cesa2013gang,gentile2014online}. Please note that we do not cluster over the context rather we cluster the arms into groups.




\subsection{Types of Bandits based on goal}

In bandit literature, based on the goal we can divide bandits into several categories. To illustrate his we put forward a simple scenario let us consider a stochastic bandit scenario where there are $K$ arms labeled $i=1,2,\ldots,K$ with their expected means of reward distributions ($D_i$) be denoted by $r_i$. Also let there be single optimal arm $*$ such that $r^* = \max_{i\in\A}r_i$. 



\subsubsection{Cumulative regret minimization}
In cumulative regret minimization the goal of the bandit is to minimize the cumulative regret which is the total loss suffered by the learner throughout the time horizon $T$ for not choosing the optimal arm. Formally, we can define the cumulative regret as,

\begin{eqnarray}
R_{T} = \sum_{t=1}^{T}r^* - \sum_{i\neq *}r_{i}n_{i,T} \label{eqn:chap1:regret}
\end{eqnarray}

where, $n_{i,T}$ is the number of times the learner has chosen arm $i$ over the entire horizon $T$. We can further reduce equation \ref{eqn:chap1:regret} to obtain,

\begin{align*}
R_{T} = \sum_{t=1}^{T}r^* - \sum_{i\neq *}r_{i}n_{i,T} = \sum_{i=1}^{K}\Delta_{i}n_{i,T}
\end{align*}

where $\Delta_{i}=r^* - r_i$ is called the gap between the optimal and the sub-optimal arm.

\subsubsection{Simple regret minimization}
In simple regret minimization the goal of the bandit is to minimize the instantaneous regret that is suffered at any  timestep by the learner. Formally, the simple regret at $t$-th timestep where $J_n\in\A$ is the recommendation by the learner at timestep $t$ is defined,

\begin{align*}
SR_{t} = r^* - r_{J_{n}} = \Delta_{J_n}
\end{align*}

where, $\Delta_{J_n}$ is the instantaneous gap between the expected mean of he optimal arm and the recommended arm by the learner.

\subsubsection{Weak Regret minimization}
In the non-stochastic scenario, when the distribution associated with each arm changes, the notion of regret is defined differently than cumulative regret. In this scenario, considering that there is a single best arm, the learner is more interested in minimizing the worst-case regret. Formally, for any sequence of actions $\left( j_1, \ldots , j_T \right)$ over the time horizon $T$, the weak regret for single best action is defined as the difference between,
\begin{align*}
G_{\max}(j_1,\ldots,j_T) - G_{\pi}(T)
\end{align*}
where, $G_{\max}(j_1,\ldots,j_T) = \max_{i\in\A}\sum_{t=1}^{T}x_{i_t}$ is the return of the globally best action over the entire horizon $T$ and $G_{\pi}(T)$ is the return following the policy $\pi$ over the horizon $T$ instead of choosing $j_1,\ldots,j_T$.



\subsection{Collaborative Bandits}

Distributed bandits are specific setup of MAB where a network of bandits collaborate with each other to identify the optimal arm(s) (see \citet{awerbuch2008competitive,liu2010distributed,szorenyi2013gossip,hillel2013distributed}). In our setting we can assign each of the $p$ clusters to individual bandits and at the end of each round they can share information synchronously to identify the optimal arm. This naturally results in a speedup of operation and helps in identifying the best arm faster. The clustering in this case is typically done over the feature space \citet{bui2012clustered}, \citet{cesa2013gang}, \citet{gentile2014online}.

\subsection{Bandits with Corrupt Feedback}


\subsection{Conservative Bandits}

\section{Outline of the thesis}
\label{outline}
In this chapter, we gave an overview of the various types of bandits available in the literature and also discussed about the main objectives of the thesis and our contributions. In this section, we give a general outline of the thesis that is to follow. In chapter \ref{chap:SMAB} we give a detailed overview of the stochastic multi-armed bandit model and the latest available algorithms in this setting. In the next chapter \ref{chap:EUCBV} we introduce our algorithm Efficient UCB Variance (EUCBV) for the stochastic multi-armed bandit model. We give theoretical guarantees on the performance of EUCBV and also show in numerical simulations that it indeed performs very well as compared to the state-of-the-art algorithms. In the subsequent chapter \ref{chap:tbandit1} we introduce a new variant of pure exploration multi-armed stochastic bandit called the thresholding bandit problem. We analyze the connections between thresholding bandit problem and pure exploration problem and also discuss several existing algorithms in both the settings that are relevant to carefully analyze the thresholding bandit problem. Then in chapter \ref{chap:tbandit2} we introduce our solution for the thresholding bandit problem, called the Augemented UCB (AugUCB) algorithm. We analyze our algorithm AugUCB and derive theoretical guarantees for it as well as show in numerical experiments that it indeed outperforms several state-of-the-art algorithms in the thresholding bandit setting. Finally, in chapter \ref{chap:psbandit} we introduce the piecewise-stochastic bandit model which is a new variant that strides between the stochastic and adversarial setting. We discuss extensively on this setting and also provide our solution to this setting and show in numerical simulations that our solution is very close to the optimal solution. 


%%%%%%%%%%%%%%%%%%%%%%%%%%%%%%%%%%%%%%%%%%%%%%%%%%%%%%%%%%%%

%%%%%%%%%%%%%%%%%%%%%%%%%%%%%%%%%%%%%%%%%%%%%%%%%%%%%%%%%%%%

\chapter{Stochastic Multi-armed Bandits}
\label{chap:SMAB}

\section{Introduction to SMAB}
\label{sec:intro}
In this chapter, we deal with the stochastic multi-armed bandit (SMAB) setting. In its classical form, stochastic MABs represent a sequential learning problem where a learner is exposed to a finite set of actions (or arms) and needs to choose one of the actions at each timestep. After choosing (or pulling) an arm the learner receives a reward, which is conceptualized as an independent random draw from stationary distribution associated with the selected arm. Also, note that in SMAB, the distribution associated with each arm is fixed throughout the entire duration of the horizon denoted by $T$.

\begin{algorithm}[!th]
\caption{SMAB formulation}
\label{alg:SMAB}
\begin{algorithmic}
\State {\bf Input:} Time horizon $T$, $K$ number of arms with unknown parameters of reward distribution
\State \For{ each timestep $t=1,2,\ldots, T$}
\State The learner chooses an arm $i\in\A$, where $A$ is the set of arms and $|A|=K$.
\State The learner observes the reward $X_{i,t}\sim^{i.i.d} D_{i}$ where, $D_{i}$ is the distribution associated with the arm $i$. 
\State \EndFor
\end{algorithmic}
\end{algorithm}
	 


\section{Notations and assumptions}
\label{sec:notations}
\begin{assumption}
\label{SMAB:assm:1}
In the considered SMAB setting we assume the optimal arm to be unique and it is denoted by $*$.
\end{assumption}

\begin{assumption}
\label{SMAB:assm:2}
We assume the rewards of all arms are bounded in $[0,1]$.
\end{assumption}


\textbf{Notations:} The mean of the reward distribution $D_i$ associated with an arm $i$ is denoted by $r_i$ whereas the mean of the reward distribution of the optimal arm $*$ is denoted by $r^*$ such that $r_i < r^*, \forall i\in \A$, where $\A$ is the set of arms such that $|\A|=K$. We denote the individual arms labeled $i$, where  $i=1,\ldots,K$. We denote the sample mean of the rewards for an arm $i$ at time instant $t$ by $\hat{r}_{i}(t)=\frac{1}{z_{i}(t)}\sum_{\ell=1}^{z_i(t)} X_{i,\ell}$, where $X_{i,\ell}$ is the reward sample received when arm $i$ is pulled for the $\ell$-th time, and $z_i(t)$ is the number of times arm $i$ has been pulled until timestep $t$. We denote the true variance of an arm by $\sigma_i^{2}$ while $\hat{v}_{i}(t)$ is the estimated variance, i.e., $\hat{v}_{i}(t)=\frac{1}{z_i(t)}\sum_{\ell=1}^{z_{i}(t)}(X_{i,\ell}-\hat{r}_{i})^{2}$. Whenever there is no ambiguity about the underlying  time index $t$, for simplicity we neglect $t$ from the notations and simply use  $\hat{r}_i, \hat{v}_i,$ and $z_i$ to denote the respective quantities. Also, $\Delta$ denotes the minimum gap such that $\Delta=\min_{i\in\A}\lbrace \Delta_{i}\rbrace$.


%For simplicity, we assume that the optimal arm is unique and denote it by ${*}$.
%We denote an arbitrary round of EUCBV by $m$.

\section{Problem Definition}
\label{sec:probDef}
With the formulation of SMAB stated in algorithm \ref{alg:SMAB}, the learner faces the task of balancing exploitation and exploration. In other words, should the learner pull the arm which currently has the best-known estimates (exploit) or explore arms more thoroughly to ensure that a correct decision is being made. This is termed as the \textit{exploration-exploitation dilemma}, one of the fundamental challenges of Reinforcement learning.


	The objective in the stochastic bandit problem is to minimize the cumulative regret, which is defined as follows:
\begin{align*}
R_{T}=r^{*}T - \sum_{i\in \A} r_{i}z_{i}(T),
\end{align*}
where $T$ is the number of timesteps, and  $z_{i}(T)$ is the number of times the algorithm has chosen arm $i$ up to timestep $T$.
The expected regret of an algorithm after $T$ timesteps can be written as,
\begin{align*}
\E[R_{T}]= \sum_{i=1}^{K} \E[z_i (T)] \Delta_i,
\end{align*}
where $\Delta_{i}=r^{*}-r_{i}$ is the gap between the means of the optimal arm and the $i$-th arm.

\section{Motivation}
\label{sec:motivation}
There has been a significant amount of research in the area of stochastic MABs. One of the earliest work can be traced to \citet{thompson1933likelihood}, which deals with  the problem of choosing between two treatments to administer on patients who come in sequentially. In \citet{thompson1935theory} this work was extended to include more general cases of finitely many treatments. In recent years the SMAB setting has garnered extensive popularity because of its simple learning  model and its practical applications in a wide-range of industries, including, but not limited to, mobile channel allocations, online advertising and computer simulation games. Some of these problems have been already discussed in chapter \ref{chap:intro}, section \ref{motivation} and an interested reader can refer to it.
	

\section{Related Work in SMAB}
\label{sec:related}
\subsection{Lower Bound in SMAB}    
    
    SMAB problems have been extensively studied in several earlier works such as \citet{thompson1933likelihood},  \citet{thompson1935theory}, \citet{robbins1952some} and \citet{lai1985asymptotically}. Lai and Robbins in  \citet{lai1985asymptotically} established an asymptotic lower bound for the cumulative regret. It showed that for any consistent allocation strategy, we can have
\begin{align*}
\liminf_{T \to \infty}\frac{\E[R_{T}]}{\log T}\geq\sum_{\{i:r_{i}<r^{*}\}}\frac{(r^{*}-r_{i})}{KL(Q_{i}||Q^{*})}
\end{align*}    
where $KL(Q_{i}||Q^{*})$ is the Kullback-Leibler divergence between the reward densities $Q_{i}$ and $Q^{*}$, corresponding to arms with mean $r_{i}$ and $r^{*}$, respectively.

\subsection{The Upper Confidence Bound Approach}
    
    Over the years SMABs have seen several algorithms with strong regret guarantees. For further reference, an interested reader can look into \citet{bubeck2012regret}. In the next few subsections, we will explicitly focus on the upper confidence bound algorithms which is a type of non-Bayesian algorithm widely used in SMAB setting. The upper confidence bound or UCB algorithms balance the exploration-exploitation dilemma by linking the uncertainty in the estimate of an arm with the number of times an arm is pulled and therefore ensuring sufficient exploration. 
    
\subsubsection{UCB1 Algorithm}    
    
    
    One of the earliest among these algorithms is UCB1 algorithm proposed first in \citet{agrawal1995sample} and subsequently analyzed in \citet{auer2002finite}. The UCB1 algorithm (as stated in \citet{auer2002finite}) is mentioned in algorithm \ref{alg:ucb1}.
    
\begin{algorithm}[!th]
\caption{UCB1}
\label{alg:ucb1}
\begin{algorithmic}[1]
\State \textbf{Input:} $K$ number of arms with unknown parameters of reward distribution
\State Pull each arm once
 \For{$t=K+1,..., T$}
\State Pull the arm such that $\argmax_{i\in A}\bigg\lbrace\hat{r}_{i} + \sqrt{\dfrac{2\log (t)}{z_i}}\bigg\rbrace$
\State $t:=t+1 $
 \EndFor
\end{algorithmic}
\end{algorithm}
    
    The intuition behind this algorithm is simple and it follows from the ideas of concentration inequalities in probability measure theory. The term $\sqrt{\dfrac{2\log (t)}{z_i}}$ is called the confidence interval of the arm $i$ and it signifies a measure of uncertainty over the arm $i$ based on the history of observed rewards for that arm. Therefore, lesser the confidence interval, higher is our confidence that the estimated mean $\hat{r}_i$ is lying close to the expected mean $r_i$ of the arm $i$. Also, note that the confidence interval decreases at the rate of $O\left( \dfrac{1}{\sqrt{z_i}}\right)$ which signifies the rate of convergence of $\hat{r}_i$ to $r_i$ and depends on the number of time the arm has been pulled.
    
    UCB1 has a gap-dependent regret upper bound of  $O\left(\frac{K\log T}{\Delta}\right)$, where $\Delta = \min_{i:\Delta_i>0} \Delta_i$. This result is asymptotically order-optimal for the class of distributions considered. But, the worst case gap-independent regret bound of UCB1 is found to be  $O \left(\sqrt{KT\log T}\right)$. 
    
\subsubsection{UCB-Improved Algorithm}        
    
\begin{algorithm}[!th]
\caption{UCB-Improved}
\label{alg:ucbi}
\begin{algorithmic}[1]
\State {\bf Input:} Time horizon $T$, $K$ number of arms with unknown parameters of reward distribution
\State {\bf Initialization:} Set $B_{0}:= \A$ and $\epsilon_{0}:=1$.
\For{$m=0,1,..\big \lfloor \dfrac{1}{2}\log_{2} \dfrac{T}{e}\big\rfloor$}    
\State Pull each arm in $B_m$, $n_{m}=\bigg\lceil\dfrac{2\log{( T\epsilon_{m}^{2})}}{\epsilon_{m}}\bigg\rceil$ number of times.
%so that the total  it has been pulled is
\ArmElim
\State For each $i \in B_{m}$, delete arm ${i}$ from $B_{m}$ if,
\begin{align*}
\hat{r}_{i} + \sqrt{\dfrac{\log{(T\epsilon_{m}^{2})}}{2 n_{m}}}  < \max_{{j}\in B_{m}}\bigg\lbrace\hat{r}_{j} -\sqrt{\dfrac{\log{( T\epsilon_{m}^{2})}}{2 n_{m}}} \bigg\rbrace
\end{align*}
\EndArmElim
%\ResParam
\State Set $\epsilon_{m+1}:=\dfrac{\epsilon_{m}}{2}$, Set $B_{m+1}:=B_{m}$
%\EndResParam
\State Stop if $|B_{m}|=1$ and pull ${i}\in B_{m}$ till $n$ is reached.
\EndFor
\end{algorithmic}
\end{algorithm}
    
    The UCB-Improved  stated in algorithm \ref{alg:ucbi}, proposed in \citet{auer2010ucb}, is a round-based variant of UCB1. An algorithm is \textit{round-based} if it pulls all the arms equal number of times in each round and then eliminates one or more arms that it deems to be sub-optimal. Note, that in this algorithm the confidence interval term is $\sqrt{\dfrac{\log{( T\epsilon_{m}^{2})}}{2 n_{m}}}$ which is constant in the $m$-th round as $n_m$ is fixed for that round and all arms are being pulled an equal number of times in each round. This is unlike UCB1 algorithm where the confidence interval term depends on $z_i$ which is a random variable. Also, note that in UCB-Improved the knowledge of horizon is required before-hand to calculate the confidence intervals whereas no such input is required for UCB1. An illustrative flowchart depicting the main steps is given in Figure \ref{fig:ucbimp}.
    
\begin{figure}[!th]
\includegraphics[scale=0.45]{Chapter2/img/Ucb-Imp.png}
\caption{Flowchart of UCB-Improved}
\label{fig:ucbimp}
\end{figure}
    
    UCB-Improved incurs a gap-dependent regret bound of $O\left(\frac{K\log (T\Delta^{2})}{\Delta}\right)$, which is better than that of UCB1. On the other hand, the worst case gap-independent regret bound of UCB-Improved is $O\left(\sqrt{KT\log K}\right)$.Empirically, UCB-Improved is out-performed by UCB1 in almost all environments. This stems from the fact that UCB-Improved is pulling all arms equal number of times in each round and hence spends a significant number of pulls in initial exploration as opposed to UCB1 thereby incurring higher regret.
        
    
\subsubsection{MOSS Algorithm}    

\begin{algorithm}[!th]
\caption{MOSS}
\label{alg:moss}
\begin{algorithmic}[1]
\State \textbf{Input:} Time horizon $T$, $K$ number of arms with unknown parameters of reward distribution
\State Pull each arm once
 \For{$t=K+1,..., T$}
\State Pull the arm such that $\argmax_{i\in \A}\bigg\lbrace\hat{r}_{i} + \sqrt{\dfrac{\max\lbrace 0,\log(\frac{T}{K z_i})\rbrace}{z_i}}\bigg\rbrace$
\State $t:=t+1 $
 \EndFor
\end{algorithmic}
\end{algorithm}
    
    In the later work of \citet{audibert2009minimax}, the authors propose the MOSS algorithm which stands for Minimax Optimal Strategy in the Stochastic case (see algorithm \ref{alg:moss}). The confidence interval of MOSS is designed in such a way so as to divide the horizon $T$ proportionally between the number of arms $K$ and the number of pulls $z_i$ that each arm is pulled. As the sub-optimal arms are pulled more number of times their confidence interval decreases, indicating that they have been explored sufficiently, forcing MOSS to explore other arms and quickly converging on the optimal arm. Theoretically, \citet{audibert2009minimax}  showed that the worst case gap-independent regret bound of MOSS is $O\left( \sqrt{KT} \right)$ which improves upon UCB1 by a factor of order $\sqrt{\log T}$. However, the gap-dependent regret of MOSS is $O\left( \frac{K^{2}\log\left(T\Delta^{2}/K\right)}{\Delta}\right)$ and in certain regimes, this can be worse than even UCB1 (see \citet{audibert2009minimax,lattimore2015optimally}). 
    
\subsubsection{OCUCB Algorithm}   

\begin{algorithm}[!th]
\caption{OCUCB}
\label{alg:ocucb}
\begin{algorithmic}[1]
\State \textbf{Input:} Time horizon $T$, $K$ number of arms with unknown parameters of reward distribution, exploration parameter $\alpha$ and $\psi$
\State Pull each arm once
 \For{$t=K+1,..., T$}
\State Pull the arm such that $\argmax_{i\in \A}\bigg\lbrace\hat{r}_{i} + \sqrt{\dfrac{\alpha\log(\psi\frac{T}{t})}{z_i}}\bigg\rbrace$
\State $t:=t+1 $
 \EndFor
\end{algorithmic}
\end{algorithm}

Recently in \citet{lattimore2015optimally}, the authors proposed the Optimally Confident UCB (OCUCB) (see algorithm \ref{alg:ocucb}) which incorporates the increasing timestep $t$ in the confidence interval along with the fixed horizon $T$ and exploration parameters $\psi$ and $\alpha$. The authors showed that the algorithm OCUCB achieves order-optimal gap-dependent regret bound of $O\left(\sum_{i=2}^{K}\frac{\log\left(T/H_i\right)}{\Delta_i}\right)$ where $H_i=\sum_{j=1}^{K}\min\left\lbrace \frac{1}{\Delta_i^2},\frac{1}{\Delta_j^2}\right\rbrace$, and a gap-independent regret bound of $O\left( \sqrt{KT}\right)$. This is the best known gap-dependent and gap-independent regret bounds in the stochastic MAB framework. However, unlike our proposed EUCBV algorithm (in chapter \ref{chap:EUCBV}), OCUCB does not take into account the variance of the arms; as a result, empirically  we find  that our algorithm outperforms OCUCB in all the environments considered.



\subsubsection{UCB-Variance Algorithm}

\begin{algorithm}[!th]
\caption{UCBV}
\label{alg:ucbv}
\begin{algorithmic}[1]
\State \textbf{Input:} $K$ number of arms with unknown parameters of reward distribution
\State Pull each arm once
 \For{$t=K+1,..., T$}
\State Pull the arm such that $\max_{i\in A}\bigg\lbrace\hat{r}_{i} + \sqrt{\dfrac{2\hat{v}_i\log (t)}{s_i}} + \dfrac{3\log (t)}{2}\bigg\rbrace$
\State $t:=t+1 $
 \EndFor
\end{algorithmic}
\end{algorithm}


    In contrast to the above work, the UCB-Variance (UCBV) algorithm in \citet{audibert2009exploration} utilizes variance estimates to compute the confidence intervals for each arm. In UCBV (see algorithm \ref{alg:ucbv}) the confidence interval term is given by $\sqrt{\dfrac{2\hat{v}_i\log (t)}{s_i}} + \dfrac{3\log (t)}{3}$ where $\hat{v}_i$ denotes the empirical variance of the arm $i$. Hence, the confidence interval makes sure that the arms whose variances are high are pulled more often to get a better estimates of their $\hat{r}_i$.
    
    UCBV has a gap-dependent regret bound of $O\left(\frac{K\sigma_{\max}^{2}\log T}{\Delta}\right)$, where $\sigma_{\max}^{2}$ denotes the maximum variance among all the arms $i\in \A$. Its gap-independent regret bound can be inferred to be same as that of UCB1 i.e $O \left(\sqrt{KT\log T}\right)$. Empirically, \citet{audibert2009exploration} showed that UCBV outperforms UCB1 in several scenarios. 


\subsection{Bayesian Approach}

\begin{algorithm}[!th]
\caption{Bernoulli Thompson Sampling}
\label{alg:ts}
\begin{algorithmic}
\State {\bf Input:} $K$ number of arms with unknown parameters of reward distribution
\State {\bf Initialization:} For each arm $i:=1$ to $K$ set $S_i =0$ and $F_i =0$
\State \For{$t=1,..,T$}
\State \For{$i=1,..,K$}
\State Sample $\theta_{i}(t)$ from the $Beta(S_i+1,F_i+1)$ distribution.
\EndFor
\State Play the arm $i(t):=\argmax_i\theta_i(t)$ and observe reward $X_{i,t}$.
\If{$X_{i,t}=1$}
$S_i (t) = S_i (t) + 1$
\Else{$F_i (t) = F_i (t) + 1$}
\EndIf
\EndFor
\end{algorithmic}
\end{algorithm}

    
    Another notable design principle which has recently gained a lot of popularity is the Thompson Sampling (TS) algorithm (\citep{thompson1933likelihood}, \citep{agrawal2011analysis})  and  Bayes-UCB (BU) algorithm \citep{kaufmann2012bayesian}. This TS is stated in algorithm \ref{alg:ts}. The TS algorithm is initialized with a uniform prior and it maintains a posterior reward distribution for each arm; at each round, the algorithm samples values from these distributions and the arm corresponding to the highest sample value is chosen. Although TS is found to perform extremely well when the reward distributions are Bernoulli, it is established that with Gaussian priors the worst-case regret can be as bad as $\Omega \left( \sqrt{KT\log T}\right)$ \citep{lattimore2015optimally}. The BU algorithm is an extension of the TS algorithm that takes quartile deviations into consideration while choosing arms.

\subsection{Information Theoretic Approach}
    
    The final design principle we state is the information theoretic approach of DMED  \citep{honda2010asymptotically} and KLUCB \citep{garivier2011kl},\citep{cappe2013kullback} algorithms. The algorithm KLUCB uses Kullbeck-Leibler divergence to compute the upper confidence bound for the arms. KLUCB is stable for a short horizon and is known to reach the \citet{lai1985asymptotically} lower bound in the special case of Bernoulli distribution. However, \citet{garivier2011kl} showed that KLUCB, MOSS and UCB1 algorithms are empirically outperformed by UCBV in the exponential distribution as they do not take the variance of the arms into consideration.
    
\subsection{Discussion on The Various Confidence Intervals}

A comparative analysis of the confidence interval of the UCB algorithms is discussed in table \ref{tab:conf-comp}. 

\begin{table}[!ht]
\caption{Confidence interval of different algorithms}
\label{tab:conf-comp}
\begin{center}
\begin{tabular}{|p{5em}|p{10em}|p{4em}|p{12em}|}
\hline
Algorithm  &  Confidence interval & Horizon as input & Remarks \\
\hline
\hline
UCB1        & $\sqrt{\dfrac{2\log (t)}{z_i}}$ & No & Loose confidence interval leading to high regret upper bounds.\\%\midrule
\hline
\hline
UCBV        & $\sqrt{\dfrac{2\hat{v}_i\log (t)}{s_i}} + \dfrac{3\log (t)}{2}$ & No & Confidence interval uses variance estimation.\\
\hline
\hline
UCB-Imp 		& $\sqrt{\dfrac{\log{( T\epsilon_{m}^{2})}}{2 n_{m}}}$ & Yes & Same confidence interval for all arms in a particular round.\\%\midrule
\hline
\hline
MOSS	     	& $\sqrt{\dfrac{\max\lbrace 0,\log(\frac{T}{K z_i})\rbrace}{z_i}}$ & Yes & Confidence interval is based on dividing the horizon proportionally between $K$ arms and $z_i$ pulls for each arm.\\%\midrule
\hline
\hline
OCUCB     	& $\sqrt{\dfrac{2\log(\frac{2T}{t})}{z_i}}$ & Yes & Tightest confidence interval with exploration parameter $\alpha=2$, $\psi=2$ leading to order-optimal regret bounds.\\\midrule
\end{tabular}
\end{center}
%\vspace*{-2em}
\end{table} 

\section{Summary}
\label{chap2:conc}
In this chapter, we looked at the stochastic multi-armed bandit (SMAB) setting and discussed how it is important in the general reinforcement learning setup. We also looked at the various state-of-the-art algorithms in the literature for the SMAB setting and discussed the advantages and disadvantages of them. The regret bounds that have been proven for the said algorithms have also been discussed at length and their confidence intervals have also been compared against each other. In the next chapter, we provide our solution to this SMAB setting which achieves an almost order-optimal regret bound.

%%%%%%%%%%%%%%%%%%%%%%%%%%%%%%%%%%%%%%%%%%%%%%%%%%%%%%%%%%%%




%%%%%%%%%%%%%%%%%%%%%%%%%%%%%%%%%%%%%%%%%%%%%%%%%%%%%%%%%%%%
\chapter{Efficient UCB Variance: An almost optimal algorithm in SMAB setting}
\label{chap:EUCBV}

\section{Introduction}
\label{Chapter3:intro}
\section{Introduction}
\label{sec:introduction}

\noindent Dennard's law~\cite{dennard:74} states that the power density would remain constant with technology scaling even with the increase in MOS-device density.  With modern deep nanometer devices, optimization of power consumption has become more complex than optimization for delay or area. This is primarily due to the dominance of leakage power in the sub-100$n$$m$ technology. Techniques like supply voltage scaling and threshold voltage scaling have been used in past to reduce dynamic power consumption while maintaining/improving timing of critical path. However, in sub-100$n$$m$ regime, the exponential increase in sub-threshold leakage due to  threshold voltage scaling has caused leakage power to dominate in microprocessors~\cite{borkar:99,kim:03}. It is also to be noted that leakage power dissipation during idle state does not contribute to any useful computation. In addition, excessive leakage power dissipation can also cause power wastage leading to a  potential {\em thermal runaway}\cite{virat}. This has led to an extensive research on leakage power optimizations at different levels of the VLSI design flow under aggressive timing constraints. 

\noindent Gate-sizing and threshold voltage ($V_t$)-sizing are two efficient techniques that are employed at the design stage for reduction of leakage power under timing constraints. In particular, gate-sizing has been very effective in the early and middle stages of the physical synthesis flow.
% * <sristisravan@gmail.com> 2017-06-28T07:45:15.739Z:
% 
% Cite some papers for these statements
% 
% ^.
However, in the post-route stage, gate sizing often necessitates 
incremental placement, which can result in increase of turn-around-time of chip production~\cite{feng:09}. On the other hand, $V_t$ sizing provides enough room for significant optimization in power/timing without any effect on placement. Though, increasing the threshold voltage for a gate (say $g$) by  $V_t$ sizing can result in reducing the leakage power, it can significantly increase the delay. Hence, either gate-sizing or $V_t$ sizing or a mix of both is used to make the optimal choice of each gate depending on the design stage that the chip is present\cite{virat}.

\noindent A standard cell library consists of different versions of the same cell, one for each threshold voltage. However, the number of versions for a given cell is finite and it is limited to the discrete values of the threshold voltages specified by the foundry. Thus, it is imperative to identify cells from the standard cell library that have appropriate gate-size/$V_t$ within the set of available values to minimize the overall leakage power of the circuit without violating the timing constraints.
This optimization problem is called the {\em Timing Constrained Discrete Sizing Problem} (TC-DSP). In this paper, we focus on the TC-DSP for leakage power minimization in digital circuits. 

\noindent In addition to the sizing problem, the rapidly shrinking fabrication technology has resulted in an increase in the design optimization effort due to various factors such as multiple process, voltage and temperature corners. This in turn has resulted in an increase in the design effort (design productivity) of chips to match the estimated parameter values at the design phase to what is desirable in the  post fabrication phase. Fast convergence to high quality solutions is thereby required so as to increase design productivity and meet the consumer demand. Practically, the problem of increasing design productivity becomes important during the optimization phase of the VLSI design; as the execution time of this phase spans over hours to days. %For example, reducing optimization time for a given parameter from 2 minutes to 1 second gives more than 2 orders of magnitude speed-up. But from a practical point of view, this reduction is not valuable as the actual reduction is only 119 seconds, which is comparatively very small in the overall design time, that spans several days or months. On the other hand, reducing optimization time from 5 days to 1 hour also gives a $2 \times$ speed-up as explained in the previous case. But in this case, the actual reduction in running time is 4 days and 23 hours, which can significantly improve the design productivity. 
% * <sristisravan@gmail.com> 2017-06-28T07:43:33.360Z:
% 
% 119 seconds
% 
% ^ <slpskp@cse.iitm.ac.in> 2017-06-28T14:34:53.507Z.

\indent ITRS 2011 \cite{itrs:2011} highlights the intensity of research carried out on low power design technology improvements. However, the designers are not able to effectively leverage all the optimization choices due to runtime constraints. For example, consider a design with three cells and a standard cell library with three $V_t$ and ten gate-size choices. If we just consider $V_t$ scaling for power optimization, the search space is $3^3$. Further, if the designer uses gate-sizing, the search space increases to $3^{10}$. Finally, if a mixed sizing technique is employed, the search space increases to $3^{30}$. Thus we can see that, design productivity heavily suffers due to a large exploration space available at each phase of optimization. The authors in  \cite{kahngtalk} list the following three hypothetical steps to improve design productivity without compromising power goals namely:  \begin{itemize}
    \item  \textbf{Optimized back-end:} mixed height library floorplan/placement, ultimate place/route, power/clock distribution;
    \item  \textbf{Modeling/sign-off criteria:} tightened Back End Of Line (BEOL) corners, reduced guardband and Adaptive Voltage Scaling (AVS) aware sign-off; and,
    \item  \textbf{Bespoke,design-specific-flow:} predictive one-pass flow, optimal tool usage.
 \end{itemize} 
 Consider the following pseudocode which represents the template for any iterative greedy gate-sizing optimization algorithm as seen in \cite{hu:12},\cite{hu:13},\cite{mok:12}, and \cite{reiman:13}. The algorithm takes as inputs: the Netlist $C$; containing $N$ gates, that needs to be optimized, a multi-$V_t$ standard cell library, the target frequency $F$ and a function $\alpha(C)$. The function $\alpha(C)$ takes as inputs the circuit C, the timing values for each logic element got by performing a Static Timing Analysis (STA), and other foundry specific parameters for the different logic elements in C, to compute a cost for every gate in C.
\begin{algorithm}
%\algsetup{linenosize=\tiny}
\scriptsize
\LinesNumbered
\title{Greedy Leakage optimization algorithm}
\caption{The template for an iterative greedy leakage optimization algorithm }
\label{alg:naive}
    \KwIn{1) Netlist of the given circuit $C$ containing $N$ gates represented as a Directed Acyclic Graph (DAG). 2) Target frequency $F$. 3) A multi-$V_t$ standard cell library containing multiple cells with different threshold voltages and sizes for every gate; and, 4) Cost function $\alpha(C)$ that computes a cost for every gate in $C$.}
    \KwOut{A leakage optimized netlist running at the target frequency $F$.}
 $N \leftarrow gate\ count$\; 

    \textit{Initial Configuration:} Assign to each of the $N$ gates a cell from the standard cell library matching its functionality. The $V_t$ and $size$ of each gate shall vary with different methods employing this template\;
Set iteration count to $1$\;
    Run Static Timing Analysis (STA) and compute $\alpha(C)$\; 
    A $\leftarrow$ list of gates sorted in decreasing order according to cost computed by $\alpha(C)$\;
   
    \While{A not empty } {
        \textit{Gate replacement:} Consider the first element of A (the gate g with the largest cost) and replace it with its unexplored slower choices (increase $V_t$ or decrease the $size$) so as to reduce the leakage power. Mark the choice as explored for gate g\;
        Estimate the new power and the new delay of the circuit $C$ by performing STA\;
        \If{delay violated} { 
        \textit{Backtrack procedure:} 
            Undo the gate replacement and delete it from A\;
        }
        \Else {
            Increment iteration count\;
            Compute $\alpha(C)$\;
	    If all the choices for gate g, available in the standard cell library, have been explored delete it from A\;
        }
        Sort the gates in A according to decreasing order of cost computed by $\alpha(C)$\;
    }
\end{algorithm}

\noindent It can be observed that the convergence of the above algorithm crucially depends on the initial configuration, the cost function $\alpha(C)$ which is used to determine the candidate gate for replacement and the number of STA calls.  Our proposed leakage optimization technique (\textit{MLTimer}) aims at resolving these bottlenecks. The main contributions of this work are as follows:
\begin{itemize} %check if ieee uses itemize or enumerate??
\item It has been empirically observed that there exists significant correlation between the timing slacks of gates in the current iteration to the gate replacements in the successive iterations. The outcome of this replacement is the fast convergence of the leakage optimization algorithm in lesser number of iterations;
     \item  It can also be seen that a smart one-pass tool that can leverage the right optimization technique at the appropriate stage of the flow can improve design productivity significantly. A novel Support Vector Machine (SVM) based classifier, which provides a good initial design configuration is used to generate a leakage optimal design at the end;
     \item The \textit{MLTimer} algorithm, which leverages the high iterative correlation between timing slacks in the current iteration and the gate replacements in the successive iterations, thereby enabling  {\em lazy timing analysis} to significantly improve the runtime; and, 
\item An {\em Adaptive Window sizing (AWS) scheme for decision making} to perform {\em multiple  replacements} between successive timing updates.

%\item A lazy adaptive heuristic,  that uses the solution provided by the learning algorithm to provide a leakage optimized solution while meeting the delay constraints.
%\item To the best of our knowledge, this is the first work to use learning to solve the leakage optimization problem.
%\item The solutions produced by our algorithm {\em Learntimer} are at least {\em an order of magnitude} faster while retaining the same solution quality as that of iterative greedy heuristics.
%\item We demonstrate the efficiency of our algorithm on the ISPD2012 benchmarks.
\end{itemize}
To the best of our knowledge, this is the first work that employs ML technique to solve the well known leakage optimization problem. We demonstrate the efficiency of our algorithm on both the ISPD2012 benchmarks and a homegrown RISC-V (SHAKTIC) processor that runs linux~\cite{riscv}. 


% \begin{figure*}[!ht]
%  \begin{center}
%  \includegraphics[scale=0.45]{fig/predict}
% \captionsetup{singlelinecheck=off}
% \caption [The figure shows the impact of the three hypothetical design steps on solution quality and design time. The proposed steps are Optimized back-end: $(mixed height library floorplan/placement, ultimate place/route, power/clock distribution)$ Modeling/sign-off criteria $(tightened BEOL corners, reduced guardband and AVS aware sign-off)$.\#1: Bespoke,design-specific-flow $(predictive one-pass flow, optimal tool usage)$ ]{The figure shows the impact of the three hypothetical design steps on solution quality (QoR) and design time. The proposed steps are 

%  \label{fig:kahng}
%  \end{center}
% \end{figure*}
 
The rest of the manuscript is organized as follows: Section~\ref{sec:background}  presents the problem along with the literature survey. Section \ref{sec:motivation} describes the need for a learning based leakage optimization technique. Section~\ref{sec:proposed} describes our proposed algorithm while the experimental setup is highlighted in Section~\ref{sec:experiment}. The results are presented in ~\ref{sec:results}. Section~\ref{sec:conclusion} concludes the paper.
% * <sristisravan@gmail.com> 2017-06-28T07:46:32.613Z:
% 
% What does STA mean?
% 
% ^.


In this chapter, we look at a novel variant of the UCB algorithm (referred to as Efficient-UCB-Variance (EUCBV)) for minimizing cumulative regret in the stochastic multi-armed bandit (SMAB) setting. EUCBV incorporates the arm elimination strategy proposed in UCB-Improved \citep{auer2010ucb} while taking into account the variance estimates to compute the arms' confidence bounds, similar to UCBV \citep{audibert2009exploration}. Through a theoretical analysis we establish that EUCBV incurs a \emph{gap-dependent} regret bound of {$O\left( \dfrac{K\sigma^2_{\max} \log (T\Delta^2 /K)}{\Delta}\right)$} after $T$ trials, where $\Delta$ is the minimal gap between optimal and sub-optimal arms; the above bound is an improvement over that of existing state-of-the-art UCB algorithms (such as UCB1, UCB-Improved, UCBV,  MOSS). Further, EUCBV incurs a \emph{gap-independent} regret bound of {$O\left(\sqrt{KT}\right)$}  which is an improvement over that of UCB1, UCBV and UCB-Improved, while being comparable with that of MOSS and OCUCB. Through an extensive numerical study, we show that EUCBV significantly outperforms the popular UCB variants (like MOSS, OCUCB, etc.) as well as Thompson sampling and Bayes-UCB algorithms. 

    The rest of the chapter is organized as follows. We elaborate our contributions in Section~\ref{sec:contri} and in Section~\ref{sec:eucbv} we present the  EUCBV algorithm. Our main theoretical results are stated in Section~\ref{sec:results}, while the proofs are established in Section~\ref{sec:proofTheorem}. Section~\ref{sec:expt} contains results and discussions from our numerical experiments and finally we summarize in Section \ref{sec:conc}.
    %and Appendix \ref{sec:app:EUCBV} contains the proofs of the lemmas that have been used for proving the main result.
    
%scriptsize


\section{Our Contributions}
\label{sec:contri}
The main contributions of the thesis are as follows:-
\begin{enumerate}
\item We propose the MLTimer alogrithm that uses gate-sizing for reducing the leakage power consumption of a digital design. We propose a smart one-pass tool that can leverage the right optimization technique at the appropriate stage of the flow thereby improving design productivity. A key observation reported in MLTimer is that there exists significant correlation between the timing slacks of gates in the current iteration to the gate replacements in successive iterations. MLTimer leverages this observation to reduce the number of STA runs thereby reducing the overall time taken for optimization.

\item We propose the Karna algorithm which uses gate-sizing for reducing the information leakage via the power side-channel of a digital design. We show that each region in a given design leaks information differently. Thus, it is sufficient to optimize gates in the highly sensitive regions to reduce information leakage. Karna leverages this observation and optimizes gates in these sensitive regions to reduce the power side-channel vulnerability. 

%\item We proposed a general framework of bandit algorithms that combines change-point detection algorithm with aggregation of expert strategies in order to define efficient pulling strategies in the context of the piecewise stochastic distributions. The algorithms that we proposed for the piecewise stochastic setting are actively adaptive algorithms which perform very similarly to the oracle algorithm which has access to the changepoints and suffers no additional delay in adapting to the changing environment. 
\end{enumerate}
 

\section{Algorithm: Efficient UCB Variance}
\label{sec:eucbv}

\begin{algorithm}[!h]
\caption{EUCBV}
\label{alg:eucbv}
\begin{algorithmic}
\State {\bf Input:} Time horizon $T$, exploration parameters $\rho$ and $\psi$.
\State {\bf Initialization:} Set $m:=0$, $B_{0}:=\mathcal{A}$, $\epsilon_{0}:=1$, $M=\big \lfloor \frac{1}{2}\log_{2} \frac{T}{e}\big\rfloor$, $n_{0}=\big\lceil\frac{\log{(\psi T\epsilon_{0}^{2})}}{2\epsilon_{0}}\big\rceil$ and  $N_{0}=Kn_{0}$.
\State Pull each arm once
\For{$t=K+1,..,T$}	
\State Pull arm $i\in \argmax_{j\in B_{m}}\bigg\lbrace \hat{r}_{j} + \sqrt{\frac{\rho(\hat{v}_{j}+2)\log{(\psi T\epsilon_{m})}}{4 z_{j}}} \bigg\rbrace$, where $z_j$ is the number of times arm $j$ has been pulled.
%\State $t:=t+1$
\ArmElim
\State For each arm $i \in B_{m}$, remove arm $i$ from $B_{m}$ if,
\begin{align*}
%%%%%%%%%%%%%%%%%%%%%%%
%& \hat{r}_{i} + \sqrt{\frac{\rho\hat{v}_{i}\log{(\psi T\epsilon_{m})}}{4 z_{i}} + \frac{\rho\log{(\psi T\epsilon_{m})}}{4 z_{i}}} < \max_{{j}\in B_{m}}\bigg\lbrace\hat{r}_{j} -\sqrt{\frac{\rho\hat{v}_{j}\log{(\psi T\epsilon_{m})}}{4 z_{j}} + \frac{\rho\log{(\psi T\epsilon_{m})}}{4 z_{j}}} \bigg\rbrace
%%%%%%%%%%%%%%%%%%%%%%%
 \hat{r}_{i} + & \sqrt{\frac{\rho(\hat{v}_{i}+2)\log{(\psi T\epsilon_{m})}}{4 z_{i}}}  
  < \max_{{j}\in B_{m}}\bigg\lbrace\hat{r}_{j} -\sqrt{\frac{\rho(\hat{v}_{j}+2)\log{(\psi T\epsilon_{m})}}{4 z_{j}}} \bigg\rbrace
\end{align*}
\EndArmElim

\If{$t\geq N_{m}$ and $m\leq M$}
\ResParam
\State $\epsilon_{m+1}:=\frac{\epsilon_{m}}{2}$\vspace{0.5ex}
\State $B_{m+1}:=B_{m}$
\State $n_{m+1}:=\bigg\lceil\frac{\log{(\psi T\epsilon_{m+1}^{2})}}{2\epsilon_{m+1}}\bigg\rceil$
\State $N_{m+1}:=t+|B_{m+1}| n_{m+1}$
\State $m:=m+1$
\EndResParam
\EndIf
\State Stop if $|B_{m}|=1$ and pull ${i}\in B_{m}$ till $T$ is reached.
\EndFor
\end{algorithmic}
%\vspace*{-0.42em}
\end{algorithm}
%\vspace*{-0.42em}

\textbf{The algorithm:} Earlier round-based arm elimination algorithms like Median Elimination \citep{even2006action} and UCB-Improved mainly suffered from two basic problems: \\
\begin{inparaenum}[\bfseries(i)]
\item \textit{Initial exploration:} Both of these algorithms pull each arm equal number of times in each round, and hence waste a significant number of pulls in initial explorations. \\
\item \textit{Conservative arm-elimination:} In UCB-Improved, arms are eliminated conservatively, i.e, only after $\epsilon_{m}<\frac{\Delta_{i}}{2}$, 
% the sub-optimal arm $i$ is discarded with high probability. 
where the quantity $\epsilon_{m}$ is initialized to $1$ and halved after every round. In the worst case scenario when $K$ is large, and the gaps are uniform  ($r_{1}=r_{2}=\cdots=r_{K-1}<r^{*}$) and small this results in very high regret.\\
\end{inparaenum}
%For any round $m$ UCB-Improved pulls all arms $n_{m}=\left\lceil \frac{ 2\log(T\epsilon_{m})}{\epsilon_{m}} \right\rceil$ number of times. The quantity $\epsilon_{m}$ is initialized to $1$ and halved after every round.
\\
	The EUCBV algorithm, which is mainly based on the arm elimination technique of the UCB-Improved algorithm,  remedies these by employing exploration regulatory factor $\psi$ and arm elimination parameter $\rho$ for aggressive elimination of sub-optimal arms. Along with these, similar to CCB \citep{liu2016modification} algorithm, EUCBV uses optimistic greedy sampling whereby at every timestep it only pulls the arm with the highest upper confidence bound rather than pulling all the arms equal number of times in each round. Also, unlike the UCB-Improved, UCB1, MOSS and OCUCB algorithms (which are based on mean estimation) EUCBV employs mean and variance estimates (as in \citet{audibert2009exploration}) for arm elimination. Further, we allow for arm-elimination at every time-step, which is in contrast to the earlier work (e.g., \citet{auer2010ucb}; \citet{even2006action}) where the arm elimination takes place only at the end of the respective exploration rounds. 






\section{Main Results} 
\label{sec:results}
\section{Results}
\label{sec:results}
% \begin{table}[!ht]
% \centering
% \caption{My caption}
% \label{my-label}
% \begin{tabular}{|p{1.2cm}|l|l|l|l|l|l|}
% \hline
% \multirow{2}{*}{\begin{tabular}[c]{@{}c@{}}Feature \\ Name\end{tabular}} & \multicolumn{2}{c}{Vt Classifier}                                       & \multicolumn{4}{|c|}{Size classifier}                                                                                                   \\ \cline{2-7} 
%                                                                          & \multicolumn{1}{c|}{1} & \multicolumn{1}{c|}{2} & \multicolumn{1}{c|}{ 1} & \multicolumn{1}{c|}{ 2} & \multicolumn{1}{c|}{3} & \multicolumn{1}{c|}{4} \\ \hline
% Sub-circuit                                                              & \multicolumn{1}{l|}{}        &                               &                               &                               &                              &                              \\ \hline
% Gate Type                                                                & \multicolumn{1}{l|}{}        &                               &                               &                               &                              &                              \\ \hline
% LNS                                                  & \multicolumn{1}{l|}{}        &                               &                               &                               &                              &                              \\ \hline
% \#Fanins                                                                 & \multicolumn{1}{l|}{}        &                               &                               &                               &                              &                              \\ \hline
% \#Fanouts                                                                & \multicolumn{1}{l|}{}        &                               &                               &                               &                              &                              \\ \hline
% \begin{tabular}[c]{@{}l@{}}\#Negative \\ Slack Paths\end{tabular}                                          & \multicolumn{1}{l|}{}        &                               &                               &                               &                              &                              \\ \hline
% Slack                                                                    & \multicolumn{1}{l|}{}        &                               &                               &                               &                              &                              \\ \hline
% \end{tabular}
% \end{table}
% \begin{table*}[!ht]
% \centering
% \caption{result3}
% \label{results3}
% \begin{tabular}{|l|c|l|l|l|l|l|l|l|l|l|l|}
% \hline
% \multicolumn{1}{|c|}{\begin{tabular}[c]{@{}c@{}}Benchmark \\  Name\end{tabular}} & \begin{tabular}[c]{@{}c@{}}Number \\ Of gates\end{tabular} & \multicolumn{1}{c|}{\begin{tabular}[c]{@{}c@{}}Target \\ Delay\end{tabular}} & \begin{tabular}[c]{@{}l@{}}Inital \\ Delay\end{tabular} & \multicolumn{2}{c|}{SVM}                                         & \multicolumn{2}{c|}{Final}                                      & \multicolumn{2}{c|}{Igor Markov}                               & \multicolumn{2}{c|}{Flach}                                     \\ \hline
% \multicolumn{1}{|c|}{}                                                           &                                                            & \multicolumn{1}{c|}{}                                                        & \multicolumn{1}{c|}{}                                   & \multicolumn{1}{c|}{Delay (ns)} & \multicolumn{1}{c|}{Power (W)} & \multicolumn{1}{c|}{Delay (ns)} & \multicolumn{1}{c|}{Power(W)} & \multicolumn{1}{c|}{Delay(ns)} & \multicolumn{1}{c|}{Power(W)} & \multicolumn{1}{c|}{Delay(ns)} & \multicolumn{1}{c|}{Power(W)} \\ \hline
% DMA\_fast                                                                        & 25.3K                                                      &                                                                              &                                                         &                                 &                                &                                 &                               &                                &                                                0.299    &       &                               \\ \hline
% DMA\_slow                                                                        & 25.3K                                                      &                                                                              &                                                         &                                 &                                &                                 &                               &                                &                                                       0.145   &     &                               \\ \hline
% pci\_fast                                                              & 33.2K                                                      &                                                                              &                                                         &                                 &                                &                                 &                               &                                &                                                     0.183     &     &                               \\ \hline
% pci\_slow                                                              & 33.2K                                                      &                                                                              &                                                         &                                 &                                &                                 &                               &                                &                                                  0.111         &    &                               \\ \hline
% des\_perf\_fast                                                                  & 111K                                                       &                                                                              &                                                         &                                 &                                &                                 &                               &                                &                                                         1.842   &   &                               \\ \hline
% des\_perf\_slow                                                                  & 111K                                                       &                                                                              &                                                         &                                 &                                &                                 &                               &                                &                                                          0.614   &  &                               \\ \hline
% vga\_lcd\_fast                                                                   & 165K                                                       &                                                                              &                                                         &                                 &                                &                                 &                               &                                &                                                            0.471  & &                               \\ \hline
% vga\_lcd\_slow                                                                   & 165K                                                       &                                                                              &                                                         &                                 &                                &                                 &                               &                                &                                                            0.351  & &                               \\ \hline
% b19\_fast                                                                        & 219K                                                       &                                                                              &                                                         &                                 &                                &                                 &                               &                                &                                                           0.771   & &                               \\ \hline
% b19\_slow                                                                        & 219K                                                       &                                                                              &                                                         &                                 &                                &                                 &                               &                                &                                                             0.583 & &                               \\ \hline
% leon3mp\_fast                                                                    & 649K                                                       &                                                                              &                                                         &                                 &                                &                                 &                               &                                &                                                             1.487 & &                               \\ \hline
% leon3mp\_slow                                                                    & 649K                                                       &                                                                              &                                                         &                                 &                                &                                 &                               &                                &                                                            1.341  & &                               \\ \hline
% netcard\_fast                                                                    & 959K                                                       &                                                                              &                                                         &                                 &                                &                                 &                               &                                &                                                            1.861  & &                               \\ \hline
% netcard\_slow                                                                    & 959K                                                       &                                                                              &                                                         &                                 &                                &                                 &                               &                                &                                                           1.770  &  &                               \\ \hline
% \end{tabular}
% \end{table*}


% Please add the following required packages to your document preamble:
% \usepackage{multirow}
% Please add the following required packages to your document preamble:
% \usepackage{multirow}
% \begin{table}[]
% \centering
% \caption{My caption}
% \label{my-label}
% \begin{tabular}{|l|l|l|l|l|l|}
% \hline
% \multirow{3}{*}{Benchmark} & \multicolumn{5}{c|}{Runtime}                                                            \\ \cline{2-6} 
%                            & \multirow{2}{*}{Igor Markov} & \multirow{2}{*}{Flach} & \multicolumn{3}{c|}{\textit{MLTimer}} \\ \cline{4-6} 
%                            &                              &                        & SVM  & Delay Recovery  & Total  \\ \hline
% DMA\_fast                  &                              &                        &      &                 &        \\ \hline
% DMA\_slow                  &                              &                        &      &                 &        \\ \hline
% pci\_fast        &                              &                        &      &                 &        \\ \hline
% pci\_brdige32\_slow        &                              &                        &      &                 &        \\ \hline
% vga\_lcd\_fast             &                              &                        &      &                 &        \\ \hline
% vga\_lcd\_slow             &                              &                        &      &                 &        \\ \hline
% des\_perf\_fast            &                              &                        &      &                 &        \\ \hline
% des\_perf\_slow            &                              &                        &      &                 &        \\ \hline
% b19\_fast                  &                              &                        &      &                 &        \\ \hline
% b19\_slow                  &                              &                        &      &                 &        \\ \hline
% leon3mp\_fast              &                              &                        &      &                 &        \\ \hline
% leon3mp\_slow              &                              &                        &      &                 &        \\ \hline
% netcard\_fast              &                              &                        &      &                 &        \\ \hline
% netcard\_slow              &                              &                        &      &                 &        \\ \hline
% \end{tabular}
% \end{table}


% Please add the following required packages to your document preamble:
% \usepackage{multirow}
% Please add the following required packages to your document preamble:
% \usepackage{multirow}
% Please add the following required packages to your document preamble:
% \usepackage{multirow}
\begin{table*}[!t]
\caption{Leakage power and Runtime comparisons between the baseline greedy algorithm and the \textit{MLTimer} algorithm on the ISPD 2012 benchmarks. Implementation 1 is the baseline implementation(non-SVM,non-adaptive timing analysis), Implementation 2 is with SVM and non-adaptive timing analysis, Implementation 3 is with non-SVM and adaptive timing analysis and Implementation 4 is with SVM and adaptive timing analysis. It can be seen that using just SVM improves the solution quality greatly, while using just the adaptive timing analysis improves the runtime. A combination of both improves the runtime and solution qualtiy.}
\label{tab:tab5}

\begin{tabular}{|l|l|l|l|l|l|l|l|l|l|}
\hline
\multirow{2}{*}{Benchmarks} & \multirow{2}{*}{\#Gates} & \multicolumn{2}{l|}{Implementation 1}                                                                                                         & \multicolumn{2}{l|}{Implementation 2}                                                                                                           & \multicolumn{2}{l|}{Implementation 3}                                                                                                        & \multicolumn{2}{l|}{Implementation 4}                                                                                                        \\ \cline{3-10} 
                            &                          & \begin{tabular}[c]{@{}l@{}}Run-\\ time \\ (mins)\end{tabular} & \begin{tabular}[c]{@{}l@{}}Leakage \\ Power\\ (W)\end{tabular} & \begin{tabular}[c]{@{}l@{}}Run-\\ time\\ (mins)\end{tabular} & \begin{tabular}[c]{@{}l@{}}Leakage \\ Power\\ \\ (W)\end{tabular} & \begin{tabular}[c]{@{}l@{}}Run-\\ time\\ (mins)\end{tabular} & \begin{tabular}[c]{@{}l@{}}Leakage\\  Power\\ (W)\end{tabular} & \begin{tabular}[c]{@{}l@{}}Run-\\ time\\ (mins)\end{tabular} & \begin{tabular}[c]{@{}l@{}}Leakage \\ Power\\ (W)\end{tabular} \\ \hline
DMA\_fast                   & 23,000                   & 16                                                            & 0.79                                                           & 14.00                                                        & 0.30                                                              & 14.00                                                        & 0.79                                                           & 13.00                                                        & 0.30                                                           \\ \hline
pci\_bridge32\_fast         & 30,000                   & 37                                                            & 0.25                                                           & 17.00                                                        & 0.14                                                              & 17.00                                                        & 0.24                                                           & 17.00                                                        & 0.14                                                           \\ \hline
des\_perf\_fast             & 102,000                  & 219                                                           & 1.73                                                           & 164.00                                                       & 1.80                                                              & 190.00                                                       & 1.73                                                           & 130.00                                                       & 1.80                                                           \\ \hline
vga\_lcd\_fast              & 148,000                  & 384                                                           & 2.80                                                           & 139.00                                                       & 0.47                                                              & 207.00                                                       & 2.72                                                           & 77.00                                                        & 0.47                                                           \\ \hline
b19\_fast                   & 213,000                  & 547                                                           & 2.13                                                           & 239.00                                                       & 0.75                                                              & 366.00                                                       & 2.13                                                           & 174.00                                                       & 0.75                                                           \\ \hline
leon3mp\_fast               & 540,000                  & 2,046                                                         & 4.00                                                           & 875.00                                                       & 1.49                                                              & 716.00                                                       & 4.00                                                           & 639.00                                                       & 1.49                                                           \\ \hline
netcard\_fast               & 861,000                  & 1,033                                                         & 2.09                                                           & 519.00                                                       & 1.77                                                              & 609.00                                                       & 2.07                                                           & 306.00                                                       & 1.77                                                           \\ \hline
\end{tabular}
\end{table*}

\begin{table*}[!ht]
%\centering
\caption{Leakage power comparisons with ISPD 2012 contest winners and other state of the art works. We use geometric mean to calculate the efficiency of our proposed solution. We exclude the infeasible solutions in our mean calculation. All the solutions reported below have no timing violations.}
\label{tab:tab6}

    \begin{tabular}{|l|l|l|p{1.2cm}|p{1.6cm}|p{1.6cm}|p{1cm}|l|p{1.2cm}|}
\hline
\multirow{2}{*}{Benchmark} & \multirow{2}{*}{\begin{tabular}[c]{@{}l@{}}Number \\ of gates\end{tabular}} & \multicolumn{5}{c|}{Leakage Power (W)} & \multicolumn{2}{c|}{Runtime (mins)}\\ \cline{3-9} 
    &  & \cite{hu:12}  & NTUgs & UFRGSgs & Powervalve & \textbf{Ours} & \cite{hu:12} & \textbf{Ours}\\ \hline
    \texttt{DMA\_fast} & 23,000 & 0.30  & 0.51 & 0.32 & 0.31 & 0.30 & 13.90 & 13.30\\ \hline
    \texttt{DMA\_slow} & 23,000  & 0.15  & 0.21 & 0.16 & 0.15 & 0.14 & 9.90 & 7.51 \\ \hline
    \texttt{pci\_fast} & 30,000 & 0.18  & 0.51 & 0.17 & 0.23 & 0.14 & 13.00 & 17.10
     \\ \hline
    \texttt{pci\_slow} & 30,000 & 0.11   & 0.20 & 0.12 & 0.12 & 0.09 & 10.20 & 9.32 \\ \hline
    \texttt{des\_perf\_fast} & 102,000 & 1.84 & 2.39 & 3.52 & 2.32 & 1.80  & 82.70 & 130.40 \\ \hline
    \texttt{des\_perf\_slow} & 102,000 & 0.61 & 0.67 & 0.88 & 0.70 & 0.64 & 70.10 & 43.50 \\ \hline
    \texttt{vga\_lcd\_fast} & 148,000 & 0.47 & 0.76 & 0.58 & 0.77 & 0.47 & 45.60 & 77.32\\ \hline
    \texttt{vga\_lcd\_slow} & 148,000 & 0.35 & 0.42 & 0.38 & 0.39 & 0.37 & 87.50 & 50.40 \\ \hline
    \texttt{b19\_fast} & 213,000 & 0.77 & 2.71 & - & 4.49 & 0.75 & 206.50 & 174.11 \\ \hline
    \texttt{b19\_slow} & 213,000 & 0.58 & 0.63 & 0.61 & 0.74 & 0.61 & 213.90 & 102.20\\ \hline
    \texttt{leon3mp\_fast} & 540,000 & 1.49 & -&  - & 4.94 & 1.49 & 1,323.20 & 639.40\\ \hline
    \texttt{leon3mp\_slow} & 540,000 & 1.34 & 1.42 & 1.79 & 2.96 & 1.30 & 1,274.20 & 325.13  \\ \hline
    \texttt{netcard\_fast} & 861,000 & 1.86 & 2.01 & 2.30 & 2.97 & 1.86 & 1,096.90 & 306.57\\ \hline
    \texttt{netcard\_slow} & 861,000 & 1.77 & 1.77 & 1.97 & 1.94 & 1.77 & 299.90 & 164.14\\ \hline
Geometric mean &  & $1.03\times$ & $1.52\times$ & $1.13\times$ & $1.57\times$ &   & $1.44\times$ & \\ \hline
\end{tabular}
\end{table*}

% Please add the following required packages to your document preamble:
% \usepackage{multirow}
\begin{table*}[!ht]
\centering
\caption{Leakage power comparisons with \cite{hu:13} on the  ISPD 2013 contest benchmark. All the solutions reported below are violation free. It can be observed that \texttt{MLTimer} outperforms \cite{hu:13} both with respect to leakage power and runtime on the larger benchmarks. The detailed results for other benchmarks were not reported in \cite{hu:13}.}
\label{tab:tab34}
\begin{tabular}{|l|l|l|l|l|l|}
\hline
\multirow{2}{*}{Benchmark} & \multirow{2}{*}{Gates} & \multicolumn{2}{l|}{\texttt{MLTimer}}                                                                                                  & \multicolumn{2}{l|}{\cite{hu:13}}                                                                                                  \\ \cline{3-6} 
                           &                        & \begin{tabular}[c]{@{}l@{}}Run-\\ time\\ (mins)\end{tabular} & \begin{tabular}[c]{@{}l@{}}Leakage\\ Power\\ (mW)\end{tabular} & \begin{tabular}[c]{@{}l@{}}Run-\\ time\\ (mins)\end{tabular} & \begin{tabular}[c]{@{}l@{}}Leakage \\ Power\\ (mW)\end{tabular} \\ \hline
usb\_phy\_fast             & 510                    & 0.48                                                         & 2.03                                                           & \textbf{0.21}                                                & \textbf{1.56}                                                   \\ \hline
usb\_phy\_slow             & 510                    & \textbf{0.11}                                                & 1.13                                                           & 0.17                                                         & \textbf{1.07}                                                   \\ \hline
pci\_bridge32\_fast        & 28,000                 & 20.83                                                        & 116.87                                                         & \textbf{12.00}                                               & \textbf{101.90}                                                 \\ \hline
pci\_bridge32\_slow        & 28,000                 & 6.78                                                         & 58.91                                                          & \textbf{5.39}                                                & \textbf{58.83}                                                  \\ \hline
fft\_fast                  & 31,000                 & 40.00                                                        & 320.37                                                         & \textbf{32.58}                                               & \textbf{305.29}                                                 \\ \hline
fft\_slow                  & 31,000                 & 25.00                                                        & 96.69                                                          & \textbf{17.40}                                               & \textbf{93.10}                                                  \\ \hline
cordic\_slow               & 42,000                 & \textbf{94.40}                                               & \textbf{397.81}                                                & 98.39                                                        & 511.91                                                          \\ \hline
des\_perf\_slow            & 104,000                & 88.18                                                        & 386.41                                                         & \textbf{62.30}                                               & \textbf{375.80}                                                 \\ \hline
edit\_dist\_fast           & 121,000                & \textbf{163.10}                                              & \textbf{572.12}                                                & 170.60                                                       & 619.30                                                          \\ \hline
edit\_dist\_slow           & 121,000                & \textbf{56.34}                                               & \textbf{423.50}                                                & 107.20                                                       & 465.60                                                          \\ \hline
matrix\_mult\_slow         & 153,000                & \textbf{139.80}                                              & \textbf{482.23}                                                & 212.60                                                       & 499.90                                                          \\ \hline
netcard\_fast              & 884,000                & \textbf{372.70}                                              & \textbf{5,157.93}                                              & 716.80                                                       & 5271.80                                                         \\ \hline
netcard\_slow              & 884,000.               & \textbf{297.12}                                              & \textbf{5,102.25}                                              & 439.60                                                       & 5183.89                                                         \\ \hline
GEOMETRIC MEAN             &                &                                               &                                               & 1.005                                                       & 1.005                                                         \\ \hline

\end{tabular}
\end{table*}

% \begin{table*}[!ht]
% \begin{center}
% \label{results3}
% \begin{tabular}{|p{2.1cm}|p{2cm}|p{2cm}|p{2cm}|p{2cm}|p{1.5cm}|}
% \hline
% Benchmark & $\#$ gates & %\multicolumn{2}{|c|}
% {\textit{MLTimer}} &%\multicolumn{2}{|c|} 
% {Igor Markov} ~\cite{hu:12} & Improvement \\
% \hline
%   &  & Leakage Power (W)      %& Running Time    
%   & Leakage Power (W) & \\% & Running Time \\ 
 
% \hline
% %USB\_PHY &536 &$<$1s &$<$1s & - \\
% \hline
% DMA\_fast & 25.3K & 0.08W %& 17m
% & 0.299W & 73\% \\ %& 13m\\
% \hline
% %DMA\_slow & 25.3 & 0.134W %& 1m44s
% %& 0.145W & 7\%  \\% & 9.9m\\
% %\hline
% pci\_bridge	&	33.2K	&	0.1331W %&	1m29s
% &	0.183W & 27\%\\%	&	13m \\
% \hline
% %pci\_bridge\_slow	&	33.2K	&	0.07W	%&	8m
% %&	0.111W & 36\% \\%	&	11m \\
% %\hline 
% b19	&	219K  &		.58W	%&	9h
% &	0.771W & 24\% \\%	&	206m \\
% \hline
% %b19\_slow 	&	219K &		0.486W%	&	9h	
% %&	0.583W & 19.21\% \\	%&	213m \\
% %\hline
% Des\_perf & 165K & .546W %& 1331m       
% & .471W & \-15.3\% \\%    & 45m \\
% \hline
% netcard & 959k & 1.8W %& 2046m 
% & 1.861W & 6.01 \% \\% & 1096m \\
% \hline
% leon3mp & 649K & 2W %&  2816m 
% & 1.487W & \-34.5\% \\ %& 1323.2 \\
% \hline
% Average & & & & 21.37\% \\ \hline
% %leon3mp\_slow & 649K & 2W &  2816m & 1.487W & 1323.2 \\
% %\hline
% \end{tabular}
% \caption{Leakage and Running Time Comparisons for ISPD benchmarks between \textit{MLTimer} and Igor Markov. In the table, {\bf h}, {\bf m} and {\bf s} stand for hours, minutes and seconds respectively.}


% \end{center}
% \end{table*}
\subsection{Comparisons with state-of-the-art}

The performance of our proposed algorithm is shown in Table~\ref{tab:tab5}. A simple greedy algorithm, implemented for obtaining the final $V_t$ and $size$ values, serves as the baseline algorithm. It can be seen that our \textit{MLTimer} implementation outperforms the baseline algorithm by 46\% in terms of solution quality. It can also be seen that the SVM module improves the solution quality and the adaptive timing analysis module improves the runtime. 

We compare the performance of our algorithm with ~\cite{hu:12} which is the best performing heuristic based algorithm reported so far in the literature. We use the ISPD 2012 benchmark set and SHAKTIC to quantify the performance our algorithm. In comparing with the state-of-the-art techniques we make the following observations:
\begin{itemize}
\item Our solution outperforms the top 3 submissions of the ISPD 2012 contest NTUgs, UFRGSgs and Powervalve by 52\%,13\% and 57\% respectively.
\item Our solution outperforms \cite{hu:12} both in terms of average runtime and solution quality by 44\% and 3\% respectively. Table~\ref{tab:tab6} highlights the performance of \textit{MLTimer} algorithm in terms of runtime and solution quality. This is because as most of the circuits share a large  number of repeating sub-circuits whose value is accurately predicted by the SVM engine and hence these gates do not undergo delay and power recovery algorithm leading to savings in runtime. 
\item It can be seen from Table~\ref{tab:tab34} that our tool outperforms \cite{hu:13} which is an extension of \cite{hu:12}. It can be observed that while \textit{MLTimer} underperforms for the smaller benchmarks, it significantly outperforms \cite{hu:13} on the larger benchmarks. Although the overall improvement in solution quality is around 0.004\%, the improvement in the larger benchmarks is around 53\% for the runtime and 10\% for solution quality.
\item In Table~\ref{tab:tab9} we compare our implementation with a commercial synthesis tool and our implementation of \cite{hu:12}. It can be observed our proposed solution performs significantly better than the commercial tool in terms of leakage power. 
\end{itemize}

\begin{table}[!t]
    \caption{The Table comparing the performance of \textit{MLTimer} versus a commercial synthesis tool on SHAKTIC. We see that the solution quality is 57\% better than that of the tool.}
    \label{tab:tab9}

    \centering
    \begin{tabular}{|l|l|l|l|l|l|l|}
        \hline
        \textbf{Metric}           & \multicolumn{2}{c|}{Commercial Tool}                                                                                     &        &            & \multicolumn{2}{c|}{Percentage Improvement} \\ \hline
                         & \begin{tabular}[c]{@{}l@{}}$LV_t$\\  synthesis\end{tabular} & \begin{tabular}[c]{@{}l@{}}Mixed $V_t$ \\ synthesis\end{tabular} & \cite{hu:12} & \textit{MLTimer} & Tool              & \cite{hu:12}       \\ \hline
                    %         \textbf{Runtime (mins)}    & 38                                                       & 14                                                            & 87     & 64         & -78.12           &  26.44       \\ \hline
                             \textbf{Leakage power (W)} & 5                                                        & 1                                                             & 0.59  & 0.43       & 57        & 27        \\ \hline
    \end{tabular}

\end{table}


\subsection{Analysis of the Learning Module}


\begin{table}[!t]
\caption{Table showing the weights assigned to each feature at each stage of the $V_t$ and $size$ classifiers. An extremely low magnitude implies that the corresponding feature does not contribute significantly to the output and can thus be discarded. However it can be seen that none of the features chosen fall into that category.}
\label{tab:tab7}

    \centering

\begin{tabular}{|l|l|l|l|l|l|l|l|}
\hline
    \multirow{2}{*}{\textbf{Feature}}       & \multicolumn{2}{c|}{$\mathbf{V_t}$} & \multicolumn{5}{c|}{\textbf{size}}             \\ \cline{2-8} 
                               & 1          & 2          & 1     & 2     & 3     & 4     & 5     \\ \hline
    \textbf{Sub-circuit}                    & -0.88      & -0.43      & -0.58 & 1.32  & -0.08 & 0.44  & 0.16  \\ \hline
    \textbf{Gate type}                      & -0.19      & -0.43      & -0.96 & -0.81 & 0.42  & -0.40 & 0.29  \\ \hline
    \textbf{LNS}                            & 1.09       & 0.33       & -0.07 & -0.11 & 0.76  & 0.76  & 0.08  \\ \hline
    \textbf{Number of Fanins}               & 2.64       & 0.04       & 3.27  & -2.38 & -1.10 & -0.34 & -1.26 \\ \hline
    \textbf{Number of Fanouts}              & -2.92      & -0.29      & -4.08 & -0.53 & 1.82  & 0.50  & -1.07 \\ \hline
    \textbf{Number of Negative Slack Paths} & 0.33       & -0.16      & 0.50  & 0.40  & -0.22 & 0.21  & -0.68 \\ \hline
    \textbf{Slack}                          & 1.89       & 0.35       & 1.22  & -3.20 & -1.64 & -1.41 & 0.37  \\ \hline
\end{tabular}

\end{table}


The learning module forms a critical component of our framework as it serves to reduce the runtime by using a simple SVM model that uses seven features.  A complex ML model with large number of redundant features might cause runtime overheads due to i) complex training procedure ii) complicated inference procedure, and; iii) reduced interpretability of the ML model. Hence there is a need to eliminate the redundant features in order to simplify the learning module. Logistic regression was performed to estimate the importance of the chosen features. The Logistic Regression model was initially trained on the set of chosen features and the importance of each feature,  obtained via the coefficient assigned by the model,  is quantified in Table~\ref{tab:tab7}.  It can be observed that none of the feature weights have extremely low value and hence cannot be eliminated.


%As mentioned earlier, an improperly trained learning engine could initialize the netlist to a sub-optimal configuration leading to more delay and power recovery cycles than necessary thereby increasing the runtime overhead. The thresholding function plays an important role in predicting the final choice ($V_t$/$size$) for a given cell. We use the b19\_fast benchmark to show the impact of varying the thresholding function on the solution quality of the SVM engine. We show the impact of the thresholding function in table ~\ref{results4}. We see that as the thresholding function increases the runtime goes up. This is because the number of gates that are marked unsure increases causing more delay and power optimizations. We us class probability to determine the class label ($V_t$/$size$). We use a thresholding value of $0.75$ for both the $V_t$ classifiers while we use a thresholding value of x and y for the $size$ classifiers. 





% \begin{table*}[!t]
% \parbox{.3\linewidth}{
% \begin{center}

% \begin{tabular}{|p{3cm}|p{1.3cm}|}
% \hline
% Metric & Number \\ \hline
% Total gates & 333\\
% \hline
% Combinational gate types &  11 \\ \hline
% Sequential gates & 1 \\ \hline
% $V_t$ choices & 3 \\ \hline
% $size$ choices & 10 \\ \hline
% $V_{cc}$ and $gnd$ cells & 2 \\ \hline
% %leon3mp\_slow & 649K & -6401 &  -6479 & 1W & 47s \\
% %\hline
% \end{tabular}
% \caption{Library statistics}
% \label{tab:lib}
% \end{center}
% %\end{table*}
% }
% \hfill
% \parbox{.6\linewidth}{
% %\begin{table*}[!t]
% \begin{center}

% \begin{tabular}{|p{2cm}|p{1cm}|p{1cm}|p{1.6cm}|p{1.6cm}|p{1.6cm}|}
% \hline
% Benchmark & \#Input & \#Output & \#Comb cell & \#Seq cell & \#Total cell \\
% \hline
% DMA &  683 & 276 & 23109 & 2192 & 25301\\ \hline
% pci & 160 & 201& 29844& 3359& 33203\\ \hline
% des\_perf &  234 & 140 & 102427 & 8802& 111229\\ \hline
% vga\_lcd &  85 & 99 &  147812 & 17079 & 164891\\ \hline
% b19 &  22 & 25 &  212674 & 6594 & 219268\\ \hline
% leon3mp & 254 & 79 & 540352 &  108839 & 649191\\ \hline
% netcard &  1836 & 10 & 860949 & 97831 & 958780\\ \hline

% %leon3mp\_slow & 649K & -6401 &  -6479 & 1W & 47s \\
% %\hline
% \end{tabular}
% \caption{Benchmark statistics}
% \label{tab:benchmark}
% \end{center}
% }
% \end{table*}


% Please add the following required packages to your document preamble:
% \usepackage{booktabs}
% \usepackage{multirow}
% Please add the following required packages to your document preamble:
% \usepackage{multirow}
% Please add the following required packages to your document preamble:
% \usepackage{multirow}
% Please add the following required packages to your document preamble:
% \usepackage{multirow}

% \begin{table*}[!t]
% \parbox{.5\linewidth}{
% \begin{center}

% \label{tab:log}
% \begin{tabular}{|p{2.5cm}|p{3cm}|}
% \hline
% Feature & Weight \\
% \hline
% Gate footprint &-0.8832794232852276 \\ \hline
% Gateid & -0.1914242984939835 \\ \hline
% Local negative slack & 1.090781658200261 \\ \hline
% \#Fanins & 2.642338600894165  \\ \hline
% \#Fanouts & -2.92081747517804  \\ \hline
% \#Negative slack paths & 0.3349856667382776  \\ \hline
% Slack & 1.894864001133556 \\ \hline

% %leon3mp\_slow & 649K & -6401 &  -6479 & 1W & 47s \\
% %\hline
% \end{tabular}
% \caption{Feature Weights for the first $V_t$ classifier }
% \end{center}
% %\end{table*}
% }
% \hfill
% %\begin{table*}[!h]
% \parbox{.5\linewidth}{
% \begin{center}

% \label{tab:log2}
% \begin{tabular}{|p{2.5cm}|p{3cm}|}
% \hline
% Feature & Weight \\
% \hline
% Gate footprint & 0.03119793516364029
% \\ \hline
% Gateid &  0.346427255804566\\ \hline
% Local negative slack & -0.01019723367101055\\ \hline
% \#Fanins & -0.007490175140922838 \\ \hline
% \#Fanouts & -0.2180352793079041 \\ \hline
% \#Negative slack paths & -0.06062286295401311  \\ \hline
% Slack & 0.663389798807628 \\ \hline

% %leon3mp\_slow & 649K & -6401 &  -6479 & 1W & 47s \\
% %\hline
% \end{tabular}
% \caption{Feature Weights for the second stage $V_t$ classifier }
% \end{center}
% }
% \end{table*}

The efficiency of the SVM engine is analyzed in Table~\ref{tab:tab8}. We see that on an average the SVM engine is able to recover a significant amount of power in a short amount of time. However, It can be observed that the solution provided by the SVM engine is not optimal hence  the delay and leakage power recovery steps are used to further optimize the solution provided by the learning step.

 \begin{table*}[!t]
  \caption{Leakage and Running Time Comparisons for ISPD benchmarks and ShaktiC with just SVM. In the table, {\bf h}, {\bf m} and {\bf s} stand for hours, minutes and seconds respectively. It can be seen that with the exception of leon3mp our SVM implementation is able to recover significant delay and power.} \
\label{tab:tab8}

     \begin{center}
 \begin{tabular}{|p{4.2cm}|p{2cm}|p{2.2cm}|p{2cm}|p{2cm}|p{2cm}|}
 \hline
    \textbf{Benchmark} & \textbf{Gate count} & \textbf{Initial Worst Negative Slack (WNS)} & \multicolumn{3}{|c|}{ \textit{MLTimer}}  \\
 \hline
   &   & & WNS (ps) &  Leakage Power (W)      & Running Time          \\
 \hline
     \texttt{ DMA\_fast} & 25,300&  -1485 & -774 &0.09  & 3s \\
 \hline
     \texttt{pci\_bridge32\_fast}	& 33,200& -1881 & -2284	&	0.18   &	3s	 \\
 \hline
     \texttt{des\_perf\_fast} &  102,000 & -669 & -1029 &.316 & 1m        \\
 \hline
     \texttt{vga\_lcd\_fast} & 148,000 & -1254 & -2964 &.29 & 1m          \\
\hline

     \texttt{b19\_fast}	&	219,000 & -2835 & -1738	&	1.6 	&	15s	\\ \hline
     \texttt{leon3mp\_fast} & 649,000 & -6401 & -3913 & 21 &  47s \\
 \hline

     \texttt{netcard\_fast} & 959,000 & -4102 &-3268 & 8 & 1m  \\ \hline
     \texttt{ShaktiC} & 174,756 & -5199 & -1067 & 0.67 & 1m \\ 
 \hline
 \end{tabular}
 \end{center}

 \end{table*}
\section{Conclusion}
\label{sec:conclusion}
Leakage optimization  techniques have been studied extensively for more than a decade.  However, the lack of a robust algorithm that is optimal in terms of both execution time and solution quality motivates research in this area. It is seen that varying window size adaptively according to the status of the timing updates produces faster solutions than for a fixed window size. The proposed \textit{MLTimer} algorithm improves the running-time considerably while still retaining the solution quality of a greedy heuristic. It is observed that for large circuits \textit{MLTimer} with initial configuration provided by SVM performs significantly better than when used with power optimal configuration as initial solution.  Extending the concepts involved in the construction of \textit{MLTimer} to other steps of EDA including placement and routing is an interesting direction for future work.
% * <sristisravan@gmail.com> 2017-06-28T10:13:29.636Z:
% 
% Check "... the lack of a robust heuristic that optimal in terms of both.... "
% 
% ^.

%\begin{table*}[t]
% \begin{center}
% \caption{Leakage and Running Time Comparisons for ISPD benchmarks with just SVM and delay recovery. In the table, {\bf h}, {\bf m} and {\bf s} stand for hours, minutes and seconds respectively.}
% \label{results2}
% \begin{tabular}{|p{1.7cm}|p{2cm}|p{2cm}|p{2cm}|p{2cm}|}
% \hline
% Benchmark & $\#$ gates & \multicolumn{3}{|c|}{\textit{MLTimer}}  \\
% \hline
%   &  & Delay(ps) &  Leakage Power (W)      & Running Time          \\
% \hline
% %USB\_PHY &536 &$<$1s &$<$1s & - \\
% \hline
% DMA_fast & 25.3K & &0.08W & 17m \\
% \hline
% DMA_slow & 25.3 & &0.134W & 1m44s \\
% \hline
% pci_bridge_fast	& 33.2K&	&	0.1331W &	1m29s	 \\
% \hline
% pci_bridge_slow	& 	33.2K&	&	0.07W	&	8m	\\
% \hline 
% b19_fast	&	219K  &	&	.58W	&	9h	\\
% \hline
% b19_slow 	&	219K & &		0.486W	&	9h	 \\
% \hline
% Des_perf_fast &  165K & &.546W & 1331m        \\
% \hline
% Des_perf_slow &  165K & & .546W & 1331m         \\
% \hline
% vga_lcd_slow & 165K & & .546W & 1331m          \\
% \hline
% vga_lcd_slow & 165K &  &.546W & 1331m          \\
% \hline
% netcard_fast & 959k & & 1.8W & 2046m  \\
% \hline
% netcard_slow & 959k & & 1.8W & 2046m \\
% \hline
% leon3mp_fast & 649K & & 2W &  --- \\
% \hline
% leon3mp_slow & 649K & & 2W &  --- \\
% \hline
% \end{tabular}
% \end{center}
%\end{table*}






\section{Proofs}
\label{sec:proofTheorem}
%\subsection{Lemma 1}
%\label{sec:proofTheorem:Lemma1}
We first present a few technical lemmas that is required  to prove the result in Theorem \ref{Result:Theorem:1}.

\begin{lemma}
\label{proofTheorem:Lemma:1}
If $T\geq K^{2.4}$, $\psi=\frac{T}{ K^2}$, $\rho=\frac{1}{2}$ and $m\leq \frac{1}{2} \log_2\left(\frac{T}{e}\right) $, then,
\begin{align*}
\dfrac{\rho m \log(2)}{\log(\psi T) - 2m\log( 2)} \leq \frac{3}{2}.
\end{align*}
\end{lemma}



\begin{lemma}
\label{proofTheorem:Lemma:2}
If $T\geq K^{2.4}$, $\psi=\frac{T}{ K^2}$, $\rho =\frac{1}{2}$, $m_i = min\lbrace m|\sqrt{4\epsilon_{m} } < \frac{\Delta_i}{4} \rbrace $ and $c_{i} =\sqrt{\frac{\rho (\hat{v}_i + 2)\log (\psi T\epsilon_{m_{i}})}{4 z_i}}$, then,
%\begin{align*}
\center $c_{i} < \frac{\Delta_i}{4}$.
%\end{align*}
\end{lemma}



\begin{lemma}
\label{proofTheorem:Lemma:3}
If $m_i = min\lbrace m|\sqrt{4\epsilon_{m} } < \frac{\Delta_i}{4} \rbrace $,  $c_{i} = \sqrt{\frac{\rho (\hat{v}_i + 2) \log (\psi T\epsilon_{m_{i}})}{4 z_{i}}}$ and $n_{m_i} = \frac{\log{(\psi T\epsilon_{m_{i}})}}{2\epsilon_{m_{i}}}$ then we can show that,
\begin{align*}
\mathbb{P}(\hat{r}_{i}> r_{i} + c_{i})\le \dfrac{2}{(\psi  T\epsilon_{m_{i}})^{\frac{3\rho}{2}}}.
\end{align*}
\end{lemma}



%\begin{lemma}
%\label{proofTheorem:Lemma:3}
%If $m_i = min\lbrace m|\sqrt{4\epsilon_{m} } < \frac{\Delta_i}{4} \rbrace $,  $\bar{c}_i=\sqrt{\frac{\rho (\sigma_{i}^{2}+\sqrt{\epsilon_{m_{i}}} + 2)\log(\psi T\epsilon_{m_{i}})}{4z_i}}$ and $n_{m_i} = \frac{\log{(\psi T\epsilon_{m_{i}})}}{2\epsilon_{m_{i}}}$ then we can show that,
%\begin{align*}
%\mathbb{P}\left( \hat{r}_{i} > r_{i}+ \bar{c}_i\right) 
%+ \mathbb{P}\left( \hat{v}_{i}\geq \sigma_{i}^{2}+\sqrt{\epsilon_{m_{i}}}\right) \leq \dfrac{2}{(\psi  T\epsilon_{m_{i}})^{\frac{3\rho}{2}}}.
%\end{align*}
%\end{lemma}



\begin{lemma}
\label{proofTheorem:Lemma:4}
If $m_i = min\lbrace m|\sqrt{4\epsilon_{m} } < \frac{\Delta_i}{4} \rbrace $, $\psi=\frac{T}{ K^2}$, $\rho=\frac{1}{2}$, $c_{i} =\sqrt{\frac{\rho(\hat{v}_i + 2)\log (\psi T\epsilon_{m_{i}})}{4 z_{i}}}$ and $n_{m_i}=\frac{\log{(\psi T\epsilon_{m_{i}}^{2})}}{2\epsilon_{m_{i}}}$ then in the $m_i$-th round, 
\begin{align*}
\Pb\lbrace c^{*} > c_i \rbrace  \leq \dfrac{182 K^4}{T^{\frac{5}{4}}\sqrt{\epsilon_{m_i}}}.
\end{align*}
\end{lemma}



\begin{lemma}
\label{proofTheorem:Lemma:5}
If $m_i = min\lbrace m|\sqrt{4\epsilon_{m} } < \frac{\Delta_i}{4} \rbrace $,$\psi=\frac{T}{ K^2}$, $\rho=\frac{1}{2}$, $c_{i} =\sqrt{\frac{\rho (\hat{v}_i + 2)\log (\psi T\epsilon_{m_{i}})}{4 z_i}}$ and $n_{m_i}=\frac{\log{(\psi T\epsilon_{m_{i}}^{2})}}{2\epsilon_{m_{i}}}$ then in the $m_i$-th round, 
\begin{align*}
\Pb\lbrace z_i < n_{m_i} \rbrace  \leq \dfrac{182 K^4}{T^{\frac{5}{4}}\sqrt{\epsilon_{m_i}}}.
\end{align*}
\end{lemma}



%\begin{lemma}
%\label{proofTheorem:Lemma:6}
%For $T\geq K^{2.4}$, $\epsilon_{m_i}\geq \sqrt{\frac{e}{T}}$, $\psi=\frac{T}{K^2}$ and $\rho=\frac{1}{2}$,  
%\begin{align*}
%\dfrac{6K}{(\psi T \epsilon_{m_i})^{\frac{3\rho}{2}}} > \dfrac{K\log T}{(\psi T)^{3\rho}}\sum_{m=0}^{m_i}\dfrac{1}{\epsilon_{m_i}^{3\rho + 1}}
%\end{align*}
%\end{lemma}



%\begin{lemma}
%\label{proofTheorem:Lemma:6}
%For all bounded rewards in $[0,1]$, $\frac{\Delta_i}{4} \geq \frac{\Delta_i}{4\sigma_i^2 + 4} $.
%\end{lemma}



\begin{lemma}
\label{proofTheorem:Lemma:6}
For two integer constants $c_1$ and $c_2$, if $20 c_1 \leq c_2$ then,
\begin{align*}
c_1 \frac{4\sigma_i^2 + 4}{\Delta_i}\log\bigg( \frac{T\Delta_i^2}{K}\bigg) \leq c_2 \frac{\sigma_i^2}{\Delta_i}\log\bigg( \frac{T\Delta_i^2}{K}\bigg).
\end{align*}
\end{lemma}


%\begin{lemma}
%\label{proofTheorem:Lemma:8}
%If $m_*$ be the first round that the optimal arm $*$ gets eliminated, then we can show that the regret is upper bounded by,
%
%\begin{align*}
%\sum_{m_{*}=0}^{max_{j\in \A^{'}}m_{j}}\sum_{i\in \A^{''}:m_{i}>m_{*}}\bigg(\dfrac{388 K}{(\psi  T\epsilon_{m_{*}})^{\frac{3\rho}{2}}} \bigg).T\max_{j\in \A^{''}:m_{j}\geq m_{*}}{\Delta}_{j} \\
%%%%%%%%%%%%%%%%%%%%%%%%%
% \leq\sum_{i\in \A^{'}}\dfrac{C_2^{'} K^{\frac{5}{2}}}{\sqrt{T\Delta_i}} +\sum_{i\in \A^{''}\setminus \A^{'}}\dfrac{C_2^{'} K^{\frac{5}{2}}}{\sqrt{T b}}
%\end{align*}
%
%\end{lemma}


The proofs of lemmas \ref{proofTheorem:Lemma:1} - \ref{proofTheorem:Lemma:6} can be found in Appendix ~\ref{App:Lemma:1}, ~\ref{App:Lemma:2}, ~\ref{App:Lemma:3}, ~\ref{App:Lemma:4}, ~\ref{App:Lemma:5} and
 ~\ref{App:Lemma:6} respectively.

%The proofs of all the Lemmas can be found in Appendix ~\ref{App:Lemma:1} - Appendix ~\ref{App:Lemma:9} respectively.

\subsection*{Proof of Theorem 1}
\label{sec:proofTheorem:Theorem1}
\begin{customproof}{1}
For each sub-optimal arm ${i}\in\mathcal{A}$, let $m_{i}=\min{\left\lbrace m|\sqrt{4\epsilon_{m_i}} < \frac{\Delta_{i}}{4}\right\rbrace}$. Also, let $\A^{'}=\lbrace i\in \A: \Delta_{i} > b \rbrace$ and $\A^{''}=\lbrace i\in \A: \Delta_{i} > 0 \rbrace$. Note that as all rewards are bounded in $[0,1]$, it implies that $0\leq \sigma_i^2 \leq \frac{1}{4},\forall i\in \A$. Now, as in \citet{auer2010ucb}, we bound the regret under the following two cases: 
\begin{itemize}
\item {Case $(a)$}: some sub-optimal arm ${i}$ is not eliminated in round $m_{i}$ or before and the optimal arm ${*}\in B_{m_{i}}$
\item {Case $(b)$}: an arm ${i}\in B_{m_i}$ is eliminated in round $m_{i}$ (or before), or there is no optimal arm $*\in B_{m_i}$
\end{itemize} 
The details of each case are contained in the following sub-sections.

%Note that in in round $m_i$ as $\sqrt{4\epsilon_{m_i}} < \dfrac{\Delta_{i}}{4}$ implies that $\sqrt{4\epsilon_{m_i}} < \dfrac{\Delta_{i}}{4\sigma_i^2}$, since $\sigma_i^2\in (0,1]$

\textbf{Case $(a)$:}
For simplicity, let $c_{i} := \sqrt{\frac{\rho (\hat{v}_i + 2) \log (\psi T\epsilon_{m_{i}})}{4 z_{i}}}$ denote the length of the confidence interval corresponding to arm $i$ in round $m_i$. Thus, in round $m_i$ (or before) whenever $z_i \geq n_{m_{i}}\ge\frac{\log{(\psi T\epsilon_{m_{i}}^{2})}}{2\epsilon_{m_{i}}}$, by applying Lemma \ref{proofTheorem:Lemma:2} we obtain $c_{i} < \frac{\Delta_{i}}{4}$.
%\begin{align*}
%	c_{i} < \dfrac{\Delta_{i}}{4} 
%\end{align*}
Now, the sufficient conditions for arm $i$ to get eliminated by an optimal arm in round $m_i$ is given by
	\begin{eqnarray}
	\hat{r}_{i} \leq r_{i} + c_{i} \text{, } 
 	\hat{r}^{*} \geq r^{*} - c^{*} \text{, } c_{i} \geq c^* \text{ and } z_i \geq n_{m_i} \label{eq:armelim-casea}.
	\end{eqnarray}

Indeed, in round $m_i$ suppose (\ref{eq:armelim-casea}) holds, then we have
%	 
  \begin{align*}
\hat{r}_{i} + c_{i}&\leq r_{i} + 2c_{i} 
= r_{i} + 4c_{i} - 2c_{i} \\
 &< r_{i} + \Delta_{i} - 2c_{i}
 \leq r^{*} -2c^{*} 
 \leq \hat{r}^{*} - c^{*}
  \end{align*}
  so that a sub-optimal arm ${i} \in \A^{'}$ gets eliminated.	
Thus, the probability of the complementary event of these four conditions in (\ref{eq:armelim-casea}) yields a bound on the probability that arm $i$ is not eliminated in round $m_i$. Following the proof of Lemma 1 of \citet{audibert2009exploration} we can show that a bound on the complementary of the first condition is given by,

\begin{align}
\mathbb{P}(\hat{r}_{i}> r_{i} + c_{i})
&\leq \mathbb{P}\left( \hat{r}_{i} > r_{i}+ \bar{c}_i\right) 
+ \mathbb{P}\left( \hat{v}_{i}\geq \sigma_{i}^{2}+\sqrt{\epsilon_{m_{i}}}\right)\label{eq:prob_eq2}
\end{align}
where 
\begin{align*}
\bar{c}_i=\sqrt{\dfrac{\rho (\sigma_{i}^{2}+\sqrt{\epsilon_{m_{i}}} + 2)\log(\psi T\epsilon_{m_{i}})}{4n_{m_i}}}.
\end{align*}

%%%%%%%%%%%%%%%%%%%%%%%%%%%%%%%%%%
% Shifted as Lemma
%%%%%%%%%%%%%%%%%%%%%%%%%%%%%%%%%%
%Note that, substituting $ n_{m_i} \geq \frac{\log{(\psi T\epsilon_{m_{i}})}}{2\epsilon_{m_{i}}}$, $\bar{c}_i$ can be simplified to obtain,
%\begin{align}
%\bar{c}_i
%\leq \sqrt{\dfrac{\rho\epsilon_{m_{i}}(\sigma_{i}^{2}+\sqrt{\epsilon_{m_{i}}} + 2)}{2}}\leq \sqrt{ \epsilon_{m_{i}}}.
%\label{si_bar_equn}
%\end{align}
%%
%The first term in the LHS of (\ref{eq:prob_eq2}) can be bounded using the Bernstein inequality as below:
%\begin{align}
%&\mathbb{P}\left( \hat{r}_{i} > r_{i}+ \bar{c}_i\right)\nonumber 
%\le \exp\left(- \dfrac{(\bar{c}_i)^2 z_{i}}{2\sigma_i^2 + \frac{2}{3}\bar{c}_i} \right)\nonumber 
%%%%%%%%%%%%%%%%
%\\
%& \overset{(a)}{\le} \exp\left(- \rho \left(\dfrac{3\sigma_{i}^{2}+3\sqrt{\epsilon_{m_{i}}} + 6}{6\sigma_i^2 + 2\sqrt{\epsilon_{m_i}}} \right)\log(\psi  T\epsilon_{m_{i}}\right)\nonumber \\
%%%%%%%%%%%%%%%%
%% &\le \exp\left(- \rho (\sigma_{i}^{2}+\sqrt{\epsilon_{m_{i}}} + 2)\log(\psi  T\epsilon_{m_{i}})\right)\nonumber \\
%%%%%%%%%%%%%%%%
%& \overset{(b)}{\leq} \exp\left(- \rho \log(\psi  T\epsilon_{m_{i}})\right) 
%%%%%%%%%%%%%%%%
%\le \dfrac{1}{(\psi  T\epsilon_{m_{i}})^{\rho}}
%\label{lhs1_equn}
%\end{align}
%where, $(a)$ is obtained by substituting equation \ref{si_bar_equn} and $(b)$ occurs because for all $\sigma_{i}^2 \in [0,1]$, $\left(\dfrac{3\sigma_{i}^{2}+3\sqrt{\epsilon_{m_{i}}} + 6}{6\sigma_i^2 + 2\sqrt{\epsilon_{m_i}}}\right) \geq 1$ .
%
% 
%The second term in the LHS of (\ref{eq:prob_eq2}) can be simplified as follows:
%\begin{align}
%&\mathbb{P}\bigg\lbrace \hat{v}_{i}\geq \sigma_{i}^{2}+\sqrt{\epsilon_{m_{i}}}\bigg\rbrace\nonumber\\
%%%%%%%%%%%%%%%%%%%
%&\leq \mathbb{P}\bigg\lbrace \dfrac{1}{n_{i}}\sum_{t=1}^{n_{i}}(X_{i,t}-r_{i})^{2}-(\hat{r}_{i}-r_{i})^{2}\geq \sigma_{i}^{2}+\sqrt{\epsilon_{m_{i}}}\bigg\rbrace\nonumber\\
%%%%%%%%%%%%%%%%%%%
%&\leq \mathbb{P}\bigg\lbrace \dfrac{\sum_{t=1}^{n_{i}}(X_{i,t}-r_{i})^{2}}{n_{i}}\geq \sigma_{i}^{2}+\sqrt{\epsilon_{m_{i}}} \bigg\rbrace\nonumber\\
%%%%%%%%%%%%%%%%%%%
%&\overset{(a)}{\leq} \mathbb{P}\bigg\lbrace \dfrac{\sum_{t=1}^{n_{i}}(X_{i,t}-r_{i})^{2}}{n_{i}}\geq \sigma_{i}^{2} + \bar{c}_i\bigg\rbrace \nonumber\\
%%%%%%%%%%%%%%%%%%%
%&\overset{(b)}{\leq} \exp\left(- \rho \left(\dfrac{3\sigma_{i}^{2}+3\sqrt{\epsilon_{m_{i}}} + 6}{6\sigma_i^2 + 2\sqrt{\epsilon_{m_i}}} \right)\log(\psi  T\epsilon_{m_{i}})\right)
%%%%%%%%%%%%%%%%%%
%\le \dfrac{1}{(\psi  T\epsilon_{m_{i}})^{\rho}}
%\label{lhs2_equn}
%\end{align}
%where inequality $(a)$ is obtained using (\ref{si_bar_equn}), while $(b)$ follows from the Bernstein inequality.
  
%Thus, using (\ref{lhs1_equn}) and (\ref{lhs2_equn}) in (\ref{eq:prob_eq2}) we obtain $\mathbb{P}(\hat{r}_{i}> r_{i} + c_{i})\le \dfrac{2}{(\psi  T\epsilon_{m_{i}})^{\rho}}$. 

From Lemma \ref{proofTheorem:Lemma:3} we can show that $\mathbb{P}(\hat{r}_{i}> r_{i} + c_{i})\leq\mathbb{P}\left( \hat{r}_{i} > r_{i}+ \bar{c}_i\right) + \mathbb{P}\left( \hat{v}_{i}\geq \sigma_{i}^{2}+\sqrt{\epsilon_{m_{i}}}\right) \leq \frac{2}{(\psi  T\epsilon_{m_{i}})^{\frac{3\rho}{2}}}$. Similarly, $\mathbb{P}\lbrace\hat{r}^{*} < r^{*} - c^{*}\rbrace \leq \frac{2}{(\psi  T\epsilon_{m_{i}})^{\frac{3\rho}{2}}}$. Summing the above two contributions, the probability that a sub-optimal arm ${i}$ is not eliminated on or before $m_{i}$-th round by the first two conditions in  (\ref{eq:armelim-casea}) is,  
\begin{eqnarray}
\bigg(\dfrac{4}{(\psi T\epsilon_{m_{i}})^{\frac{3\rho}{2}}} \bigg). \label{eq:arm:elim:c1}
\end{eqnarray}
 

Again, from Lemma \ref{proofTheorem:Lemma:4} and Lemma \ref{proofTheorem:Lemma:5} we can bound the probability of the  complementary of the event $c_{i} \geq c^* $ and $ z_i \geq n_{m_i}$ by,

\begin{eqnarray}
\dfrac{182 K^4}{T^{\frac{5}{4}}\sqrt{\epsilon_{m_i}}} + \dfrac{182 K^4}{T^{\frac{5}{4}}\sqrt{\epsilon_{m_i}}}\leq \dfrac{364 K^4}{T^{\frac{5}{4}}\sqrt{\epsilon_{m_i}}}. \label{eq:arm:elim:c2}
\end{eqnarray}

Also, for eq. $(\ref{eq:arm:elim:c1})$ we can show that for any $\epsilon_{m_i}\in[\sqrt{\frac{e}{T}},1]$
\begin{eqnarray}
\bigg(\dfrac{4}{(\psi T\epsilon_{m_{i}})^{\frac{3\rho}{2}}} \bigg) &\overset{(a)}{\leq} \bigg(\dfrac{4}{(\frac{T^2}{K^2}\epsilon_{m_{i}})^{\frac{3}{4}}} \bigg)\leq \bigg(\dfrac{4 K^{\frac{3}{2}}}{(T^\frac{3}{2} \epsilon_{m_i}^{\frac{1}{4}}\sqrt{\epsilon_{m_{i}}})}\bigg) \nonumber \\
%%%%%%%%%%%%%%%%%%%%%%%
&\overset{(b)}{\leq} \bigg(\dfrac{4 K^{\frac{3}{2}}}{(T^{\frac{3}{2}-\frac{1}{8}}\sqrt{\epsilon_{m_{i}}})}  \bigg)
\leq \dfrac{4 K^4}{T^{\frac{5}{4}}\sqrt{\epsilon_{m_i}}}. \label{eq:arm:elim:c3}
\end{eqnarray}

Here, in $(a)$ we substitute the values of $\psi$ and $\rho$ and $(b)$ follows from the identity $\epsilon_{m_i}^{\frac{1}{4}}\geq (\frac{e}{T})^{\frac{1}{8}} $ as $\epsilon_{m_i}\geq \sqrt{\frac{e}{T}}$.

Summing up over all arms in $\A^{'}$ and bounding the regret for all the \textit{four} arm elimination conditions in (\ref{eq:armelim-casea}) by $(\ref{eq:arm:elim:c2}) + (\ref{eq:arm:elim:c3})$ for each arm $i\in \A^{'}$ trivially by $T\Delta_{i}$, we obtain
	\begin{align*}
&\sum_{i\in \A^{'}}\bigg(\dfrac{4 K^4 T\Delta_i}{T^{\frac{5}{4}}\sqrt{\epsilon_{m_i}}}\bigg) + \sum_{i\in \A^{'}}\bigg(\dfrac{364 K^4 T\Delta_i}{T^{\frac{5}{4}}\sqrt{\epsilon_{m_i}}}\bigg)\\
%%%%%%%%%%%%%%%%%%%%%%%%%%%%%
&\overset{(a)}{\leq}\sum_{i\in \A^{'}}\bigg(\dfrac{368 K^4 T\Delta_{i}}{T^{\frac{5}{4}}\left(\frac{\Delta_{i}^{2}}{4.16}\right)^{\frac{1}{2}}}\bigg)
%%%%%%%%%%%%%%%%%%%%%%%%%%%%%%%
\overset{(b)}{\leq} \sum_{i\in \A^{'}}\bigg(\dfrac{C_1 K^4}{(T)^{\frac{1}{4}}}\bigg).\\  
%%%%%%%%%%%%%%%%%%%%%%%%%%%%%%%
	\end{align*}

%   \begin{align*}
%&\sum_{i\in \A^{'}}\bigg(\dfrac{388 K T\Delta_{i}}{(\psi T\epsilon_{m_{i}})^{\frac{3\rho}{2}}}\bigg)
%\leq\sum_{i\in \A^{'}}\bigg(\dfrac{388 K T\Delta_{i}}{(\psi T\dfrac{\Delta_{i}^{2}}{4.16})^{\frac{3\rho}{2}}}\bigg)\\
%%%%%%%%%%%%%%%%%%%%%%%%%%%%%%%
%&\leq \sum_{i\in \A^{'}}\bigg(\dfrac{388.2^{2+2\frac{3\rho}{2}}.16^{\frac{3\rho}{2}} K T^{1-\frac{3\rho}{2}}}{\psi^{\frac{3\rho}{2}}\Delta_{i}^{2\frac{3\rho}{2} -1}}\bigg)\\  
%%%%%%%%%%%%%%%%%%%%%%%%%%%%%%%
%& \overset{(a)}{\leq} \sum_{i\in \A^{'}}\bigg(\dfrac{388.2^{2+\frac{3}{2}}.16^{\frac{3}{4}} K T^{1-\frac{3}{4}}}{(\frac{T}{K^2})^{\frac{3}{4}}\Delta_{i}^{2.\frac{3}{4} -1}}\bigg)\leq \sum_{i\in \A^{'}}\dfrac{C_1 K^{\frac{5}{2}}}{\sqrt{T\Delta_i}}  
%   \end{align*}
%Here in $(a)$ we substitute the values of $\rho$ and $\psi$ and $C_1$ denotes a constant integer value.\\
Here, $(a)$ happens because $\sqrt{4\epsilon_{m_i}} < \frac{\Delta_i}{4}$, and in $(b)$, $C_1$ denotes a constant integer value.\\


%%%%%%%%%%%%%%%%%%%%%%%%%%%%%%%%%%%%%
% Case (b)
%%%%%%%%%%%%%%%%%%%%%%%%%%%%%%%%%%%%%
\textbf{Case $(b)$:} Here, there are two sub-cases to be considered.
% \subsection*{Case $b$: \textit{An arm ${i}\in B_{m_i}$ is eliminated in round $m_{i}$ or before or there is no $*\in B_{m_i}$}}

\noindent
\textbf{Case $(b1)$ (\textit{${*}\in B_{m_{i}}$ and each ${i}\in \A^{'}$ is  eliminated on or before $m_{i}$ }): } Since we are eliminating a sub-optimal arm ${i}$ on or before round $m_{i}$, it is pulled no longer than, 
 \begin{align*}
 z_{i} < \bigg\lceil\dfrac{\log{(\psi T\epsilon_{m_{i}}^{2})}}{2\epsilon_{m_{i}}}\bigg\rceil
 \end{align*}
%\hspace*{4em}
%%$, since $\sqrt{\rho_{a}\epsilon_{m_{i}}}\leq\dfrac{\Delta_{i}}{2}
So, the total contribution of ${i}$  till round $m_{i}$ is given by, 
\begin{align*}
&\Delta_{i}\bigg\lceil\dfrac{\log{(\psi T\epsilon_{m_{i}}^{2})}}{2\epsilon_{m_{i}}}\bigg\rceil
\overset{(a)}{\leq}    \Delta_{i}\bigg\lceil\dfrac{\log{(\psi T(\dfrac{\Delta_{i}}{16 \times 256})^{4})}}{2(\dfrac{\Delta_{i}}{4\sqrt{4}})^{2}}\bigg\rceil \\
%%%%%%%%%%%%%%%%%%%%%%%%%%%%%%
&\leq   \Delta_{i}\bigg(1+\dfrac{32\log{(\psi T(\dfrac{\Delta_{i}^{4}}{16384})}}{\Delta_{i}^{2}}\bigg)
\leq \Delta_{i}\bigg(1+\dfrac{32\log{(\psi T\Delta_{i}^{4})}}{\Delta_{i}^{2}}\bigg) .
\end{align*} 

Here, $(a)$ happens because $\sqrt{4\epsilon_{m_{i}}} < \frac{\Delta_{i}}{4}$. Summing over all arms in $\A^{'}$ the total regret is given by, 
\begin{align*}
&\sum_{i\in \A^{'}}\Delta_{i}\bigg(1+\dfrac{32\log{(\psi T\Delta_{i}^{4}})}{\Delta_{i}^{2}}\bigg) = \sum_{i\in \A^{'}}\bigg(\Delta_{i} +\dfrac{32\log{(\psi T\Delta_{i}^{4}})}{\Delta_{i}}\bigg) \\
%%%%%%%%%%%%%%%%%%%%%%%%%%%
&\overset{(a)}{\leq} \sum_{i\in \A^{'}} \left(\Delta_{i}+\dfrac{64\log{( \frac{T\Delta_{i}^{2}}{K})}}{\Delta_{i}}\right)\\
%%%%%%%%%%%%%%%%%%%%%%%%%%%
&\overset{(b)}{\leq} \sum_{i\in \A^{'}} \left(\Delta_{i} +\dfrac{16(4\sigma_i^2 + 4)\log{( \frac{T\Delta_{i}^{2}}{K})}}{\Delta_{i}}\right)\\
&%%%%%%%%%%%%%%%%%%%%%%%%%%%
\overset{(c)}{\leq} \sum_{i\in \A^{'}} \left(\Delta_{i} +\dfrac{320\sigma_i^2\log{( \frac{T\Delta_{i}^{2}}{K})}}{\Delta_{i}}\right).\\
\end{align*}

We obtain $(a)$ by substituting the value of $\psi$, $(b)$ from $0\leq\sigma_i^2 \leq\frac{1}{4},\forall i\in \A$ and $(c)$ from Lemma \ref{proofTheorem:Lemma:6}.\\

\noindent
\textbf{Case $(b2)$ (\textit{Optimal arm ${*}$ is eliminated by a sub-optimal arm):  }} Firstly, if conditions of Case $a$ holds then the optimal arm ${*}$ will not be eliminated in round $m=m_{*}$ or it will lead to the contradiction that $r_{i}>r^{*}$. In any round $m_{*}$, if the optimal arm ${*}$ gets eliminated then for any round from $1$ to $m_{j}$ all arms ${j}$ such that $m_{j}< m_{*}$ were eliminated according to assumption in Case $a$. Let the arms surviving till $m_{*}$ round be denoted by $\A^{'}$. This leaves any arm $a_{b}$ such that $m_{b}\geq m_{*}$ to still survive and eliminate arm ${*}$ in round $m_{*}$. Let such arms that survive ${*}$ belong to $\A^{''}$. Also maximal regret per step after eliminating ${*}$ is the maximal $\Delta_{j}$ among the remaining arms ${j}$ with $m_{j}\geq m_{*}$.  Let $m_{b}=\min\left\lbrace m|\sqrt{4\epsilon_{m}}<\frac{\Delta_{b}}{4}\right\rbrace$. Hence, the maximal regret after eliminating the arm ${*}$ is upper bounded by, 

\begin{align*}
&\sum_{m_{*}=0}^{max_{j\in \A^{'}}m_{j}}\sum_{i\in \A^{''}:m_{i}>m_{*}}\bigg(\dfrac{368 K^4}{(T^{\frac{5}{4}}\sqrt{\epsilon_{m_{*}}})} \bigg).T\max_{j\in \A^{''}:m_{j}\geq m_{*}}{\Delta}_{j}\\
%%%%%%%%%%%%%%%%%%%%%%%%%%%%
&\leq\sum_{m_{*}=0}^{max_{j\in \A^{'}}m_{j}}\sum_{i\in \A^{''}:m_{i}>m_{*}}\bigg(\dfrac{368 K^4 \sqrt{4}}{(T^{\frac{5}{4}}\sqrt{\epsilon_{m_{*}}})} \bigg).T.4\sqrt{\epsilon_{m_{*}}}\\
%%%%%%%%%%%%%%%%%%%%%%%%%%%%
&\overset{(a)}{\leq}\sum_{m_{*}=0}^{max_{j\in \A^{'}}m_{j}}\sum_{i\in \A^{''}:m_{i}>m_{*}}\bigg(\dfrac{C_2 K^4}{T^{\frac{1}{4}}\epsilon_{m_{*}}^{\frac{1}{2}-\frac{1}{2}}} \bigg)\\
%%%%%%%%%%%%%%%%%%%%%%%%%%%%
&\leq\sum_{i\in \A^{''}:m_{i}>m_{*}}\sum_{m_{*}=0}^{\min{\lbrace m_{i},m_{b}\rbrace}}\bigg(\dfrac{C_2 K^4}{T^{\frac{1}{4}}} \bigg)\\
%%%%%%%%%%%%%%%%%%%%%%%%%%%%
&\leq\sum_{i\in \A^{'}}\bigg(\dfrac{C_2 K^4}{T^{\frac{1}{4}}} \bigg)+\sum_{i\in \A^{''}\setminus \A^{'}}\bigg(\dfrac{C_2 K^4}{T^{\frac{1}{4}}} \bigg).\\
\end{align*}
Here at $(a)$, $C_2$ denotes an integer constant.



%\begin{align*}
%\sum_{m_{*}=0}^{max_{j\in \A^{'}}m_{j}}\sum_{i\in \A^{''}:m_{i}>m_{*}}\bigg(\dfrac{388 K}{(\psi  T\epsilon_{m_{*}})^{\frac{3\rho}{2}}} \bigg).T\max_{j\in \A^{''}:m_{j}\geq m_{*}}{\Delta}_{j}
%\end{align*}
%
%Again applying Lemma \ref{proofTheorem:Lemma:8} we can show that the above expression is upper bounded by 
%\begin{align*}
%\sum_{i\in \A^{'}}\dfrac{C_2^{'} K^{\frac{5}{2}}}{\sqrt{T\Delta_i}} +\sum_{i\in \A^{''}\setminus \A^{'}}\dfrac{C_2^{'} K^{\frac{5}{2}}}{\sqrt{T b}}
%\end{align*}

%%%%%%%%%%%%%%%%%%%%%%%%%%%%%%%%
%Moved to Appendix as Lemma 9
%%%%%%%%%%%%%%%%%%%%%%%%%%%%%%%%

%\begin{align*}
%&\sum_{m_{*}=0}^{max_{j\in \A^{'}}m_{j}}\sum_{i\in \A^{''}:m_{i}>m_{*}}\bigg(\dfrac{388 K}{(\psi  T\epsilon_{m_{*}})^{\frac{3\rho}{2}}} \bigg).T\max_{j\in \A^{''}:m_{j}\geq m_{*}}{\Delta}_{j}\\
%%%%%%%%%%%%%%%%%%%%%%%%%%%%%
%&\leq\sum_{m_{*}=0}^{max_{j\in \A^{'}}m_{j}}\sum_{i\in \A^{''}:m_{i}>m_{*}}\bigg(\dfrac{388 K\sqrt{4}}{(\psi  T\epsilon_{m_{*}})^{\frac{3\rho}{2}}} \bigg).T.4\sqrt{\epsilon_{m_{*}}}\\
%%%%%%%%%%%%%%%%%%%%%%%%%%%%%
%&\leq\sum_{m_{*}=0}^{max_{j\in \A^{'}}m_{j}}\sum_{i\in \A^{''}:m_{i}>m_{*}}C_2 K\bigg(\dfrac{T^{1-\frac{3\rho}{2}}}{\psi^{\frac{3\rho}{2}}\epsilon_{m_{*}}^{\frac{3\rho}{2}-\frac{1}{2}}} \bigg)\\
%%%%%%%%%%%%%%%%%%%%%%%%%%%%%
%&\leq\sum_{i\in \A^{''}:m_{i}>m_{*}}\sum_{m_{*}=0}^{\min{\lbrace m_{i},m_{b}\rbrace}}\bigg(\dfrac{C_2 K T^{1-\frac{3\rho}{2}}}{\psi^{\frac{3\rho}{2}}2^{-(\frac{3\rho}{2} -\frac{1}{2})m_{*}}} \bigg)\\
%%%%%%%%%%%%%%%%%%%%%%%%%%%%%
%&\leq\sum_{i\in \A^{'}}\bigg(\dfrac{C_2 K T^{1-\frac{3\rho}{2}}}{\psi^{\frac{3\rho}{2}}2^{-(\frac{3\rho}{2} -\frac{1}{2})m_{*}}} \bigg)+\sum_{i\in \A^{''}\setminus \A^{'}}\bigg(\dfrac{C_2 K T^{1-\frac{3\rho}{2} }}{\psi^{\frac{3\rho}{2}}2^{-(\frac{3\rho}{2} -\frac{1}{2})m_{b}}} \bigg)\\
%%%%%%%%%%%%%%%%%%%%%%%%%%%%%
%&\leq\sum_{i\in \A^{'}}\bigg(\dfrac{C_2 K T^{1-\frac{3\rho}{2}}.2^{\frac{\frac{3\rho}{2}}{2}-\frac{1}{4}}}{\psi^{\frac{3\rho}{2}}\Delta_{i}^{\frac{3\rho}{2} -\frac{1}{2}}} \bigg)+\sum_{i\in \A^{''}\setminus \A^{'}}\bigg(\dfrac{C_2 K T^{1-\frac{3\rho}{2}}.2^{\frac{\frac{3\rho}{2}}{2}-\frac{1}{4}}}{\psi^{\frac{3\rho}{2}}b^{\frac{3\rho}{2} -\frac{1}{2}}} \bigg)\\
%%%%%%%%%%%%%%%%%%%%%%%%%%%%%
%&\leq\sum_{i\in \A^{'}}\bigg(\dfrac{ C_2 K 2^{\frac{\frac{3\rho}{2}}{2}+\frac{19}{4}}.T^{1-\frac{3\rho}{2} } }{\psi^{\rho}\Delta_{i}^{2\frac{3\rho}{2} -1}} \bigg)+\sum_{i\in \A^{''}\setminus \A^{'}}\bigg(\dfrac{C_2 K 2^{\frac{\frac{3\rho}{2}}{2}+\frac{19}{4}}.T^{1-\frac{3\rho}{2}} }{\psi^{\frac{3\rho}{2} }b^{2\frac{3\rho}{2}-1}} \bigg)\\
%%%%%%%%%%%%%%%%%%%%%%%%%%%%%
%&\overset{(a)}{\leq}\sum_{i\in \A^{'}}\bigg(\dfrac{C_2^{'} K .T^{1-\frac{3}{4}}}{(\frac{T}{K^2})^{\frac{3}{4}}\Delta_{i}^{2.\frac{3}{4} -1}} \bigg)+\sum_{i\in \A^{''}\setminus \A^{'}}\bigg(\dfrac{C_2^{'} K T^{1-\frac{3}{4}}}{(\frac{T}{K^2})^{\frac{3}{4}}b^{2.\frac{3}{4}-1}} \bigg)\\
%%%%%%%%%%%%%%%%%%%%%%%%%%%%%
%&\leq\sum_{i\in \A^{'}}\dfrac{C_2^{'} K^{\frac{5}{2}}}{\sqrt{T\Delta_i}} +\sum_{i\in \A^{''}\setminus \A^{'}}\dfrac{C_2^{'} K^{\frac{5}{2}}}{\sqrt{T b}}
%%%%%%%%%%%%%%%%%%%%%%%%%%%%%
%\end{align*}
%In the above simplification, $(a)$ is obtained by substituting the values of $\psi$ and $\rho$.

Finally, summing up the regrets in \textbf{Case a} and \textbf{Case b}, the total regret is given by
\begin{align*}
\E [R_{T}] \leq &\sum\limits_{i\in \A :\Delta_{i} > b}\bigg\lbrace \dfrac{C_0 K^{4}}{T^{\frac{1}{4}}} + \bigg(\Delta_{i}+\dfrac{320\sigma_i^2\log{(\frac{T\Delta_{i}^{2}}{K})}}{\Delta_{i}}\bigg)\bigg \rbrace\\ 
  & +\sum\limits_{i\in \A :0 < \Delta_{i}\leq b} \dfrac{C_2 K^{4}}{T^{\frac{1}{4}}} + \max_{i\in \A :0 < \Delta_{i}\leq b}\Delta_{i}T
\end{align*}

where $C_0, C_1, C_2$ are integer constants s.t. $C_0 = C_1 + C_2$.
\end{customproof}




\section{Experiments}
\label{sec:expt}
In this section, we conduct extensive empirical evaluations of EUCBV against several other popular MAB  algorithms. We use expected cumulative regret as the metric of comparison. The comparison is conducted against the following algorithms: KLUCB+ \citep{garivier2011kl}, DMED \citep{honda2010asymptotically}, MOSS \citep{audibert2009minimax}, UCB1 \citep{auer2002finite}, UCB-Improved \citep{auer2010ucb}, Median Elimination \citep{even2006action}, Thompson Sampling (TS) \citep{agrawal2011analysis}, OCUCB \citep{lattimore2015optimally}, Bayes-UCB (BU) \citep{kaufmann2012bayesian} and UCB-V \citep{audibert2009exploration}\footnote{The implementation for KLUCB, Bayes-UCB and DMED were taken from \citet{CapGarKau12}}. The parameters of EUCBV algorithm for all the experiments are set as follows: $\psi=\frac{T}{K^2}$ and $\rho =0.5$ (as in Corollary \ref{Result:Corollary:1}). Note that KLUCB+ empirically outperforms KLUCB (as shown in \citet{garivier2011kl}).

\begin{figure}[!th]
    \centering
    \begin{tabular}{cc}
    \setlength{\tabcolsep}{0.1pt}
    \subfigure[0.25\textwidth][Expt-$1$: $20$ Bernoulli-distributed arms ]
    %with $r_{i_{{i}\neq {*}}}=0.07$ and $r^{*}=0.1$
    {
    		\pgfplotsset{
		tick label style={font=\Large},
		label style={font=\Large},
		legend style={font=\Large},
		ylabel style={yshift=5pt},
		%legend style={legendshift=32pt},
		}
        \begin{tikzpicture}[scale=0.7]
      	\begin{axis}[
		xlabel={timestep},
		ylabel={Cumulative Regret},
		grid=major,
        %clip mode=individual,grid,grid style={gray!30},
        clip=true,
        %clip mode=individual,grid,grid style={gray!30},
  		legend style={at={(0.5,1.5)},anchor=north, legend columns=3} ]
      	% UCB
		\addplot table{Chapter3/results/NewExpt/Expt1/UCBV01_comp_subsampled.txt};
		\addplot table{Chapter3/results/NewExpt/Expt1/EUCBV01_comp_subsampled.txt};
		\addplot table{Chapter3/results/NewExpt/Expt1/KLUCB01_comp_subsampled.txt};
		\addplot table{Chapter3/results/NewExpt/Expt1/MOSS01_comp_subsampled.txt};
		\addplot table{Chapter3/results/NewExpt/Expt1/DMED01_comp_subsampled.txt};
		\addplot table{Chapter3/results/NewExpt/Expt1/UCB01_comp_subsampled.txt};
		\addplot table{Chapter3/results/NewExpt/Expt1/TS01_comp_subsampled.txt};
		\addplot table{Chapter3/results/NewExpt/Expt1/OCUCB01_comp_subsampled.txt};
		\addplot table{Chapter3/results/NewExpt/Expt1/BU01_comp_subsampled.txt};
      	\legend{UCB-V,EUCBV,KLUCB+,MOSS,DMED,UCB1,TS,OCUCB,BU}      	
      	\end{axis}
      	\end{tikzpicture}
  		\label{fig:1}
    }
    &
    \subfigure[0.25\textwidth][Expt-$2$: $3$ Group Mean Setting ]
    %with $r_{i_{{i}\neq {*}:1-33}}=0.1$, $r_{i_{{i}\neq {*}:34-99}}=0.6$, $r^{*}_{i=100}=0.9$ and $\sigma_{i=1:100}^{2} = 0.3$
    {
    		\pgfplotsset{
		tick label style={font=\Large},
		label style={font=\Large},
		legend style={font=\Large},
		ylabel style={yshift=5pt},
		}
        \begin{tikzpicture}[scale=0.7]
        \begin{axis}[
		xlabel={timestep},
		ylabel={Cumulative Regret},
        %clip mode=individual,grid,grid style={gray!30},
       	grid=major,
       	clip=true,
  		legend style={at={(0.5,1.5)},anchor=north, legend columns=3} ]
      	% UCB
		\addplot table{Chapter3/results/NewExpt/Expt2/UCBV01_comp_subsampled.txt};
		\addplot table{Chapter3/results/NewExpt/Expt2/EUCBV01_comp_subsampled.txt};
		\addplot table{Chapter3/results/NewExpt/Expt2/KLUCB01_comp_subsampled.txt};
		\addplot table{Chapter3/results/NewExpt/Expt2/MOSS01_comp_subsampled.txt};
		\addplot table{Chapter3/results/NewExpt/Expt2/UCBR01_comp_subsampled.txt};
		\addplot table{Chapter3/results/NewExpt/Expt2/UCB01_comp_subsampled.txt};
		\addplot table{Chapter3/results/NewExpt/Expt2/TS01_comp_subsampled.txt};
		\addplot table{Chapter3/results/NewExpt/Expt2/OCUCB01_comp_subsampled.txt};
		\addplot table{Chapter3/results/NewExpt/Expt2/BU01_comp_subsampled.txt};      	
      	\legend{UCB-V,EUCBV,KLUCB-G+,MOSS,UCB-Imp,UCB1,TS-G,OCUCB,BU-G}
      	\end{axis}
      	\end{tikzpicture}
   		\label{fig:2}
    }
    \end{tabular}
    \caption{A comparison of the cumulative regret incurred by the various bandit algorithms. }
    \label{fig:karmed}
    \vspace*{-1em}
\end{figure}
% For the purpose of performance comparison


\textbf{Experiment-1 (Bernoulli with uniform gaps):} This experiment is conducted to observe the performance of EUCBV over a short horizon. The horizon $T$ is set to $60000$. The testbed comprises of $20$ Bernoulli distributed arms with expected rewards of the arms as $r_{1:19}=0.07$ and $r^{*}_{20}=0.1$ and these type of cases are frequently encountered in web-advertising domain (see \cite{garivier2011kl}). The regret is averaged over $100$ independent runs and is shown in Figure \ref{fig:1}. EUCBV, MOSS, OCUCB, UCB1, UCB-V, KLUCB+, TS, BU and DMED are run in this experimental setup. Not only do we observe that EUCBV performs better than all the non-variance based algorithms such as MOSS, OCUCB, UCB-Improved and UCB1, but it also outperforms UCBV because of the choice of the exploration parameters. Because of the small gaps and short horizon $T$, we do not compare with UCB-Improved and Median Elimination for this test-case. 

\textbf{Experiment-2 (Gaussian $3$ Group Mean Setting):} This experiment is conducted to observe the performance of EUCBV over a large horizon in Gaussian distribution testbed. This setting comprises of a large horizon of $T = 3\times 10^{5}$ timesteps and a large set of arms. This testbed comprises of $100$ arms involving Gaussian reward distributions with expected rewards of the arms in $3$ groups, $r_{1:66}=0.07$, $r_{67:99}=0.01$ and $r^{*}_{100}=0.09$ with variance set as $\sigma_{1:66}^{2} = 0.01,\sigma_{67:99}^{2} = 0.25$ and $\sigma^{2}_{100}=0.25$. The regret is averaged over $100$ independent runs and is shown in Figure \ref{fig:2}. From the results in Figure \ref{fig:2}, we observe that since the gaps are small and the variances of the optimal arm and the arms farthest from the optimal arm are the highest, EUCBV, which allocates pulls proportional to the variances of the arms, outperforms all the non-variance based algorithms MOSS, OCUCB, UCB1, UCB-Improved and Median-Elimination ($\epsilon=0.1,\delta=0.1$). The performance of Median-Elimination is extremely weak in comparison with the other algorithms and its plot is not shown in Figure \ref{fig:2}. We omit its plot in order to more clearly show the difference between EUCBV, MOSS and OCUCB. Also note that the order of magnitude in the y-axis (cumulative regret) of Figure \ref{fig:2} is $10^4$. KLUCB-Gauss+ (denoted by KLUCB-G+), TS-G and BU-G are initialized with Gaussian priors. Both KLUCB-G+ and UCBV which is a variance-aware algorithm perform much worse than TS-G and EUCBV. The performance of DMED is similar to KLUCB-G+ in this setup and its plot is omitted. 


\begin{figure}[!h]
    \centering
    \begin{tabular}{cc}
    \subfigure[0.25\textwidth][Expt-$3$: Failure of TS]
    {
    		\pgfplotsset{
		tick label style={font=\Large},
		label style={font=\Large},
		legend style={font=\Large},
		ylabel style={yshift=5pt},
		}
        \begin{tikzpicture}[scale=0.7]
      	\begin{axis}[
		ylabel={Cumulative Regret},
		xlabel={timestep},
		grid=major,
        %clip mode=individual,grid,grid style={gray!30},
        clip=true,
        %clip mode=individual,grid,grid style={gray!30},
  		legend style={at={(0.5,1.3)},anchor=north, legend columns=3} ]
      	% UCB
		\addplot table{Chapter3/results/NewExpt/Expt3/UCBV01_comp_subsampled.txt};
		\addplot table{Chapter3/results/NewExpt/Expt3/EUCBV01_comp_subsampled.txt};
		\addplot table{Chapter3/results/NewExpt/Expt3/MOSS01_comp_subsampled.txt};
		\addplot table{Chapter3/results/NewExpt/Expt3/TS01_comp_subsampled.txt};
		\addplot table{Chapter3/results/NewExpt/Expt3/OCUCB01_comp_subsampled.txt};
		\addplot table{Chapter3/results/NewExpt/Expt3/BU01_comp_subsampled.txt};
      	\legend{UCBV,EUCBV,MOSS,TS-G,OCUCB,BU-G} 
      	\end{axis}
      	\end{tikzpicture}
  		\label{fig:3}
    }
    &
    \subfigure[0.25\textwidth][Expt-$4$: $3$ Group Variance Setting]
    %with $r_{i_{{i}\neq {*}}}=0.05$ and $r^{*}=0.1$
    {
    	\pgfplotsset{
		tick label style={font=\Large},
		label style={font=\Large},
		legend style={font=\Large},
		ylabel style={yshift=5pt},
		}
        \begin{tikzpicture}[scale=0.7]
        \begin{axis}[
		xlabel={timestep},
		ylabel={Cumulative Regret},
        %clip mode=individual,grid,grid style={gray!30},
		grid=major,
		clip=true,
  		legend style={at={(0.5,1.3)},anchor=north, legend columns=3} ]
        % UCB
		\addplot table{Chapter3/results/NewExpt/Expt41/UCBV01_comp_subsampled.txt};
		\addplot table{Chapter3/results/NewExpt/Expt41/EUCBV01_comp_subsampled.txt};
		\addplot table{Chapter3/results/NewExpt/Expt41/MOSS01_comp_subsampled.txt};
		\addplot table{Chapter3/results/NewExpt/Expt41/TS01_comp_subsampled.txt};
		\addplot table{Chapter3/results/NewExpt/Expt41/OCUCB01_comp_subsampled.txt};
		\addplot table{Chapter3/results/NewExpt/Expt41/BU01_comp_subsampled.txt};
      	\legend{UCBV,EUCBV,MOSS,TS-G,OCUCB,BU-G} 
      	\end{axis}
        \end{tikzpicture}
        \label{fig:4}
    }
	\end{tabular}
	\label{fig:furtherExpt1}
    \caption{Further Experiments with EUCBV}
    \vspace*{-1em}
\end{figure}
%\vspace*{-0.5em}



\textbf{Experiment-3 (Failure of TS):} This experiment is conducted to demonstrate that in certain environments when the horizon is large, gaps are small and the variance of the optimal arm is high, the Bayesian algorithms (like TS) do not perform well but EUCBV performs exceptionally well. This experiment is conducted on $100$ Gaussian distributed arms such that expected rewards of the arms $r_{1:10}=0.045$, $r_{11:99}=0.04$, $r^{*}_{100}=0.05$ and the variance is set as $\sigma_{1:10}^{2}=0.01$,   $\sigma_{100}^{2}=0.25$ and $T=4\times 10^5$. The variance of the arms $i=11:99$ are chosen uniform randomly between $[0.2,0.24]$. TS and BU with Gaussian priors fail because here the chosen variance values are such that only variance-aware algorithms with appropriate exploration factors will perform  well or otherwise it will get bogged down in costly exploration. The algorithms that are not variance-aware will spend a significant amount of pulls trying to find the optimal arm. The result is shown in Figure \ref{fig:3}. Predictably EUCBV, which allocates pulls proportional to the variance of the arms, outperforms its closest competitors TS-G, BU-G, UCBV, MOSS and OCUCB. The plots for KLUCB-G+, DMED, UCB1, UCB-Improved and Median Elimination are omitted from the figure as their performance is extremely weak in comparison with other algorithms. We omit their plots to clearly show how EUCBV outperforms its nearest competitors. Note that EUCBV by virtue of its aggressive exploration parameters outperforms UCBV in all the experiments even though UCBV is a variance-based algorithm. The performance of TS-G is also weak and this is in line with the observation in \citet{lattimore2015optimally} that the worst case regret of TS when Gaussian prior is used is $\Omega\left( \sqrt{KT\log T}\right)$.



\textbf{Experiment-4 (Gaussian $3$ Group Variance setting):} This experiment is conducted to show that when the gaps are uniform and variance of the arms are the only discriminative factor then the EUCBV performs extremely well over a very large horizon and over a large number of arms. This testbed comprises of $100$ arms with Gaussian reward distributions, where the expected rewards of the arms are $r_{1:99}=0.09$ and $r^{*}_{100}=0.1$. The variances of the arms are divided into $3$ groups. The group $1$ consist of arms $i=1:49$ where the variances are chosen uniform randomly between $[0.0,0.05]$, group $2$ consist of arms $i=50:99$ where the variances are chosen uniform randomly between $[0.19,0.24]$ and for the optimal arm $i=100$ (group $3$) the variance is set as $\sigma_{*}^{2}=0.25$. We report the cumulative regret averaged over $100$ independent runs. The horizon is set at $T=4\times 10^{5}$ timesteps. We report the performance of MOSS,BU-G, UCBV, TS-G and OCUCB who are the closest competitors of EUCBV over this uniform gap setup. From the results in Figure \ref{fig:4}, it is evident that the growth of regret for EUCBV  is much lower than that of TS-G, MOSS, BU-G, OCUCB and UCBV. Because of the poor performance of KLUCB-G+ in the last two experiments we do not implement it in this setup. Also, note that for optimal performance BU-G, TS-G and KLUCB-G+ require the knowledge of the type of distribution to set their priors . Also, in all the experiments with Gaussian distributions EUCBV significantly outperforms all the Bayesian algorithms initialized with Gaussian priors.




\section{Summary and Future Works}
\label{sec:conc}
In this chapter, we studied the EUCBV algorithm which takes into account the empirical variance of the arms and employs aggressive exploration parameters in conjunction with non-uniform arm selection (as opposed to UCB-Improved) to eliminate sub-optimal arms. Our theoretical analysis conclusively established that EUCBV exhibits an order-optimal gap-independent regret bound of $O\left(\sqrt{KT}\right)$. Empirically, we show that EUCBV performs superbly across diverse experimental settings and outperforms most of the bandit algorithms in an SMAB setup. Our experiments showed that EUCBV is extremely stable for large horizons and performs consistently well across different types of distributions. 





%%%%%%%%%%%%%%%%%%%%%%%%%%%%%%%%%%%%%%%%%%%%%%%%%%%%%%%%%%%%






%%%%%%%%%%%%%%%%%%%%%%%%%%%%%%%%%%%%%%%%%%%%%%%%%%%%%%%%%%%%
\chapter{Thresholding Bandits}
\label{chap:tbandit1}

\section{Introduction to Thresholding Bandits}
\label{tbandit:intro1}

Pure-exploration MAB problems are unlike their traditional (exploration vs.\ exploitation)  counterparts, the SMABs, where the  objective is to minimize the cumulative regret. The cumulative regret is the total loss incurred by the learner for not playing the optimal arm throughout the time horizon $T$. The SMABs were extensively discussed in Chapter \ref{chap:SMAB} and \ref{chap:EUCBV}. An interested reader can read through the previous chapters or can continue from here. Though we re-use the ideas from SMABs, the goal of pure exploration setup is distinctly different from that of cumulative regret minimization of SMABs and the required algorithms to understand this setup and our proposed algorithm are mentioned in this chapter itself.

	In pure-exploration problems a learning algorithm, until time $T$, can invest entirely on exploring the arms without being concerned about the loss incurred while exploring; the objective is to minimize the probability that the arm recommended at time $T$ is not the best arm.  In this paper, we further consider a combinatorial version of the pure-exploration MAB, called the thresholding bandit problem (TBP).  Here, the learning algorithm is provided with a threshold $\tau$, and the objective, after exploring for $T$ rounds, is to  output all arms $i$ whose $r_{i}$ is above $\tau$. 
It is important to emphasize that the \emph{thresholding} bandit problem is different from the \emph{threshold} bandit setup studied in \cite{abernethy2016threshold}, where the learner receives an unit reward whenever the value of an observation is above a threshold. 



\section{Notations}
\label{tbandit:notations}
To benefit the reader, we again recall the notations we stated in chapter \ref{chap:SMAB} and also a few additional notations. $\mathcal{A}$ denotes the set of arms, and $|\mathcal{A}|=K$ is the number of arms in $\mathcal{A}$. For arm $i\in\mathcal{A}$, we use $r_{i}$ to denote the true mean of the distribution from which the rewards are sampled, while $\hat{r}_{i}(t)$ denotes the estimated mean at time $t$. Formally, using $z_i(t)$ to denote the number of times arm $i$ has been pulled until time $t$, we have $\hat{r}_{i}(t)=\frac{1}{z_{i}(t)}\sum_{b=1}^{z_i(t)} X_{i,b}$, where $X_{i,b}$ is the reward sample received when arm $i$ is pulled for the $b$-th time. %
Similarly, we use $\sigma_{i}^{2}$ to denote the true variance of the reward distribution corresponding to arm $i$, while $\hat{v}_{i}(t)$ is the estimated variance, i.e., $\hat{v}_{i}(t)=\frac{1}{z_i(t)}\sum_{b=1}^{z_{i}(t)}(X_{i,b}-\hat{r}_{i})^{2}$. Whenever there is no ambiguity about the underlaying  time index $t$, for simplicity we neglect $t$ from the notations and simply use  $\hat{r}_i, \hat{v}_i,$ and $z_i, $ to denote the respective quantities.  Let  $\Delta_{i}=|\tau-r_{i}|$ denote the distance of the true mean from the threshold $\tau$. Also, the rewards are assumed to take values in $[0,1]$.

%%%%%%%%%%
%1-sub-gaussian assumption removed
%%%%%%%%%%
%Along the lines of \cite{locatelli2016optimal} we assume that all the reward distributions are $1$-sub-Gaussian (note that,  $1$-sub-Gaussian includes Gaussian distributions with variance less than $1$, distributions supported on an interval of length less than 2, etc).



\section{Problem Definition}
\label{tbandit:probDef}
Formally, the problem we consider is the following. First, we define the set $S_{\tau}=\lbrace i\in \mathcal{A}: r_{i}\geq \tau \rbrace$. Note that, $S_\tau$ is the set of all arms whose reward mean is greater than $\tau$. Let 
$S_\tau^c$ denote the complement of $S_\tau$, i.e.,  $S_{\tau}^{c}=\lbrace i\in \mathcal{A}: r_{i} < \tau \rbrace$. Next, let $\hat{S}_{\tau}=\hat{S}_{\tau}(T)\subseteq \mathcal{A}$ denote the recommendation of a learning algorithm (under consideration) after $T$ time units of exploration, while $\hat{S}_{\tau}^c$ denotes its complement.

The performance of the learning agent is measured by the accuracy with which it can classify the arms into $S_{\tau}$ and $S_{\tau}^{c}$ after time horizon $T$. Equivalently, using $\mathbb{I}(E)$ to denote the indicator of an event $E$, the \emph{loss} $\mathcal{L}(T)$ is defined as
\begin{align*}
\Ls (T) = \mathbb{I}\big(\lbrace S_{\tau}\cap \hat{S}_{\tau}^{c}\neq \emptyset\rbrace    \cup    \lbrace\hat{S}_{\tau}\cap S_{\tau}^{c}\neq \emptyset\rbrace\big).
\end{align*}			
Finally, the goal of the learning agent is to minimize the expected loss:
\begin{align*}
\E[\Ls(T)] = \Pb\big(\lbrace S_{\tau}\cap \hat{S}_{\tau}^{c} \neq \emptyset \rbrace  \cup   \lbrace \hat{S}_{\tau}\cap S_{\tau}^{c} \neq \emptyset\rbrace\big).
\end{align*}
Note that the expected loss is simply the \emph{probability of mis-classification} (i.e., error), that occurs either if a good arm is rejected or a bad arm is accepted as a good one.


\section{Motivation}
\label{tbandit:motivation}
The above TBP formulation has several applications, for instance, from areas ranging from anomaly detection and classification (see  \citet{locatelli2016optimal}) to industrial application. Particularly in industrial applications, a learners objective is to choose (i.e., keep in operation) all machines whose productivity is above a threshold. The TBP also finds applications in mobile communications (see \citet{audibert2010best})  where the users are to be allocated only those channels whose quality is above an acceptable threshold. Again, some of these problems have been already discussed in chapter \ref{chap:intro}, section \ref{motivation} and an interested reader can refer to it. In some cases the TBP problem is more relevant than the variants of $p$-best problem (identifying the best $p$ arms from $K$ given arms). As explained in \citet{locatelli2016optimal}, the $p$-best problem is a "contest" whereas the TBP is an  "exam" and in many domains, one requires the idea of "efficiency" or "correctness" threshold above which one wants to utilize an option rather than simply selecting the $p$-best options.

%where the learner has to keep all those workers active whose productivity is above a particular threshold $\tau$, or allocating channels whose quality is above a threshold for Mobile Communications 
% or in crowd-sourcing while hiring workers the TBP problem 

%
%	1. \emph{Product Selection:} A company wants to introduce a new product in market and there is a clear separation of the test phase from the commercialization phase. In this case the company tries to minimize the loss it might incur in the commercialization phase by testing as much as possible in the test phase. So from the several variants of the product that are in the test phase the learning agent must suggest the product variant(s) that are above a particular threshold $\tau$ at the end of the test phase that have the highest probability of minimizing loss in the commercialization phase. A similar problem has been discussed for single best product variant identification without threshold in \cite{bubeck2011pure}. 
%
%	2. \emph{Mobile Phone Channel Allocation:} Another similar problem as above concerns channel allocation for mobile phone communications (\cite{audibert2009exploration}). Here there is a clear separation between the allocation phase and communication phase whereby in the allocation phase a learning algorithm has to explore as many channels as possible to suggest the best possible set of channel(s) that are above a particular threshold $\tau$. The threshold depends on the subscription level of the customer. With higher subscription the customer is allowed better channel(s) with the $\tau$ set high. Each evaluation of a channel is noisy and the learning algorithm must come up with the best possible suggestion within a very small  number of attempts.
%
%	3. \emph{Anomaly Detection and Classification:} Thresholding bandit can also be used for anomaly detection and classification where we define a cutoff level $\tau$ and for any samples above this cutoff gets classified as an anomaly. For further reading we point the reader to section 3 of \cite{locatelli2016optimal}.
%
%

\section{Related Work in Pure Exploration}
\label{tbandit:prevRes}
Significant amount of literature is available on the stochastic MAB setting with respect to minimizing the cumulative regret. Chapter \ref{chap:SMAB} and \ref{chap:EUCBV} deals with that. In this work we are particularly interested in \emph{pure-exploration MABs},  where the focus in primarily on simple regret rather than the cumulative regret. The relationship between cumulative regret and simple regret is proved in \citet{bubeck2011pure} where the authors prove that minimizing the simple regret necessarily results in maximizing the cumulative regret.
The pure exploration problem has been explored  mainly under the following two settings:
	
\subsection{Fixed Budget setting} 

Here the learning algorithm has to suggest the best arm(s) within a fixed time-horizon $T$, that is usually given as an input. The objective is to maximize the probability of returning the best arm(s).  This is the scenario we consider in this chapter. Some of the important algorithms used in pure exploration setting are discussed in the next part.

\subsubsection{UCB-Exploration Algorithm}

\begin{algorithm}[!h]
\caption{UCBE}
\label{alg:ucbe}
\begin{algorithmic}[1]
\State \textbf{Input: } The budget $T$
\State Pull each arm once
\For{$t=K+1,..., T$}
\State Pull the arm such that $\argmax_{i\in \A}\bigg\lbrace\hat{r}_{i} + \sqrt{\dfrac{a}{n_i}}\bigg\rbrace$, where $a = \dfrac{25(T-K)}{36 H_1}$ and $H_1 = \sum_{i=1}^{K}\dfrac{1}{\Delta_i^2}$.
\State $t:=t+1 $
\EndFor
\end{algorithmic}
\end{algorithm}

In \citet{audibert2010best} the authors propose the  UCBE and the Successive Reject (SR) algorithm, and prove simple-regret guarantees for the problem of identifying the single best arm.  In the combinatorial fixed budget setup \citet{gabillon2011multi} propose the GapE and GapE-V algorithms that suggest, with high probability, the best $m$ arms at the end of the time budget. 


\subsubsection{Successive Reject Algorithm}


\begin{algorithm}[h!]
\caption{Successive Reject(SR)}
\label{alg:sr}
\begin{algorithmic}[1]
\State \textbf{Input: } The budget $T$
\State \textbf{Initialization: } $n_0 = 0$
\State \textbf{Definition: } $\bar{\log K} = \dfrac{1}{2} + \sum_{i=2}^{K}\dfrac{1}{i}$, $n_k = \dfrac{1}{\bar{\log K}}\dfrac{T-K}{K + 1 - m}$
\For{For each phase $m=1,..., K-1$}
\State For each $i \in B_{m}$, select arm $i$ for $n_k - n_{k-1}$ rounds.
\State Let $B_{m+1} = B_m\setminus \argmin_{i\in B_m} \hat{r}_i$
(remove one element from $B_m$ , if there
is a tie, select randomly the arm to dismiss among the worst arms).
\State $m:=m+1 $
\EndFor
\State Output the single remaining $i\in B_{m}$.
\end{algorithmic}
\end{algorithm}


Similarly, \citet{bubeck2013multiple} introduce the  Successive Accept Reject (SAR) algorithm, which is an extension of the SR algorithm; SAR is a round based algorithm whereby at the end of each round an arm is either accepted or rejected (based on certain confidence conditions) until the top $m$ arms are suggested at the end of the budget with high probability. A similar combinatorial setup was explored in \citet{chen2014combinatorial} where the authors propose the Combinatorial Successive Accept Reject (CSAR) algorithm, which is similar in concept to SAR but with a more general setup. 

\subsection{Fixed Confidence setting} 

In this setting the learning algorithm has to suggest the best arm(s) with a fixed confidence (given as input) with as fewer number of attempts as possible. The single best arm identification has been studied in \citet{even2006action}, while for the combinatorial setup \citet{kalyanakrishnan2012pac} have proposed the LUCB algorithm which, on termination, returns  $m$ arms which are at least $\epsilon$ close to the true top-$m$ arms with probability at least $1-\delta$. For a detail survey of this setup we refer the reader to \citet{jamieson2014best}. 

\subsection{Unified Setting}
Apart from these two settings some unified approaches has also been suggested in \citet{gabillon2012best} which proposes the algorithms UGapEb and UGapEc which can work in both the above two settings. The thresholding bandit problem is a specific instance of the pure-exploration setup of \citet{chen2014combinatorial}. 



	
	



\section{Related Work in Thresholding Bandits}
\label{tbandit:prevResAPT}
In the latest work of \citet{locatelli2016optimal} Anytime Parameter-Free Thresholding (APT) algorithm comes up with an improved anytime guarantee than CSAR for the thresholding bandit problem. APT is stated in algorithm \ref{alg:apt}. 

\begin{algorithm}[!th]
\caption{APT}
\label{alg:apt}
\begin{algorithmic}
\State {\bf Input:} Time horizon $T$, threshold $\tau$, tolerance factor $\epsilon\geq 0$
\State Pull each arm once
\State \For{$t=K+1,..,T$}
\State Pull arm $j\in\argmin_{i\in A}\big\lbrace \left(|\hat{r}_{i} - \tau | + \epsilon\right)\sqrt{n_i}\big\rbrace$ and observe the reward for arm $j$.
\EndFor
\State \textbf{Output:} $\hat{S}_{\tau}=\lbrace i: \hat{r}_{i}\geq \tau \rbrace$.
\end{algorithmic}
\end{algorithm}

The APT algorithm is very simple to implement and the logic behind the arm pull directly follows from the challenges in the TBP setting discussed before. Note, that the most difficult arms to discriminate are the arms whose expected means are lying close to the threshold $\tau$, hence APT pulls those arms whose sample means $\hat{r}_i$ are lying close to the threshold and the arms which has not been pulled often. The second condition is satisfied by the $\sqrt{n_i}$ term which acts very similar to the confidence interval term discussed for UCBE (see algorithm \ref{alg:ucbe}). The tolerance level $\epsilon\geq 0$ gives the algorithm a degree of flexibility in pulling the arms close to the threshold.


\section{Summary}
\label{tbandit:conc}
In this chapter, we looked at the pure exploration MAB and thresholding bandit (TBP) setting which is a special case of combinatorial pure exploration MAB. We then looked at the various state-of-the-art algorithms in the literature for the pure-exploration setting and discussed the advantages and disadvantages of them. Then we looked at the latest algorithm for the TBP setting. The expected loss that has been proven for the said algorithms have also been discussed at length and their exploration parameters have also been compared against each other. In the next chapter, we provide our solution to this TBP setting which uses variance estimation to find the set of arms above the threshold.


%%%%%%%%%%%%%%%%%%%%%%%%%%%%%%%%%%%%%%%%%%%%%%%%%%%%%%%%%%%%%%%




%%%%%%%%%%%%%%%%%%%%%%%%%%%%%%%%%%%%%%%%%%%%%%%%%%%%%%%%%%%%%%%
\chapter{Augmented UCB for TBP}
\label{chap:tbandit2}

\section{Introduction}
\label{tbandit:intro2}
In this chapter we look at the Augmented-UCB (AugUCB) algorithm for a fixed-budget version of the thresholding bandit problem (TBP), where the objective is to identify a set of arms whose expected mean is above a threshold. A key feature of AugUCB is that it uses both mean and variance estimates to eliminate arms that have been sufficiently explored; to the best of our knowledge this is the first algorithm to employ such an approach for the considered TBP.  Theoretically, we obtain an upper bound on the loss (probability of mis-classification) incurred by AugUCB. Although UCBEV in literature provides a better guarantee, it is important to emphasize that UCBEV has access to problem complexity (whose computation requires arms' mean and variances), and hence is not realistic in practice; this is in contrast to AugUCB whose implementation does not require any such complexity inputs. We conduct extensive simulation experiments to validate the performance of AugUCB. Through our simulation work, we establish that AugUCB, owing to its utilization of variance estimates, performs significantly better than the state-of-the-art APT, CSAR and other non variance-based algorithms.

The rest of the chapter is organized as follows. We elaborate our contributions in Section~\ref{tbandit:contribution} and  we present the AugUCB algorithm in Section~\ref{tbandit:algorithm}. Our main theoretical result on expected loss is stated in Section~\ref{tbandit:results}. Section~\ref{tbandit:expt} contains numerical simulations on various testbeds to show the performance of AugUCB against state-of-the-art algorithms and finally, we summarize in Section~\ref{tbandit:conclusion}.


\section{Our Contribution}
\label{tbandit:contribution}
We propose the Augmented UCB (AugUCB) algorithm for the fixed-budget setting of a specific combinatorial, pure-exploration, stochastic MAB called the thresholding bandit problem.
%In this paper we propose AugUCB algorithm for the fixed-budget, comp thresholding bandit problem.
 AugUCB essentially combines the approach of UCB-Improved, CCB \citep{liu2016modification} and APT algorithms. Our algorithm takes into account the empirical variances of the arms along with mean estimates; to the best of our knowledge this is the first variance-based algorithm for the considered TBP. 
Thus, we also address an open problem discussed in \cite{auer2010ucb} of designing an algorithm that can eliminate arms based on variance estimates. In this regard, note that both CSAR and APT are not variance-based algorithms. 

\begin{table}[b]
\caption{AugUCB vs.\ State of the art}
\label{tab:regret-bds}
\begin{center}
\begin{tabular}{|p{2.3cm}|p{8.4cm}|}
% \toprule
\hline
Algorithm  & Upper Bound on Expected Loss \\
% \midrule
\hline
\hline
AugUCB      &$ \exp\left(- \dfrac{T}{4096 \log(K\log K)H_{\sigma,2}} + \log\left(2KT\right) \right) $ \\
\hline
\hline
UCBEV		&$\exp\left(-\dfrac{1}{512}\frac{T-2K}{H_{\sigma,1}} + \log\left(6KT\right)\right)$ \\
%\midrule
\hline
\hline
APT         &$\exp\left(-\dfrac{T}{64 H_1}+2\log((\log(T)+1)K)\right)$ \\
% \midrule
\hline
\hline
CSAR		&$\exp\left(-\dfrac{T-K}{72\log(K)H_{CSAR,2}}+2\log(K)\right)$ \\
%\midrule
\hline

%\bottomrule
\end{tabular}
\end{center}
\end{table}

Our theoretical contribution comprises 
 proving an upper bound on the expected loss incurred by AugUCB (Theorem~\ref{tbandit:Result:Theorem:1}).
In Table \ref{tab:regret-bds} we compare the upper bound on the losses incurred by the various algorithms, including AugUCB. The terms $H_1, H_2$, $H_{CSAR,2}, H_{\sigma,1}$ and $H_{\sigma,2}$ represent various problem complexities, and are as defined in Section~\ref{tbandit:results}. From Section~\ref{tbandit:results} we note that, for all $K\ge8$, we have
\begin{align*}
\log\left(K\log K\right) H_{\sigma,2} > \log(2K) H_{\sigma,2} \ge H_{\sigma,1}.
\end{align*}
%; relation between these quantities are also given in Section~\ref{results} 
%The term containing $H_{\sigma,2}$ is comparable to the similar terms (containing $H_{\sigma,1}$) for the error probability of GapE-V \cite{gabillon2011multi} algorithm which we modify to perform in the TBP problem and name it as UCBEV.
Thus, it follows that the upper bound for UCBEV is better than that for AugUCB.
 %The error probability of UCBEV for single bandit multi-armed case is given in Table \ref{tab:regret-bds}. We see that $\log(\frac{3}{16} K\log K) H_2^{\sigma} > \log(2K) H_2^{\sigma} \ge H_1^{\sigma}$ and hence our algorithm is weaker with respect to UCBEV for single  multi-armed bandit scenario.
 However, implementation of UCBEV algorithm requires $H_{\sigma,1}$ as input, whose computation is not realistic in practice. In contrast, our AugUCB algorithm requires no such complexity factor as input. 
%Theoretically, we can compare the first term (containing $H_2$) of our expected loss and see that for all $K\geq 4$, $ H_2 \log(\frac{3}{16} K\log K) > (\log K)H_{CSAR,2}\geq H_1 $ and hence our result is weaker than CSAR and APT.

Proceeding with the comparisons, we emphasize that the upper bound for  AugUCB is, in fact, not comparable with that of APT and CSAR; this is because the complexity term $H_{\sigma,2}$ is not explicitly comparable with either $H_1$ or $H_{CSAR,2}$. However, through extensive simulation experiments we find that AugUCB significantly outperforms both APT, CSAR and other non variance-based algorithms. AugUCB also outperforms UCBEV under explorations where non-optimal values of $H_{\sigma,1}$  are used. In particular, we consider experimental scenarios comprising large number of arms, with the variances of arms in $S_\tau$ being large. AugUCB, being variance based, exhibits superior performance under these settings.  
%


%Empirically we show that for a large number of arms when the variance of the arms lying above $\tau$ are high, our algorithm performs better than all other algorithms, except the algorithm UCBEV which has access to the underlying problem complexity and also is a variance-aware algorithm. 
%
%AugUCB requires one input parameter and the exact choice for the parameter is derived in Theorem \ref{Result:Theorem:1}. Also, unlike SAR or CSAR, AugUCB does not have explicit accept or reject sets rather the arm elimination condition simply removes arm(s) if it is sufficiently sure that the mean of the arms are very high or very low about the threshold based on mean and variance estimation thereby re-allocating the remaining budget among the surviving arms. This although is a tactic similar to SAR or CSAR, but here at any round, an arbitrary number of arms can be accepted or rejected thereby improving upon SAR and CSAR which accepts/rejects one arm in every round. Also their round lengths are non-adaptive and they pull all the arms equal number of times in each round. 
%At every timestep AugUCB pulls the arm that minimizes thereby making this an anytime algorithm whereby we need not finish every round. 
%Irrespective of this case AugUCB also employs elimination of arms based on mean estimation only and is the first such algorithm which uses elimination by both mean and variance estimation simultaneously.

%The remainder of the paper is organized as follows. In section \ref{tbandit:algorithm} we present our AugUCB algorithm. 
%Section \ref{tbandit:results} contains our main theorem on expected loss, while section \ref{tbandit:expt} contains simulation experiments. We finally draw our conclusions in section \ref{tbandit:conclusion}.
%in section \ref{notation} we introduce the notations and the


\section{Augmented-UCB Algorithm}
\label{tbandit:algorithm}
\textbf{The Algorithm:} The Augmented-UCB (AugUCB) algorithm is presented in Algorithm~\ref{alg:augucb}.
AugUCB is essentially based on the arm elimination method of the UCB-Improved \cite{auer2010ucb}, but adapted to the thresholding bandit setting proposed in \cite{locatelli2016optimal}. However, unlike the UCB improved (which is based on mean estimation) our algorithm employs \emph{variance estimates} (as in \cite{audibert2009exploration}) for arm elimination; to the best of our knowledge this is the first variance-aware  algorithm for the thresholding bandit problem. Further, we allow for arm-elimination at each time-step, which is in contrast to the earlier work (e.g., \cite{auer2010ucb,chen2014combinatorial}) where the arm elimination task is deferred to the end of the respective exploration rounds. The details are presented below.

% In algorithm \ref{alg:augucb}, hence referred to as AugUCB, we have two exploration parameters, $\rho_{\mu}$ and $\rho_v$ which are the arm elimination parameters. $\psi_{m}$ is the exploration regulatory factor. 
%The main approach is based on the UCB-Improved algorithm with modifications suited for the thresholding bandit problem. 
The active set $B_{0}$ is initialized with all the arms from $\mathcal{A}$. We divide the entire budget $T$ into rounds/phases like in UCB-Improved, CCB, SAR and CSAR. At every time-step AugUCB checks for arm elimination conditions, while updating parameters at the end of each round. As suggested by \cite{liu2016modification} to make AugUCB to overcome too much early exploration, we no longer pull all the arms equal number of times in each round. Instead, we choose an arm in the active set $B_m$ that minimizes $(|\hat{r}_{i} - \tau |-2s_i)$ where 
%$\min_{i\in B_{m}}\big\lbrace |\hat{r}_{i} - \tau | - 2\sqrt{\frac{\rho_v\psi_m \hat{V}_{i} \log ( T \epsilon_{m})}{4 n_{i}} + \frac{\rho_v\psi_m \log{( T\epsilon_{m})}}{4 n_{i}}} \big\rbrace $
\begin{small}
\begin{align*}
s_i & = \sqrt{\frac{\rho\psi_m (\hat{v}_{i}+1) \log ( T \epsilon_{m})}{4 n_{i}}} %+ \frac{\rho\psi_m \log{( T\epsilon_{m})}}{4 n_{i}}}.
\end{align*}
\end{small} 
with $\rho$ being the arm elimination parameter and $\psi_{m}$ being the exploration regulatory factor.
%  in the active set $B_{m}$. 
The above condition ensures that an arm closer to the threshold $\tau$ is pulled; 
%and with suitable choice of $\rho_{\mu}$ and $\rho_v$ we can fine tune the exploration. 
parameter $\rho$ can be used to fine tune the elimination interval.
The choice of exploration factor, $\psi_m$,
% $\psi_m=\frac{T\epsilon_m}{(\log(\frac{3}{16} K\log K))^{2}}$ 
comes directly from \cite{audibert2010best} and \cite{bubeck2011pure} where it is  stated that in pure exploration setup, the exploring factor must be linear in $T$ (so that an exponentially small probability of error is achieved) rather than being logarithmic in $T$ (which is more suited for minimizing cumulative regret).

\begin{algorithm}[t!]
\caption{AugUCB}
\label{alg:augucb}
\begin{algorithmic}
\State {\bf Input:} Time budget $T$; parameter $\rho$; 
% $\rho_{\mu}$, $\rho_v$ 
  threshold $\tau$
\State {\bf Initialization:} $B_{0}=\mathcal{A}$; $m=0$; $\epsilon_{0}=1$;
\begin{small}
\begin{align*}
M&=\left\lfloor \frac{1}{2}\log_{2} \frac{T}{e}\right\rfloor; 
\hspace{2mm}\psi_{0}=\frac{T\epsilon_{0}}{128\Big(\log(\frac{3}{16}K\log K)\Big)^2}; \\
\ell_{0}&=\left\lceil \frac{2\psi_0\log( T\epsilon_{0})}{\epsilon_{0}} \right\rceil;
\hspace{2mm}N_{0}=K\ell_{0}
\end{align*}
\end{small}
%$M=\left\lfloor \frac{1}{2}\log_{2} \frac{T}{e}\right\rfloor $,  
%$\psi_{0}=\frac{T\epsilon_{0}}{(\log(\frac{3}{16}K\log K)^2}$,
% $\ell_{0}=\left\lceil \frac{2\psi\log( T\epsilon_{0})}{\epsilon_{0}} \right\rceil$ and 
% $N_{0}=K\ell_{0} $. Pull each arm once.
\State Pull each arm once
\vspace{-2mm}
\State \For{$t=K+1,..,T$}
\State Pull arm $j\in\argmin_{i\in B_{m}}\Big\lbrace |\hat{r}_{i} - \tau | - 2s_{i}\Big\rbrace$
% \State where $s_j=\sqrt{\frac{\rho\psi_{m}\hat{v}_{j}\log{( T\epsilon_{m})}}{4 n_{j}} + \frac{\rho\psi_{m} \log{(T\epsilon_{m})}}{4 n_{j}}}$
\State $t\leftarrow t+1$ 
\vspace{-4mm}
%\ArmElim
%\State For each arm $i \in B_{m}$, remove arm ${i}$ from $B_{m}$ if
%\begin{align*}
%\hat{r}_{i} + c_i  < \tau - c_i \mbox{ or } \hat{r}_{i} - c_i  > \tau + c_i \\
%\text{where $c_i=\sqrt{\frac{\rho_{\mu}\psi_{m}\log{( T\epsilon_{m})}}{2 n_{i}}}$}
%\end{align*}
%\EndArmElim
%\ArmElimV
%\State \For{$i\in B_m$}
%\State For each arm $i \in B_{m}$, remove arm ${i}$ from $B_{m}$ if
\State \For{$i\in B_m$}
\vspace{-4mm}
\State \If{$(\hat{r}_{i} + s_i  < \tau - s_i)$ or $(\hat{r}_{i} - s_i > \tau + s_i)$}
\State $B_m\leftarrow B_m\backslash\{i\}$\hspace{4mm} (Arm deletion)
\EndIf
\EndFor
%\begin{align*}
%\hat{r}_{i} + s_i  < \tau - s_i,\hspace{1mm} \mbox{ or } \hspace{1mm}\hat{r}_{i} - s_i  > \tau + s_i \\
%% \text{where $s_i=\sqrt{\frac{\rho\psi_{m}\hat{v}_{i}\log{( T\epsilon_{m})}}{4 n_{i}} + \frac{\rho\psi_{m} \log{(T\epsilon_{m})}}{4 n_{i}}}$}
%\end{align*}
%\EndFor
%\EndArmElimV
\vspace{-2mm}
\State \If{$t\geq N_{m}$ and $m \leq M$}
%\ResetParam
\State \textbf{Reset Parameters}
\State $\epsilon_{m+1}\leftarrow\frac{\epsilon_{m}}{2}$
\State $B_{m+1} \leftarrow B_{m}$
\State $\psi_{m+1}\leftarrow \frac{T\epsilon_{m+1}}{128(\log(\frac{3}{16}K\log K))^{2}}$
\State $\ell_{m+1}\leftarrow\left\lceil \frac{2\psi_{m+1}\log( T\epsilon_{m+1})}{\epsilon_{m+1}} \right\rceil$
\State $N_{m+1} \leftarrow t + |B_{m+1}|\ell_{m+1}$
\State $m \leftarrow m+1$
%\EndResetParam
\EndIf
\EndFor
\State \textbf{Output:} $\hat{S}_{\tau}=\lbrace i: \hat{r}_{i}\geq \tau \rbrace$.
\end{algorithmic}
\end{algorithm}


%Also because of the said condition, like \cite{liu2016modification} we also claim that AugUCB is an anytime algorithm.



\section{Theoretical Results}
\label{tbandit:results}
\subsubsection{Problem Complexity}

Let us begin by recalling the following definitions of the  \emph{problem complexity} as introduced in \cite{locatelli2016optimal}:
\begin{align*}
H_{1} = \sum_{i=1}^{K}\dfrac{1}{\Delta_{i}^{2}} \hspace{1mm}\text{     and }  \hspace{1mm}
H_{CSAR,2} =\min_{i\in \mathcal{A}}\dfrac{i}{{\Delta_{(i)}^{2}}} 
\end{align*}
where $(\Delta_{(i)}: i\in\mathcal{A})$ is obtained by arranging $(\Delta_i:i\in\mathcal{A})$ in an increasing order. Also, from \cite{chen2014combinatorial} we have
\begin{align*}
H_{CSAR,2}=\max_{i\in\mathcal{A}}\frac{i}{\Delta_{(i)}^2}.
\end{align*}
$H_{CSAR,2}$ is the complexity term appearing in the bound for the CSAR algorithm. The relation between the above complexity terms are as follows (see \cite{locatelli2016optimal}):
%
%$H_1$ and $H_2$ is same as the problem complexity defined in \cite{locatelli2016optimal} for the thresholding bandit problem while $H_{CSAR,2}=\max_{i}\frac{i}{\Delta_{(i)}^2}$ is defined in \cite{chen2014combinatorial}. Also we know from \cite{locatelli2016optimal} that,
\begin{align*}
H_{1}\leq \log(2K)H_{2} \mbox{ and }
 H_1 \leq \log(K)H_{CSAR,2}.
\end{align*}

As ours is a variance-aware algorithm, we require $H_{1}^{\sigma}$ (as defined in \cite{gabillon2011multi}) that incorporates reward variances into its expression as given below:
\begin{align*}
 H_{\sigma,1}=\sum_{i=1}^{K}\frac{\sigma_{i}+\sqrt{\sigma_{i}^{2}+(16/3)\Delta_{i}}}{\Delta_{i}^{2}}.
\end{align*}
Finally, analogous to $H_{CSAR,2}$, in this paper we introduce the complexity term $H_{\sigma,2}$, which is given by
%and $H_{2}^{\sigma}$ (introduced in this paper) as,
\begin{align*}
%& H_{1}^{\sigma}=\sum_{i=1}^{K}\frac{\sigma_{i}+\sqrt{\sigma_{i}^{2}+(16/3)\Delta_{i}}}{\Delta_{i}^{2}}\\
H_{\sigma,2}=\max_{i\in \mathcal{A}} \frac{i}{\tilde{\Delta}_{(i)}^{2}}%& H_{2}^{\sigma}=\min_{i\in \mathcal{A}} i\frac{12\sigma_{(i)}^{2} + \Delta_{(i)}}{12\Delta_{(i)}^{2}}
\end{align*}
where $\tilde{\Delta}_{i}^{2}=\frac{\Delta_{i}^{2}}{\sigma_{i}+\sqrt{\sigma_{i}^{2}+(16/3)\Delta_{i}}}$, and $(\tilde{\Delta}_{(i)})$ is an increasing ordering of $(\tilde{\Delta}_{i})$. Following the results in \cite{audibert2010best}, we can show that
\begin{align*}
H_{\sigma,2}\le H_{\sigma,1}\le\overline{\log}(K) H_{\sigma,2} \le \log(2K) H_{\sigma,2}.
\end{align*}


%Similar to the relation between $H_1$ and $H_2$, it can be shown that
%%which also gives us that 
%$H_{2}^{\sigma} \leq H_{1}^{\sigma} \leq \log(2K) H_{2}^{\sigma}$.
%
%Also, from \cite{audibert2010best} we know that,
%\begin{align*}
%\sum_{i=1}^{K}\tilde{\Delta}_{i}^{-2} = \tilde{\Delta}_{(2)}^{-2} + \sum_{i=2}^{K}\frac{1}{i}i\tilde{\Delta}_{(i)}^{-2} &\leq \bar{\log K}\min_{i}i\tilde{\Delta}_{(i)}^{-2}\\
%& \leq \log(2K) H_{2}^{\sigma}, \text{ as $\bar{\log K} \leq \log(2K)$}
%\end{align*}


\subsubsection{Proof of expected loss of AugUCB}
Our main result is summarized in the following theorem where we prove an  upper bound on the expected loss. 
\begin{theorem}
\label{tbandit:Result:Theorem:1}
For $K\geq 4$ and
%with $\rho_{\mu}=\frac{1}{8}$ and 
$\rho={1}/{3}$,
the expected loss of the AugUCB algorithm is given by,
%\begin{small}
\begin{align*}
\E[\Ls(T)]
%\exp\bigg( -\frac{T\log (2 K\sqrt{\log K})}{2H_2 K (\log K)^{3/2}} + \log\bigg(K\big(\log_2\frac{T}{e}+1\big)\bigg)\bigg)\\
%& + \exp\bigg(- \frac{5T\log ( K\sqrt{\log K})}{H_{2}^{\sigma} K(\log K)^{3/2}}  + \log\bigg(K\big(\log_2\frac{T}{e}+1\big)\bigg)\bigg).
& \leq 2KT
% \bigg\lbrace\exp\bigg( -\frac{T}{ 64 H_2 a}\bigg)
% + 2
 \exp\bigg(- \frac{T}{4096 \log( K\log K) H_{\sigma,2}} \bigg).
 %\bigg\rbrace
\end{align*}

\end{theorem}

\begin{proof}\textbf{(Proof Outline)}
%\begin{discussion}
%\label{tbandit:discussion}
The proof comprises of two modules. In the first module we investigate the necessary conditions for arm elimination within a specified number of rounds, which is motivated by the technique in \cite{auer2010ucb}. Bounds on the arm-elimination probability is then obtained; however, since we use variance estimates, we invoke the Bernstein inequality (as in \cite{audibert2009exploration}, see \ref{app:bernstein}) rather that the Chernoff-Hoeffding bounds (which is appropriate for the UCB-Improved \citep{auer2010ucb}, see \ref{app:chernoff}). In the second module, as in \cite{locatelli2016optimal}, we first define a favourable event that will yield an upper bound on the expected loss. Using union bound, we then incorporate the result from module-1 (on the arm elimination probability), and finally derive the result through a series of simplifications.
%In the final module we conclude by combining the results for the first two modules. 
%\end{discussion}
The details of the proof as stated in the proof outline are as follows. 


\textbf{Arm Elimination:} Recall the notations used in the algorithm, Also, for each arm $i\in\mathcal{A}$, define $m_{i}=\min\left\lbrace m| \sqrt{\rho\epsilon_{m}}<\frac{\Delta_{i}}{2}\right\rbrace$. In the $m_i$-th round, whenever $n_i=\ell_{m_i}\ge\frac{2\psi_{m_i}\log{(T\epsilon_{m_{i}})}}{\epsilon_{m_{i}}}$, we obtain (as $\hat{v}_i\in[0,1]$)

\begin{align}
\label{tbandit:si_bound_equn}
s_i 
&\le \sqrt{\frac{\rho(\hat{v}_i+1)\epsilon_{m_i}}{8}}
% +\frac{\rho\epsilon_{m_i}}{8}}
  \le \frac{\sqrt{\rho\epsilon_{m_i}}}{2} < \frac{\Delta_i}{4}.
\end{align}

First, let us consider a bad arm $i\in\mathcal{A}$ (i.e., $r_i<\tau$). We note that, in the $m_i$-th round  whenever 
$\hat{r}_i \le r_i +2s_i$, then arm $i$ is eliminated as a bad arm. This is easy to verify as follows: using (\ref{tbandit:si_bound_equn}) we obtain,
\begin{align*}
\hat{r}_{i}\leq r_{i} + 2s_{i} 
%&= r_{i} + 4s_{i} - 2s_{i} \\
< r_{i} + \Delta_{i} - 2s_{i} 
= \tau - 2s_{i} % \geq \tau + s_{i}
\end{align*}
which is precisely one of the elimination conditions in Algorithm~\ref{alg:augucb}. Thus, the probability that a bad arm is not eliminated correctly in the $m_i$-th round (or before) is given by

%%%%%%%%%%%%%%%%% Favorable event is defined here
%We note that in the $g_i$-th round arm $i$ can be pulled no more than $\ell_{g_i}$ number of times. 
%
%
%According to the algorithm, the number of rounds is $m=\lbrace 0,1,2,.. M\rbrace $ where $M=\bigg\lfloor \frac{1}{2}\log_{2} \frac{T}{e}\bigg\rfloor$. So, $\epsilon_{m}\geq 2^{-M}\geq \sqrt{\frac{e}{T}}$. Also each round $m$ consists of $|B_{m}|\ell_{m}$ timesteps where $\ell_{m} = \left\lceil\frac{2\psi_{m}\log( T \epsilon_{m})}{\epsilon_{m}}\right\rceil$, $B_{m}$ is the set of all surviving arms and let $a=(\log(\frac{3}{16} K\log K))$.
%
%
%Let $c_{i} = \sqrt{\frac{\rho_{\mu}\psi_{m} \log{(T\epsilon_{m})}}{2 n_{i}}}$ denote the confidence interval, where $n_{i}$ is the number of times an arm $i$ is pulled. Let $\mathcal{A}^{'}=\lbrace i\in \mathcal{A}|\Delta_{i}\geq b\rbrace$, for $b\geq \sqrt{\frac{e}{T}}$. Define $m_{i}=\min\lbrace m| \sqrt{\rho_{\mu}\epsilon_{m}}<\frac{\Delta_{i}}{2}\rbrace$.
%% Let $m_{i}$ be the minimum round such that an arm $i$ gets eliminated such that. 
%
%% Let $s_{i}=\sqrt{\frac{\rho_v\psi_{g} \hat{V_{i}} \log{( T\epsilon_{g})}}{4 n_{i}} + \frac{\rho_v\psi_{g} \log{( T\epsilon_{g})}}{4 n_{i}}}$ and 
%% $g_{i}$ be the minimum round that an arm $i$ gets eliminated such that $g_{i}=min\lbrace g| \sqrt{\rho_{v}\epsilon_{g}}<\frac{\Delta_{i}}{2}\rbrace$. 
%%In this proof sub-optimal arms refer to the arms whose $r_{i}$ is lower than the threshold $\tau$.
%
%%At the end of any round $\max\lbrace m_{i},g_{i}\rbrace$, for any arm $i$, two cases are possible.
%
%Let $\xi_{1}$ and $\xi_{2}$ be the favorable event such that,
%\begin{align*}
%\xi_{1}&=\bigg\lbrace \forall i\in \mathcal{A}, \forall m=0,1,2,..,M: |\hat{r_i} - r_i| \leq 2c_i\bigg\rbrace\\
%\xi_{2}&=\bigg\lbrace \forall i\in \mathcal{A}, \forall m=0,1,2,..,M: |\hat{r_i} - r_i| \leq  2s_i\bigg\rbrace
%\end{align*}
%
%So, $\xi_{1}$ and $\xi_{2}$ signifies the event any arm $i$ will get eliminated from $B_m$.
%%%%%%%%%%%%%%%%%%%%%%







%%%%%%%%%%%%%%%%%
%\subsubsection{\textit{Arm i is not eliminated on or before round $\max\lbrace m_{i},g_{i}\rbrace$}}
%
%For any arm $i$, if it is eliminated from active set $B_{m_{i}}$ then one of the below two events has to occur,
%%\begin{small}
%\begin{align}
%\hat{r}_{i} + c_{i} < \tau - c_{i}, \label{eq:armelim-casea}\\
%\hat{r}_{i} - c_{i} > \tau + c_{i}, \label{eq:armelim-caseb}
%\end{align}
%%\end{small}
%For (\ref{eq:armelim-casea}) we can see that it eliminates arms that have performed poorly and removes them  from $B_{m_{i}}$. Similarly, (\ref{eq:armelim-caseb}) eliminates arms from $B_{m_{i}}$ that have performed very well compared to threshold $\tau$.
%
%%Each round consists of $|B_{m_{i}}|\ell_{m_{i}}$ timesteps. 
%In the $m_{i}$-th round an arm $i$ can be pulled no more than $\ell_{m_{i}}$ times. So when $n_{i}=\ell_{m_{i}}$, putting the value of $\ell_{m_{i}}\ge\frac{2\psi_{m_i}\log{( T\epsilon_{m_{i}})}}{\epsilon_{m_{i}}}$ in $c_{i}$ we get, 
%%\begin{small}
%\begin{align*}
%c_{i}
%&=\sqrt{\frac{\rho_{\mu}\psi_{m_i}\epsilon_{m_{i}}\log ( T\epsilon_{m_{i}})}{2 n_{i}}}
%\le\sqrt{\frac{\rho_{\mu}\psi_{m_i}\epsilon_{i}\log ( T\epsilon_{m_{i}})}{2*2 \psi_{m_i} \log( T\epsilon_{m_{i}})}}\\
%& \le\frac{\sqrt{\rho_{\mu}\epsilon_{m_{i}}}}{2}
%% % \leq \sqrt{\rho_{\mu}\epsilon_{m_{i}+1}} 
%< \frac{\Delta_{i}}{4} \text{, as }\rho_{\mu}\in (0,1].
%\end{align*}
%%\end{small}
%Again, for ${i} \in \mathcal{A}^{'}$ for the  elimination condition in (\ref{eq:armelim-casea}), 
%%\begin{small}
%%\begin{align*}
%%\hat{r}_{i} + c_{i}&\leq r_{i} + 2c_{i} = r_{i} + 4c_{i} - 2c_{i} \\
%%&< r_{i} + \Delta_{i} - 2c_{i} = \tau -2c_{i} \leq \tau - c_{i}
%%\end{align*}
%%\end{small}
%%\begin{small}
%\begin{align*}
%\hat{r}_{i} &\leq r_{i} + 2c_{i} = r_{i} + 4c_{i} - 2c_{i} \\
%&< r_{i} + \Delta_{i} - 2c_{i} = \tau -2c_{i}.
%\end{align*}
%%\end{small}
%Similarly, for ${i} \in \mathcal{A}^{'}$ for the  elimination condition in (\ref{eq:armelim-caseb}), 
%%\begin{small}
%\begin{align*}
%\hat{r}_{i} &\geq r_{i} - 2c_{i} = r_{i} - 4c_{i} + 2c_{i} \\
%&> r_{i} - \Delta_{i} + 2c_{i}= \tau + 2c_{i}.
%\end{align*}
%%\end{small}
%
%
%%Now, arm elimination condition is being checked at every timestep, in the $m_{i}$-th round as soon as $n_{i}=\ell_{m_{i}}$, arm $i$ gets eliminated. 
%Applying Chernoff-Hoeffding bound and considering independence of complementary of the event in (\ref{eq:armelim-casea}),
%%\begin{small}
%\begin{align*}
%%\mathbb{P}\lbrace\hat{r}_{i}\geq r_{i} - 2c_{i}\rbrace &\leq exp(-2(\tau + 2c_{i})^{2}n_{i})\\
%&\mathbb{P}\lbrace\hat{r}_{i}> r_{i} + 2c_{i}\rbrace \leq \exp(-4 c_{i}^{2}n_{i})\\
%&\leq \exp(-8 * \dfrac{\rho_{\mu}\psi_{m_i}\log ( T\epsilon_{m_{i}})}{2 n_{i}} *n_{i})\\
%&\leq \exp\big(-4\rho_{\mu}\psi_{m_i}\log ( T\epsilon_{m_{i}})\big)\\
%&\leq \exp\left(-\rho_{\mu}\frac{T\epsilon_{m_{i}}}{32 a^2}\log ( T\epsilon_{m_{i}})\right),\\
%&\text{putting the value of $\psi_{m_i}=\frac{T\epsilon_{m_i}}{128(\log(\frac{3}{16} K\log K))^{2}}$}
%\end{align*}
%%\end{small}
%Similarly for the condition in (\ref{eq:armelim-caseb}), $\mathbb{P}\lbrace\hat{r}_{i}< r_{i} - 2c_{i}\rbrace\leq \exp\left(-\frac{T\rho_{\mu}\epsilon_{m_{i}}}{32 a^2 }\log ( T\epsilon_{m_{i}})\right)$.
%
%Summing the above two expressions, the probability that arm ${i}$ is not eliminated on or before $m_{i}$-th is $\left(2\exp\left(-\frac{T\rho_{\mu}\epsilon_{m_{i}}}{32 a^2 }\log ( T\epsilon_{m_{i}})\right)\right)$. 
%%%%%%%%%%%%%%%%%%%%%

%%%%%%%%%%%%
%Again for any arm $i$, if it is eliminated from active set $B_{g_{i}}$ then the below two events have to come true,
%%\begin{small}
%\begin{align}
%\hat{r}_{i} + s_{i} < \tau - s_{i}, \label{eq:armelim-var-casea}\\
%\hat{r}_{i} - s_{i} > \tau + s_{i}, \label{eq:armelim-var-caseb}
%\end{align}
%%\end{small}
%%
%% For \ref{eq:armelim-var-casea} we can see that it eliminates arms that have performed poorly and removes them them from $B_{g_{i}}$. Similarly, \ref{eq:armelim-var-caseb} eliminates arms from $B_{g_{i}}$ that have performed very well compared to threshold $\tau$.
%%But, we know that $\epsilon_{m_{i}}=\epsilon_{g_{i}}$ and round consist of $|B_{g_{i}}|\ell_{g_{i}}$ timesteps. 
%In the $g_{i}$-th round an arm $i$ can be pulled no more than $\ell_{g_{i}}$ times. So when $n_{i}=\ell_{g_{i}}$, putting the value of $\ell_{g_{i}}\ge\frac{2\psi_{m_i}\log{( T\epsilon_{g_{i}})}}{\epsilon_{g_{i}}}$ in $s_{i}$ we get, 
%%\begin{small}
%\begin{align*}
%s_{i}&=\sqrt{\dfrac{\rho_v \psi_{g_i} \hat{V}_{i} \epsilon_{g_{i}}\log ( T\epsilon_{g_{i}})}{4 n_{i}} + \dfrac{\rho_v \psi_{g_i}\log{( T\epsilon_{g_{i}})}}{4 n_{i}}} \\
%&\leq \sqrt{\dfrac{\rho_v\psi_{g_i} \epsilon_{g_{i}}\log ( T\epsilon_{g_{i}})}{4*2 \log(\psi_{g_i} T\epsilon_{g_{i}})} + \dfrac{\rho_v \psi_{g_i}\epsilon_{g_{i}} \log{( T\epsilon_{g_{i}})}}{4*2\psi_{g_i} \log( T\epsilon_{g_{i}})} } \text{, as }\hat{V}_{i}\in [0,1].\\
%& \leq \sqrt{\dfrac{\rho_v \epsilon_{g_{i}}}{8} + \dfrac{\rho_v \epsilon_{g_{i}}}{8} } \leq \dfrac{\sqrt{\rho_v \epsilon_{g_{i}}}}{2}< \dfrac{\Delta_{i}}{4} \text{, as }\rho_v\in (0,1].
%%& \leq \sqrt{\rho_v \epsilon_{g_{i}+1}} < \dfrac{\Delta_{i}}{4} \text{, as }\rho_v\in (0,1].
%\end{align*}
%%\end{small}
%
%Again, for ${i} \in \mathcal{A}^{'}$ for the elimination condition in (\ref{eq:armelim-var-casea}),
%%\begin{small}
%\begin{align*}
%\hat{r}_{i} &\leq r_{i} + 2s_{i} = r_{i} + 4s_{i} - 2s_{i} \\
%&< r_{i} + \Delta_{i} - 2s_{i} = \tau -2s_{i} % \leq \tau - s_{i}
%\end{align*}
%%\end{small} 
%
%
%Also, for ${i} \in \mathcal{A}^{'}$ for the elimination condition in (\ref{eq:armelim-var-caseb}), 
%%\begin{small}
%\begin{align*}
%\hat{r}_{i}&\geq r_{i} - 2s_{i} = r_{i} - 4s_{i} + 2s_{i} \\
%&> r_{i} - \Delta_{i} + 2s_{i}\geq \tau + 2s_{i} % \geq \tau + s_{i}
%\end{align*}
%%\end{small}
%%%%%%%%%%%%%%%%


%Since, arm elimination condition is being checked at every timestep, in the $g_{i}$-th round as soon as $n_{i}=\ell_{g_{i}}$, arm $i$ gets eliminated. 
% Applying Bernstein inequality and considering independence of complementary of the event in (\ref{eq:armelim-var-casea}),
%\begin{small}
\noindent
\begin{align}
\mathbb{P}(\hat{r}_{i}> r_{i} + 2s_{i})
% &= \mathbb{P}\bigg( \hat{r}_{i} > r_{i}+ 2\sqrt{\dfrac{\rho\psi_{m_i} \hat{v}_{i}\log( T\epsilon_{m_{i}}) + \rho\psi_{m_i} \log{( T\epsilon_{m_{i}})}}{4n_{i}} } \bigg)\nonumber\\
&\leq \mathbb{P}\left( \hat{r}_{i} > r_{i}+ 2\bar{s}_i\right)  % \label{eq:prob_eq1}\\ 
+ \mathbb{P}\left( \hat{v}_{i}\geq \sigma_{i}^{2}+\sqrt{\rho\epsilon_{m_{i}}}\right)\label{tbandit:eq:prob_eq2}
\end{align}
where 
\begin{align*}
\bar{s}_i=\sqrt{\dfrac{\rho\psi_{m_i} (\sigma_{i}^{2}+\sqrt{\rho\epsilon_{m_{i}}} + 1)\log( T\epsilon_{m_{i}})}{4n_{i}}}
\end{align*}
%\end{small}
Note that, substituting $n_i=\ell_{m_i}\ge \frac{2\psi_{m_i}\log{(T\epsilon_{m_{i}})}}{\epsilon_{m_{i}}}$, $\bar{s}_i$ can be simplified to obtain,
\begin{align}
2\bar{s}_i
% &\le 2\sqrt{\dfrac{\rho\psi_{m_i} (\sigma_{i}^{2}+\sqrt{\rho\epsilon_{m_{i}}})\log( T\epsilon_{m_{i}})}{\frac{8\psi_{m_i}\log( T \epsilon_{m_{i}})}{\epsilon_{m_{i}}}} }
%+ \dfrac{\rho\psi_{m_i} \log{( T\epsilon_{m_{i}})}}{\frac{8\psi_{m_i}\log( T \epsilon_{m_{i}})}{\epsilon_{m_{i}}}}}
\leq \dfrac{\sqrt{\rho\epsilon_{m_{i}}(\sigma_{i}^{2}+\sqrt{\rho\epsilon_{m_{i}}} + 1)}}{2}\leq \sqrt{\rho \epsilon_{m_{i}}}.
\label{tbandit:si_bar_equn}
\end{align}

%Now, we know that in the $g_{i}$-th round,
%%\begin{small}
%\begin{align*}
%& 2\sqrt{\dfrac{\rho_v\psi_{g_i} [\sigma_{i}^{2}+\sqrt{\rho_{v}\epsilon_{g_{i}}}]\log( T\epsilon_{g_{i}})}{4n_{i}} + \dfrac{\rho_v\psi_{g_i}  \log{(T\epsilon_{g_{i}})}}{4 n_{i}}}\\ &\leq  2\sqrt{\dfrac{\rho_v\psi_{g_i} [\sigma_{i}^{2}+\sqrt{\rho_{v}\epsilon_{g_{i}}}]\log( T\epsilon_{g_{i}})}{\frac{8\psi_{g_i}\log( T \epsilon_{g_{i}})}{\epsilon_{g_{i}}}} + \dfrac{\rho_v\psi_{g_i} \log{( T\epsilon_{g_{i}})}}{\frac{8\psi_{g_i}\log( T \epsilon_{g_{i}})}{\epsilon_{g_{i}}}}}\\
%& \leq \dfrac{\sqrt{\rho_v \epsilon_{g_{i}}[\sigma_{i}^{2}+\sqrt{\rho_{v}\epsilon_{g_{i}}} + 1]}}{2}\leq \sqrt{\rho_v \epsilon_{g_{i}}}
%\end{align*}
%%\end{small}
%--------------------

The first term in the LHS of (\ref{eq:prob_eq2}) can be bounded using the Bernstein inequality as below:
\begin{align}
&\mathbb{P}\left( \hat{r}_{i} > r_{i}+ 2\bar{s}_i\right)\nonumber \\
&\le \exp\left(- \dfrac{(2\bar{s}_i)^2 n_i}{2\sigma_i^2+\frac{4}{3}\bar{s}_i}\right)\nonumber \\
& \le \exp\left(- \dfrac{\rho\psi_{m_i} (\sigma_{i}^{2}+\sqrt{\rho\epsilon_{m_{i}}} + 1)\log( T\epsilon_{m_{i}})}{2\sigma_i^2+\frac{2}{3}\sqrt{\rho \epsilon_{m_{i}}}}\right)\nonumber \\
& \overset{(a)}{\leq} \exp\left(- \dfrac{3\rho T\epsilon_{m_i}}{256 a^2} \left(\dfrac{\sigma_{i}^{2}+\sqrt{\rho\epsilon_{m_{i}}}+1}{3\sigma_{i}^{2}+\sqrt{\rho \epsilon_{m_{i}}}}\right) \log( T\epsilon_{m_{i}}) \right) \nonumber \\
&:= \exp(-Z_i) 
\label{tbandit:lhs1_equn}
\end{align}
where, for simplicity, we have used $\alpha_i$ to denoted the exponent in the inequality $(a)$.
Also, note that $(a)$ is obtained by using  $\psi_{m_i}=\frac{T\epsilon_{m_i}}{128a^{2}}$, where $a=(\log(\frac{3}{16} K\log K))$.
%For the term in (\ref{eq:prob_eq1}), by applying Bernstein inequality, we can write as,
%\begin{small}
%\begin{align*}
%&\mathbb{P}\bigg( \hat{r}_{i}> r_{i} + \bigg(2\sqrt{\frac{\rho_v\psi_{g_i} [\sigma_{i}^{2}+\sqrt{\rho_{v}\epsilon_{g_{i}}} + 1]\log( T\epsilon_{g_{i}})}{4n_{i}}  } \bigg)\bigg)\\
%%%%%%%%%%%%%%%%%%%%%%%%
% &\leq \exp\bigg(- \dfrac{\bigg(2\sqrt{\frac{\rho_v\psi_{g_i} [\sigma_{i}^{2}+\sqrt{\rho_{v}\epsilon_{g_{i}}} +1]\log( T\epsilon_{g_{i}})}{4n_{i}}}\bigg)^{2}n_{i}}{2\sigma_{i}^{2}+\frac{4}{3}\sqrt{\frac{\rho_v\psi_{g_i} [\sigma_{i}^{2}+\sqrt{\rho_{v}\epsilon_{g_{i}}}+1]\log( T\epsilon_{g_{i}})}{4n_{i}}}}\bigg) \\
%%%%%%%%%%%%%%%%%%%%%%%
%&\leq \exp\bigg(- \dfrac{\bigg(\rho_v\psi_{g_i} [\sigma_{i}^{2}+\sqrt{\rho_{v}\epsilon_{g_{i}}} + 1]\log( T\epsilon_{g_{i}})\bigg)}{2\sigma_{i}^{2}+\frac{2}{3}\sqrt{\rho_v \epsilon_{g_{i}}}} \bigg)\\
% &\leq \exp\bigg(- \dfrac{3\rho_v\psi_{g_i}}{2} \bigg(\dfrac{\sigma_{i}^{2}+\sqrt{\rho_{v}\epsilon_{g_{i}}}+1}{3\sigma_{i}^{2}+\sqrt{\rho_v \epsilon_{g_{i}}}}\bigg) \log( T\epsilon_{g_{i}}) \bigg)\\
%%%%%%%%%%%%%%%%%%%%%%%
% &\leq \exp\left(- \dfrac{3\rho_v T\epsilon_{g_i}}{256 a^2} \left(\dfrac{\sigma_{i}^{2}+\sqrt{\rho_{v}\epsilon_{g_{i}}}+1}{3\sigma_{i}^{2}+\sqrt{\rho_v \epsilon_{g_{i}}}}\right) \log( T\epsilon_{g_{i}}) \right),
% &\text{ putting the value of $\psi_{g_i}=\frac{T\epsilon_{g_i}}{128(\log(\frac{3}{16} K\log K))^{2}}$}
%%%%%%%%%%%%%%%%%%%%%%%%%%%%%%%%%%%%%%%%%%%%%%%%%%%%%%%%%%%%%%%%%%%%%%%%%%%%%%%%%%
%\begin{align*}
%&\mathbb{P}\bigg\lbrace \hat{r}_{i}> r_{i} + \bigg(2\sqrt{\frac{\rho_v\psi_{g_i} [\sigma_{i}^{2}+\sqrt{\rho_{v}\epsilon_{g_{i}}} + 1]\log( T\epsilon_{g_{i}})}{4n_{i}}  } \bigg)\bigg\rbrace\\
%&\leq \exp\bigg(- \dfrac{\bigg(2\sqrt{\frac{\rho_v\psi_{g_i} [\sigma_{i}^{2}+\sqrt{\rho_{v}\epsilon_{g_{i}}}]\log( T\epsilon_{g_{i}})}{4n_{i}} + \frac{\rho_v\psi_{g_i} \log{( T\epsilon_{g_{i}})}}{4 n_{i}}}\bigg)^{2}n_{i}}{2\sigma_{i}^{2}+\frac{4}{3}\sqrt{\frac{\rho_v\psi_{g_i} [\sigma_{i}^{2}+\sqrt{\rho_{v}\epsilon_{g_{i}}}]\log( T\epsilon_{g_{i}})}{4n_{i}}+\frac{\rho_v\psi_{g_i} \log{( T\epsilon_{g_{i}})}}{4 n_{i}}}}\bigg) \\
%&\leq \exp\bigg(- \dfrac{\bigg(\rho_v\psi_{g_i} [\sigma_{i}^{2}+\sqrt{\rho_{v}\epsilon_{g_{i}}} + 1]\log( T\epsilon_{g_{i}})\bigg)}{2\sigma_{i}^{2}+\frac{2}{3}\sqrt{\rho_v \epsilon_{g_{i}}}} \bigg)\\
%&\leq \exp\bigg(- \dfrac{3\rho_v\psi_{g_i}}{2} \bigg(\dfrac{\sigma_{i}^{2}+\sqrt{\rho_{v}\epsilon_{g_{i}}}+1}{3\sigma_{i}^{2}+\sqrt{\rho_v \epsilon_{g_{i}}}}\bigg) \log( T\epsilon_{g_{i}}) \bigg)\\
%&\leq \exp\left(- \dfrac{3\rho_v T\epsilon_{g_i}}{16 K\log K} \left(\dfrac{\sigma_{i}^{2}+\sqrt{\rho_{v}\epsilon_{g_{i}}}+1}{3\sigma_{i}^{2}+\sqrt{\rho_v \epsilon_{g_{i}}}}\right) \log( T\epsilon_{g_{i}}) \right),\\
%&\text{ putting the value of $\psi_{g_i}=\frac{T\epsilon_{g_i}}{128(\log(\frac{3}{16} K\log K))^{2}}$}
%%%%%%%%%%%%%%%%%%%%%%%%%%%%%%%%%%%%%%%%%%%%%%%%%%%%%%%%%%%%%%%%%%%%%%%%%%%%%%%%%%
%\end{align*}
%\end{small}
% where  the last inequality is obtained using 
% \begin{align*}
% \psi_{m_i}=\frac{T\epsilon_{m_i}}{128(\log(\frac{3}{16} K\log K))^{2}}.
% \end{align*}
 
 The second term in the LHS of (\ref{tbandit:eq:prob_eq2}) can be simplified as follows:
% For the term in , by applying Bernstein inequality, we can write as,
%\begin{small}
\begin{align}
&\mathbb{P}\bigg\lbrace \hat{v}_{i}\geq \sigma_{i}^{2}+\sqrt{\rho\epsilon_{m_{i}}}\bigg\rbrace\nonumber\\
&\leq \mathbb{P}\bigg\lbrace \dfrac{1}{n_{i}}\sum_{t=1}^{n_{i}}(X_{i,t}-r_{i})^{2}-(\hat{r}_{i}-r_{i})^{2}\geq \sigma_{i}^{2}+\sqrt{\rho\epsilon_{m_{i}}}\bigg\rbrace\nonumber\\
&\leq \mathbb{P}\bigg\lbrace \dfrac{\sum_{t=1}^{n_{i}}(X_{i,t}-r_{i})^{2}}{n_{i}}\geq \sigma_{i}^{2}+\sqrt{\rho\epsilon_{m_{i}}} \bigg\rbrace\nonumber\\
&\overset{(a)}{\leq} \mathbb{P}\bigg\lbrace \dfrac{\sum_{t=1}^{n_{i}}(X_{i,t}-r_{i})^{2}}{n_{i}}\geq \sigma_{i}^{2} + 2\bar{s}_i\bigg\rbrace \nonumber\\
% &\bigg(2\sqrt{\dfrac{\rho_v\psi_{g_i} [\sigma_{i}^{2}+\sqrt{\rho_{v}\epsilon_{g_{i}}}]\log( T\epsilon_{g_{i}})}{4n_{i}}+\frac{\rho_v\psi_{g_i}  \log{(T\epsilon_{g_{i}})}}{4 n_{i}}}\bigg)\bigg\rbrace\\
&\overset{(b)}{\leq} \exp\bigg(- \dfrac{3\rho\psi_{m_i}}{2} \bigg(\dfrac{\sigma_{i}^{2}+\sqrt{\rho\epsilon_{m_{i}}}+1}{3\sigma_{i}^{2}+\sqrt{\rho \epsilon_{m_{i}}}}\bigg) \log( T\epsilon_{m_{i}}) \bigg)\nonumber \\
%&\leq \exp\bigg(- \dfrac{3\rho_vT\epsilon_{g_i}}{256 a^2 } \bigg(\dfrac{\sigma_{i}^{2}+\sqrt{\rho_{v}\epsilon_{g_{i}}}+1}{3\sigma_{i}^{2}+\sqrt{\rho_v \epsilon_{g_{i}}}}\bigg) \log( T\epsilon_{g_{i}}) \bigg)
& = \exp(-Z_i)
%&\text{ putting the value of $\psi_{g_i}=\frac{T\epsilon_{g_i}}{128(\log(\frac{3}{16} K\log K))^{2}}$}
\label{tbandit:lhs2_equn}
\end{align}
%\end{small}
where inequality $(a)$ is obtained using (\ref{tbandit:si_bar_equn}), while $(b)$ follows from the Bernstein inequality. 
  
Thus, using (\ref{tbandit:lhs1_equn}) and (\ref{tbandit:lhs2_equn}) in (\ref{tbandit:eq:prob_eq2}) we obtain $\mathbb{P}(\hat{r}_{i}> r_{i} + 2s_{i})\le 2\exp(-Z_i)$.
% \begin{small}
% \begin{align*}
%& \mathbb{P}(\hat{r}_{i}> r_{i} + 2s_{i}) \le\\ 
%&2\exp\left(- \frac{3T\rho_v\epsilon_{g_{i}}}{256 a^2 } \left(\frac{\sigma_{i}^{2}+\sqrt{\rho_{v}\epsilon_{g_{i}}}+1}{3\sigma_{i}^{2}+\sqrt{\rho_v \epsilon_{g_{i}}}}\right) \log( T\epsilon_{g_{i}}) \right)
% \end{align*}
% \end{small}
 %
Proceeding similarly, for a good arm $i\in\mathcal{A}$, the probability that it is not correctly eliminated in the $m_i$-th round (or before) is also bounded by $\mathbb{P}(\hat{r}_{i}< r_{i} - 2s_{i})\le 2\exp(-Z_i)$. In general, for any $i\in\mathcal{A}$ we have
\begin{align}
\Pb(|\hat{r}_i-r_i|>2s_i) 
&\le4\exp(-Z_i).
\label{tbandit:final_bound_equn}
\end{align}
  
  
%Similarly, the condition for the complementary event for the elimination case \ref{eq:armelim-var-caseb} holds such that $\mathbb{P}\lbrace\hat{r}_{i}< r_{i} - 2s_{i}\rbrace \leq 2\exp\left(- \frac{3T\rho_v\epsilon_{g_{i}}}{256 a^2 } \left(\frac{\sigma_{i}^{2}+\sqrt{\rho_{v}\epsilon_{g_{i}}}+1}{3\sigma_{i}^{2}+\sqrt{\rho_v \epsilon_{g_{i}}}}\right) \log( T\epsilon_{g_{i}}) \right)$.


\textbf{Favourable Event:} Following the notation in \cite{locatelli2016optimal} we define the event
\begin{align*}
\xi&=\bigg\lbrace \forall i\in \mathcal{A}, \forall t=1,2,..,T: |\hat{r_i} - r_i| \leq  2s_i\bigg\rbrace.
\end{align*}
Note that, on $\xi$ each arm $i\in \mathcal{A}$  is eliminated correctly in the $m_i$-th round (or before). Thus, it follows that $\mathbb{E}[\mathcal{L}(T)]\le P(\xi^c)$. Since $\xi^c$ can be expressed as an union of the events $(|\hat{r}_i-r_i|>2s_i)$ for all $i\in\mathcal{A}$ and all $t=1,2,\cdots,T$, using union bound we can write
\begin{align*}
&\mathbb{E}[\mathcal{L}(T)] 
\le \sum_{i\in\mathcal{A}}\sum_{t=1}^T \Pb(|\hat{r}_i-r_i|>2s_i) \\
&\le \sum_{i\in\mathcal{A}}\sum_{t=1}^T 4 \exp(-Z_i) \\
&\le 4T\sum_{i\in\mathcal{A}} \exp\left(- \dfrac{3\rho T\epsilon_{m_i}}{256 a^2} \left(\dfrac{\sigma_{i}^{2}+\sqrt{\rho\epsilon_{m_{i}}}+1}{3\sigma_{i}^{2}+\sqrt{\rho \epsilon_{m_{i}}}}\right) \log( T\epsilon_{m_{i}}) \right) \\
&\overset{(a)}{\le} 4T \sum_{i\in\mathcal{A}} \exp\left(- \frac{3T\Delta_{i}^{2}}{4096 a^2} \left(\frac{4\sigma_{i}^{2}+\Delta_{i}+4}{12\sigma_{i}^{2}+\Delta_{i}}\right) \log( \frac{3}{16} T\Delta_{i}^{2}) \right) \\
&\overset{(b)}{\le} 4T \sum_{i\in\mathcal{A}}\exp\bigg(- \frac{12T\Delta_{i}^{2}}{(12\sigma_{i}+ 12\Delta_{i})}\frac{\log (\frac{3}{16} K\log K)}{4096 a^2 } \bigg) \\
&\overset{(c)}{\le} 4T \sum_{i\in\mathcal{A}} \exp\bigg(- \frac{T\Delta_{i}^{2}\log ( \frac{3}{16} K\log K)}{4096 (\sigma_{i} + \sqrt{\sigma_{i}^{2} + (16/3)\Delta_{i}}) a^2} \bigg) \\
& \overset{(d)}{\le} 4T \sum_{i\in\mathcal{A}} \exp\bigg(- \frac{T\log ( \frac{3}{16} K\log K)}{4096 \tilde{\Delta}_i^{-2} a^2} \bigg) \\
& \overset{(e)}{\le}4T \sum_{i\in\mathcal{A}} \exp\bigg(- \frac{T\log ( \frac{3}{16} K\log K)}{4096 \max_{j}(j\tilde{\Delta}_{(j)}^{-2}) (\log(\frac{3}{16} K\log K))^{2}} \bigg) \\
& \overset{(f)}{\le}4KT \exp\bigg(- \frac{T}{4096 \log(K\log K)H_{\sigma,2}}\bigg).
\end{align*}
The justification for the above simplifications are as follows:
%\begin{itemize}

\noindent
 $\bullet$ $(a)$ is obtained by noting that in round $m_i$ we have 
 $\frac{\Delta_i}{4}\leq\sqrt{\rho\epsilon_{m_{i}}}<\frac{\Delta_i}{2}.$

\noindent
 $\bullet$ For $(b)$, we note that the function $x\mapsto x\exp(-Cx^2)$, where $x\in[0,1]$, is  decreasing on $[1/\sqrt{2C},1]$ for any $C>0$ (see \cite{bubeck2011pure,auer2010ucb}). Thus, using $C=\lfloor T/e\rfloor$ and $\min_{j\in \mathcal{A}}\Delta_j =\Delta =\sqrt{\frac{K\log K}{T}} > \sqrt{\frac{e}{T}}$,
%\forall i\in \mathcal{A}$ 
we obtain (b).

\noindent
 $\bullet$
To obtain $(c)$ we have used the inequality $\Delta_i\le \sqrt{\sigma_{i}^{2} + (16/3)\Delta_{i}}$ (which holds because $\Delta_i\in[0,1]$).

\noindent
 $\bullet$
 $(d)$ is obtained simply by substituting $\tilde{\Delta}_i=\frac{\Delta_{i}^{2}}{\sigma_{i}+\sqrt{\sigma_{i}^{2}+(16/3)\Delta_{i}}}$ and $a=\log(\frac{3}{16} K\log K)$.
 
 \noindent
 $\bullet$
 Finally, to obtain $(e)$ and $(f)$, note that 
%\begin{align*}
$\tilde{\Delta}_i^{-2}\le i\tilde{\Delta}_i^{-2} \le \max_{j\in\mathcal{A}}j\Delta_{(j)}^{-2}=H_{\sigma,2}.$
%\end{align*}
%\end{itemize}
\end{proof}

%Again  summing the above expressions, the probability that an arm ${i}$ is not eliminated on or before $g_{i}$-th round based on the (\ref{eq:armelim-var-casea}) and (\ref{eq:armelim-var-caseb}) elimination condition is  $4\exp\left(- \frac{3T\rho_v\epsilon_{g_{i}}}{256 a^2 } \left(\frac{\sigma_{i}^{2}+\sqrt{\rho_{v}\epsilon_{g_{i}}}+1}{3\sigma_{i}^{2}+\sqrt{\rho_v \epsilon_{g_{i}}}}\right) \log( T\epsilon_{g_{i}}) \right)$. 
  
%%%%%%%%%%%%%%%%%%%%%%%%%%%%%%%%%%%%%%%%%%%%%%%%%%%%%%%%%%%%%%%%%%%%%%%%%%%%%%%%%%%%%%
%Not Required for probability of error for AugUCB
%%%%%%%%%%%%%%%%%%%%%%%%%%%%%%%%%%%%%%%%%%%%%%%%%%%%%%%%%%%%%%%%%%%%%%%%%%%%%%%%%%%%%%

%We start with an upper bound on the number of plays $\delta_{\max\lbrace m_{i}, g_{i}\rbrace}$ in the $\max\lbrace m_{i}, g_{i}\rbrace$-th round. We know that the total number of arms surviving in the $\max\lbrace m_{i}, g_{i}\rbrace$-th arm is, 
%
%\begin{small}
%\begin{align*}
%&|B_{\max\lbrace m_{i}, g_{i}\rbrace}|=2K\exp\bigg(-4\rho_{\mu}\log (\psi T\epsilon_{m_{i}}^{2})\bigg)\\ 
%& + 4K\exp\bigg(- \frac{3\rho_v}{2} \big(\frac{\sigma_{i}^{2}+\sqrt{\rho_{v}\epsilon_{g_{i}}}+1}{3\sigma_{i}^{2}+\sqrt{\rho_v \epsilon_{g_{i}}}}\big) \log(\psi T\epsilon_{g_{i}}^{2}) \bigg)
%\end{align*}     
%\end{small}
%
%
%Again for AugUCB, we know that the number of pulls allocated for each surviving arm $i$ in the $m_{i}$-th round is $\ell_{m_{i}}=\frac{2\log (\psi T \epsilon_{m_{i}}^{2})}{\epsilon_{m_{i}}}$ or for the $g_{i}$-th round is $\ell_{g_{i}}=\frac{2\log (\psi T \epsilon_{g_{i}}^{2})}{\epsilon_{g_{i}}}$. Therefore, the proportion of plays $\delta_{\max\lbrace m_{i}, g_{i}\rbrace}$ in the $\max\lbrace m_{i}, g_{i}\rbrace$-th round can be written as,
%
%\begin{small}
%\begin{align*}
%&\delta_{\max\lbrace m_{i}, g_{i}\rbrace}=(|B_{m_{i}}|.\ell_{m_{i}}) + (|B_{g_{i}}|.\ell_{g_{i}})\\
%&\leq 2K\exp\bigg(-4\rho_{\mu}\log (\psi T\epsilon_{m_{i}}^{2})\bigg).\dfrac{2\log (\psi T \epsilon_{m_{i}}^{2})}{\epsilon_{m_{i}}}\\
% & + 4K\exp\bigg(- \dfrac{3\rho_v}{2} \bigg(\dfrac{\sigma_{i}^{2}+\sqrt{\rho_{v}\epsilon_{g_{i}}}+1}{3\sigma_{i}^{2}+\sqrt{\rho_v \epsilon_{g_{i}}}}\bigg) \log(\psi T\epsilon_{g_{i}}^{2})\bigg).\dfrac{2\log (\psi T \epsilon_{g_{i}}^{2})}{\epsilon_{g_{i}}} \\
%& \leq \dfrac{4K\log (\psi T \epsilon_{m_{i}}^{2})}{\epsilon_{m_{i}}}\exp\bigg(-4\rho_{\mu}\log (\psi T\epsilon_{m_{i}}^{2})\bigg)\\
%& + \dfrac{8K\log (\psi T \epsilon_{g_{i}}^{2})}{\epsilon_{g_{i}}}\exp\bigg(- \dfrac{3\rho_v}{2} \bigg(\dfrac{\sigma_{i}^{2}+\sqrt{\rho_{v}\epsilon_{g_{i}}}+1}{3\sigma_{i}^{2}+\sqrt{\rho_v \epsilon_{g_{i}}}}\bigg) \log(\psi T\epsilon_{g_{i}}^{2}) \bigg)
%\end{align*}
%\end{small}

%Now, in the $\max\lbrace m_{i}, g_{i}\rbrace$-th round $\sqrt{\rho_{\mu}\epsilon_{m_{i}}}\leq \frac{\Delta_{i}}{2}$ or $\sqrt{\rho_v\epsilon_{g_{i}}}\leq \frac{\Delta_{i}}{2}$. Hence,
%
%\begin{small}
%\begin{align*}
%&\delta_{\max\lbrace m_{i},g_{i}\rbrace} \leq \dfrac{4K\log (\psi T \frac{\Delta_{i}^{4}}{16\rho_{\mu}^{2}})}{\frac{\Delta_{i}^{2}}{4\rho_{\mu}}}\exp\bigg(-4\rho_{\mu}\log (\psi T\frac{\Delta_{i}^{4}}{16\rho_{\mu}^{2}})\bigg)\\
%& + \dfrac{8K\log (\psi T \frac{\Delta_{i}^{4}}{16\rho_{v}^{2}})}{\frac{\Delta_{i}^{2}}{4\rho_{v}}}\exp\bigg(- \dfrac{3\rho_v}{2} \bigg(\dfrac{\sigma_{i}^{2}+\frac{\Delta_{i}}{2}+1}{3\sigma_{i}^{2}+\frac{\Delta_{i}}{2}}\bigg) \log(\psi T\frac{\Delta_{i}^{4}}{16\rho_{v}^{2}}) \bigg)\\
%%%%%%%%%%%%%%%%%%%%%%%%%%%%%%%%%%%%%%%%
%&\leq 16 C_1\exp\bigg(-4\rho_{\mu}\log (\psi T\frac{\Delta_{i}^{4}}{16\rho_{\mu}^{2}})\bigg)\\
%& + 32C_2\exp\bigg(- \dfrac{3\rho_v}{2} \bigg(\dfrac{2\sigma_{i}^{2}+\Delta_{i}+2}{6\sigma_{i}^{2}+\Delta_{i}}\bigg) \log(\psi T\frac{\Delta_{i}^{4}}{16\rho_{v}^{2}}) \bigg)\\
%&\text{where $C_1=\frac{K\rho_{\mu}\log (\psi T \frac{\Delta_{i}^{4}}{16\rho_{\mu}^{2}})}{\Delta_{i}^{2}}$ and $C_2= \frac{K\rho_v\log (\psi T \frac{\Delta_{i}^{4}}{16\rho_{v}^{2}})}{\Delta_{i}^{2}}$}\\
%%%%%%%%%%%%%%%%%%%%%%%%%%%%%%%%%%%%%%%%
%&\leq 16 C_1\exp\bigg(-4\rho_{\mu}\log (\psi T\frac{\Delta_{i}^{4}}{16\rho^{2}})\bigg)
% + 32C_2\exp\bigg(- \dfrac{3\rho_v}{2} \log(\psi T\frac{\Delta_{i}^{4}}{16\rho_{v}^{2}}) \bigg)
%\end{align*}
%\end{small}
%
%%Summing over all rounds $m=0,1,..,M$,
%Now, putting the values of $\psi$, $\rho_{\mu}$, $\rho_v$ and taking $\Delta_{i}\geq\min_{i\in A}\Delta=\sqrt{\frac{K\log K}{T}}\geq \sqrt{\frac{e}{T}},\forall i\in A$( see \cite{auer2010ucb}), 
%
%\begin{small}
%\begin{align*}
%& \delta_{\max\lbrace m_{i}, g_{i}\rbrace}= \bigg\lbrace 16 C_1\exp\bigg(-4\rho_{\mu}\log (\psi T\frac{\Delta_{i}^{4}}{16\rho_{\mu}^{2}})\bigg)\\
%& + 32C_2\exp\bigg(- \frac{3\rho_v}{2} \log(\psi T\frac{\Delta_{i}^{4}}{16\rho_{v}^{2}}) \bigg) \bigg\rbrace\\
%%%%%%%%%%%%%%%%%%%%%
%&\leq \bigg\lbrace  \frac{2K\log ( T^2 \frac{4\Delta_{i}^{4}}{\log K})}{\Delta_{i}^{2}}\exp\bigg(-\frac{1}{2}\log ( T^2\frac{4\Delta_{i}^{4}}{\log K})\bigg)\\
%& + \frac{32K\log ( T^2 \frac{9\Delta_{i}^{4}}{\log K})}{3\Delta_{i}^{2}}\exp\bigg(- \frac{1}{2} \log( T^2 \frac{9\Delta_{i}^{4}}{\log K}) \bigg) \bigg\rbrace\\
%%%%%%%%%%%%%%%%%%%%%
%&\leq \bigg\lbrace  \frac{4K\log ( T \frac{2\Delta_{i}^{2}}{\sqrt{\log K}})}{\Delta_{i}^{2}}\exp\bigg(-\log ( T\frac{2\Delta_{i}^{2}}{\sqrt{\log K}})\bigg)\\
%& + \frac{64K\log ( T \frac{3\Delta_{i}^{2}}{\sqrt{\log K}})}{3\Delta_{i}^{2}}\exp\bigg(- \log( T \frac{3\Delta_{i}^{2}}{\sqrt{\log K}}) \bigg) \bigg\rbrace\\
%%%%%%%%%%%%%%%%%%%%%
%&\leq \bigg\lbrace  \frac{4KT\log ( \frac{2 K\log K}{\sqrt{\log K}})}{K\log K}\exp\bigg(-\log ( \frac{2K\log K}{\sqrt{\log K}})\bigg)\\
%& + \frac{64TK\log (\frac{3 K\log K}{\sqrt{\log K}})}{3 K\log K}\exp\bigg(- \log( \frac{3 K\log K}{\sqrt{\log K}}) \bigg) \bigg\rbrace\\
%%%%%%%%%%%%%%%%%%%%
%&\leq \bigg\lbrace  \frac{2T\log (2 K\sqrt{\log K})}{K (\log K)^{3/2}}
% + \frac{22T\log ( K\sqrt{\log K})}{ K(\log K)^{3/2}}\bigg) \bigg\rbrace\\
%\end{align*}
%\end{small}
%Now we know that till $m_i$-th round $2c_i > \frac{\Delta_i}{2}$  or till $g_i$ th round $2s_i > \frac{\Delta_i}{2}$. Hence, for the $i$-th arm we can bound the probability of error for any round $m$ by applying Chernoff-Hoeffding and Bernstein inequality,
%\begin{small}
%\begin{align*}
% \Pb\lbrace \xi_1\rbrace  + \Pb\lbrace \xi_2 \rbrace &\geq 1-(\Pb\lbrace |\hat{r}_i -r_i| > 2c_i \rbrace + \Pb\lbrace |\hat{r}_i -r_i| > 2s_i \rbrace)\\ 
%&\geq 1-\left(\Pb\lbrace |\hat{r}_i - r_i| > \frac{\Delta_i}{2} \rbrace + \Pb\lbrace |\hat{r}_i - r_i| > \frac{\Delta_i}{2} \rbrace\right) \\
%&\geq 1-\big(2\exp( -\frac{\Delta_{i}^{2}}{4}n_i ) + 2\exp(- \frac{\Delta_{i}^{2}}{8\sigma_{i}^{2}+ \frac{4}{3}\Delta_i}n_i)\big)\\
%&\geq 1-\bigg(2\exp( -\frac{\Delta_{i}^{2}}{4}\delta_{m_{i}} ) + 2\exp(- \frac{\Delta_{i}^{2}}{8\sigma_{i}^{2}+ \frac{4}{3}\Delta_i}\delta_{g_{i}})\bigg)
%\end{align*}
%\end{small}
%Now, we know that $\E[\Ls(T)]\le1- (\Pb\lbrace \xi_1\rbrace  + \Pb\lbrace \xi_2 \rbrace) $. Summing over all arms $K$ and over all rounds $m=0,1,2,..,M$ we get that,
%\begin{small}
%\begin{align*}
%&\E[\Ls(T)] \leq \sum_{i=1}^{K}\sum_{m=0}^{M}\bigg\lbrace 2\exp\bigg( -\frac{\Delta_{i}^{2}}{4}.\frac{2T\log (2 K\sqrt{\log K})}{K (\log K)^{3/2}}\bigg)\\
%& + 2\exp\bigg(- \frac{\Delta_{i}^{2}}{8\sigma_{i}^{2}+ \frac{4}{3}\Delta_i}.\frac{22T\log ( K\sqrt{\log K})}{ K(\log K)^{3/2}} \bigg)\bigg\rbrace\\
%%%%%%%%%%%%%%%%
%&\E[\Ls(T)] \leq K\left\lceil\log_2\frac{T}{e}\right\rceil\bigg\lbrace\exp\bigg( -\frac{1}{i\max_{i}\Delta_{i}^{-2}}.\frac{T\log (2 K\sqrt{\log K})}{2K (\log K)^{3/2}}\bigg)\\
%& + \exp\bigg(- \frac{3}{i\max_i(6\sigma_{i}^{2}+ \Delta_i)\Delta_{i}^{-2}}.\frac{5T\log ( K\sqrt{\log K})}{ K(\log K)^{3/2}} \bigg)\bigg\rbrace\\
%%%%%%%%%%%%%%%%
%&\E[\Ls(T)] \leq K\left(\log_2\frac{T}{e}+1\right)\bigg\lbrace\exp\bigg( -\frac{T\log (2 K\sqrt{\log K})}{2 H_2 K (\log K)^{3/2}}\bigg)\\
%& + \exp\bigg(- \frac{5T\log ( K\sqrt{\log K})}{H_{2}^{\sigma} K(\log K)^{3/2}} \bigg)\bigg\rbrace\\
%\end{align*}
%\end{small}
%%%%%%%%%%%%%%%%%%%%%%%%%%%%%%%%%%%%%%%%%%%%%%%%%%%%%%%%%%%%%%%%%%%%%%%%%%%%%%%%%%%%%%
%Not Required for probability of error for AugUCB
%%%%%%%%%%%%%%%%%%%%%%%%%%%%%%%%%%%%%%%%%%%%%%%%%%%%%%%%%%%%%%%%%%%%%%%%%%%%%%%%%%%%%%

%Hence, for the $i$-th arm we can bound the probability of it getting eliminated till $\max\lbrace m_i , g_i  \rbrace$-th round by,
%%\begin{small}
%\begin{align*}
% & \Pb\lbrace \text{$i\in \mathcal{A}^{'}$ getting eliminated on or before round $\max\lbrace m_i, g_i\rbrace$} \rbrace \\
%&\geq 1-(\Pb\lbrace |\hat{r}_i -r_i| > 2c_i \rbrace + \Pb\lbrace |\hat{r}_i -r_i| > 2s_i \rbrace)\\
%&\geq 1- \bigg( \left(2\exp\left(-\frac{T\rho_{\mu}\epsilon_{m_{i}}}{32 a^2}\log ( T\epsilon_{m_{i}})\right)\right)\\
%& + 4\exp\left(- \frac{3T\rho_v\epsilon_{g_{i}}}{256 a^2 } \left(\frac{\sigma_{i}^{2}+\sqrt{\rho_{v}\epsilon_{g_{i}}}+1}{3\sigma_{i}^{2}+\sqrt{\rho_v \epsilon_{g_{i}}}}\right) \log( T\epsilon_{g_{i}}) \right)\bigg)
%\end{align*}
%%\end{small}
%Now, in the $m_i$-th round or in the $g_i$-th round we know that $\frac{\Delta_i}{4}\leq\sqrt{\epsilon_{m_{i}}\rho_{\mu}}<\frac{\Delta_i}{2}$ or  $\frac{\Delta_i}{4}\leq\sqrt{\epsilon_{g_{i}}\rho_{v}}<\frac{\Delta_i}{2}$.
%%\begin{small}
%\begin{align*}
%&\Pb\lbrace \text{$i\in \mathcal{A}^{'}$ getting eliminated on or before round $\max\lbrace m_i, g_i\rbrace$} \rbrace\\
%%%%%%%%%%%%%%%%%%%%%%%%%%%%%%%%%%%%%%%%%%%%%%%%%%%%%%%
%& \geq 1- \bigg( 2\exp\left(-\frac{T\rho_{\mu}\frac{\Delta_{i}^{2}}{16\rho_{\mu}}}{32 a^2 }\log ( T\frac{\Delta_{i}^{2}}{16\rho_{\mu}})\right)\\
%& + 4\exp\left(- \frac{3T\rho_v\frac{\Delta_{i}^{2}}{16\rho_{v}}}{256 a^2} \left(\frac{\sigma_{i}^{2}+\frac{\Delta_{i}}{4}+1}{3\sigma_{i}^{2}+\frac{\Delta_{i}}{4}}\right) \log( T\frac{\Delta_{i}^{2}}{16\rho_{v}}) \right)\bigg)\\
%%%%%%%%%%%%%%%%%%%%%%%%%%%%%%%%%%%%%%%%%%%%%%%%%%%%%%%	
%&\geq 1-\bigg( 2\exp\left(-\frac{T\Delta_{i}^{2}}{64a}\log( \frac{T\Delta_{i}^{2}}{2})\right) \\
%& + 4\exp\left(- \frac{3T\Delta_{i}^{2}}{4096 a^2} \left(\frac{4\sigma_{i}^{2}+\Delta_{i}+4}{12\sigma_{i}^{2}+\Delta_{i}}\right) \log( \frac{3}{16} T\Delta_{i}^{2}) \right)\bigg),\\
%&\text{putting the values of $\rho_{\mu}$ and $\rho_{v}$.}
%\end{align*}
%%\end{small}
%Again, $\Pb\lbrace \xi_1 \cup \xi_2 \rbrace\geq 1- \sum_{i=1}^{K}\sum_{m=0}^{\max\lbrace m_{i} ,g_{i}\rbrace}\Pb\lbrace i\in \mathcal{A}^{'}$ getting eliminated on or before round $\max\lbrace m_i, g_i\rbrace \rbrace $.
%Also, $\E[\Ls(T)]\le 1- \Pb\lbrace \xi_1 \cup \xi_2 \rbrace $. We know from \cite{bubeck2011pure} and \cite{auer2010ucb} that the function $x\in [0,1]\mapsto x\exp(-Cx^2)$ is  decreasing on $[1/\sqrt{2C},1]$ for any $C>0$. So, taking $C=\lfloor \sqrt{e/T}\rfloor$ and putting $\min_{i\in \mathcal{A}}\Delta_i =\Delta =\sqrt{\frac{K\log K}{T}} > \sqrt{\frac{e}{T}},\forall i\in \mathcal{A}$ we get that,
%%and summing over all arms $K$ and over all rounds $m=0,1,2,..,\max\lbrace m_{i} ,g_{i}\rbrace$
%%\begin{small}
%\begin{align*}
%&\E[\Ls(T)] \leq \sum_{i=1}^{K}\sum_{m=0}^{\max\lbrace m_{i} ,g_{i}\rbrace}\bigg\lbrace \bigg( 2\exp\left(-\frac{T\Delta_{i}^{2} \log(\frac{T\Delta_{i}^{2}}{2})}{64 a^2 }\right) \\
%& + 4\exp\left(- \frac{3T\Delta_{i}^{2}}{4096 a^2 } \left(\frac{4\sigma_{i}^{2}+\Delta_{i}+4}{12\sigma_{i}^{2}+\Delta_{i}}\right) \log( \frac{3}{16} T\Delta_{i}^{2}) \right)\bigg\rbrace\\
%%%%%%%%%%%%%%%%%
%& \leq K\sum_{m=0}^{M}\bigg\lbrace 2\exp\bigg( -\frac{T}{\min_{i}i\Delta_{(i)}^{-2}}.\frac{\log (\frac{1}{2} K\log K)}{64 a^2 }\bigg)\\
%& + 4\exp\bigg(- \frac{12T\Delta_{i}^{2}}{(12\sigma_{i}+ 12\Delta_{i})}.\frac{\log (\frac{3}{16} K\log K)}{4096 a^2 } \bigg)\bigg\rbrace\\
%%%%%%%%%%%%%%%%
%&\leq K\left(\log_2\frac{T}{e}+1\right)\bigg\lbrace\exp\bigg( -\frac{T\log ( \frac{1}{2} K\log K)}{ 64 H_2 a^2}\bigg)\\
%& + 2\exp\bigg(- \frac{T\Delta_{i}^{2}\log ( \frac{3}{16} K\log K)}{4096 (\sigma_{i} + \sqrt{\sigma_{i}^{2} + (16/3)\Delta_{i}}) a^2} \bigg)\bigg\rbrace\\
%%%%%%%%%%%%%%%%
%&\leq K\left(\log_2\frac{T}{e}+1\right)\bigg\lbrace\exp\bigg( -\frac{T\log ( \frac{1}{2} K\log K)}{ 64 H_2 (\log(\frac{3}{16} K\log K))^{2}}\bigg)\\
%& + 2\exp\bigg(- \frac{T\log ( \frac{3}{16} K\log K)}{4096 \min_{i}i\tilde{\Delta}_{(i)}^{-2} (\log(\frac{3}{16} K\log K))^{2}} \bigg)\bigg\rbrace\\
%%%%%%%%%%%%%%%%
%&\leq K\left(\log_2\frac{T}{e}+1\right)\bigg\lbrace\exp\bigg( -\frac{T}{ 64 H_2 (\log(\frac{3}{16} K\log K))}\bigg)\\
%& + 2\exp\bigg(- \frac{T}{4096 H_{2}^{\sigma} (\log(\frac{3}{16} K\log K))} \bigg)\bigg\rbrace\\
%\end{align*}
%\end{small}


%	Next we specialize the result of Theorem \ref{Result:Theorem:1} in Corollary \ref{Result:Corollary:1}.
%
%\subsection{Corollary 2}
%
%
%\begin{corollary}
%\label{Result:Corollary:1}
%For $c_{0}=\sqrt{T}$, $\psi=\frac{T}{\log (K)}$, $\rho_{\mu}=\frac{1}{8}$ and $\rho_v=\frac{2}{3}$, the simple regret of AugUCB is given by,
%\begin{small}
%\begin{align*}
%& SR_{AugUCB} \leq \sum_{i=1}^{K} \Delta_{i}\bigg\lbrace\exp\bigg(-\log ( 2T\frac{\Delta_{i}^{2}}{\sqrt{\log K}})-\dfrac{T}{2 H_{2}}\\
%& + \log \big( \dfrac{4\gamma K\log ( 2T \frac{\Delta_{i}^{2}}{\sqrt{\log K}})}{T\Delta_{i}^{2}}\log_{2}\dfrac{T}{e} \big) \bigg)\\
%& +  \exp\bigg(- \bigg(\dfrac{2\sigma_{i}^{2}+\Delta_{i}+2}{6\sigma_{i}^{2}+\Delta_{i}}\bigg)\log( 3T\frac{\Delta_{i}^{2}}{8\sqrt{\log K}}) -\dfrac{3T}{32 H_{2}}\\
%& + \log\big ( \dfrac{64\gamma K\log ( 3T \frac{\Delta_{i}^{2}}{8\sqrt{\log K}})}{3T\Delta_{i}^{2}}\log_{2}\dfrac{T}{e} \big)  \bigg)\bigg\rbrace
%\end{align*}
%\end{small}
%\end{corollary}
%
%\begin{proof}
%Putting $c_{0}=\sqrt{T}$, $\psi=\frac{T}{\log (K)}$, $\rho_{\mu}=\frac{1}{8}$ and $\rho_v=\frac{2}{3}$ in the result obtained in Theorem \ref{Result:Theorem:1}, we get
%\begin{small}
%\begin{align*}
%& SR_{AugUCB} \leq \sum_{i=1}^{K} \Delta_{i}\bigg\lbrace \exp\bigg(-4\rho\log (\psi T\frac{\Delta_{i}^{4}}{16\rho^{2}})-\dfrac{c_{0}\sqrt{T}}{16\rho H_{2}}\\
%& + \log \big( 16\gamma C_1\log_{2}\dfrac{T}{e} \big) \bigg) + \exp\bigg(- \dfrac{3\rho_v}{2} \bigg(\dfrac{2\sigma_{i}^{2}+\Delta_{i}+2}{6\sigma_{i}^{2}+\Delta_{i}}\bigg)\log(\psi T\frac{\Delta_{i}^{4}}{16\rho_{v}^{2}})\\
%& -\dfrac{c_{0}\sqrt{T}}{16\rho_v H_{2}} + \log\big ( 32\gamma C_2\log_{2}\dfrac{T}{e} \big)  \bigg)\bigg\rbrace\\
%%%%%%%%%%%%%%%%%%
%&\leq \sum_{i=1}^{K} \Delta_{i}\bigg\lbrace\exp\bigg(-\dfrac{1}{2}\log ( T^{2}\frac{4\Delta_{i}^{4}}{\log K})-\dfrac{T}{2 H_{2}}\\
%& + \log \big( \dfrac{2\gamma K\log ( T^{2} \frac{4\Delta_{i}^{4}}{\log K})}{T\Delta_{i}^{2}}\log_{2}\dfrac{T}{e} \big) \bigg)\\
%& + \exp\bigg(-  \bigg(\dfrac{2\sigma_{i}^{2}+\Delta_{i}+2}{6\sigma_{i}^{2}+\Delta_{i}}\bigg)\log( T^{2}\frac{\Delta_{i}^{4}}{16.\frac{4}{9}\log K}) -\dfrac{c_{0}\sqrt{T}}{16.\frac{2}{3} H_{2}}\\
%& + \log\big ( \dfrac{32\gamma\rho_v K\log ( T^{2} \frac{\Delta_{i}^{4}}{16.\frac{2}{9}\log K})}{T\Delta_{i}^{2}}\log_{2}\dfrac{T}{e} \big)  \bigg)\bigg\rbrace\\
%%%%%%%%%%%%%%%%%%
%&\leq \sum_{i=1}^{K} \Delta_{i}\bigg\lbrace\exp\bigg(-\log ( 2T\frac{\Delta_{i}^{2}}{\sqrt{\log K}})-\dfrac{T}{2 H_{2}}\\
%& + \log \big( \dfrac{4\gamma K\log ( 2T \frac{\Delta_{i}^{2}}{\sqrt{\log K}})}{T\Delta_{i}^{2}}\log_{2}\dfrac{T}{e} \big) \bigg)\\
%& +  \exp\bigg(- \bigg(\dfrac{2\sigma_{i}^{2}+\Delta_{i}+2}{6\sigma_{i}^{2}+\Delta_{i}}\bigg)\log( 3T\frac{\Delta_{i}^{2}}{8\sqrt{\log K}}) -\dfrac{3T}{32 H_{2}}\\
%& + \log\big ( \dfrac{64\gamma K\log ( 3T \frac{\Delta_{i}^{2}}{8\sqrt{\log K}})}{3T\Delta_{i}^{2}}\log_{2}\dfrac{T}{e} \big)  \bigg)\bigg\rbrace
%\end{align*} 
%\end{small}
%\end{proof}


\section{Numerical Experiments}
\label{tbandit:expt}

In this section, we empirically compare the  performance of AugUCB against APT, UCBE, UCBEV, CSAR and the uniform-allocation (UA) algorithms. A brief note about these algorithms are as follows:
%\begin{itemize}

$\bullet$ APT: This algorithm is from \cite{locatelli2016optimal}; we set $\epsilon=0.05$, which is the margin-of-error within which APT suggests the set of good arms.

$\bullet$ AugUCB: This is the Augmented-UCB algorithm proposed in this paper; as in Theorem \ref{tbandit:Result:Theorem:1} we set $\rho=\frac{1}{3}$.

$\bullet$ UCBE: This is a modification of the algorithm in \cite{audibert2009exploration} (as it was originally proposed for the best arm identification problem); here, we set $a=\frac{T-K}{H_1}$, and at each time-step an arm $i\in\argmin\left\lbrace |\hat{r}_{i} -\tau|-\sqrt{\frac{a}{n_{i}}} \right\rbrace$ is pulled.

$\bullet$ UCBEV: This is a modification of the algorithm in \cite{gabillon2011multi} (proposed for the TopM problem); its implementation is identical to UCBE, but with $a = \frac{T-2K}{H_{\sigma,1}}$. As mentioned earlier, note that UCBEV's implementation would not be possible in real scenarios, as it requires computing the problem complexity $H_{\sigma,1}$. However, for theoretical reasons we show the best performance achievable by UCBEV. In experiment 6 we perform further explorations of UCBEV with alternate settings of $a$.

$\bullet$ CSAR:  Modification of the successive-reject algorithm in \cite{chen2014combinatorial}; here, we reject the arm farthest from $\tau$ after each round. 

$\bullet$ UA: The naive strategy where at each time-step an arm is uniformly sampled from $\mathcal{A}$ (the set of all arms); however, UA is known to be optimal if all arms are equally difficult to classify. 
%\end{itemize}


\noindent
Motivated by the settings considered in \cite{locatelli2016optimal}, 
we design six different experimental scenarios that are obtained by varying the arm means and variances.  Across all experiments consists of $K=100$  arms (indexed $i=1,2,\cdots,100$) of which ${S}_\tau=\{6,7,\cdots,10\}$, where we have fixed $\tau=0.5$. In all the experiments, each algorithm is run independently for $10000$ time-steps. At every time-step, the output set,  $\hat{S}_\tau$, suggested by each algorithm is recorded; the output is counted as an error if $\hat{S}_\tau\ne S_\tau$. In Figure~1, for each experiment, we have reported the percentage of error incurred by the different algorithms as a function of time; Error percentage is obtained by repeating each experiment independently  for $500$ iterations, and then respectively computing the fraction of errors. The details of the considered experiments are as follows.

\textbf{Experiment-1:} The reward distributions are Gaussian with  means  $r_{1:4}=0.2+(0:3)\cdot0.05$, $r_{5}=0.45$, $r_{6}=0.55$, $r_{7:10}=0.65+(0:3)\cdot0.05$ and $r_{11:100}=0.4$. Thus, the means of the first $10$ arms follow an arithmetic progression. The remaining arms have identical means; this setting is chosen because now a significant budget is required in exploring these arms, thus increasing the problem complexity.

 The corresponding variances are $\sigma_{1:5}^{2}=0.5$ and $\sigma_{6:10}^{2}=0.6$, while $\sigma_{11:100}^{2}$ is chosen independently and uniform in the  interval $[0.38,0.42]$; note that, the variances of the arms in $S_\tau$ are higher than those of the other arms. The corresponding  results are shown in Figure \ref{Fig:budgetExpt1}, from where we see that UCBEV, which has access to the problem complexity while being variance-aware, outperforms all other algorithm (including UCBE which also has access to the problem complexity but does not take into account the variances of the arms).  Interestingly, the performance of our AugUCB (without requiring any complexity input) is comparable with UCBEV, while it outperforms UCBE, APT and the other non variance-aware algorithms that we have considered. 	

\begin{figure}[th!]
    \centering
    \begin{tabular}{cc}
    \subfigure[0.32\textwidth][Expt-$1$: Arithmetic Progression (Gaussian)]
    {
    		\pgfplotsset{
		tick label style={font=\Large},
		label style={font=\Large},
		legend style={font=\Large},
		}
        \begin{tikzpicture}[scale=0.6]
      	\begin{axis}[
		xlabel={Time-step},
		ylabel={Error Percentage},
		grid=major,
        %clip mode=individual,grid,grid style={gray!30},
        clip=true,
        %clip mode=individual,grid,grid style={gray!30},
  		legend style={at={(0.5,1.3)},anchor=north, legend columns=3} ]
      	% UCB
		\addplot table{Chapter5/results/budgetTestAP/APT12_comp_subsampled.txt};
		\addplot table{Chapter5/results/budgetTestAP/AugUCBV1_comp_subsampled.txt};
		\addplot table{Chapter5/results/budgetTestAP/UCBEM1_comp_subsampled.txt};
		\addplot table{Chapter5/results/budgetTestAP/UCBEMV1_comp_subsampled.txt};
		\addplot table{Chapter5/results/budgetTestAP/SR1_comp_subsampled.txt};
		\addplot table{Chapter5/results/budgetTestAP/UA1_comp_subsampled.txt};
		
      	\legend{APT,AugUCB,UCBE,UCBEV,CSAR,UA}
      	\end{axis}
      	\end{tikzpicture}
  		\label{Fig:budgetExpt1}
    }
    &
    \subfigure[0.32\textwidth][Expt-$2$: Geometric Progression (Gaussian)]
    {
    	\pgfplotsset{
		tick label style={font=\Large},
		label style={font=\Large},
		legend style={font=\Large},
		}
        \begin{tikzpicture}[scale=0.6]
        \begin{axis}[
		xlabel={Time-step},
		ylabel={Error Percentage},
        %clip mode=individual,grid,grid style={gray!30},
		grid=major,
		clip=true,
  		legend style={at={(0.5,1.3)},anchor=north, legend columns=3} ]
        % UCB
		\addplot table{Chapter5/results/budgetTestGP/APT12_comp_subsampled.txt};
		\addplot table{Chapter5/results/budgetTestGP/AugUCBV1_comp_subsampled.txt};
		\addplot table{Chapter5/results/budgetTestGP/UCBEM1_comp_subsampled.txt};
		\addplot table{Chapter5/results/budgetTestGP/UCBEMV1_comp_subsampled.txt};
		\addplot table{Chapter5/results/budgetTestGP/SR1_comp_subsampled.txt};
		\addplot table{Chapter5/results/budgetTestGP/UA1_comp_subsampled.txt};
        \legend{APT,AugUCB,UCBE,UCBEV,CSAR,UA}
      	\end{axis}
      	\label{Fig:budgetExpt2}
        \end{tikzpicture}
    }
    \end{tabular}

	\begin{tabular}{cc}
	\centering
    \subfigure[0.32\textwidth][Expt-$3$: Three Group Setting (Gaussian)]
    {
    		\pgfplotsset{
		tick label style={font=\Large},
		label style={font=\Large},
		legend style={font=\Large},
		}
        \begin{tikzpicture}[scale=0.6]
        \begin{axis}[
		xlabel={Time-step},
		ylabel={Error Percentage},
        %clip mode=individual,grid,grid style={gray!30},
       	grid=major,
       	clip=true,
  		legend style={at={(0.5,1.3)},anchor=north, legend columns=3} ]
      	% UCB
		\addplot table{Chapter5/results/budgetTestGR1/APT1_comp_subsampled.txt};
		\addplot table{Chapter5/results/budgetTestGR1/AugUCB1_comp_subsampled.txt};
		\addplot table{Chapter5/results/budgetTestGR1/UCBEM1_comp_subsampled.txt};
		\addplot table{Chapter5/results/budgetTestGR1/UCBEMV1_comp_subsampled.txt};
		\addplot table{Chapter5/results/budgetTestGR1/SR1_comp_subsampled.txt};
		\addplot table{Chapter5/results/budgetTestGR1/UA1_comp_subsampled.txt};
        \legend{APT,AugUCB,UCBE,UCBEV,CSAR,UA}
      	\end{axis}
      	\end{tikzpicture}
   		\label{Fig:budgetExpt3} 
    }
    &
    \subfigure[0.32\textwidth][Expt-$4$: Two Group Setting (Gaussian) ]
    {
    	\pgfplotsset{
		tick label style={font=\Large},
		label style={font=\Large},
		legend style={font=\Large},
		}
        \begin{tikzpicture}[scale=0.6]
        \begin{axis}[
		xlabel={Time-step},
		ylabel={Error Percentage},
        %clip mode=individual,grid,grid style={gray!30},
		grid=major,
		clip=true,
  		legend style={at={(0.5,1.3)},anchor=north, legend columns=3} ]
        % UCB
		\addplot table{Chapter5/results/budgetTestGR2/APT1_comp_subsampled.txt};
		\addplot table{Chapter5/results/budgetTestGR2/AugUCBV1_comp_subsampled.txt};
		\addplot table{Chapter5/results/budgetTestGR2/UCBEM1_comp_subsampled.txt};
		\addplot table{Chapter5/results/budgetTestGR2/UCBEMV1_comp_subsampled.txt};
		\addplot table{Chapter5/results/budgetTestGR2/SR1_comp_subsampled.txt};
		\addplot table{Chapter5/results/budgetTestGR2/UA1_comp_subsampled.txt};
        \legend{APT,AUgUCB,UCBE,UCBEV,CSAR,UA}
        %\legend{APT,AugUCB,UCBE,UCBEV,CSAR,Unif Alloc}
      	\end{axis}
      	\label{Fig:budgetExpt4}
        \end{tikzpicture}
    }
    \end{tabular}

	\begin{tabular}{cc}
    \subfigure[0.32\textwidth][Expt-$5$: Two Group Setting (Advance) ]
    {
    	\pgfplotsset{
		tick label style={font=\Large},
		label style={font=\Large},
		legend style={font=\Large},
		}
        \begin{tikzpicture}[scale=0.6]
        \begin{axis}[
		xlabel={Time-step},
		ylabel={Error Percentage},
        %clip mode=individual,grid,grid style={gray!30},
		grid=major,
		clip=true,
  		legend style={at={(0.5,1.3)},anchor=north, legend columns=3} ]
        % UCB
		\addplot table{Chapter5/results/budgetTestGR4/APT1_comp_subsampled.txt};
		\addplot table{Chapter5/results/budgetTestGR4/AugUCB1_comp_subsampled.txt};
		\addplot table{Chapter5/results/budgetTestGR4/UCBEM1_comp_subsampled.txt};
		\addplot table{Chapter5/results/budgetTestGR4/UCBEMV1_comp_subsampled.txt};
		\addplot table{Chapter5/results/budgetTestGR4/SR1_comp_subsampled.txt};
		\addplot table{Chapter5/results/budgetTestGR4/UA1_comp_subsampled.txt};
        \legend{APT,AUgUCB,UCBE,UCBEV,CSAR,UA}
      	\end{axis}
      	\label{Fig:budgetExpt5}
        \end{tikzpicture}
    }
    &
    \subfigure[0.32\textwidth][Expt-$6$: Two Group Setting (Advance) ]
    {
    	\pgfplotsset{
		tick label style={font=\Large},
		label style={font=\Large},
		legend style={font=\Large},
		}
        \begin{tikzpicture}[scale=0.6]
        \begin{axis}[
		xlabel={Time-step},
		ylabel={Error Percentage},
        %clip mode=individual,grid,grid style={gray!30},
		grid=major,
		clip=true,
  		legend style={at={(0.5,1.3)},anchor=north, legend columns=2} ]
        % UCB
		\addplot table{Chapter5/results/budgetTestGR3/testUCBEMV1_0.25_comp_subsampled.txt};
		\addplot table{Chapter5/results/budgetTestGR4/AugUCB1_comp_subsampled.txt};
		\addplot table{Chapter5/results/budgetTestGR3/testUCBEMV1_256_comp_subsampled.txt};
		\addplot table{Chapter5/results/budgetTestGR4/UCBEMV1_comp_subsampled.txt};
        \legend{UCBEV($0.25$), AugUCB, UCBEV($256$), UCBEV($1$)}
      	\end{axis}
      	\label{Fig:budgetExpt6}
        \end{tikzpicture}
    }
    \end{tabular}
    \caption{Performances of the various TBP algorithms in terms of error percentage vs. time-step, for  six different experimental scenarios.}
    \label{fig:budgetExpt}
    \vspace{-6mm}
\end{figure}

	
\textbf{Experiment-2:} We again consider  Gaussian reward distributions. However, here the means of the first $10$ arms constitute a geometric progression. Formally, the reward means are $r_{1:4}=0.4-(0.2)^{1:4}$, $r_{5}=0.45$, $r_{6}=0.55$, $r_{7:10}=0.6+(0.2)^{5-(1:4)}$ and $r_{11:100}=0.4$; the arm variances are as in experiment-$1$. The corresponding results are shown in Figure \ref{Fig:budgetExpt2}.  We again observe AugUCB outperforming the other algorithms, except UCBEV. 
	

\textbf{Experiment-3:} Here, the first
$10$ arms are partitioned into three groups, with all arms in a group being assigned the same mean; the reward distributions are again Gaussian. Specifically, the reward means are $r_{1:3}=0.1$, $r_{4:7}=\lbrace 0.35, 0.45, 0.55, 0.65\rbrace$ and $r_{8:10}=0.9$; as before,  $r_{11:100}=0.4$ and all the variances are as in Experiment-$1$. The results for this scenario are presented in Figure \ref{Fig:budgetExpt3}. The observations are inline with those made in the previous experiments. 


	
\textbf{Experiment-4:} The setting is similar to that considered in Experiment-3, but with the first $10$ arms partitioned into two groups; the respective means are $r_{1:5}=0.45$, $r_{6:10}=0.55$. The corresponding results are shown in Figure \ref{Fig:budgetExpt4}, from where the good performance of AugUCB is again validated.


\textbf{Experiment-5:} This is again the two group setting involving Gaussian reward distributions. The reward means are as in Experiment-4, while the variances are  $\sigma_{1:5}^{2}=0.3$ and $\sigma_{6:10}^{2}=0.8$;  $\sigma_{11:100}^{2}$ are independently and uniformly chosen in the interval $[0.2,0.3]$.  The corresponding results are shown in Figure \ref{Fig:budgetExpt5}.
 We refer to this setup as \emph{Advanced} because here the chosen variance values are such that only  variance-aware algorithms will perform well.Hence, we see that UCBEV performs very well in comparison with the other algorithms. However,  it is interesting to note that the performance of  AugUCB catches-up with UCBEV as the time-step increases, while significantly outperforming the other non-variance aware algorithms.


\textbf{Experiment-6:} We use the same setting as in Experiment-5, but conduct more exploration of UCBEV with different values of the exploration parameter $a$. The corresponding results are shown in Figure \ref{Fig:budgetExpt6}. As studied in \cite{locatelli2016optimal}, we implement UCBEV with $ a_{i} = 4^{i} \frac{T-2K}{H_{\sigma,1}}$ for $i = -1,0,4$. Here, $a_{0}$ corresponds to UCBEV($1$) (in Figure \ref{Fig:budgetExpt6}) which is UCBEV run with the optimal choice of $H_{\sigma ,1}$. For other choices of $a_i$ we see that UCBEV($a_i$) is significantly outperformed by AugUCB. 
	
Finally, note that in all the above experiments, the CSAR algorithm, although performs well initially, quickly exhausts its budget and saturates at a higher error percentage. This is because it pulls all arms equally in each round, with the round lengths being non-adaptive.








\section{Summary and Future Works}
\label{tbandit:conclusion}
We proposed the AugUCB algorithm for a fixed-budget, pure-exploration TBP. Our algorithm employs both mean and variance estimates for arm elimination. This, to our knowledge is the first variance-based algorithm for the specific TBP that we have considered. We first prove an upper bound on the expected loss incurred by AugUCB. We then conduct simulation experiments to validate the performance of AugUCB. In comparison with APT, CSAR and other non variance-based algorithms, we find that the performance of AugUCB is significantly better. Further, the performance of AugUCB is comparable with UCBEV (which is also variance-based), although the latter exhibits a slightly better performance.  However, UCBEV is not implementable in practice as it requires computing problem complexity, $H_{\sigma,1}$, while AugUCB (requiring no such inputs) can be easily deployed in real-life scenarios. It would be an interesting future work to design an anytime version of the AugUCB algorithm. 

%Although UCBEV provides a better guarantee, it is important to emphasize that UCBEV has access to the problem complexity, and is hence not realistic in practice. This is in contrast to AugUCB whose implementation does not require any such complexity inputs. 
%Finally, through extensive simulation experiments we have validated the performance of AugUCB.

%From a theoretical viewpoint we conclude the expected loss AugUCB is more than UCBEV (which has access to problem complexity). From the numerical experiments on settings with large number of arms with different mean and variance, we observed that AugUCB outperforms all the non-variance aware algorithms.
%This is also the first paper to apply elimination by variance estimation in the TBP problem by modifying UCB-Improved and CCB algorithms. 
% It would be interesting future research to come up with an anytime version of AugUCB algorithm. This is also the first paper to apply elimination by variance estimation in the TBP problem by modifying UCB-Improved and CCB algorithms. 


%%%%%%%%%%%%%%%%%%%%%%%%%%%%%%%%%%%%%%%%%%%%%%%%%%%%%%%%%%%%


%%%%%%%%%%%%%%%%%%%%%%%%%%%%%%%%%%%%%%%%%%%%%%%%%%%%%%%%%%%%
\chapter{Single Expert Algorithms for Changepoint Detection}
\label{chap:psbandit}

\section{Introduction}
\label{psbandit:intro}
In this chapter, we consider the piece-wise stochastic multi-armed bandit problem, an interesting variation of the stochastic multi-armed bandit (SMAB) problem in sequential decision making which was discussed in detail in chapter \ref{chap:SMAB}. In this setting,  a learning algorithm is provided with a set of decisions (or arms) with reward distributions unknown to the learner. The learning proceeds in an iterative fashion, where in each round, the algorithm chooses an arm and receives a stochastic reward that is drawn from a distribution specific to the arm selected. There exist a finite number of changepoints such that the reward distribution of arms changes at those changepoints. Given the goal of maximizing the cumulative reward, the learner faces the \textit{exploration-exploitation-changepoint} dilemma, as opposed to the simple \textit{exploration-exploitation} dilemma, i.e., in each round should the algorithm select the arm which has the highest observed reward so far (\textit{exploitation}), or should the algorithm choose a new arm to gain more knowledge of the expected reward of the arms and thereby avert a sub-optimal greedy decision (\textit{exploration}) and finally keep track of the \textit{changepoints} and adapt accordingly.

    The rest of the chapter is organized as follows. We first state the notations, definitions, and assumptions required for this setting in section~\ref{psbandit:notations}. Then we define our problem statement in section~\ref{psbandit:probDef} and in section~\ref{psbandit:related} we discuss the related works in this setting. We elaborate our contributions in section~\ref{psbandit:contribution} and in section~\ref{psbandit:algorithm} we present the changepoint detection algorithms. Section~\ref{psbandit:results} contains our main result, Section~\ref{psbandit:expt} contains numerical simulations where we test our proposed algorithms and finally we summarize in section \ref{psbandit:conclusion}. 



\section{Notations, Assumptions and Definitions}
\label{psbandit:notations}
Again, to benefit the reader we state the notations used in this chapter. $T$ denotes the time horizon. $\A$ denotes the set of arms with individual arm is denoted by $i$ such that $i=1,\ldots, K$. We assume that the total number of arms is constant throughout the time horizon and $|\A|=K$. In this setting each arm has a piece-wise stationary distribution associated with it. Let $c_0,c_1,c_2,\ldots,c_G$ denotes $G$ such changepoints which belong to the set $\G$ and $|\G|=G$. Here, $c_0$ is a virtual changepoint which starts at $t=1$. 
We denote each of the \textit{non-overlapping reward generating processes} as $\rho_{c_j}$, $j\in\mathbb{N}$. Note that the learner does not know these changepoints or the total number of changepoints in the environment. 

%We assume a global switching model where at particular breakpoint all the arms' distribution changes.

	The distribution associated with individual arm $i$ is denoted by $D_{i,\rho_{c_j:c_{j+1}}}$ for the segment $\rho_{c_j:c_{j+1}}$ whereas the reward drawn from that distribution for the $t$-th time instant is denoted by $X_{i,t}$. We assume all rewards are bounded in $[0,1]$. $n_{i,t_0:t}$ denotes the number of times arm $i$ has been pulled from $t_0$ to $t$ timesteps. $r_{i,t_{c_j:c_{j+1}}}$ denotes the expected mean of the reward distribution $D_{i,\rho_{c_j:c_{j+1}}}$ and $\hat{r}_{i,t_0:t_p}$ denotes the sample mean for the arm $i$ from $t_0$ to $t_p$ timesteps.



\begin{definition}
\label{Def:tcj}
We define $t_{c_j}$ for a $c_j\in\G$ between two consecutive segments $\rho_{c_{j-1}:cj}$ and $\rho_{c_j:c_{j+1}}$ as the first time instant that the changepoint $c_j$ happens $\forall i\in\A$, that is,

\begin{align*}
t_{c_j}& = \min\left\lbrace t\in[1,T],\forall i\in\A : r_{i,t_{c_{j-1}:c_j}} \neq r_{i,t_{c_{j}:c_{j+1}}}   \right\rbrace\\
%%%%%%%%%%%%%%%%%%%%%%%%%%
%&= \min\left\lbrace t\in[1,T],\forall i\in\A: \Delta^{chg}_{i,c_j} \geq \Delta_0 \right\rbrace
\end{align*}

\end{definition}

\begin{assumption}
\label{assm:global}
In this work we have assumed the global changepoint setting and hence $t_{c_j}$ is common across all the arms and the learner does not know the $t_{c_j},\forall c_j\in\G$. 
\end{assumption}

%Also, note that we have assumed that all changepoint gaps $\Delta^{chg}_{i,c_j}$ are above $\Delta_0$.




%\begin{definition}
%\label{Def:d-chg-gap}
%We define a changepoint $c_j$ at time $t_{c_j}$ to be a $d$-optimal changepoint such that for an arm $i\in\A$,
%\begin{align*}
%\left|t_{c_{j-1}} - t_{c_j}\right| \geq d 
%\end{align*}
%
%%\Delta^{chg}_{d,c_j}=\lbrace \exists i\in\A : |\mu_{i,\rho_{c_{j-1}:c_j}}-\mu_{i,\rho_{c_j:c_{j+1}}}|\geq \Delta^{chg}_{i,d}, \text{ where } \\
%%d\geq\dfrac{C\log(t_{c_j}-t_{c_{j-1}})}{\Delta^{chg}_{i,t_{c_j}}}
%
%where $d=\dfrac{C\log(t_{c_j}-t_{c_{j-1}})}{(\Delta^{chg}_{i,c_j})^2}$ and $C$ is an integer constant.
%\end{definition}
%
%\begin{assumption}
%\label{assm:gap}
%We assume that for every two segments $\rho_{c_{j-1}:c_j}$ and $\rho_{c_j:c_{j+1}}$, for $j=1,2,\ldots,G$ there exists atleast one arm $i\in\A$ such that $t_{c_j}$ is a $d$-optimal changepoint.
%\end{assumption}

\begin{assumption}
\label{assm:space-gap}
We assume that for every two consecutive segments $\rho_{c_{j-1}:c_j}$ and $\rho_{c_j:c_{j+1}}, \forall j=1,2,\ldots,G$ all the three  changepoints $t_{c_{j-1}}, t_{c_j}$ and $t_{c_{j+1}}$ satisfy the following condition,
\begin{align*}
\dfrac{d(t_{c_j} - t_{c_{j-1}})}{t_{c_{j+1}} - t_{c_j}} \leq \dfrac{\epsilon_0}{1-\epsilon_0}
\end{align*}
where $d(t_{c_j} - t_{c_{j-1}})$ denotes the delay in detecting a changepoint at $t_{c_j}$ by the learner and $ \epsilon_0 \in (0,1)$.
\end{assumption}

\begin{definition}
\label{Def:chg-gap}
We define the changepoint gap at $t_{c_j}$ for an arm $i\in\A$ between the segments $\rho_{c_{j-1}:c_j}$ and $\rho_{c_j:c_{j+1}}$ as,
\begin{align*}
\Delta^{chg}_{i,c_j}=|r_{i,t_{c_{j-1}:c_j}}-r_{i,t_{c_j:c_{j+1}}}|.
\end{align*}
\end{definition}

\begin{discussion}
\label{dis:gap-delay}
Thus, assumption \ref{assm:space-gap} makes sure that all the changepoints $t_{c_j},\forall j=1,2,\ldots,G$ are sufficiently far apart to detect a change. A good learner tries to minimize the delay ($d$) in detection of a changepoint at $t_{c_j},\forall c_{j}\in\G$ by achieving atmost a delay of $d\left(t_{c_j} - t_{c_{j-1}}\right) \leq O\left( \dfrac{\log( t_{c_j} - t_{c_{j-1}} )}{\Delta_{i,c_j}^{2}}\right)$. Note, that when the gaps are same such that for all $i\in\A$, $\Delta_{i,c_j}^{chg}=\Delta_{\epsilon_0,c_j}^{chg}$ and $d\left(t_{c_j} - t_{c_{j-1}}\right) = \left( \dfrac{\log( t_{c_j} - t_{c_{j-1}} )}{\Delta_{\epsilon_0,c_j}^{2}}\right)$, then due to assumption \ref{assm:space-gap} we get,
\begin{align*}
\Delta_{\epsilon_0,c_j}^{chg}\geq \sqrt{\dfrac{1-\epsilon_0}{\epsilon_0}\dfrac{\log(t_{c_j} - t_{c_{j-1}})}{(t_{c_{j+1}} - t_{c_{j}})}}.
\end{align*} 

Hence, $\Delta_{\epsilon_0,c_j}^{chg}$ is the minimum gap possible that can be detected by the learner given the changepoints $t_{c_{j-1}}, t_{c_j}$ and $t_{c_{j+1}}$ follow assumption \ref{assm:space-gap}.

% that is atleast $\epsilon_0 \geq \dfrac{t_{c_j} - t_{c_{j-1}}}{t_{c_{j+1}} - t_{c_{j-1}}}$. Hence, smaller the $\epsilon_0$, harder it is to detect a changepoint at $t_{c_j}$ and a larger succeeding segment $\rho_{c_j:c_{j+1}}$ is required.
\end{discussion}


%as the preceding segment $t_{c_{j+1}} - t_{c_{j}}$ is smaller and less observation is available from the segment $\rho_{c_{j-1}:c_j}$ 

\begin{definition}
\label{Def:e-chg-gap}
We define a changepoint gap $\Delta^{chg}_{i,c_j}$ at changepoint $c_j\in\G$ for an arm $i\in\A$ to be an $\epsilon_0$-optimal changepoint such that,
\begin{align*}
\Delta^{chg}_{i,c_j} \geq \Delta_{\epsilon_0,c_j}^{chg}.
\end{align*}
\end{definition}


\begin{assumption}
\label{assm:chg-gap}
In this paper we assume that $\forall c_{j}\in\G$ the changepoint gaps $\Delta^{chg}_{i,c_j},\forall i\in\A$ are $\epsilon_0$-optimal.
\end{assumption}


\begin{definition}
\label{Def:opt-gap}
We define the optimality gap $\Delta^{opt}_{i,c_j}$ for an arm $i_{t'}\neq i^*_{t'},\forall t'\in[t_{c_{j-1}},t_{c_j}]$ as,
\begin{align*}
\Delta^{opt}_{i,c_j}= r_{i^*,t_{c_{j-1}:c_j}}-r_{i,t_{c_{j-1}:c_j}}.
\end{align*}
\end{definition}

Also, recall from Definition \ref{Def:chg-gap} that the changepoint gap $\Delta^{chg}_{i,c_j}=|r_{i,t_{c_{j-1}:c_j}}-r_{i,t_{c_j:c_{j+1}}}|$.



\begin{definition}
\label{Def:ratio}
We define $\Psi_{i,c_j}$ as the ratio between $\Delta^{opt}_{i,c_j}$ and $\Delta^{chg}_{i,c_j}$ for the $i$-th arm in $\A$ and the $c_j$-th changepoint in $\G$ such that,

\begin{align*}
\Psi_{i,c_j} = \dfrac{\Delta^{opt}_{i,c_{j}}}{\Delta^{chg}_{i,c_j}}
\end{align*}

\end{definition}

 
%	We define each expert or forecaster as $f_{j}\in \M_{t_0:t_p}$, where $\M_{t_0:t_p}$ is the set of all forecasters from time $t_0$ to $t_p$. $\M^+_{t}$ is the set of new forecasters introduced at time $t$. Also, we define $\hat{L}_{f_i,t_0:t_p}=\sum_{s=t_0}^{t_p}\ell_{f_i ,s}$ as the true cumulative loss suffered by the expert $f_i$ from $t_0$ to $t_p$-th timestep and $\hat{L}_{f_i,t_0:t_p}$ as the estimated cumulative loss suffered by an expert $f_i$ from $t_0$ to $t_p$-th timestep such that $\hat{L}_{f_i,t}=\sum_{s=t_0}^{t_p}\hat{\ell}_{f_i ,s}$. Similarly, we define $L_{i,t_0:t_p}$ and $\hat{L}_{i,t_0:t_p}$ as the true loss and the estimated loss suffered by arm $i$. The weight of an expert $f_i$ at time $t$ is defined as $w_{f_i,t}$ and $\eta$ is defined as a parameter for exploration. Also let $i_{f_j,t}$ be the action suggested by expert $f_j$ at time $t$.


\section{Problem Definitions}
\label{psbandit:probDef}
Given the exploration-exploitation-changepoint dilemma, the objective of the learner in the piecewise-stochastic bandit problem is to minimize the cumulative regret until time $t$, which is defined as follows:
\begin{align*}
R_{t}=\sum_{t'=1}^t\mu_{i^*_{t'}} - \sum_{t'=1}^t\mu_{I_t'}
%R_{t}=\sum_{t'=1}^t\mu_{i^*_{t'}} - \sum_{i=1}^K \left(\mu_{i}n_{i_{t'}\neq i^*_{t'},\forall t'=1:t}\right),
\end{align*}
where $t$ is the number of rounds, $\mu_{i^*_{t'}}$ is the optimal arm at the $t'$ timestep and $\mu_{I_{t'}}$ is the arm chosen by the learner at the $t'$ timestep. Let $n_{i_{t'}\neq i^*_{t'},\forall t'=1:t}$ is the number of times the learner has chosen arm $i$ up to round $t$ when it was not the optimal arm $i^*_{t'}$. The expected regret of an algorithm after $t$ rounds can be written as,

\begin{align*}
\E[R_{t}]&= \E\left[\sum_{t'=1}^t\mu_{i^*_{t'}} - \sum_{t'=1}^t\mu_{I_t}\right]\\
%%%%%%%%%%%%%%%%%%%%%%%%%%%%%%%%%%%%%%%%
&=\sum_{i\neq i^*}^K\left(\mu_{i^*} - \mu_i\right) \E[n_{i_{t'}\neq i^*_{t'},\forall t'=1:t}] \\
%%%%%%%%%%%%%%%%%%%%%%%%%%%%%%%%%%%%%%%%
&\overset{(a)}{=}\sum_{i\neq i^*}^K\sum_{j=1}^{G}\Delta^{opt}_{i,c_j}\E[n_{i_{t'}\neq i^*_{t'},\forall t'=t_{c_{j-1}}:t_{c_j}}]
\end{align*}
where $(a)$ is obtained from assumption \ref{assm:chg-gap} and $\Delta^{opt}_{i,c_j}$ is defined as in Definition \ref{Def:opt-gap}. 


\section{Related Works}
\label{psbandit:related}
Multi-armed bandits (MAB) have been extensively studied in the \textit{stochastic setting} where the distribution associated with each arm is fixed throughout the time horizon. This setting has been extensively discussed in chapter~\ref{chap:SMAB}. Starting from the seminal works of \citet{thompson1933likelihood}, \citet{robbins1952some} and finally in \citet{lai1985asymptotically} the authors provide the asymptotic lower bound for the class of algorithms considered.  The upper confidence bound (UCB) algorithms, which are a type of index-based frequentist strategy, were first proposed in \citet{agrawal1995sample} and the first finite-time analysis for the stochastic setting for this class of algorithms was proved in \citet{auer2002finite}. Over the years several strong UCB type algorithms were proposed for the stochastic setting such as MOSS \citep{audibert2009minimax}, UCBV \citep{audibert2009exploration}, UCB-Improved \citep{auer2010ucb}, OCUCB \citep{lattimore2015optimally}, etc. Further, some of the Bayesian strategies which were proposed for this setting includes the Thompson Sampling \citep{agrawal2012analysis},\citep{agrawal2013further} and Bayes-UCB \citep{kaufmann2012bayesian} algorithms.

Another setting that has greatly motivated the first studies in bandit literature is the \textit{adversarial setting} which was briefly described in chapter~\ref{chap:intro}. In this setting, at every timestep, an adversary chooses the reward for each arm and then the learner selects an arm without the knowledge of the adversary's choice. The adversary may or may not be oblivious to the learner's strategy and this forces the learner to employ a randomized algorithm to confuse the adversary. Previous works on this have focused on constructing different types of exponential weighting algorithms that are based on the Hedge algorithm that has been proposed before in \citet{littlestone1994weighted},\citet{freund1995desicion} and analyzed in \citet{auer1995gambling}. Further variants of this strategy called EXP3 \citep{auer2002nonstochastic} and EXP3IX \citep{kocak2014efficient} have also been proposed which incorporates different strategies for exploration to minimize the loss of the learner.

Striding between these two contrasting settings is the piece-wise stochastic multi-armed bandit setting where there are a finite number of changepoints when the distribution associated with each arm changes abruptly. Hence, this setting is neither as pessimistic as adversarial setting nor as optimistic as the stochastic setting. Therefore, the two broad class of algorithms mentioned before fail to perform optimally in this setting. Several interesting solutions have been proposed before for this setting which can be broadly divided into two categories, passively adaptive and actively adaptive strategies. Passively adaptive strategies like Discounted UCB (DUCB) \citep{kocsis2006discounted}, Switching Window UCB (SW-UCB)  \citep{garivier2011upper} and Discounted Thompson Sampling (DTS) \citep{raj2017taming} do not actively try to locate the changepoints but rather try to minimize their losses by concentrating on past few observations. Similarly, algorithms like Restarting Exp3 (RExp3) \citep{DBLP:journals/corr/BesbesGZ14} behave pessimistically as like Exp3 but restart after pre-determined phases. Hence, RExp3 can also be termed as a passively adaptive algorithm. On the other hand, actively adaptive strategies like Adapt-EVE \citep{hartland2007change}, Windowed-Mean Shift \citep{yu2009piecewise}, EXP3.R \citep{allesiardo2017non}, CUSUM-UCB \citep{liu2017change} try to locate the changepoints and restart the chosen bandit algorithms. Also, there are Bayesian strategies like Global Change-Point Thompson Sampling (GCTS)\citep{mellor2013thompson} which uses Bayesian changepoint detection to locate the changepoints. 

\section{Our Contribution}
\label{psbandit:contribution}
In this chapter, we propose an actively adaptive upper confidence bound (UCB) algorithm, referred to as Improved Changepoint Detector (ImpCPD), that tries to locate the changepoints in the distribution and adapt accordingly to minimize the cumulative regret. Our solution to the piecewise stochastic problem also depends on restarting where we employ ImpCPD which tries to minimize the cumulative regret by balancing exploration-exploitation and a changepoint detector which tries to predict a changepoint with high probability. When the changepoint is detected ImpCPD is again restarted with all the past history of actions and rewards erased. For the changepoint detection we propose a novel algorithm CPDI which works as a subroutine in conjunction with the main ImpCPD algorithm.

%We give a general framework such that any of algorithms that has been discussed in chapter \ref{chap:SMAB} and chapter \ref{chap:EUCBV} can be employed to minimize the cumulative regret whereas 

%, CPD(\ref{alg:CPD2}) and CPD(\ref{alg:CPD3}) based on the three dominating ideas of construction on confidence interval, the union bound involving Chernoff-Hoeffding bound (see Appendix \ref{sec:app:Conc}) as in UCB1 \citep{auer2002finite}, the peeling argument as used in MOSS\citep{audibert2009minimax} and DUCB \citep{garivier2011upper} and finally the Laplace method which has been previously explored in UCB-$\delta$ \citep{abbasi2011improved}.

	Also, in contrast to other existing works we take minimal assumption on the environment considered. The three assumptions taken, that is Assumption\ref{assm:global}, Assumption\ref{assm:space-gap} and Assumption\ref{assm:chg-gap} are the standard assumptions taken by all algorithms which is required for detecting significant changes. Furthermore, among the actively adaptive algorithms, CUSUM-UCB requires only Bernoulli rewards whereas our algorithm is applicable to all sub-Gaussian distribution, EXP3.R is more pessimistic compared to our approach as it employs exponential weighting algorithm that does not fully leverage the piecewise stochastic structure of the environment. 
	
	Empirically, we show that ImpCPD significantly outperforms several algorithms in piecewise stochastic setting. In all the environment considered ImpCPD performs better than all the passively adaptive algorithms and most of the actively adaptive algorithms. 
	
	%Also, our proposed algorithms does not require any additional exploration parameters to be tuned. Further, both of the proposed algorithms are anytime algorithm which does not require horizon $T$ as input. 

\section{Changepoint Detection Algorithms}
\label{psbandit:algorithm}
\subsection{Proposed Algorithms}

In this section, we first introduce the three changepoint detection algorithms CPD(\ref{alg:CPD1}), CPD(\ref{alg:CPD2}) and CPD(\ref{alg:CPD3}) which uses three different confidence intervals which are carefully constructed using three different approaches.  CPD(\ref{alg:CPD1}) uses a simple union bound using Chernoff-Hoeffding inequality whereas CPD(\ref{alg:CPD2}) uses the peeling trick and CPD(\ref{alg:CPD3}) uses the Laplace method which results in a confidence interval that is valid uniformly over time.

Then we introduce the single expert changepoint detection algorithm in Algorithm \ref{alg:SECPD1} which calls one of this CPD algorithms at every timestep while running an expert bandit algorithm which is restarted once a changepoint is detected. Finally, we introduce in Algorithm \ref{alg:SECPD2} which uses the UCB-Improved \citep{auer2010ucb}  style phases where at the end of each phase the algorithm calls one of the CPD to detect the changepoints. Naturally, SECPD2 results in more speedup than SECPD1 which employs the changepoint detection algorithms at every timestep at the cost of only additional logarithmic regret.  

\begin{algorithm}[!ht]
\caption{Changepoint-Detection-1($t_0$, $t_p$) (CPD1)}
\label{alg:CPD1}
\begin{algorithmic}
\For{$k=1,..,K$}
\For{$t' = t_0 ,..,t_p$}
\State \If{$\hat{\mu}_{k,t_0:t'} + \sqrt{\dfrac{\log(\frac{t_p}{\sqrt{\delta}})}{n_{k,t_0:t'}}} < \hat{\mu}_{k,t'+1:t_p} - \sqrt{\dfrac{\log(\frac{t_p}{\sqrt{\delta}})}{n_{k,t'+1:t_p}}}$}
\State Return True
\Else{$\hat{\mu}_{k,t_0:t'} - \sqrt{\dfrac{\log(\frac{t_p}{\sqrt{\delta}})}{n_{k,t_0:t'}}} > \hat{\mu}_{k,t'+1:t_p} + \sqrt{\dfrac{\log(\frac{t_p}{\sqrt{\delta}})}{n_{k,t'+1:t_p}}}$}
\State Return True
\EndIf
\EndFor
\EndFor
\end{algorithmic}
\end{algorithm}

\begin{algorithm}[!ht]
\caption{Changepoint-Detection-2($t_0$, $t_p$) (CPD2)}
\label{alg:CPD2}
\begin{algorithmic}
\State {\bf Input:} Exploration parameter $B>1$
\For{$k=1,..,K$}
\For{$t' = t_0 ,..,t_p$}
\State \If{$\hat{\mu}_{k,t_0:t'} + \sqrt{\dfrac{B\log(\frac{t_p}{\sqrt{\delta}})}{n_{k,t_0:t'}}} < \hat{\mu}_{k,t'+1:t_p} - \sqrt{\dfrac{B\log(\frac{t_p}{\sqrt{\delta}})}{n_{k,t'+1:t_p}}}$}
\State Return True
\Else{$\hat{\mu}_{k,t_0 , t'} - \sqrt{\dfrac{B\log(\frac{t_p}{\sqrt{\delta}})}{n_{k,t_0:t'}}} > \hat{\mu}_{k,t'+1:t_p} + \sqrt{\dfrac{B\log(\frac{t_p}{\sqrt{\delta}})}{n_{k,t'+1:t_p}}}$}
\State Return True
\EndIf
\EndFor
\EndFor
\end{algorithmic}
\end{algorithm}

\begin{algorithm}[!ht]
\caption{Changepoint-Detection-3($t_0$, $t_p$) (CPD3)}
\label{alg:CPD3}
\begin{algorithmic}
\For{$k=1,..,K$}
\For{$t' = t_0 ,..,t_p$}
\State \If{$\hat{\mu}_{k,t_0:t'} + \sqrt{\frac{(n_{i,t_0:t'}+1)\log(\frac{(n_{i,t_0:t'}+1)}{\sqrt{\delta}})}{2n_{i,t_0:t'}^2}} < \hat{\mu}_{k,t'+1:t_p} - \sqrt{\frac{(n_{i,t'+1:t}+1)\log(\frac{(n_{i,t'+1:t}+1)}{\sqrt{\delta}})}{2n_{i,t'+1:t}^2}}$}
\State Return True
\Else{$\hat{\mu}_{k,t_0 , t'} - \sqrt{\frac{(n_{i,t_0:t'}+1)\log(\frac{(n_{i,t_0:t'}+1)}{\sqrt{\delta}})}{2n_{i,t_0:t'}^2}} > \hat{\mu}_{k,t'+1:t_p} + \sqrt{\frac{(n_{i,t'+1:t}+1)\log(\frac{(n_{i,t'+1:t}+1)}{\sqrt{\delta}})}{2n_{i,t'+1:t}^2}}$}
\State Return True
\EndIf
\EndFor
\EndFor
\end{algorithmic}
\end{algorithm}


%\begin{algorithm}[!ht]
%\caption{Changepoint-Detection-4($t_0$, $t_p$) (CPD4)}
%\label{alg:CPD4}
%\begin{algorithmic}
%\State \textbf{Initialization: } $m=0$, $\epsilon_0 = 1$, $B>1$
%\State \textbf{Definition: } $n_m = \dfrac{B\log(\frac{t}{{\delta}})}{\epsilon_m}$
%\State \If{$tp == n_{m+1}$}
%\For{$k=1,..,K$}
%\For{$t' = t_0 ,..,t_p$}
%\State \If{$\hat{\mu}_{k,t_0:t'} + \sqrt{\dfrac{\log(\frac{t}{{\delta}} )}{2t}} < \hat{\mu}_{k,t':t_p} - \sqrt{\dfrac{\log(\frac{t}{{\delta}} )}{2t}}$}
%\State Return True
%\Else{$\hat{\mu}_{k,t_0 , t'} - \sqrt{\dfrac{\log(\frac{t}{{\delta}} )}{2t}} > \hat{\mu}_{k,t':t_p} + \sqrt{\dfrac{\log(\frac{t}{{\delta}} )}{2t}}$}
%\State Return True
%\EndIf
%\EndFor
%\EndFor
%\EndIf
%\end{algorithmic}
%\end{algorithm}

\begin{algorithm}[!ht]
\caption{Single Expert with change-point detection (SECPD-1)}
\label{alg:SECPD1}
\begin{algorithmic}
\State {\bf Input:} Time horizon $T$; 
\State {\bf Initialization:} $t_0 = 1$, $t_p = 1$, $\M=\lbrace 0\rbrace$;
\State {\bf New Expert:} Start a new expert $f_{t_0}$ and add it to $\M$.
\State Pull each arm once
\State \For{$t=K+1,..,T$}
\State Play the arm $i_{t}$ suggested by $f_{t_0}$, observe reward $x_{i_t,t}$.

\If{ (Changepoint-Detection($t_0$, $t_p$)) == True}\hspace*{4em} /* Call CPD\ref{alg:CPD1} or CPD\ref{alg:CPD2} or CPD\ref{alg:CPD3} */
\State {\bf Reset Parameters:} $t_0=1$, $t_p = 1$, $\M=\lbrace 0\rbrace$\hspace*{4em}/* Changepoint detected, forget old expert  */
\State  Start a new expert $f_{t_0}$ and add it to $\M$\hspace*{4em}/* Add a new expert*/
\Else{
\State Update the local model of $f_{t_0}$
\State $t_p = t_p + 1$}
\EndIf
\EndFor
\end{algorithmic}
\end{algorithm}


\begin{algorithm}[!ht]
\caption{Single Expert with change-point detection (SECPD-2)}
\label{alg:SECPD2}
\begin{algorithmic}

\State {\bf Input:} Time horizon $T$, parameter exploration $\delta$; 
\State {\bf Initialization:} $t_0 = 1$, $t_p = 1$, $\M=\lbrace 0\rbrace$, $m=0$, $\epsilon_0 = 1$, $n_0 = \frac{B\log (\frac{t_p}{\sqrt{\delta}})}{\epsilon_0}$, $p_{0}=Kn_0$;
\State {\bf New Expert:} Start a new expert $f_{t_0}$ and add it to $\M$.
\State Pull each arm once
\State \For{$t=K+1,..,T$}
\State Play the arm $i_{t}$ suggested by $f_{t_0}$, observe reward $x_{i_t,t}$.
\State \State Update the local model of $f_{t_0}$
\State $t_p = t_p + 1$
\If{($t_p \neq p_{m}$)}
\State \State Update the local model of $f_{e}$
\State $t_p = t_p + 1$
\Else{
\If{ (Changepoint-Detection($t_0$, $t_p$)) == True}\hspace*{4em} /* Call CPD\ref{alg:CPD1} or CPD\ref{alg:CPD2} or CPD\ref{alg:CPD3} */
\State {\bf Reset Parameters:} $t_0=1$, $t_p = 1$, $\M=\lbrace 0\rbrace$\hspace*{4em}/* Changepoint detected, forget old expert  */
\State Start a new expert $f_{t_0}$ and add it to $\M$\hspace*{4em}/* Add a new expert */
\State $m=0$, $\epsilon_0 = 1$, $n_0 = \dfrac{K B\log (\frac{t_p}{\sqrt{\delta}})}{\epsilon_0}$;
\Else{
\State {\bf Update Parameters:} $\epsilon_{m+1} = \max{\left\lbrace\sqrt{\frac{e}{t_p}},\frac{\epsilon_m}{B}\right\rbrace}$,  $n_{m+1} = \frac{B\log (\frac{t_p}{\sqrt{\delta}})}{\epsilon_{m+1}}$, $p_{m+1}=t_p + Kn_{m+1}$, $m=m+1$;
}
\EndIf}
\EndIf
%\State\If{($t_p \neq n_m$)}
%\State Update the local model of $f_{t_0}$
%\State $t_p = t_p + 1$
%\EndIf
\EndFor

\end{algorithmic}
\end{algorithm}


%\begin{algorithm}[!ht]
%\caption{Aggregate Expert with change-point detection (AECPD-1)}
%\label{alg:AECPD1}
%\begin{algorithmic}
%
%\State {\bf Input:} Time horizon $T$, parameter exploration $\delta$; 
%\State {\bf Initialization:} $t_0 = 1$, $e = 1$, $t_p = 1$, $\M =\lbrace 0\rbrace$, $m=0$, $\epsilon_0 = 1$, $n_0 = \frac{B\log (\frac{t_p}{\sqrt{\delta}})}{\epsilon_0}$, $p_{0}=Kn_{0}$;
%\State {\bf New Expert:} Start a new expert $f_{e}$ and add it to $\M $.
%\State \For{$t=1,..,T$}
%\State Play the arm $i_{t}$ suggested by $f_{e}$, observe reward $x_{i_t,t}$.
%\If{($t_p \neq p_{m}$)}
%\State \State Update the local model of $f_{e}$
%\State $t_p = t_p + 1$
%\Else{
%\If{ (Changepoint-Detection($t_0$, $t_p$)) == True}\hspace*{4em} /* Call CPD\ref{alg:CPD1} or CPD\ref{alg:CPD2} or CPD\ref{alg:CPD3} */
%\State {\bf Reset Parameters:} $t_0=1$, $t_p = 1$, $\M =\lbrace f_{e}(\text{model})\rbrace$\hspace*{0.0em}/*Changepoint detected, store old expert*/
%\State $e=e+1$
%\State Start a new expert $f_{e}$ and add it to $\M $\hspace*{4em}/* Add a new expert */
%\State $m=0$, $\epsilon_0 = 1$, $n_0 = \dfrac{K B\log (\frac{t_p}{\sqrt{\delta}})}{\epsilon_0}$;
%\ElsIf{(Model-Overlap($\M$,$f_e$)==True)}
%\State $f_e = f_{suggest}$
%\Else{
%\State {\bf Update Parameters:} $\epsilon_{m+1} = \max{\left\lbrace\sqrt{\frac{e}{t_p}},\frac{\epsilon_m}{B}\right\rbrace}$,  $n_{m+1} =\frac{B\log (\frac{t_p}{\sqrt{\delta}})}{\epsilon_{m+1}}$, $p_{m+1}=t_p + Kn_{m+1}$, $m=m+1$;
%}
%\EndIf}
%\EndIf
%\EndFor
%
%\end{algorithmic}
%\end{algorithm}
%
%%\If{(Model-Overlap($\M$,$f_e$)==True)}
%%\State $f_e = f_{suggest}$
%%\State \State Update the local model of $f_{e}$
%%\State $t_p = t_p + 1$
%%\Else{
%
%\begin{algorithm}[!ht]
%\caption{Aggregation of Expert with change-point detection (EAggrCPD)}
%\label{alg:EAggrCPD}
%\begin{algorithmic}
%\State {\bf Input:} Time horizon $T$, parameter exploration $\eta$; 
%\State {\bf Initialization:} $t_0 = 1$, $t_p = 1$, $M_{t_0}=\lbrace 0\rbrace$, $\hat{L}_{i,t_0}=0,\forall i\in\A$;
%\State \For{$t=1,..,T$}
%\State {\bf New Expert:} Start a new expert $f_{t_p}$ and add it to $\M_{t_0:t_p}$.
%\State Set $\hat{L}_{f_{t_p},t_p}=0 , w_{f_{t_p},0} = \dfrac{1}{t_p}\sum_{f_j\in M_{t_0:t_p}}w_{f_j,t_p -1}\exp(-\eta\hat{L}_{f_j,t_0:t_p -1})$
%\State \For{$i=1,..,K$}
%\State $H_{i_{f_j,t_p}}=\sum_{f_j\in \M_{t_0:t_p}}(w_{f_j,t_p -1})\mathbb{I}[i_{f_j,t_p}=i] $
%\EndFor
%
%\State \For{$i=1,..,K$}
%\State $\hat{p}_{i,t_p} = \dfrac{H_i\exp(-\eta\hat{L}_{i,t_p -1})}{\sum_{i\in \A}H_i\exp(-\eta\hat{L}_{i,t_0:t_p -1})}$ 
%\EndFor
%
%\State Play the arm $i_{t}$ according to the probability $\hat{p}_{i,t}$, observe reward $x_{i_t,t}$.
%\State $\hat{L}_{i_t,t_0:t_p} = \hat{L}_{i_t,t_0:t_p} + \dfrac{1-x_{i_t,t_p}}{\hat{p}_{i_t,t_p}}$
%
%
%\If{ (Changepoint-Detection($t_0$, $t_p$)) == True}
%\State {\bf Reset Parameters:} $t_0=1$, $t_p = 1$, $M_{t_0}=\lbrace 0\rbrace$, $\hat{L}_{i,t_0}=0,\forall i\in\A$ \hspace*{4em}/*Changepoint Detected*/
%\Else{
%\State {\bf Update Weights:} \For{$j=1,..,|M_{t_0:t_p}|$}
%\State \If{$i_{t}=i_{f_j,t_p}$}
%\State $\hat{L}_{f_j,t_0:t_p} = \hat{L}_{f_j,t_0:t_p} + \dfrac{1-x_{i_t,t}}{\hat{p}_{i_t,t}}$
%\State $w_{f_j,t_p}=\exp(-\eta\hat{L}_{f_j,t_0:t_p})$
%\State Update the local model of $f_j$
%\State $t_p = t_p + 1$
%\EndIf
%\EndFor}
%\EndIf
%\EndFor
%\end{algorithmic}
%\end{algorithm}



%\subsection{Proposed Algorithm 2 (Aggregation of Experts with CPD)

 



%\section{Theoretical Results}
%\label{tbandit:results}
%\section{Results}
\label{sec:results}
% \begin{table}[!ht]
% \centering
% \caption{My caption}
% \label{my-label}
% \begin{tabular}{|p{1.2cm}|l|l|l|l|l|l|}
% \hline
% \multirow{2}{*}{\begin{tabular}[c]{@{}c@{}}Feature \\ Name\end{tabular}} & \multicolumn{2}{c}{Vt Classifier}                                       & \multicolumn{4}{|c|}{Size classifier}                                                                                                   \\ \cline{2-7} 
%                                                                          & \multicolumn{1}{c|}{1} & \multicolumn{1}{c|}{2} & \multicolumn{1}{c|}{ 1} & \multicolumn{1}{c|}{ 2} & \multicolumn{1}{c|}{3} & \multicolumn{1}{c|}{4} \\ \hline
% Sub-circuit                                                              & \multicolumn{1}{l|}{}        &                               &                               &                               &                              &                              \\ \hline
% Gate Type                                                                & \multicolumn{1}{l|}{}        &                               &                               &                               &                              &                              \\ \hline
% LNS                                                  & \multicolumn{1}{l|}{}        &                               &                               &                               &                              &                              \\ \hline
% \#Fanins                                                                 & \multicolumn{1}{l|}{}        &                               &                               &                               &                              &                              \\ \hline
% \#Fanouts                                                                & \multicolumn{1}{l|}{}        &                               &                               &                               &                              &                              \\ \hline
% \begin{tabular}[c]{@{}l@{}}\#Negative \\ Slack Paths\end{tabular}                                          & \multicolumn{1}{l|}{}        &                               &                               &                               &                              &                              \\ \hline
% Slack                                                                    & \multicolumn{1}{l|}{}        &                               &                               &                               &                              &                              \\ \hline
% \end{tabular}
% \end{table}
% \begin{table*}[!ht]
% \centering
% \caption{result3}
% \label{results3}
% \begin{tabular}{|l|c|l|l|l|l|l|l|l|l|l|l|}
% \hline
% \multicolumn{1}{|c|}{\begin{tabular}[c]{@{}c@{}}Benchmark \\  Name\end{tabular}} & \begin{tabular}[c]{@{}c@{}}Number \\ Of gates\end{tabular} & \multicolumn{1}{c|}{\begin{tabular}[c]{@{}c@{}}Target \\ Delay\end{tabular}} & \begin{tabular}[c]{@{}l@{}}Inital \\ Delay\end{tabular} & \multicolumn{2}{c|}{SVM}                                         & \multicolumn{2}{c|}{Final}                                      & \multicolumn{2}{c|}{Igor Markov}                               & \multicolumn{2}{c|}{Flach}                                     \\ \hline
% \multicolumn{1}{|c|}{}                                                           &                                                            & \multicolumn{1}{c|}{}                                                        & \multicolumn{1}{c|}{}                                   & \multicolumn{1}{c|}{Delay (ns)} & \multicolumn{1}{c|}{Power (W)} & \multicolumn{1}{c|}{Delay (ns)} & \multicolumn{1}{c|}{Power(W)} & \multicolumn{1}{c|}{Delay(ns)} & \multicolumn{1}{c|}{Power(W)} & \multicolumn{1}{c|}{Delay(ns)} & \multicolumn{1}{c|}{Power(W)} \\ \hline
% DMA\_fast                                                                        & 25.3K                                                      &                                                                              &                                                         &                                 &                                &                                 &                               &                                &                                                0.299    &       &                               \\ \hline
% DMA\_slow                                                                        & 25.3K                                                      &                                                                              &                                                         &                                 &                                &                                 &                               &                                &                                                       0.145   &     &                               \\ \hline
% pci\_fast                                                              & 33.2K                                                      &                                                                              &                                                         &                                 &                                &                                 &                               &                                &                                                     0.183     &     &                               \\ \hline
% pci\_slow                                                              & 33.2K                                                      &                                                                              &                                                         &                                 &                                &                                 &                               &                                &                                                  0.111         &    &                               \\ \hline
% des\_perf\_fast                                                                  & 111K                                                       &                                                                              &                                                         &                                 &                                &                                 &                               &                                &                                                         1.842   &   &                               \\ \hline
% des\_perf\_slow                                                                  & 111K                                                       &                                                                              &                                                         &                                 &                                &                                 &                               &                                &                                                          0.614   &  &                               \\ \hline
% vga\_lcd\_fast                                                                   & 165K                                                       &                                                                              &                                                         &                                 &                                &                                 &                               &                                &                                                            0.471  & &                               \\ \hline
% vga\_lcd\_slow                                                                   & 165K                                                       &                                                                              &                                                         &                                 &                                &                                 &                               &                                &                                                            0.351  & &                               \\ \hline
% b19\_fast                                                                        & 219K                                                       &                                                                              &                                                         &                                 &                                &                                 &                               &                                &                                                           0.771   & &                               \\ \hline
% b19\_slow                                                                        & 219K                                                       &                                                                              &                                                         &                                 &                                &                                 &                               &                                &                                                             0.583 & &                               \\ \hline
% leon3mp\_fast                                                                    & 649K                                                       &                                                                              &                                                         &                                 &                                &                                 &                               &                                &                                                             1.487 & &                               \\ \hline
% leon3mp\_slow                                                                    & 649K                                                       &                                                                              &                                                         &                                 &                                &                                 &                               &                                &                                                            1.341  & &                               \\ \hline
% netcard\_fast                                                                    & 959K                                                       &                                                                              &                                                         &                                 &                                &                                 &                               &                                &                                                            1.861  & &                               \\ \hline
% netcard\_slow                                                                    & 959K                                                       &                                                                              &                                                         &                                 &                                &                                 &                               &                                &                                                           1.770  &  &                               \\ \hline
% \end{tabular}
% \end{table*}


% Please add the following required packages to your document preamble:
% \usepackage{multirow}
% Please add the following required packages to your document preamble:
% \usepackage{multirow}
% \begin{table}[]
% \centering
% \caption{My caption}
% \label{my-label}
% \begin{tabular}{|l|l|l|l|l|l|}
% \hline
% \multirow{3}{*}{Benchmark} & \multicolumn{5}{c|}{Runtime}                                                            \\ \cline{2-6} 
%                            & \multirow{2}{*}{Igor Markov} & \multirow{2}{*}{Flach} & \multicolumn{3}{c|}{\textit{MLTimer}} \\ \cline{4-6} 
%                            &                              &                        & SVM  & Delay Recovery  & Total  \\ \hline
% DMA\_fast                  &                              &                        &      &                 &        \\ \hline
% DMA\_slow                  &                              &                        &      &                 &        \\ \hline
% pci\_fast        &                              &                        &      &                 &        \\ \hline
% pci\_brdige32\_slow        &                              &                        &      &                 &        \\ \hline
% vga\_lcd\_fast             &                              &                        &      &                 &        \\ \hline
% vga\_lcd\_slow             &                              &                        &      &                 &        \\ \hline
% des\_perf\_fast            &                              &                        &      &                 &        \\ \hline
% des\_perf\_slow            &                              &                        &      &                 &        \\ \hline
% b19\_fast                  &                              &                        &      &                 &        \\ \hline
% b19\_slow                  &                              &                        &      &                 &        \\ \hline
% leon3mp\_fast              &                              &                        &      &                 &        \\ \hline
% leon3mp\_slow              &                              &                        &      &                 &        \\ \hline
% netcard\_fast              &                              &                        &      &                 &        \\ \hline
% netcard\_slow              &                              &                        &      &                 &        \\ \hline
% \end{tabular}
% \end{table}


% Please add the following required packages to your document preamble:
% \usepackage{multirow}
% Please add the following required packages to your document preamble:
% \usepackage{multirow}
% Please add the following required packages to your document preamble:
% \usepackage{multirow}
\begin{table*}[!t]
\caption{Leakage power and Runtime comparisons between the baseline greedy algorithm and the \textit{MLTimer} algorithm on the ISPD 2012 benchmarks. Implementation 1 is the baseline implementation(non-SVM,non-adaptive timing analysis), Implementation 2 is with SVM and non-adaptive timing analysis, Implementation 3 is with non-SVM and adaptive timing analysis and Implementation 4 is with SVM and adaptive timing analysis. It can be seen that using just SVM improves the solution quality greatly, while using just the adaptive timing analysis improves the runtime. A combination of both improves the runtime and solution qualtiy.}
\label{tab:tab5}

\begin{tabular}{|l|l|l|l|l|l|l|l|l|l|}
\hline
\multirow{2}{*}{Benchmarks} & \multirow{2}{*}{\#Gates} & \multicolumn{2}{l|}{Implementation 1}                                                                                                         & \multicolumn{2}{l|}{Implementation 2}                                                                                                           & \multicolumn{2}{l|}{Implementation 3}                                                                                                        & \multicolumn{2}{l|}{Implementation 4}                                                                                                        \\ \cline{3-10} 
                            &                          & \begin{tabular}[c]{@{}l@{}}Run-\\ time \\ (mins)\end{tabular} & \begin{tabular}[c]{@{}l@{}}Leakage \\ Power\\ (W)\end{tabular} & \begin{tabular}[c]{@{}l@{}}Run-\\ time\\ (mins)\end{tabular} & \begin{tabular}[c]{@{}l@{}}Leakage \\ Power\\ \\ (W)\end{tabular} & \begin{tabular}[c]{@{}l@{}}Run-\\ time\\ (mins)\end{tabular} & \begin{tabular}[c]{@{}l@{}}Leakage\\  Power\\ (W)\end{tabular} & \begin{tabular}[c]{@{}l@{}}Run-\\ time\\ (mins)\end{tabular} & \begin{tabular}[c]{@{}l@{}}Leakage \\ Power\\ (W)\end{tabular} \\ \hline
DMA\_fast                   & 23,000                   & 16                                                            & 0.79                                                           & 14.00                                                        & 0.30                                                              & 14.00                                                        & 0.79                                                           & 13.00                                                        & 0.30                                                           \\ \hline
pci\_bridge32\_fast         & 30,000                   & 37                                                            & 0.25                                                           & 17.00                                                        & 0.14                                                              & 17.00                                                        & 0.24                                                           & 17.00                                                        & 0.14                                                           \\ \hline
des\_perf\_fast             & 102,000                  & 219                                                           & 1.73                                                           & 164.00                                                       & 1.80                                                              & 190.00                                                       & 1.73                                                           & 130.00                                                       & 1.80                                                           \\ \hline
vga\_lcd\_fast              & 148,000                  & 384                                                           & 2.80                                                           & 139.00                                                       & 0.47                                                              & 207.00                                                       & 2.72                                                           & 77.00                                                        & 0.47                                                           \\ \hline
b19\_fast                   & 213,000                  & 547                                                           & 2.13                                                           & 239.00                                                       & 0.75                                                              & 366.00                                                       & 2.13                                                           & 174.00                                                       & 0.75                                                           \\ \hline
leon3mp\_fast               & 540,000                  & 2,046                                                         & 4.00                                                           & 875.00                                                       & 1.49                                                              & 716.00                                                       & 4.00                                                           & 639.00                                                       & 1.49                                                           \\ \hline
netcard\_fast               & 861,000                  & 1,033                                                         & 2.09                                                           & 519.00                                                       & 1.77                                                              & 609.00                                                       & 2.07                                                           & 306.00                                                       & 1.77                                                           \\ \hline
\end{tabular}
\end{table*}

\begin{table*}[!ht]
%\centering
\caption{Leakage power comparisons with ISPD 2012 contest winners and other state of the art works. We use geometric mean to calculate the efficiency of our proposed solution. We exclude the infeasible solutions in our mean calculation. All the solutions reported below have no timing violations.}
\label{tab:tab6}

    \begin{tabular}{|l|l|l|p{1.2cm}|p{1.6cm}|p{1.6cm}|p{1cm}|l|p{1.2cm}|}
\hline
\multirow{2}{*}{Benchmark} & \multirow{2}{*}{\begin{tabular}[c]{@{}l@{}}Number \\ of gates\end{tabular}} & \multicolumn{5}{c|}{Leakage Power (W)} & \multicolumn{2}{c|}{Runtime (mins)}\\ \cline{3-9} 
    &  & \cite{hu:12}  & NTUgs & UFRGSgs & Powervalve & \textbf{Ours} & \cite{hu:12} & \textbf{Ours}\\ \hline
    \texttt{DMA\_fast} & 23,000 & 0.30  & 0.51 & 0.32 & 0.31 & 0.30 & 13.90 & 13.30\\ \hline
    \texttt{DMA\_slow} & 23,000  & 0.15  & 0.21 & 0.16 & 0.15 & 0.14 & 9.90 & 7.51 \\ \hline
    \texttt{pci\_fast} & 30,000 & 0.18  & 0.51 & 0.17 & 0.23 & 0.14 & 13.00 & 17.10
     \\ \hline
    \texttt{pci\_slow} & 30,000 & 0.11   & 0.20 & 0.12 & 0.12 & 0.09 & 10.20 & 9.32 \\ \hline
    \texttt{des\_perf\_fast} & 102,000 & 1.84 & 2.39 & 3.52 & 2.32 & 1.80  & 82.70 & 130.40 \\ \hline
    \texttt{des\_perf\_slow} & 102,000 & 0.61 & 0.67 & 0.88 & 0.70 & 0.64 & 70.10 & 43.50 \\ \hline
    \texttt{vga\_lcd\_fast} & 148,000 & 0.47 & 0.76 & 0.58 & 0.77 & 0.47 & 45.60 & 77.32\\ \hline
    \texttt{vga\_lcd\_slow} & 148,000 & 0.35 & 0.42 & 0.38 & 0.39 & 0.37 & 87.50 & 50.40 \\ \hline
    \texttt{b19\_fast} & 213,000 & 0.77 & 2.71 & - & 4.49 & 0.75 & 206.50 & 174.11 \\ \hline
    \texttt{b19\_slow} & 213,000 & 0.58 & 0.63 & 0.61 & 0.74 & 0.61 & 213.90 & 102.20\\ \hline
    \texttt{leon3mp\_fast} & 540,000 & 1.49 & -&  - & 4.94 & 1.49 & 1,323.20 & 639.40\\ \hline
    \texttt{leon3mp\_slow} & 540,000 & 1.34 & 1.42 & 1.79 & 2.96 & 1.30 & 1,274.20 & 325.13  \\ \hline
    \texttt{netcard\_fast} & 861,000 & 1.86 & 2.01 & 2.30 & 2.97 & 1.86 & 1,096.90 & 306.57\\ \hline
    \texttt{netcard\_slow} & 861,000 & 1.77 & 1.77 & 1.97 & 1.94 & 1.77 & 299.90 & 164.14\\ \hline
Geometric mean &  & $1.03\times$ & $1.52\times$ & $1.13\times$ & $1.57\times$ &   & $1.44\times$ & \\ \hline
\end{tabular}
\end{table*}

% Please add the following required packages to your document preamble:
% \usepackage{multirow}
\begin{table*}[!ht]
\centering
\caption{Leakage power comparisons with \cite{hu:13} on the  ISPD 2013 contest benchmark. All the solutions reported below are violation free. It can be observed that \texttt{MLTimer} outperforms \cite{hu:13} both with respect to leakage power and runtime on the larger benchmarks. The detailed results for other benchmarks were not reported in \cite{hu:13}.}
\label{tab:tab34}
\begin{tabular}{|l|l|l|l|l|l|}
\hline
\multirow{2}{*}{Benchmark} & \multirow{2}{*}{Gates} & \multicolumn{2}{l|}{\texttt{MLTimer}}                                                                                                  & \multicolumn{2}{l|}{\cite{hu:13}}                                                                                                  \\ \cline{3-6} 
                           &                        & \begin{tabular}[c]{@{}l@{}}Run-\\ time\\ (mins)\end{tabular} & \begin{tabular}[c]{@{}l@{}}Leakage\\ Power\\ (mW)\end{tabular} & \begin{tabular}[c]{@{}l@{}}Run-\\ time\\ (mins)\end{tabular} & \begin{tabular}[c]{@{}l@{}}Leakage \\ Power\\ (mW)\end{tabular} \\ \hline
usb\_phy\_fast             & 510                    & 0.48                                                         & 2.03                                                           & \textbf{0.21}                                                & \textbf{1.56}                                                   \\ \hline
usb\_phy\_slow             & 510                    & \textbf{0.11}                                                & 1.13                                                           & 0.17                                                         & \textbf{1.07}                                                   \\ \hline
pci\_bridge32\_fast        & 28,000                 & 20.83                                                        & 116.87                                                         & \textbf{12.00}                                               & \textbf{101.90}                                                 \\ \hline
pci\_bridge32\_slow        & 28,000                 & 6.78                                                         & 58.91                                                          & \textbf{5.39}                                                & \textbf{58.83}                                                  \\ \hline
fft\_fast                  & 31,000                 & 40.00                                                        & 320.37                                                         & \textbf{32.58}                                               & \textbf{305.29}                                                 \\ \hline
fft\_slow                  & 31,000                 & 25.00                                                        & 96.69                                                          & \textbf{17.40}                                               & \textbf{93.10}                                                  \\ \hline
cordic\_slow               & 42,000                 & \textbf{94.40}                                               & \textbf{397.81}                                                & 98.39                                                        & 511.91                                                          \\ \hline
des\_perf\_slow            & 104,000                & 88.18                                                        & 386.41                                                         & \textbf{62.30}                                               & \textbf{375.80}                                                 \\ \hline
edit\_dist\_fast           & 121,000                & \textbf{163.10}                                              & \textbf{572.12}                                                & 170.60                                                       & 619.30                                                          \\ \hline
edit\_dist\_slow           & 121,000                & \textbf{56.34}                                               & \textbf{423.50}                                                & 107.20                                                       & 465.60                                                          \\ \hline
matrix\_mult\_slow         & 153,000                & \textbf{139.80}                                              & \textbf{482.23}                                                & 212.60                                                       & 499.90                                                          \\ \hline
netcard\_fast              & 884,000                & \textbf{372.70}                                              & \textbf{5,157.93}                                              & 716.80                                                       & 5271.80                                                         \\ \hline
netcard\_slow              & 884,000.               & \textbf{297.12}                                              & \textbf{5,102.25}                                              & 439.60                                                       & 5183.89                                                         \\ \hline
GEOMETRIC MEAN             &                &                                               &                                               & 1.005                                                       & 1.005                                                         \\ \hline

\end{tabular}
\end{table*}

% \begin{table*}[!ht]
% \begin{center}
% \label{results3}
% \begin{tabular}{|p{2.1cm}|p{2cm}|p{2cm}|p{2cm}|p{2cm}|p{1.5cm}|}
% \hline
% Benchmark & $\#$ gates & %\multicolumn{2}{|c|}
% {\textit{MLTimer}} &%\multicolumn{2}{|c|} 
% {Igor Markov} ~\cite{hu:12} & Improvement \\
% \hline
%   &  & Leakage Power (W)      %& Running Time    
%   & Leakage Power (W) & \\% & Running Time \\ 
 
% \hline
% %USB\_PHY &536 &$<$1s &$<$1s & - \\
% \hline
% DMA\_fast & 25.3K & 0.08W %& 17m
% & 0.299W & 73\% \\ %& 13m\\
% \hline
% %DMA\_slow & 25.3 & 0.134W %& 1m44s
% %& 0.145W & 7\%  \\% & 9.9m\\
% %\hline
% pci\_bridge	&	33.2K	&	0.1331W %&	1m29s
% &	0.183W & 27\%\\%	&	13m \\
% \hline
% %pci\_bridge\_slow	&	33.2K	&	0.07W	%&	8m
% %&	0.111W & 36\% \\%	&	11m \\
% %\hline 
% b19	&	219K  &		.58W	%&	9h
% &	0.771W & 24\% \\%	&	206m \\
% \hline
% %b19\_slow 	&	219K &		0.486W%	&	9h	
% %&	0.583W & 19.21\% \\	%&	213m \\
% %\hline
% Des\_perf & 165K & .546W %& 1331m       
% & .471W & \-15.3\% \\%    & 45m \\
% \hline
% netcard & 959k & 1.8W %& 2046m 
% & 1.861W & 6.01 \% \\% & 1096m \\
% \hline
% leon3mp & 649K & 2W %&  2816m 
% & 1.487W & \-34.5\% \\ %& 1323.2 \\
% \hline
% Average & & & & 21.37\% \\ \hline
% %leon3mp\_slow & 649K & 2W &  2816m & 1.487W & 1323.2 \\
% %\hline
% \end{tabular}
% \caption{Leakage and Running Time Comparisons for ISPD benchmarks between \textit{MLTimer} and Igor Markov. In the table, {\bf h}, {\bf m} and {\bf s} stand for hours, minutes and seconds respectively.}


% \end{center}
% \end{table*}
\subsection{Comparisons with state-of-the-art}

The performance of our proposed algorithm is shown in Table~\ref{tab:tab5}. A simple greedy algorithm, implemented for obtaining the final $V_t$ and $size$ values, serves as the baseline algorithm. It can be seen that our \textit{MLTimer} implementation outperforms the baseline algorithm by 46\% in terms of solution quality. It can also be seen that the SVM module improves the solution quality and the adaptive timing analysis module improves the runtime. 

We compare the performance of our algorithm with ~\cite{hu:12} which is the best performing heuristic based algorithm reported so far in the literature. We use the ISPD 2012 benchmark set and SHAKTIC to quantify the performance our algorithm. In comparing with the state-of-the-art techniques we make the following observations:
\begin{itemize}
\item Our solution outperforms the top 3 submissions of the ISPD 2012 contest NTUgs, UFRGSgs and Powervalve by 52\%,13\% and 57\% respectively.
\item Our solution outperforms \cite{hu:12} both in terms of average runtime and solution quality by 44\% and 3\% respectively. Table~\ref{tab:tab6} highlights the performance of \textit{MLTimer} algorithm in terms of runtime and solution quality. This is because as most of the circuits share a large  number of repeating sub-circuits whose value is accurately predicted by the SVM engine and hence these gates do not undergo delay and power recovery algorithm leading to savings in runtime. 
\item It can be seen from Table~\ref{tab:tab34} that our tool outperforms \cite{hu:13} which is an extension of \cite{hu:12}. It can be observed that while \textit{MLTimer} underperforms for the smaller benchmarks, it significantly outperforms \cite{hu:13} on the larger benchmarks. Although the overall improvement in solution quality is around 0.004\%, the improvement in the larger benchmarks is around 53\% for the runtime and 10\% for solution quality.
\item In Table~\ref{tab:tab9} we compare our implementation with a commercial synthesis tool and our implementation of \cite{hu:12}. It can be observed our proposed solution performs significantly better than the commercial tool in terms of leakage power. 
\end{itemize}

\begin{table}[!t]
    \caption{The Table comparing the performance of \textit{MLTimer} versus a commercial synthesis tool on SHAKTIC. We see that the solution quality is 57\% better than that of the tool.}
    \label{tab:tab9}

    \centering
    \begin{tabular}{|l|l|l|l|l|l|l|}
        \hline
        \textbf{Metric}           & \multicolumn{2}{c|}{Commercial Tool}                                                                                     &        &            & \multicolumn{2}{c|}{Percentage Improvement} \\ \hline
                         & \begin{tabular}[c]{@{}l@{}}$LV_t$\\  synthesis\end{tabular} & \begin{tabular}[c]{@{}l@{}}Mixed $V_t$ \\ synthesis\end{tabular} & \cite{hu:12} & \textit{MLTimer} & Tool              & \cite{hu:12}       \\ \hline
                    %         \textbf{Runtime (mins)}    & 38                                                       & 14                                                            & 87     & 64         & -78.12           &  26.44       \\ \hline
                             \textbf{Leakage power (W)} & 5                                                        & 1                                                             & 0.59  & 0.43       & 57        & 27        \\ \hline
    \end{tabular}

\end{table}


\subsection{Analysis of the Learning Module}


\begin{table}[!t]
\caption{Table showing the weights assigned to each feature at each stage of the $V_t$ and $size$ classifiers. An extremely low magnitude implies that the corresponding feature does not contribute significantly to the output and can thus be discarded. However it can be seen that none of the features chosen fall into that category.}
\label{tab:tab7}

    \centering

\begin{tabular}{|l|l|l|l|l|l|l|l|}
\hline
    \multirow{2}{*}{\textbf{Feature}}       & \multicolumn{2}{c|}{$\mathbf{V_t}$} & \multicolumn{5}{c|}{\textbf{size}}             \\ \cline{2-8} 
                               & 1          & 2          & 1     & 2     & 3     & 4     & 5     \\ \hline
    \textbf{Sub-circuit}                    & -0.88      & -0.43      & -0.58 & 1.32  & -0.08 & 0.44  & 0.16  \\ \hline
    \textbf{Gate type}                      & -0.19      & -0.43      & -0.96 & -0.81 & 0.42  & -0.40 & 0.29  \\ \hline
    \textbf{LNS}                            & 1.09       & 0.33       & -0.07 & -0.11 & 0.76  & 0.76  & 0.08  \\ \hline
    \textbf{Number of Fanins}               & 2.64       & 0.04       & 3.27  & -2.38 & -1.10 & -0.34 & -1.26 \\ \hline
    \textbf{Number of Fanouts}              & -2.92      & -0.29      & -4.08 & -0.53 & 1.82  & 0.50  & -1.07 \\ \hline
    \textbf{Number of Negative Slack Paths} & 0.33       & -0.16      & 0.50  & 0.40  & -0.22 & 0.21  & -0.68 \\ \hline
    \textbf{Slack}                          & 1.89       & 0.35       & 1.22  & -3.20 & -1.64 & -1.41 & 0.37  \\ \hline
\end{tabular}

\end{table}


The learning module forms a critical component of our framework as it serves to reduce the runtime by using a simple SVM model that uses seven features.  A complex ML model with large number of redundant features might cause runtime overheads due to i) complex training procedure ii) complicated inference procedure, and; iii) reduced interpretability of the ML model. Hence there is a need to eliminate the redundant features in order to simplify the learning module. Logistic regression was performed to estimate the importance of the chosen features. The Logistic Regression model was initially trained on the set of chosen features and the importance of each feature,  obtained via the coefficient assigned by the model,  is quantified in Table~\ref{tab:tab7}.  It can be observed that none of the feature weights have extremely low value and hence cannot be eliminated.


%As mentioned earlier, an improperly trained learning engine could initialize the netlist to a sub-optimal configuration leading to more delay and power recovery cycles than necessary thereby increasing the runtime overhead. The thresholding function plays an important role in predicting the final choice ($V_t$/$size$) for a given cell. We use the b19\_fast benchmark to show the impact of varying the thresholding function on the solution quality of the SVM engine. We show the impact of the thresholding function in table ~\ref{results4}. We see that as the thresholding function increases the runtime goes up. This is because the number of gates that are marked unsure increases causing more delay and power optimizations. We us class probability to determine the class label ($V_t$/$size$). We use a thresholding value of $0.75$ for both the $V_t$ classifiers while we use a thresholding value of x and y for the $size$ classifiers. 





% \begin{table*}[!t]
% \parbox{.3\linewidth}{
% \begin{center}

% \begin{tabular}{|p{3cm}|p{1.3cm}|}
% \hline
% Metric & Number \\ \hline
% Total gates & 333\\
% \hline
% Combinational gate types &  11 \\ \hline
% Sequential gates & 1 \\ \hline
% $V_t$ choices & 3 \\ \hline
% $size$ choices & 10 \\ \hline
% $V_{cc}$ and $gnd$ cells & 2 \\ \hline
% %leon3mp\_slow & 649K & -6401 &  -6479 & 1W & 47s \\
% %\hline
% \end{tabular}
% \caption{Library statistics}
% \label{tab:lib}
% \end{center}
% %\end{table*}
% }
% \hfill
% \parbox{.6\linewidth}{
% %\begin{table*}[!t]
% \begin{center}

% \begin{tabular}{|p{2cm}|p{1cm}|p{1cm}|p{1.6cm}|p{1.6cm}|p{1.6cm}|}
% \hline
% Benchmark & \#Input & \#Output & \#Comb cell & \#Seq cell & \#Total cell \\
% \hline
% DMA &  683 & 276 & 23109 & 2192 & 25301\\ \hline
% pci & 160 & 201& 29844& 3359& 33203\\ \hline
% des\_perf &  234 & 140 & 102427 & 8802& 111229\\ \hline
% vga\_lcd &  85 & 99 &  147812 & 17079 & 164891\\ \hline
% b19 &  22 & 25 &  212674 & 6594 & 219268\\ \hline
% leon3mp & 254 & 79 & 540352 &  108839 & 649191\\ \hline
% netcard &  1836 & 10 & 860949 & 97831 & 958780\\ \hline

% %leon3mp\_slow & 649K & -6401 &  -6479 & 1W & 47s \\
% %\hline
% \end{tabular}
% \caption{Benchmark statistics}
% \label{tab:benchmark}
% \end{center}
% }
% \end{table*}


% Please add the following required packages to your document preamble:
% \usepackage{booktabs}
% \usepackage{multirow}
% Please add the following required packages to your document preamble:
% \usepackage{multirow}
% Please add the following required packages to your document preamble:
% \usepackage{multirow}
% Please add the following required packages to your document preamble:
% \usepackage{multirow}

% \begin{table*}[!t]
% \parbox{.5\linewidth}{
% \begin{center}

% \label{tab:log}
% \begin{tabular}{|p{2.5cm}|p{3cm}|}
% \hline
% Feature & Weight \\
% \hline
% Gate footprint &-0.8832794232852276 \\ \hline
% Gateid & -0.1914242984939835 \\ \hline
% Local negative slack & 1.090781658200261 \\ \hline
% \#Fanins & 2.642338600894165  \\ \hline
% \#Fanouts & -2.92081747517804  \\ \hline
% \#Negative slack paths & 0.3349856667382776  \\ \hline
% Slack & 1.894864001133556 \\ \hline

% %leon3mp\_slow & 649K & -6401 &  -6479 & 1W & 47s \\
% %\hline
% \end{tabular}
% \caption{Feature Weights for the first $V_t$ classifier }
% \end{center}
% %\end{table*}
% }
% \hfill
% %\begin{table*}[!h]
% \parbox{.5\linewidth}{
% \begin{center}

% \label{tab:log2}
% \begin{tabular}{|p{2.5cm}|p{3cm}|}
% \hline
% Feature & Weight \\
% \hline
% Gate footprint & 0.03119793516364029
% \\ \hline
% Gateid &  0.346427255804566\\ \hline
% Local negative slack & -0.01019723367101055\\ \hline
% \#Fanins & -0.007490175140922838 \\ \hline
% \#Fanouts & -0.2180352793079041 \\ \hline
% \#Negative slack paths & -0.06062286295401311  \\ \hline
% Slack & 0.663389798807628 \\ \hline

% %leon3mp\_slow & 649K & -6401 &  -6479 & 1W & 47s \\
% %\hline
% \end{tabular}
% \caption{Feature Weights for the second stage $V_t$ classifier }
% \end{center}
% }
% \end{table*}

The efficiency of the SVM engine is analyzed in Table~\ref{tab:tab8}. We see that on an average the SVM engine is able to recover a significant amount of power in a short amount of time. However, It can be observed that the solution provided by the SVM engine is not optimal hence  the delay and leakage power recovery steps are used to further optimize the solution provided by the learning step.

 \begin{table*}[!t]
  \caption{Leakage and Running Time Comparisons for ISPD benchmarks and ShaktiC with just SVM. In the table, {\bf h}, {\bf m} and {\bf s} stand for hours, minutes and seconds respectively. It can be seen that with the exception of leon3mp our SVM implementation is able to recover significant delay and power.} \
\label{tab:tab8}

     \begin{center}
 \begin{tabular}{|p{4.2cm}|p{2cm}|p{2.2cm}|p{2cm}|p{2cm}|p{2cm}|}
 \hline
    \textbf{Benchmark} & \textbf{Gate count} & \textbf{Initial Worst Negative Slack (WNS)} & \multicolumn{3}{|c|}{ \textit{MLTimer}}  \\
 \hline
   &   & & WNS (ps) &  Leakage Power (W)      & Running Time          \\
 \hline
     \texttt{ DMA\_fast} & 25,300&  -1485 & -774 &0.09  & 3s \\
 \hline
     \texttt{pci\_bridge32\_fast}	& 33,200& -1881 & -2284	&	0.18   &	3s	 \\
 \hline
     \texttt{des\_perf\_fast} &  102,000 & -669 & -1029 &.316 & 1m        \\
 \hline
     \texttt{vga\_lcd\_fast} & 148,000 & -1254 & -2964 &.29 & 1m          \\
\hline

     \texttt{b19\_fast}	&	219,000 & -2835 & -1738	&	1.6 	&	15s	\\ \hline
     \texttt{leon3mp\_fast} & 649,000 & -6401 & -3913 & 21 &  47s \\
 \hline

     \texttt{netcard\_fast} & 959,000 & -4102 &-3268 & 8 & 1m  \\ \hline
     \texttt{ShaktiC} & 174,756 & -5199 & -1067 & 0.67 & 1m \\ 
 \hline
 \end{tabular}
 \end{center}

 \end{table*}
\section{Conclusion}
\label{sec:conclusion}
Leakage optimization  techniques have been studied extensively for more than a decade.  However, the lack of a robust algorithm that is optimal in terms of both execution time and solution quality motivates research in this area. It is seen that varying window size adaptively according to the status of the timing updates produces faster solutions than for a fixed window size. The proposed \textit{MLTimer} algorithm improves the running-time considerably while still retaining the solution quality of a greedy heuristic. It is observed that for large circuits \textit{MLTimer} with initial configuration provided by SVM performs significantly better than when used with power optimal configuration as initial solution.  Extending the concepts involved in the construction of \textit{MLTimer} to other steps of EDA including placement and routing is an interesting direction for future work.
% * <sristisravan@gmail.com> 2017-06-28T10:13:29.636Z:
% 
% Check "... the lack of a robust heuristic that optimal in terms of both.... "
% 
% ^.

%\begin{table*}[t]
% \begin{center}
% \caption{Leakage and Running Time Comparisons for ISPD benchmarks with just SVM and delay recovery. In the table, {\bf h}, {\bf m} and {\bf s} stand for hours, minutes and seconds respectively.}
% \label{results2}
% \begin{tabular}{|p{1.7cm}|p{2cm}|p{2cm}|p{2cm}|p{2cm}|}
% \hline
% Benchmark & $\#$ gates & \multicolumn{3}{|c|}{\textit{MLTimer}}  \\
% \hline
%   &  & Delay(ps) &  Leakage Power (W)      & Running Time          \\
% \hline
% %USB\_PHY &536 &$<$1s &$<$1s & - \\
% \hline
% DMA_fast & 25.3K & &0.08W & 17m \\
% \hline
% DMA_slow & 25.3 & &0.134W & 1m44s \\
% \hline
% pci_bridge_fast	& 33.2K&	&	0.1331W &	1m29s	 \\
% \hline
% pci_bridge_slow	& 	33.2K&	&	0.07W	&	8m	\\
% \hline 
% b19_fast	&	219K  &	&	.58W	&	9h	\\
% \hline
% b19_slow 	&	219K & &		0.486W	&	9h	 \\
% \hline
% Des_perf_fast &  165K & &.546W & 1331m        \\
% \hline
% Des_perf_slow &  165K & & .546W & 1331m         \\
% \hline
% vga_lcd_slow & 165K & & .546W & 1331m          \\
% \hline
% vga_lcd_slow & 165K &  &.546W & 1331m          \\
% \hline
% netcard_fast & 959k & & 1.8W & 2046m  \\
% \hline
% netcard_slow & 959k & & 1.8W & 2046m \\
% \hline
% leon3mp_fast & 649K & & 2W &  --- \\
% \hline
% leon3mp_slow & 649K & & 2W &  --- \\
% \hline
% \end{tabular}
% \end{center}
%\end{table*}







\section{Numerical Experiments}
\label{psbandit:expt}
In this section we present two experiments in two different environments.


\begin{figure}[!th]
    \centering
    \begin{tabular}{cc}
    %\setlength{\tabcolsep}{0.1pt}
    \subfigure[\Large\textwidth][\large Expt-$1$: $3$ Bernoulli-distributed arms (From Dr. Odalric's Draft).]
    %with $r_{i_{{i}\neq {*}}}=0.07$ and $r^{*}=0.1$
    {
    		\pgfplotsset{
		tick label style={font=\normalsize},
		label style={font=\normalsize},
		legend style={font=\normalsize},
		ylabel style={yshift=12pt},
		%legend style={legendshift=32pt},
		}
        \begin{tikzpicture}[scale=0.7]
      	\begin{axis}[
		xlabel={timestep},
		ylabel={Cumulative Regret},
		grid=major,
        %clip mode=individual,grid,grid style={gray!30},
        clip=true,
        %clip mode=individual,grid,grid style={gray!30},
  		legend style={at={(0.5,1.4)},anchor=north, legend columns=3} ]
      	% UCB
		
		\addplot table{Chapter6/results/NewExpt/Expt5/comp_subsampled_DUCB01.txt};
		\addplot table{Chapter6/results/NewExpt/Expt5/comp_subsampled_ETS01.txt};
		\addplot table{Chapter6/results/NewExpt/Expt5/comp_subsampled_ETS02.txt};
		\addplot table{Chapter6/results/NewExpt/Expt5/comp_subsampled_ETS03.txt};
		\addplot table{Chapter6/results/NewExpt/Expt5/comp_subsampled_ETS1E01.txt};
		\addplot table{Chapter6/results/NewExpt/Expt5/comp_subsampled_ETS1E02.txt};
		\addplot table{Chapter6/results/NewExpt/Expt5/comp_subsampled_TS01.txt};
		\addplot table{Chapter6/results/NewExpt/Expt5/comp_subsampled_OTS01.txt};
		\addplot table{Chapter6/results/NewExpt/Expt5/comp_subsampled_DTS01.txt};
      	
      	\legend{DUCB($\gamma=1-\frac{1}{4\sqrt{T}}$),ETSDAE1,ETSDAE2,ETSDAE3,ETSD1E1,ETSD1E2,TS,OTS,DTS($\gamma=1-\frac{1}{4\sqrt{T}})$}    
      	\end{axis}
      	\end{tikzpicture}
  		\label{psbandit:fig:1}
    }
    &
    \subfigure[\Large\textwidth][\large Expt-$1$: $3$ Bernoulli-distributed arms (From Dr. Odalric's Draft).]
    %with $r_{i_{{i}\neq {*}}}=0.07$ and $r^{*}=0.1$
    {
    		\pgfplotsset{
		tick label style={font=\normalsize},
		label style={font=\normalsize},
		legend style={font=\normalsize},
		ylabel style={yshift=12pt},
		%legend style={legendshift=32pt},
		}
        \begin{tikzpicture}[scale=0.7]
      	\begin{axis}[
		xlabel={timestep},
		ylabel={Cumulative Regret},
		grid=major,
        %clip mode=individual,grid,grid style={gray!30},
        clip=true,
        %clip mode=individual,grid,grid style={gray!30},
  		legend style={at={(0.5,1.4)},anchor=north, legend columns=3} ]
      	% UCB
		
		\addplot table{Chapter6/results/NewExpt/Expt5/comp_subsampled_DUCB01.txt};
		\addplot table{Chapter6/results/NewExpt/Expt5/comp_subsampled_ETS04.txt};
		\addplot table{Chapter6/results/NewExpt/Expt5/comp_subsampled_ETS1E03.txt};
		\addplot table{Chapter6/results/NewExpt/Expt5/comp_subsampled_ETS06.txt};
		\addplot table{Chapter6/results/NewExpt/Expt5/comp_subsampled_ETS1E04.txt};
		\addplot table{Chapter6/results/NewExpt/Expt5/comp_subsampled_TS01.txt};
		\addplot table{Chapter6/results/NewExpt/Expt5/comp_subsampled_OTS01.txt};
		\addplot table{Chapter6/results/NewExpt/Expt5/comp_subsampled_DTS01.txt};
		\addplot table{Chapter6/results/NewExpt/Expt5/comp_subsampled_ETS1E06.txt};
      	
      	\legend{DUCB($\gamma=1-\frac{1}{4\sqrt{T}}$),EAggrCPD1,CPD1-E1,EAggrCPD2,CPD2-E1,TS,OTS,DTS($\gamma=1-\frac{1}{4\sqrt{T}})$,CPD2-E1-N}    
      	\end{axis}
      	\end{tikzpicture}
  		\label{psbandit:fig:2}
    }
    \end{tabular}
    \caption{Cumulative regret for various bandit algorithms on a piecewise stochastic 3-armed bandit environment. }
    \label{fig:karmed1}
\end{figure}


%\begin{figure}[!th]
%    \centering
%    \begin{tabular}{c}
%    %\setlength{\tabcolsep}{0.1pt}
%    \subfigure[\Large\textwidth][\large Expt-$1$: $3$ Bernoulli-distributed arms (From Dr. Odalric's Draft).]
%    %with $r_{i_{{i}\neq {*}}}=0.07$ and $r^{*}=0.1$
%    {
%    		\pgfplotsset{
%		tick label style={font=\normalsize},
%		label style={font=\normalsize},
%		legend style={font=\normalsize},
%		ylabel style={yshift=12pt},
%		%legend style={legendshift=32pt},
%		}
%        \begin{tikzpicture}[scale=0.7]
%      	\begin{axis}[
%		xlabel={timestep},
%		ylabel={Cumulative Regret},
%		grid=major,
%        %clip mode=individual,grid,grid style={gray!30},
%        clip=true,
%        %clip mode=individual,grid,grid style={gray!30},
%  		legend style={at={(0.5,1.4)},anchor=north, legend columns=3} ]
%      	% UCB
%		
%		\addplot table{results/NewExpt/Expt5/comp_subsampled_DUCB01.txt};
%		\addplot table{results/NewExpt/Expt5/comp_subsampled_ETS04.txt};
%		\addplot table{results/NewExpt/Expt5/comp_subsampled_ETS1E03.txt};
%		\addplot table{results/NewExpt/Expt5/comp_subsampled_TS01.txt};
%		\addplot table{results/NewExpt/Expt5/comp_subsampled_OTS01.txt};
%		\addplot table{results/NewExpt/Expt5/comp_subsampled_DTS01.txt};
%      	
%      	\legend{DUCB($\gamma=1-\frac{1}{4\sqrt{T}}$),ETSDAE4,ETSD1E3,TS,OTS,DTS($\gamma=1-\frac{1}{4\sqrt{T}})$}    
%      	\end{axis}
%      	\end{tikzpicture}
%  		\label{fig:2}
%    }
%    \end{tabular}
%    \caption{Cumulative regret for various bandit algorithms on a piecewise stochastic 3-armed bandit environment. }
%    \label{fig:karmed2}
%\end{figure}




\section{Summary and Future Works}
\label{psbandit:conclusion}
In this chapter, we looked at the stochastic multi-armed bandit (SMAB) setting and discussed how it is important in the general reinforcement learning setup. We also looked at the various state-of-the-art algorithms in the literature for the SMAB setting and discussed the advantages and disadvantages of them. The regret bounds that have been proven for the said algorithms have also been discussed at length and their confidence intervals have also been compared against each other. In the next chapter, we provide our solution to this SMAB setting which achieves an almost order-optimal regret bound.


%%%%%%%%%%%%%%%%%%%%%%%%%%%%%%%%%%%%%%%%%%%%%%%%%%%%%%%%%%%%


%%%%%%%%%%%%%%%%%%%%%%%%%%%%%%%%%%%%%%%%%%%%%%%%%%%%%%%%%%%%
% Appendices.

\appendix

%\chapter{APPENDIX}

\chapter{Appendix on Concentration Inequalities}
\label{sec:app:Conc}
\subsection{Sample space}



\subsection{Events}



\subsection{Sigma-algebra}



\subsubsection{Borel-sigma algebra}



\subsection{Measure}



\subsubsection{The probability measure}



\subsection{The triplet}



\subsection{Filtration}



\section{Martingale}



\subsection{Super-martingale}



\subsection{Sub-martingale}


\section{Convergence theorems}


\subsection{Monotone convergence theorem}



\subsection{Dominated convergence theorem}



\subsection{Fatou's Lemma}



\section{Sub-Gaussian distribution}

Let a random variable $X\in \R$ with variance as $\sigma^2$. Then $X$ is said to be $\sigma$-sub-gaussian for $\sigma\geq 0$ such that $\E[X] = 0$ and its moment generating function satisfies for all $\lambda \in \R $ the following condition,

\begin{align*}
\E[\exp{\lambda X}]\leq \exp\left( - \dfrac{\lambda^2\sigma^2}{2}\right)
\end{align*} 

Also, note that sub-gaussian distribution is a class of distribution rather than a distribution itself.

\begin{remark}
A random variable $X\in[0,1]$ is said to be $\dfrac{1}{2}-sub-gaussian$ with its moment generating function satisfying the condition,

\begin{align*}
\E[\exp{\lambda X}]\leq \exp\left( - \dfrac{\lambda^2}{8}\right), \forall \lambda \in \R
\end{align*}
  
\end{remark}

\section{Concentration Inequalities}

In this section we state some of the concentration inequalities used in the proofs in several chapters of the thesis. Concentration inequality deals with the control of the tail of the average of independent random variables from their expected mean. 


	Let, $X_1,X_2,\ldots,X_n$ be a sequence of independent random variables defined on a probability space $(\omega,\mathcal{F},\Pb)$, is bounded in $[a_i,b_i],\forall i=1,2,\ldots, n$. Let $S_n$ denote the sum of the random variables such that $S_n = X_1 + X_2 + \ldots + X_n$,  $\hat{r} = \dfrac{S_n}{n}$ and $E[S_n]=r$. Let $\mathcal{F}_n$ be an increasing sequence of $\sigma$-fields of $\mathcal{F}$ such that for each $n$, $\sigma(X_{1},\ldots,X_n)\subset \mathcal{F}_t$ and for $q>t$, $X_q$ is independent of $\mathcal{F}_n$.

\subsection{Markov's inequality}

Markov's inequality states that, for any $\epsilon > 0$, 

\begin{align*}
\Pb[S_n > \epsilon] \leq \dfrac{\E[S_n]}{\epsilon}
\end{align*}


\subsection{Chernoff-Hoeffding Bound}

Chernoff-Hoeffding gives us the following inequality regarding the sums of independent random variables $S_n$ and their deviation from their expectation $\E[S_n]=r$, for any $\epsilon > 0$,

\begin{align*}
&\Pb\lbrace S_n - n\E[S_n] \geq \epsilon \rbrace \leq \exp\left( -\dfrac{2\epsilon^2}{n \sum_{i=1}^n(a_i -b_i)}\right) \\
%%%%%%%%%%%%%%%%%%%%%%
&=\Pb\lbrace S_n - n\E[S_n] \leq - \epsilon \rbrace \leq \exp\left( -\dfrac{2\epsilon^2}{n \sum_{i=1}^n(a_i -b_i)}\right)
\end{align*}



Considering all the random variables bounded in $[0,1]$, the above two inequalities can be reduced to,

\begin{align*}
&\Pb\lbrace \left|\dfrac{S_n}{n} - \E[S_n]\right| \geq \epsilon \rbrace \leq 2\exp\left( - 2\epsilon^2 n \right) \\
%%%%%%%%%%%%%%%%%%%%%%
&=\Pb\lbrace \left|\hat{r} - r\right| \geq \epsilon \rbrace \leq 2\exp\left( - 2\epsilon^2 n \right)
\end{align*}


\subsection{Empirical Bernstein inequality}

Similar to Chernoff-Hoeffding bound, empirical Bernstein inequality gives us the following inequality regarding the sums of independent random variables $S_n$ and their deviation from their expectation $\E[S_n]=r$, for any $\epsilon > 0$,

\begin{align*}
&\Pb\lbrace S_n - n\E[S_n] \geq \epsilon \rbrace \leq \exp\left( -\dfrac{2\epsilon^2}{\left(2\sigma^2 + \frac{2 b_{\max} \epsilon}{3}\right) n \sum_{i=1}^n(a_i -b_i)}\right), \\
%%%%%%%%%%%%%%%%%%%%%%
&\Pb\lbrace S_n - n\E[S_n] \leq - \epsilon \rbrace \leq \exp\left( -\dfrac{2\epsilon^2}{\left(2\sigma^2 + \frac{2 b_{\max} \epsilon}{3}\right) n \sum_{i=1}^n(a_i -b_i)}\right)
\end{align*}



Considering all the random variables bounded in $[0,1]$, the above two inequalities can be reduced to,

\begin{align*}
&\Pb\lbrace \left|\dfrac{S_n}{n} - \E[S_n]\right| \geq \epsilon \rbrace \leq 2\exp\left( -\dfrac{2\epsilon^2 n}{\left(2\sigma^2 + \frac{2\epsilon}{3}\right)}\right) \\
%%%%%%%%%%%%%%%%%%%%%%
&=\Pb\lbrace \left|\hat{r} - r\right| \geq \epsilon \rbrace \leq 2\exp\left( -\dfrac{2\epsilon^2 n}{\left(2\sigma^2 + \frac{2\epsilon}{3}\right)}\right)
\end{align*}




%\newpage

\chapter{Appendix for EUCBV}
\label{sec:app:EUCBV}

\subsection{Proof of Lemma \ref{proofTheorem:Lemma:1}} 

\label{App:Lemma:1}

\begin{customlem}{1}
%\label{proofTheorem:Lemma:1}
If $T\geq K^{2.4}$, $\psi=\dfrac{T}{ K^2}$, $\rho=\dfrac{1}{2}$ and $m\leq \dfrac{1}{2} \log_2\left(\dfrac{T}{e}\right) $, then,
\begin{align*}
\dfrac{\rho m \log(2)}{\log(\psi T) - 2m\log( 2)} \leq \frac{3}{2}.
\end{align*}
\end{customlem}

\begin{proof}
The proof is based on contradiction. Suppose
\begin{eqnarray*}
\dfrac{\rho m \log(2)}{\log(\psi T) - 2m\log( 2)} > \frac{3}{2}.
\end{eqnarray*}
Then, with $\psi=\dfrac{T}{ K^2}$ and $\rho=\dfrac{1}{2}$, we obtain
\begin{eqnarray*}
6\log(K) 
&>& 6\log(T) - 7m\log(2) \\
&\overset{(a)}{\ge}& 6\log(T) - \frac{7}{2} \log_2\left(\frac{T}{e}\right) \log(2) \\
&=& 2.5\log(T) + 3.5 \log_2(e)\log(2)  \\
&\overset{(b)}{=}& 2.5\log(T) +3.5
\end{eqnarray*}
where $(a)$ is obtained using $m\leq \dfrac{1}{2} \log_2\left(\dfrac{T}{e}\right)$, while $(b)$ follows from the identity $\log_2(e)\log(2) =1$. Finally, for $T\ge K^{2.4}$ we obtain, $6\log(K)>6\log(K)+3.5$, which is a contradiction.
\hfill $\blacksquare$	
\end{proof}

\subsection{Proof of Lemma \ref{proofTheorem:Lemma:2}}
\label{App:Lemma:2}
\begin{customlem}{2}
%\label{proofTheorem:Lemma:2}
If $T\geq K^{2.4}$, $\psi=\dfrac{T}{ K^2}$, $\rho =\dfrac{1}{2}$, $m_i = min\lbrace m|\sqrt{4\epsilon_{m} } < \dfrac{\Delta_i}{4} \rbrace $ and $c_{i} =\sqrt{\frac{\rho (\hat{v}_i + 2)\log (\psi T\epsilon_{m_{i}})}{4 z_i}}$, then, 
\begin{align*}
c_{i} < \dfrac{\Delta_i}{4}
\end{align*}

\end{customlem}

\begin{proof}

	In the $m_i$-th round since $z_i\geq n_{m_i}$, by substituting $z_i$ with $n_{m_i}$ we can show that, 

\begin{align*}
	c_{i} &\leq \sqrt{\dfrac{\rho (\hat{v}_i + 2)\epsilon_{m_{i}}\log (\psi T\epsilon_{m_{i}})}{2\log(\psi T\epsilon_{m_{i}}^{2})}} \overset{(a)}{\leq} \sqrt{\dfrac{2\rho\epsilon_{m_{i}}\log (\frac{\psi T\epsilon_{m_{i}}^{2}}{\epsilon_{m_{i}}})}{\log(\psi T\epsilon_{m_{i}}^{2})}} \\
	%%%%%%%%%%%%%%%%%%%%%%%%%%
	& = \sqrt{\dfrac{2\rho\epsilon_{m_{i}}\log (\psi T\epsilon_{m_{i}}^{2}) - 2\rho\epsilon_{m_{i}}\log (\epsilon_{m_{i}})}{\log(\psi T\epsilon_{m_{i}}^{2})}} \\
	%%%%%%%%%%%%%%%%%%%%%%%%%%
	& \leq  \sqrt{2\rho\epsilon_{m_{i}} - \dfrac{2\rho\epsilon_{m_i}\log(\frac{1}{2^{m_i}})}{\log(\psi T \frac{1}{2^{2m_i}})}} \\
	%%%%%%%%%%%%%%%%%%%%%%%%%%
	&\leq \sqrt{2\rho\epsilon_{m_{i}} + \dfrac{2\rho\epsilon_{m_i}\log(2^{m_i})}{\log(\psi T) - \log( 2^{2m_i})}}\\
	%%%%%%%%%%%%%%%%%%%%%%%%%%
	& \leq \sqrt{2\rho\epsilon_{m_{i}} + \dfrac{2\rho\epsilon_{m_i}m_i \log(2)}{\log(\psi T) - 2m_i\log( 2)}} \\ 
	%%%%%%%%%%%%%%%%%%%%%%%%%%
	 & \overset{(b)}{\leq} \sqrt{2\rho\epsilon_{m_{i}} + 2.\frac{3}{2}\epsilon_{m_i}} 
	  < \sqrt{4\epsilon_{m_i}} < \dfrac{\Delta_{i}}{4}.
	\end{align*}
In the above simplification, $(a)$ is due to $\hat{v}_i \in [0,1]$, while $(b)$ is obtained using Lemma~\ref{proofTheorem:Lemma:1}.
% < \dfrac{\Delta_{i}}{4 \sigma_i^2} \overset{(c)}{<}
% and $(c)$ happens because $\sigma_i \in (0,1]$. 
%Similarly, it can be shown that $c^* < \frac{\Delta_i}{4}$ in round $m_i$.
\hfill $\blacksquare$	
\end{proof}


\subsection{Proof of Lemma \ref{proofTheorem:Lemma:3}}
\label{App:Lemma:3}

\begin{customlem}{3}
If $m_i = min\lbrace m|\sqrt{4\epsilon_{m} } < \frac{\Delta_i}{4} \rbrace $,  $c_{i} = \sqrt{\frac{\rho (\hat{v}_i + 2) \log (\psi T\epsilon_{m_{i}})}{4 z_{i}}}$ and $n_{m_i} = \frac{\log{(\psi T\epsilon_{m_{i}})}}{2\epsilon_{m_{i}}}$ then we can show that,
\begin{align*}
\mathbb{P}(\hat{r}_{i}> r_{i} + c_{i})\le \dfrac{2}{(\psi  T\epsilon_{m_{i}})^{\frac{3\rho}{2}}}.
\end{align*}
\end{customlem}

%\begin{customlem}{3}
%If $m_i = min\lbrace m|\sqrt{4\epsilon_{m} } < \dfrac{\Delta_i}{4} \rbrace $,  $\bar{c}_i=\sqrt{\dfrac{\rho (\sigma_{i}^{2}+\sqrt{\epsilon_{m_{i}}} + 2)\log(\psi T\epsilon_{m_{i}})}{4z_i}}$ and $n_{m_i} = \frac{\log{(\psi T\epsilon_{m_{i}})}}{2\epsilon_{m_{i}}}$ then we can show that,
%\begin{align*}
%\mathbb{P}\left( \hat{r}_{i} > r_{i}+ \bar{c}_i\right) 
%+ \mathbb{P}\left( \hat{v}_{i}\geq \sigma_{i}^{2}+\sqrt{\epsilon_{m_{i}}}\right) \leq \dfrac{2}{(\psi  T\epsilon_{m_{i}})^{\frac{3\rho}{2}}}.
%\end{align*}
%\end{customlem}

\begin{proof}

We start by recalling from equation (\ref{eq:prob_eq2}) that,

\begin{align}
\mathbb{P}(\hat{r}_{i}> r_{i} + c_{i})
&\leq \mathbb{P}\left( \hat{r}_{i} > r_{i}+ \bar{c}_i\right) 
+ \mathbb{P}\left( \hat{v}_{i}\geq \sigma_{i}^{2}+\sqrt{\epsilon_{m_{i}}}\right)\label{eq:prob_eq3}
\end{align}
where 
\begin{align*}
&c_i =\sqrt{\frac{\rho (\hat{v}_i + 2)\log (\psi T\epsilon_{m_{i}})}{4 z_i}} \text{ and } \\
&\bar{c}_i=\sqrt{\dfrac{\rho (\sigma_{i}^{2}+\sqrt{\epsilon_{m_{i}}} + 2)\log(\psi T\epsilon_{m_{i}})}{4 z_i}}.
\end{align*}

Note that, substituting $ z_i \geq n_{m_i} \geq \frac{\log{(\psi T\epsilon_{m_{i}})}}{2\epsilon_{m_{i}}}$, $\bar{c}_i$ can be simplified to obtain,
\begin{align}
\bar{c}_i
\leq \sqrt{\dfrac{\rho\epsilon_{m_{i}}(\sigma_{i}^{2}+\sqrt{\epsilon_{m_{i}}} + 2)}{2}}\leq \sqrt{ \epsilon_{m_{i}}}.
\label{si_bar_equn}
\end{align}
%
The first term in the LHS of (\ref{eq:prob_eq3}) can be bounded using the Bernstein inequality as below:
\begin{align}
&\mathbb{P}\left( \hat{r}_{i} > r_{i}+ \bar{c}_i\right)\nonumber 
\le \exp\left(- \dfrac{(\bar{c}_i)^2 z_{i}}{2\sigma_i^2 + \frac{2}{3}\bar{c}_i} \right)\nonumber 
%%%%%%%%%%%%%%%
\\
& \overset{(a)}{\le} \exp\left(- \rho \left(\dfrac{3\sigma_{i}^{2}+3\sqrt{\epsilon_{m_{i}}} + 6}{6\sigma_i^2 + 2\sqrt{\epsilon_{m_i}}} \right)\log(\psi  T\epsilon_{m_{i}}\right)\nonumber \\
%%%%%%%%%%%%%%%
% &\le \exp\left(- \rho (\sigma_{i}^{2}+\sqrt{\epsilon_{m_{i}}} + 2)\log(\psi  T\epsilon_{m_{i}})\right)\nonumber \\
%%%%%%%%%%%%%%%
& \overset{(b)}{\leq} \exp\left(- \rho \log(\psi  T\epsilon_{m_{i}})\right) 
%%%%%%%%%%%%%%%
\le \dfrac{1}{(\psi  T\epsilon_{m_{i}})^{\frac{3\rho}{2}}}
\label{lhs1_equn}
\end{align}
where, $(a)$ is obtained by substituting equation \ref{si_bar_equn} and $(b)$ occurs because for all $\sigma_{i}^2 \in [0,\frac{1}{4}]$, $\left(\frac{3\sigma_{i}^{2}+3\sqrt{\epsilon_{m_{i}}} + 6}{6\sigma_i^2 + 2\sqrt{\epsilon_{m_i}}}\right) \geq \frac{3}{2}$ .

 
The second term in the LHS of (\ref{eq:prob_eq3}) can be simplified as follows:
\begin{align}
&\mathbb{P}\bigg\lbrace \hat{v}_{i}\geq \sigma_{i}^{2}+\sqrt{\epsilon_{m_{i}}}\bigg\rbrace\nonumber\\
%%%%%%%%%%%%%%%%%%
&\leq \mathbb{P}\bigg\lbrace \dfrac{1}{n_{i}}\sum_{t=1}^{n_{i}}(X_{i,t}-r_{i})^{2}-(\hat{r}_{i}-r_{i})^{2}\geq \sigma_{i}^{2}+\sqrt{\epsilon_{m_{i}}}\bigg\rbrace\nonumber\\
%%%%%%%%%%%%%%%%%%
&\leq \mathbb{P}\bigg\lbrace \dfrac{\sum_{t=1}^{n_{i}}(X_{i,t}-r_{i})^{2}}{n_{i}}\geq \sigma_{i}^{2}+\sqrt{\epsilon_{m_{i}}} \bigg\rbrace\nonumber\\
%%%%%%%%%%%%%%%%%%
&\overset{(a)}{\leq} \mathbb{P}\bigg\lbrace \dfrac{\sum_{t=1}^{n_{i}}(X_{i,t}-r_{i})^{2}}{n_{i}}\geq \sigma_{i}^{2} + \bar{c}_i\bigg\rbrace \nonumber\\
%%%%%%%%%%%%%%%%%%
&\overset{(b)}{\leq} \exp\left(- \rho \left(\dfrac{3\sigma_{i}^{2}+3\sqrt{\epsilon_{m_{i}}} + 6}{6\sigma_i^2 + 2\sqrt{\epsilon_{m_i}}} \right)\log(\psi  T\epsilon_{m_{i}})\right)
%%%%%%%%%%%%%%%%%
\le \dfrac{1}{(\psi  T\epsilon_{m_{i}})^{\frac{3\rho}{2}}}
\label{lhs2_equn}
\end{align}
where inequality $(a)$ is obtained using (\ref{si_bar_equn}), while $(b)$ follows from the Bernstein inequality. 

Thus, using (\ref{lhs1_equn}) and (\ref{lhs2_equn}) in (\ref{eq:prob_eq3}) we obtain $\mathbb{P}(\hat{r}_{i}> r_{i} + c_{i})\le \dfrac{2}{(\psi  T\epsilon_{m_{i}})^{\frac{3\rho}{2}}}$.
\hfill $\blacksquare$	
\end{proof}


\subsection{Proof of Lemma \ref{proofTheorem:Lemma:4}}
\label{App:Lemma:4}
\begin{customlem}{4}
%\label{proofTheorem:Lemma:4}
If $m_i = min\lbrace m|\sqrt{4\epsilon_{m} } < \dfrac{\Delta_i}{4} \rbrace $, $\psi=\frac{T}{ K^2}$, $\rho=\frac{1}{2}$,  $c_{i} =\sqrt{\dfrac{\rho(\hat{v}_i + 2)\log (\psi T\epsilon_{m_{i}})}{4 z_{i}}}$ and $n_{m_i}=\dfrac{\log{(\psi T\epsilon_{m_{i}}^{2})}}{2\epsilon_{m_{i}}}$ then in the $m_i$-th round, 
%\begin{align*}
%\Pb\lbrace c^{*} > c_i \rbrace \leq \dfrac{32K\log T}{(\psi T)^{3\rho}}\sum_{m=0}^{m_i}\dfrac{1}{\epsilon_{m_i}^{3\rho + 1}}. 
%\end{align*}
\begin{align*}
\Pb\lbrace c^{*} > c_i \rbrace  \leq \dfrac{182 K^4}{T^{\frac{5}{4}}\sqrt{\epsilon_{m_i}}}.
\end{align*}
\end{customlem}

\begin{proof}
From the definition of $c_i$ we know that $c_i\propto \frac{1}{z_i}$ as $\psi$ and $T$ are constants. Therefore in the $m_i$-th round,
\begin{align*}
&\Pb\lbrace c^{*} > c_i \rbrace
%%%%%%%%%%%%%%%%%%%%%%%%%%%%%%%%%%%%%
\leq  \Pb\lbrace  z^* < z_i  \rbrace \\
%%%%%%%%%%%%%%%%%%%%%%%%%%%%%%%%%%%%%
&\leq \sum_{m=0}^{m_i}\sum_{z^* =1}^{n_{m}}\sum_{z_i =1}^{n_{m}}\bigg(\Pb\lbrace \hat{r}^* < r^* - c^{*}\rbrace + \Pb\lbrace \hat{r}_i > r_i + c_i\rbrace\bigg)
\end{align*}

%Again, we can show that
%\begin{align*}
%&\sum_{m=0}^{m_i}\sum_{z^* =1}^{n_{m_i}}\sum_{z_i =1}^{n_{m_i}}\mathbb{I}_{i}\cup\mathbb{I}_{*} \leq \sum_{m=0}^{m_i}|B_{m_i}|n_{m_i} \\
%%%%%%%%%%%%%%%%%%%%%%%%%%%%%%%%%%%%%
%&\leq \sum_{m=0}^{m_i}\dfrac{4K}{(\psi T \epsilon_{m_i})^{\frac{3\rho}{2}}}\dfrac{\log{(\psi T\epsilon_{m_{i}}^{2})}}{2\epsilon_{m_{i}}} 
%%\leq 
%%%%%%%%%%%%%%%%%%%%%%%%%%%%%%%%%%%%%%
%\overset{(a)}{\leq} \sum_{m=0}^{m_i}\dfrac{4K}{(\psi T \epsilon_{m_i})^{\frac{3\rho}{2}}}\dfrac{2\log(\frac{T}{K})}{2\epsilon_{m_i}}
%\end{align*}
%
%Here, in $(a)$ happens by substituting the value of $\psi$ and considering $\epsilon_{m_i}\in [\sqrt{\frac{e}{T}},1]$. 

Now, applying Bernstein inequality and following the same way as in Lemma \ref{proofTheorem:Lemma:3} we can show that,
\begin{align*}
&\Pb\lbrace \hat{r}^* < r^* - c^{*}\rbrace \leq \exp(- \frac{(c^{*})^2}{2\sigma_*^2 + \frac{2 c^{*}}{3}} z^*)\leq \frac{4}{(\psi T\epsilon_{m_i})^{\frac{3\rho}{2}}} \\ 
%%%%%%%%%%%%%%%%%%%%%%%%%%%%%%%%%%%
&\Pb\lbrace \hat{r}_i > r_i + c_i\rbrace \leq \exp(- \frac{(c_{i})^2}{2\sigma_i^2 + \frac{2 c_{i}}{3}} z_i)\leq \frac{4}{(\psi T\epsilon_{m_i})^{\frac{3\rho}{2}}}
\end{align*}

Hence, summing everything up, 
\begin{align*}
&\Pb\lbrace c^{*} > c_i \rbrace \\
%%%%%%%%%%%%%%%%%%%%%%%%%%%%%%%%
&\leq \sum_{m=0}^{m_i}\sum_{z^* =1}^{n_{m}}\sum_{z_i =1}^{n_{m}}\bigg(\Pb\lbrace \hat{r}^* < r^* - c^{*}\rbrace + \Pb\lbrace \hat{r}_i > r_i + c_i\rbrace\bigg)\\
%%%%%%%%%%%%%%%%%%%%%%%%%%%%%%%%
&\overset{(a)}{\leq} \sum_{m=0}^{m_i}|B_{m}|n_{m}\bigg(\Pb\lbrace \hat{r}^* < r^* - c^{*}\rbrace + \Pb\lbrace \hat{r}_i > r_i + c_i\rbrace\bigg)\\
%%%%%%%%%%%%%%%%%%%%%%%%%%%%%%%%
&\overset{(b)}{\leq} \sum_{m=0}^{m_i}\dfrac{4K}{(\psi T \epsilon_{m_i})^{\frac{3\rho}{2}}}\dfrac{\log{(\psi T\epsilon_{m}^{2})}}{2\epsilon_{m}}\times 
\\
&\bigg(\Pb\lbrace \hat{r}^* < r^* - c^{*}\rbrace + \Pb\lbrace \hat{r}_i > r_i + c_i\rbrace\bigg)\\
%%%%%%%%%%%%%%%%%%%%%%%%%%%%%%%%
&\overset{(c)}{\leq} \sum_{m=0}^{m_i}\dfrac{4K}{(\psi T \epsilon_{m})^{\frac{3\rho}{2}}}\dfrac{\log(T)}{\epsilon_{m}}\bigg[\frac{4}{(\psi T\epsilon_{m})^{\frac{3\rho}{2}}} + \frac{4}{(\psi T\epsilon_{m})^{\frac{3\rho}{2}}}  \bigg]\\
%%%%%%%%%%%%%%%%%%%%%%%%%%%%%%%%
&\leq \sum_{m=0}^{m_i}\dfrac{32K\log T}{(\psi T\epsilon_{m})^{3\rho}\epsilon_{m}} \leq 
\dfrac{32K\log T}{(\psi T)^{3\rho}}\sum_{m=0}^{m_i}\dfrac{1}{\epsilon_{m}^{3\rho + 1}} \\
%%%%%%%%%%%%%%%%%%%%%%%%%%%%%%%%
&\overset{(d)}{\leq} \sum_{m=0}^{m_i}\dfrac{32K\log T}{(\psi T)^{3\rho}}\left(\sum_{m=0}^{m_i}\dfrac{1}{\epsilon_{m}}\right)^{3\rho + 1}\\
%%%%%%%%%%%%%%%%%%%%%%%%%%%%%%%%
&\overset{(e)}{\leq} 
\dfrac{32 K\log T}{(\frac{T^2}{K^2})^{\frac{3}{2}}}\bigg[\left( 1 + \dfrac{2(2^{ \frac{1}{2}\log_{2} \frac{T}{e}}-1)}{2-1} \right)^{\frac{5}{2}}\bigg] \\
%%%%%%%%%%%%%%%%%%%%%%%%%%%%%%%%
&\leq \dfrac{182 K^4 T^{\frac{5}{4}}\log T}{T^3} \overset{(f)}{\leq} \dfrac{182 K^4}{T^{\frac{5}{4}}} \overset{(g)}{\leq} \dfrac{182 K^4}{T^{\frac{5}{4}}\sqrt{\epsilon_{m_i}}}
\end{align*}

where, $(a)$ comes from the total pulls allocated for all $i\in B_m$ till the $m$-th round, in $(b)$ the arm count $|B_m|$ can be bounded by using equation $(\ref{eq:arm:elim:c1})$ and then we substitute the value of $n_{m}$, $(c)$ happens by substituting the value of $\psi$ and considering $\epsilon_{m}\in [\sqrt{\frac{e}{T}},1]$, $(d)$ follows as $\frac{1}{\epsilon_{m}}\geq 1,\forall m $, in $(e)$ we use the standard geometric progression formula and then we substitute the values of $\rho$ and $\psi$, $(f)$ follows from the inequality $\log T \leq \sqrt{T}$ and $(g)$ is valid for any $\epsilon_{m_i}\in[\sqrt{\frac{e}{T}},1]$. 

\hfill $\blacksquare$	
\end{proof}


\subsection{Proof of Lemma \ref{proofTheorem:Lemma:5}}
\label{App:Lemma:5}
\begin{customlem}{5}
%\label{proofTheorem:Lemma:5}
If $m_i = min\lbrace m|\sqrt{4\epsilon_{m} } < \dfrac{\Delta_i}{4} \rbrace $, $\psi=\frac{T}{ K^2}$, $\rho=\frac{1}{2}$, $c_{i} =\sqrt{\frac{\rho (\hat{v}_i + 2)\log (\psi T\epsilon_{m_{i}})}{4 z_i}}$ and $n_{m_i}=\dfrac{\log{(\psi T\epsilon_{m_{i}}^{2})}}{2\epsilon_{m_{i}}}$ then in the $m_i$-th round, 
%\begin{align*}
%\Pb\lbrace z_i < n_{m_i} \rbrace \leq \dfrac{32K\log T}{(\psi T)^{3\rho}}\sum_{m=0}^{m_i}\dfrac{1}{\epsilon_{m_i}^{3\rho + 1}}.
%\end{align*}
\begin{align*}
\Pb\lbrace z_i < n_{m_i} \rbrace \leq \dfrac{182 K^4}{T^{\frac{5}{4}}\sqrt{\epsilon_{m_i}}}.
\end{align*}
\end{customlem}

\begin{proof}
Following a similar argument as in Lemma \ref{proofTheorem:Lemma:4}, we can show that in the $m_i$-th round,
\begin{align*}
&\Pb\lbrace z_i < n_{m_i} \rbrace \\
%%%%%%%%%%%%%%%%%%%%%%%%%%%%%%%%%%%%
&\leq \sum_{m=0}^{m_i}\sum_{z_i =1}^{n_{m}}\sum_{z^* =1}^{n_{m}}\bigg(\Pb\lbrace \hat{r}^* > r^* - c^{*}\rbrace + \Pb\lbrace \hat{r}_i < r_i + c_i\rbrace\bigg) \\
%%%%%%%%%%%%%%%%%%%%%%%%%%%%%%%%%%%%
&\leq \dfrac{32K\log T}{(\psi T)^{3\rho}}\sum_{m=0}^{m_i}\dfrac{1}{\epsilon_{m}^{3\rho + 1}}\leq \dfrac{182 K^4}{T^{\frac{5}{4}}\sqrt{\epsilon_{m_i}}}.
\end{align*}
\hfill $\blacksquare$	
\end{proof}

%\subsection{Proof of Lemma 6}
%\label{App:Lemma:6}
%\begin{lemma}
%%\label{proofTheorem:Lemma:5}
%For $T\geq K^{2.4}$, $\epsilon_{m_i}\geq \sqrt{\frac{e}{T}}$, $\psi=\frac{T}{K^2}$ and $\rho=\frac{1}{2}$,  
%\begin{align*}
%\dfrac{8K}{(\psi T \epsilon_{m_i})^{\frac{3\rho}{2}}} \geq \dfrac{K\log T}{(\psi T)^{3\rho}}\sum_{m=0}^{m_i}\dfrac{1}{\epsilon_{m_i}^{3\rho + 1}}
%\end{align*}
%\end{lemma}
%
%\begin{proof}
%%We prove this lemma by contradiction. Suppose,
%%
%%\begin{align}
%%\dfrac{6K}{(\psi T \epsilon_{m_i})^{\frac{3\rho}{2}}} < \dfrac{K\log T}{(\psi T)^{3\rho}}\sum_{m=0}^{m_i}\dfrac{1}{\epsilon_{m_i}^{3\rho + 1}} \label{eq:contra:1}
%%\end{align}
%%
%%Again, we know that 
%%\begin{align*}
%%\dfrac{6K}{(\psi T \epsilon_{m_i})^{\frac{3\rho}{2}}} \overset{(a)}{\leq} \dfrac{6K}{( \frac{T^2}{K^2} \epsilon_{m_i})^{\frac{3}{4}}} \overset{(b)}{\leq} \dfrac{6K}{( \frac{T^2}{K^2} \sqrt{\frac{e}{T}})^{\frac{3}{4}}}
%%< \dfrac{6K^{\frac{5}{2}}}{T^{\frac{9}{8}}}
%%\end{align*}
%%
%%where, in $(a)$ we substitute the values of $\rho$ and $\psi$ and $(b)$ happens because $\epsilon_{m_i} \geq \sqrt{\frac{e}{T}}$. 
%From the conditions stated we can show that,
%
%\begin{align*}
%\dfrac{K\log T}{(\psi T)^{3\rho}}\sum_{m=0}^{m_i}\dfrac{1}{\epsilon_{m_i}^{3\rho + 1}} &\overset{(a)}{\leq} 
%\dfrac{K\log T}{(\frac{T^2}{K^2})^{\frac{3}{2}}}\bigg[\left( 1 + \dfrac{2(2^{ \frac{1}{2}\log_{2} \frac{T}{e}}-1)}{2-1} \right)^{\frac{5}{2}}\bigg] \\
%%%%%%%%%%%%%%%%%%%%%%%%%%%%%%%%%
%&\leq \dfrac{12 K^4 T^{\frac{5}{4}}\log T}{T^3} \overset{(b)}{\leq} \dfrac{12 K^4}{T^{\frac{5}{4}}}
%\end{align*}
%
%where, in $(a)$ we substitute the values of $\rho$ and $\psi$ and $(b)$ follows from the inequality $\log T \leq \sqrt{T}$. 
%%Substituting the values in Equation \ref{eq:contra:1} we get,
%%\begin{align}
%%& \dfrac{6K^{\frac{5}{2}}}{T^{\frac{9}{8}}} < \dfrac{6K^4\log T}{T^{\frac{5}{4}}}\nonumber\\
%%& \dfrac{T^{\frac{11}{8}}}{\log T} < K^{\frac{3}{2}}\nonumber\\
%%& \dfrac{T^{\frac{11}{8}}}{\sqrt{T}} \overset{(a)}{<} K^{1.5}\nonumber\\
%%& K^{2.1} \overset{(b)}{<} K^{1.5} \label{eq:ineq1}
%%\end{align}
%%
%%where, $(a)$ occurs because of the inequality $\log T < \sqrt{T}$ and $(b)$ happens because of the condition that $T\geq K^{2.4}$. But the inequality \ref{eq:ineq1} is not possible for any $K\geq 2$. 
%So clearly we can conclude that,
%\begin{align*}
% \dfrac{32K\log T}{(\psi T)^{3\rho}}\sum_{m=0}^{m_i}\dfrac{1}{\epsilon_{m_i}^{3\rho + 1}} \leq \dfrac{12 K^4}{T^{\frac{5}{4}}}
%\end{align*}
%%\begin{align*}
%%\dfrac{6K}{(\psi T \epsilon_{m_i})^{\frac{3\rho}{2}}} > \dfrac{32K\log T}{(\psi T)^{3\rho}}\sum_{m=0}^{m_i}\dfrac{1}{\epsilon_{m_i}^{3\rho + 1}}
%%\end{align*}
%
%\hfill $\blacksquare$	
%\end{proof}

%\subsection{Proof of Lemma 6}
%\label{App:Lemma:6}
%\begin{lemma}
%For all bounded rewards in $[0,1]$, $\dfrac{\Delta_i}{4} \geq \dfrac{\Delta_i}{4\sigma_i^2 + 4} $.
%\end{lemma}
%
%\begin{proof}
%Since all rewards are bounded in $[0,1]$, we know that $0\leq\sigma_i^2 \leq \frac{1}{4}$. Hence we can show that,
%\begin{align*}
%\dfrac{\Delta_i}{4\sigma_i^2 + 4} &\leq \dfrac{\Delta_i}{4.\frac{1}{4} + 4}\\
%& \leq \dfrac{\Delta_i}{5}
%\end{align*}
% 
%But for all $\Delta_i \in [0,1]$ we know that $\dfrac{\Delta_{i}}{5} \leq \dfrac{\Delta_{i}}{4} $. Hence, 
%$\dfrac{\Delta_i}{4} \geq \dfrac{\Delta_i}{4\sigma_i^2 + 4} $.
%
%\hfill $\blacksquare$	
%\end{proof}



\subsection{Proof of Lemma \ref{proofTheorem:Lemma:6}}
\label{App:Lemma:6}
\begin{customlem}{6}
For two integer constants $c_1$ and $c_2$, if $20 c_1 \leq c_2$ then,
\begin{align*}
c_1 \dfrac{4\sigma_i^2 + 4}{\Delta_i}\log\bigg( \dfrac{T\Delta_i^2}{K}\bigg) \leq c_2 \dfrac{\sigma_i^2}{\Delta_i}\log\bigg( \dfrac{T\Delta_i^2}{K}\bigg).
\end{align*}
 
\end{customlem}

\begin{proof}
We again prove this by contradiction. Suppose, 
\begin{align*}
c_1 \dfrac{4\sigma_i^2 + 4}{\Delta_i}\log\bigg( \dfrac{T\Delta_i^2}{K}\bigg) > c_2 \dfrac{\sigma_i^2}{\Delta_i}\log\bigg( \dfrac{T\Delta_i^2}{K}\bigg).
\end{align*}

Further reducing the above two terms we can show that, 

\begin{align*}
& 4c_1\sigma_i^2 + 4c_1 > c_2\sigma_i^2\\
& \Rightarrow 4c_1.\dfrac{1}{4} + 4c_1 \overset{(a)}{>} \dfrac{c_2}{4}\\
& \Rightarrow 20 c_1 > c_2.
\end{align*}

Here, $(a)$ occurs because $0\leq\sigma_i^2 \leq \frac{1}{4},\forall i\in \A$. But, we already know that $20 c_1 \leq c_2$. Hence, 
\begin{align*}
c_1 \dfrac{4\sigma_i^2 + 4}{\Delta_i}\log\bigg( \dfrac{T\Delta_i^2}{K}\bigg) \leq c_2 \dfrac{\sigma_i^2}{\Delta_i}\log\bigg( \dfrac{T\Delta_i^2}{K}\bigg).
\end{align*}

\hfill $\blacksquare$	
\end{proof}

%\subsection{Proof of Lemma 8}
%\label{App:Lemma:8}
%\begin{lemma}
%If $m_*$ be the round that the optimal arm $*$ gets eliminated, then we can show that the regret is upper bounded by,
%
%\begin{align*}
%\sum_{m_{*}=0}^{max_{j\in \A^{'}}m_{j}}\sum_{i\in \A^{''}:m_{i}>m_{*}}\bigg(\dfrac{388 K}{(\psi  T\epsilon_{m_{*}})^{\frac{3\rho}{2}}} \bigg).T\max_{j\in \A^{''}:m_{j}\geq m_{*}}{\Delta}_{j} \\
%%%%%%%%%%%%%%%%%%%%%%%%%
% \leq\sum_{i\in \A^{'}}\dfrac{C_2^{'} K^{\frac{5}{2}}}{\sqrt{T\Delta_i}} +\sum_{i\in \A^{''}\setminus \A^{'}}\dfrac{C_2^{'} K^{\frac{5}{2}}}{\sqrt{T b}}
%\end{align*}
%
%\end{lemma}
%
%\begin{proof}
%\begin{align*}
%&\sum_{m_{*}=0}^{max_{j\in \A^{'}}m_{j}}\sum_{i\in \A^{''}:m_{i}>m_{*}}\bigg(\dfrac{364 K^4}{(T^{\frac{5}{4}}\sqrt{\epsilon_{m_{*}}})} \bigg).T\max_{j\in \A^{''}:m_{j}\geq m_{*}}{\Delta}_{j}\\
%%%%%%%%%%%%%%%%%%%%%%%%%%%%%
%&\leq\sum_{m_{*}=0}^{max_{j\in \A^{'}}m_{j}}\sum_{i\in \A^{''}:m_{i}>m_{*}}\bigg(\dfrac{364 K^4 \sqrt{4}}{(T^{\frac{5}{4}}\sqrt{\epsilon_{m_{*}}})} \bigg).T.4\sqrt{\epsilon_{m_{*}}}\\
%%%%%%%%%%%%%%%%%%%%%%%%%%%%%
%&\leq\sum_{m_{*}=0}^{max_{j\in \A^{'}}m_{j}}\sum_{i\in \A^{''}:m_{i}>m_{*}}\bigg(\dfrac{C_2 K^4}{T^{\frac{1}{4}}\epsilon_{m_{*}}^{\frac{1}{2}-\frac{1}{2}}} \bigg)\\
%%%%%%%%%%%%%%%%%%%%%%%%%%%%%
%&\leq\sum_{i\in \A^{''}:m_{i}>m_{*}}\sum_{m_{*}=0}^{\min{\lbrace m_{i},m_{b}\rbrace}}\bigg(\dfrac{C_2 K^4}{T^{\frac{1}{4}}} \bigg)\\
%%%%%%%%%%%%%%%%%%%%%%%%%%%%%
%&\leq\sum_{i\in \A^{'}}\bigg(\dfrac{C_2 K^4}{T^{\frac{1}{4}}} \bigg)+\sum_{i\in \A^{''}\setminus \A^{'}}\bigg(\dfrac{C_2 K^4}{T^{\frac{1}{4}}} \bigg)\\
%\end{align*}
%In the above simplification, $C_2$ is an integer constant.
%
%\hfill $\blacksquare$	
%\end{proof}

%\subsection{Proof of Lemma 8}
%\label{App:Lemma:8}
%\begin{lemma}
%If $m_*$ be the round that the optimal arm $*$ gets eliminated, then we can show that the regret is upper bounded by,
%
%\begin{align*}
%\sum_{m_{*}=0}^{max_{j\in \A^{'}}m_{j}}\sum_{i\in \A^{''}:m_{i}>m_{*}}\bigg(\dfrac{388 K}{(\psi  T\epsilon_{m_{*}})^{\frac{3\rho}{2}}} \bigg).T\max_{j\in \A^{''}:m_{j}\geq m_{*}}{\Delta}_{j} \\
%%%%%%%%%%%%%%%%%%%%%%%%%
% \leq\sum_{i\in \A^{'}}\dfrac{C_2^{'} K^{\frac{5}{2}}}{\sqrt{T\Delta_i}} +\sum_{i\in \A^{''}\setminus \A^{'}}\dfrac{C_2^{'} K^{\frac{5}{2}}}{\sqrt{T b}}
%\end{align*}
%
%\end{lemma}
%
%\begin{proof}
%\begin{align*}
%&\sum_{m_{*}=0}^{max_{j\in \A^{'}}m_{j}}\sum_{i\in \A^{''}:m_{i}>m_{*}}\bigg(\dfrac{388 K}{(\psi  T\epsilon_{m_{*}})^{\frac{3\rho}{2}}} \bigg).T\max_{j\in \A^{''}:m_{j}\geq m_{*}}{\Delta}_{j}\\
%%%%%%%%%%%%%%%%%%%%%%%%%%%%%
%&\leq\sum_{m_{*}=0}^{max_{j\in \A^{'}}m_{j}}\sum_{i\in \A^{''}:m_{i}>m_{*}}\bigg(\dfrac{388 K\sqrt{4}}{(\psi  T\epsilon_{m_{*}})^{\frac{3\rho}{2}}} \bigg).T.4\sqrt{\epsilon_{m_{*}}}\\
%%%%%%%%%%%%%%%%%%%%%%%%%%%%%
%&\leq\sum_{m_{*}=0}^{max_{j\in \A^{'}}m_{j}}\sum_{i\in \A^{''}:m_{i}>m_{*}}C_2 K\bigg(\dfrac{T^{1-\frac{3\rho}{2}}}{\psi^{\frac{3\rho}{2}}\epsilon_{m_{*}}^{\frac{3\rho}{2}-\frac{1}{2}}} \bigg)\\
%%%%%%%%%%%%%%%%%%%%%%%%%%%%%
%&\leq\sum_{i\in \A^{''}:m_{i}>m_{*}}\sum_{m_{*}=0}^{\min{\lbrace m_{i},m_{b}\rbrace}}\bigg(\dfrac{C_2 K T^{1-\frac{3\rho}{2}}}{\psi^{\frac{3\rho}{2}}2^{-(\frac{3\rho}{2} -\frac{1}{2})m_{*}}} \bigg)\\
%%%%%%%%%%%%%%%%%%%%%%%%%%%%%
%&\leq\sum_{i\in \A^{'}}\bigg(\dfrac{C_2 K T^{1-\frac{3\rho}{2}}}{\psi^{\frac{3\rho}{2}}2^{-(\frac{3\rho}{2} -\frac{1}{2})m_{*}}} \bigg)+\sum_{i\in \A^{''}\setminus \A^{'}}\bigg(\dfrac{C_2 K T^{1-\frac{3\rho}{2} }}{\psi^{\frac{3\rho}{2}}2^{-(\frac{3\rho}{2} -\frac{1}{2})m_{b}}} \bigg)\\
%%%%%%%%%%%%%%%%%%%%%%%%%%%%%
%&\leq\sum_{i\in \A^{'}}\bigg(\dfrac{C_2 K T^{1-\frac{3\rho}{2}}.2^{\frac{\frac{3\rho}{2}}{2}-\frac{1}{4}}}{\psi^{\frac{3\rho}{2}}\Delta_{i}^{\frac{3\rho}{2} -\frac{1}{2}}} \bigg)+\sum_{i\in \A^{''}\setminus \A^{'}}\bigg(\dfrac{C_2 K T^{1-\frac{3\rho}{2}}.2^{\frac{\frac{3\rho}{2}}{2}-\frac{1}{4}}}{\psi^{\frac{3\rho}{2}}b^{\frac{3\rho}{2} -\frac{1}{2}}} \bigg)\\
%%%%%%%%%%%%%%%%%%%%%%%%%%%%%
%&\leq\sum_{i\in \A^{'}}\bigg(\dfrac{ C_2 K 2^{\frac{\frac{3\rho}{2}}{2}+\frac{19}{4}}.T^{1-\frac{3\rho}{2} } }{\psi^{\rho}\Delta_{i}^{2\frac{3\rho}{2} -1}} \bigg)+\sum_{i\in \A^{''}\setminus \A^{'}}\bigg(\dfrac{C_2 K 2^{\frac{\frac{3\rho}{2}}{2}+\frac{19}{4}}.T^{1-\frac{3\rho}{2}} }{\psi^{\frac{3\rho}{2} }b^{2\frac{3\rho}{2}-1}} \bigg)\\
%%%%%%%%%%%%%%%%%%%%%%%%%%%%%
%&\overset{(a)}{\leq}\sum_{i\in \A^{'}}\bigg(\dfrac{C_2^{'} K .T^{1-\frac{3}{4}}}{(\frac{T}{K^2})^{\frac{3}{4}}\Delta_{i}^{2.\frac{3}{4} -1}} \bigg)+\sum_{i\in \A^{''}\setminus \A^{'}}\bigg(\dfrac{C_2^{'} K T^{1-\frac{3}{4}}}{(\frac{T}{K^2})^{\frac{3}{4}}b^{2.\frac{3}{4}-1}} \bigg)\\
%%%%%%%%%%%%%%%%%%%%%%%%%%%%%
%&\leq\sum_{i\in \A^{'}}\dfrac{C_2^{'} K^{\frac{5}{2}}}{\sqrt{T\Delta_i}} +\sum_{i\in \A^{''}\setminus \A^{'}}\dfrac{C_2^{'} K^{\frac{5}{2}}}{\sqrt{T b}}
%%%%%%%%%%%%%%%%%%%%%%%%%%%%%
%%& = \sum_{i\in \A^{'}}\bigg(\dfrac{ C_{2}(\rho) T^{1-\rho}}{\Delta_{i}^{2\rho-1}} \bigg)+\sum_{i\in \A^{''}\setminus \A^{'}}\bigg(\dfrac{C_{2}(\rho)T^{1-\rho}}{b^{2\rho -1}} \bigg) \text{, where } C_2(x) = \frac{2^{\frac{x}{2}+\frac{19}{4}}}{\psi^{x}}
%\end{align*}
%In the above simplification, $(a)$ is obtained by substituting the values of $\psi$ and $\rho$.
%
%\hfill $\blacksquare$	
%\end{proof}

\subsection{Proof of Corollary \ref{Result:Corollary:1}}
\label{App:Corollary:1}

\begin{customCorollary}{1}(\textbf{\textit{Gap-Independent Bound}})
%\label{Result:Corollary:1}
When the gaps of all the sub-optimal arms are identical, i.e., $\Delta_i =\Delta = \sqrt{\frac{K\log K}{T}}>\sqrt{\frac{e}{T}}, \forall i\in \A$ and $C_3$ being an integer constant, the
regret of EUCBV is upper bounded by the following gap-independent expression:
\begin{align*}
	\E[R_{T}]\leq  \dfrac{C_3 K^5}{T^{\frac{1}{4}}} + 320\sqrt{KT}.
\end{align*}	
\end{customCorollary}


\begin{proof}
\label{Proof:Corollary:1}
From \cite{bubeck2011pure}  we know that the function $x\in [0,1]\mapsto x\exp(-Cx^2)$ is  decreasing on $\left[\frac{1}{\sqrt{2C}},1\right ]$ for any $C>0$. Thus, we take $C=\left\lfloor \frac{T}{e}\right\rfloor$ and choose  $\Delta_{i}=\Delta=\sqrt{\frac{K\log K}{T}}>\sqrt{\frac{e}{T}}$ for all $i$.

First, let us recall the result in Theorem \ref{Result:Theorem:1} below:
\begin{align*}
\E [R_{T}] \leq &\sum\limits_{i\in \A :\Delta_{i} > b}\bigg\lbrace \dfrac{C_0 K^{4}}{T^{\frac{1}{4}}} + \bigg(\Delta_{i}+\dfrac{320\sigma_i^2\log{(\frac{T\Delta_{i}^{2}}{K})}}{\Delta_{i}}\bigg)\bigg \rbrace\\ 
  & +\sum\limits_{i\in \A :0 < \Delta_{i}\leq b} \dfrac{C_2 K^{4}}{T^{\frac{1}{4}}} + \max_{i\in \A :0 < \Delta_{i}\leq b}\Delta_{i}T.
\end{align*}

Now,  with  $\Delta_i =\Delta = \sqrt{\frac{K\log K}{T}}>\sqrt{\frac{e}{T}}$ we obtain,
	\begin{align*}
	&\sum_{i\in \A :\Delta_{i} > b}\dfrac{320\sigma_i^2\log{(\frac{T\Delta_{i}^{2}}{K})}}{\Delta_{i}} \leq  \dfrac{320\sigma_{\max}^2 K\sqrt{T}\log{(T\dfrac{K(\log K)}{T K})}}{\sqrt{K\log K}}\\ 
	&\leq  \dfrac{320\sigma_{\max}^2\sqrt{KT}\log{(\log K)}}{\sqrt{\log K}}
	\overset{(a)}{\leq} 320\sigma_{\max}^2\sqrt{KT} 
	\end{align*}		
	where $(a)$ follows from the identity $\dfrac{\log{(\log K)}}{\sqrt{\log K}}\leq 1$ for $K\geq 2$. 
	
%For the term $\sum\limits_{i\in \A :\Delta_{i} > b}\dfrac{C_0 K^{\frac{5}{2}}}{\sqrt{T\Delta_i}}$ by substituting the value of $\Delta_i=\Delta=\sqrt{\dfrac{K\log K}{T}}$ we get,
%\begin{align*}
%\dfrac{C_0 K^{\frac{5}{2}+1}}{\sqrt{T\Delta_i}} &\leq \dfrac{C_0 K^{\frac{5}{2}}}{\sqrt{T\sqrt{\dfrac{K\log K}{T}}}} \\
%%%%%%%%%%%%%%%%%%%%%%%%%%%
%\leq \dfrac{C_0 K^{\frac{5}{2}+1}}{(KT\log K)^{\frac{1}{4}}} \leq \dfrac{C_0 K^3}{T^{\frac{1}{4}}}
%\end{align*}	
%	
%Similarly for the term $\sum\limits_{i\in \A :0 < \Delta_{i}\leq b} \dfrac{C_2^{'} K^{\frac{5}{2}}}{\sqrt{T\Delta_i}}$ we can show that,
%\begin{align*}
%\sum\limits_{i\in \A :0 < \Delta_{i}\leq b} \dfrac{C_2^{'} K^{\frac{5}{2}}}{\sqrt{T\Delta_i}} &\leq \dfrac{C_2^{'} K^{\frac{5}{2}+1}}{\sqrt{T\sqrt{\dfrac{e}{T}}}} \leq \dfrac{C_2^{'} K^3}{T^{\frac{1}{4}}} 
%\end{align*}

Thus, the total worst case gap-independent bound is given by
	\begin{align*}
	\E[R_{T}] &\overset{(a)}{\leq}  \dfrac{C_3 K^5}{T^{\frac{1}{4}}} + 320\sigma_{\max}^2\sqrt{KT}\\
	&\overset{(b)}{\leq} \dfrac{C_3 K^5}{T^{\frac{1}{4}}} + 320\sqrt{KT}
	\end{align*}	
	
where, in$(a)$, $C_3$ is an integer constant such that $C_3 = C_0 + C_2 $ and $(b)$ occurs because $\sigma_i^2 \in [0,\frac{1}{4}], \forall i\in \A$.

\hfill $\blacksquare$	
\end{proof}

%%%%%%%%%%%%%%%%%%%%%%%%%%%%%%%%%%%%%%%%%%%%%%%%%%%%%%%%%%%%
% Bibliography.

\begin{singlespace}
\bibliographystyle{iitm}
\bibliography{refs}
\end{singlespace}


%%%%%%%%%%%%%%%%%%%%%%%%%%%%%%%%%%%%%%%%%%%%%%%%%%%%%%%%%%%%
% List of papers

\listofpapers

%\begin{enumerate}  
%\item Authors....  \newblock
% Title...
%  \newblock {\em Journal}, Volume,
%  Page, (year).
%\end{enumerate}  

\begin{enumerate}
\item Subhojyoti Mukherjee, K.P.~Naveen, Nandan Sudarsanam, and Balaraman Ravindran, ``\textit{Thresholding Bandit with Augmented UCB}'', \textit{Proceedings of the Twenty-Sixth International Joint Conference on
               Artificial Intelligence, {IJCAI} 2017, Melbourne, Australia, August
               19-25, 2017,2515-2521}, main conference track.
\item Subhojyoti Mukherjee, K.P.~Naveen, Nandan Sudarsanam, and Balaraman Ravindran, ``\textit{Efficient UCBV: An Almost Optimal Algorithm using Variance Estimates}'', \textit{To appear in Proceedings of the Thirty-Second Association for the Advancement of Artificial Intelligence (AAAI-18)}, main conference track.
\end{enumerate}

\end{document}

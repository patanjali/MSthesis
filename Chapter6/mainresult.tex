%\begin{lemma}
%
%\end{lemma}

\begin{theorem}
The expected cumulative regret of ImpCPD algorithm is upper bounded by,
\begin{align*}
\E[R_t] &\leq \sum_{i = 1}^K\sum_{j=1}^{G}\bigg[ 1 
%%%%%%%%%%
+ \dfrac{4T\Delta^{opt}_{i,c_j}\log(\psi T)}{(\psi T)^{\alpha}}\dfrac{M 2^{2\alpha +1}}{(\Delta^{opt}_{i,c_j})^{4\alpha +2}}
%%%%%%%%%%
+ \Delta^{opt}_{i,c_j} + \dfrac{4\log(\psi T(\Delta^{opt}_{i,c_j})^4)}{2(\Delta^{opt}_{i,c_j})}\\
%%%%%%%%%%%%%%%%%%%%%%%%%%%
& + \dfrac{2T\Delta^{opt}_{i,c_{j+1}}\log(\psi T)}{(\psi T)^{\alpha}}\dfrac{M 2^{2\alpha + 1}}{(\Delta^{chg}_{i,c_j})^{4\alpha + 2}}\\
%%%%%%%%%%%%%%%%%%%%%%%%%%%
&+ \Delta^{opt}_{\max,c_{j+1}} + \dfrac{\Delta^{opt}_{\max,c_{j+1}}\log(\psi T(\Delta^{chg}_{i,c_j})^4)}{2\Delta^{chg}_{\epsilon_0,c_j}} + \Delta^{opt}_{i,c_{j+1}} + \dfrac{\Delta^{opt}_{i,c_{j+1}}\log(\psi T(\Delta^{chg}_{i,c_j})^4)}{2(\Delta^{chg}_{i,c_j})^2}
\bigg]
\end{align*}

where $\alpha,\psi$ are exploration parameters and $M=\dfrac{1}{2}\log_{2}\dfrac{T}{e}$.
\end{theorem}

\begin{proof}

We proceed as like Theorem 3.1 in \citet{auer2010ucb} and we combine the changepoint detection sub-routine with this. 

\textbf{Step 0.(Vital Assumption for proving):} In this proof, we assume that all the changepoints are detected by the algorithm. 

\textbf{Step 1.(Define some notations):} In this proof, we define the confidence interval for the $i$-th arm as $s_i=\sqrt{\dfrac{\alpha\log(\psi T\epsilon_m^2)}{n_{i}}}$. The phase numbers are denoted by $m=0,1,\ldots,M$ where $M=\frac{1}{2}\log_{2}\frac{T}{e}$. For the sake of brevity, in this proof we write $n_i$ instead of $n_{i,t_0:t_p}$ denoting only the number of times the arm $i$ is pulled between two restarts, ie, between $t_0$ to $t_p$.

\textbf{Step 2.(Define an optimality stopping phase $m_i$):} We define an optimality stopping phase $m_i$ for a sub-optimal arm $i\in\A$ as the first phase after which the arm $i$ is no longer pulled such that,

\begin{align*}
m_i = \min\left\lbrace m: \sqrt{\alpha\epsilon_{m}} < \dfrac{\Delta^{opt}_{i,c_j}}{4} \right\rbrace
\end{align*} 

\textbf{Step 3.(Define a changepoint stopping phase $p_i$):} We define a changepoint stopping phase $p_i$ for an arm $i\in\A$ such that it is the first phase when the changepoint is detected such that,

\begin{align*}
p_i = \min\left\lbrace m: \sqrt{\alpha\epsilon_{m}} < \dfrac{\Delta^{chg}_{i,c_j}}{4} \right\rbrace
\end{align*}

\textbf{Step 4.(Regret for sub-optimal arm $i$ being pulled after $m_i$-th phase):} Note, that on or after the $m_i$-th phase a sub-optimal arm $i$ is not pulled anymore if these four conditions hold,

\begin{eqnarray}
\hat{r}_{i,c_j} < r_{i,c_j} + s_i, \hspace*{2em}  \hat{r}_{i*,c_j} > r_{i*,c_j} - s_{i*}, \hspace*{2em} s_i > s_{i*}, \hspace*{2em} n_i < n_{i*} \label{eq:arm-pull-opt}
\end{eqnarray}

Also, in the $m_i$-th phase if $n_i \geq \ell_{m_i} = \dfrac{\log(\psi T\epsilon_{m_i}^2)}{2\epsilon_{m_i}}$ then we can show that,
\begin{align*}
s_i = \sqrt{\dfrac{\alpha\log(\psi T\epsilon_{m_i}^2)}{2n_{i}}} \leq \sqrt{\dfrac{\alpha\log(\psi T\epsilon_{m_i}^2)}{2\ell_{m_i}}} \leq \sqrt{\alpha\epsilon_{m_i}\dfrac{\log(\psi T\epsilon_{m_i}^2)}{\log(\psi T\epsilon_{m_i}^2)}} \leq \dfrac{\Delta^{opt}_{i,c_j}}{4}.
\end{align*}

If indeed the four conditions in equation \ref{eq:arm-pull-opt} hold then we can show that in the $m_i$-th phase,

\begin{align*}
\hat{r}_{i,c_j} + s_i \leq \hat{r}_{i,c_j} + 4s_i - 2s_i \leq \hat{r}_{i,c_j} + \Delta^{opt}_{i,c_j} - 2s_i \leq \hat{r}_{i*,c_j} - 2s_{i*} \leq \hat{r}_{i*,c_j} - s_{i*}
\end{align*}

Hence, the sub-optimal arm $i$ is no longer pulled on or after the $m_i$-th phase. Therefore, to bound the number of pulls of the sub-optimal arm $i$, we need to bound the probability of the complementary of the four events in equation \ref{eq:arm-pull-chg}.

For the first event in equation \ref{eq:arm-pull-opt}, using Chernoff-Hoeffding bound we can bound the probability of the complementary of that event by,

\begin{align*}
\sum_{m=0}^{m_i}\sum_{n_i=1}^{\ell_{m_i}}\Pb\lbrace \hat{r}_{i,c_j} \geq  r_{i,c_j} + s_i \rbrace &\leq \sum_{m=0}^{m_i}\sum_{n_i=1}^{\ell_{m_i}}\exp\left(-2s_i^2n_i \right)
%%%%%%%%%%%%%%%%%%%%%%%%%%%%%%%%%%
\leq \sum_{m=0}^{m_i}\sum_{n_i=1}^{\ell_{m_i}}\exp\left(-2\dfrac{\alpha\log(\psi T\epsilon_{m_i}^2)}{2n_i}n_i \right)\\
%%%%%%%%%%%%%%%%%%%%%%%%%%%%%%%%%%
&\leq \sum_{m=0}^{m_i}\sum_{n_i=1}^{\ell_{m_i}}\dfrac{1}{\psi T\epsilon_{m_i}^2} \leq \sum_{m=0}^{m_i}\dfrac{\log(\psi T\epsilon_{m_i}^2)}{2\epsilon_{m_i}}\dfrac{1}{(\psi T\epsilon_{m_i}^2)^{\alpha}} = \dfrac{\log(\psi T)}{2(\psi T)^{\alpha}}\sum_{m=0}^{m_i}\dfrac{1}{\epsilon_{m_i}^{2\alpha +1}}.
\end{align*}

Similarly, for the second event in equation \ref{eq:arm-pull-opt}, we can bound the probability of its complementary event by,

\begin{align*}
\sum_{m=0}^{m_i}\sum_{n_{i*} =1}^{\ell_{m_i}}\Pb\lbrace \hat{r}_{i*,c_j} \leq  r_{i*,c_j} - s_{i*} \rbrace &\leq \sum_{m=0}^{m_i}\sum_{n_i=1}^{\ell_{m_i}}\exp\left(-2(s_{i*})^2n_{i*} \right)\leq \dfrac{\log(\psi T)}{2(\psi T)^{\alpha}}\sum_{m=0}^{m_i}\dfrac{1}{\epsilon_{m_i}^{2\alpha +1}}.
\end{align*}


Also, for the third event in equation \ref{eq:arm-pull-opt}, we can bound the probability of its complementary event by,

\begin{align*}
\sum_{m=0}^{m_i}\sum_{n_i=1}^{\ell_{m_i}}\Pb\lbrace s_i < s_{i*}\rbrace \leq \sum_{m=0}^{m_i}\sum_{n_i=1}^{\ell_{m_i}}\Pb\lbrace n_i > n_{i*}\rbrace \leq \sum_{m=0}^{m_i}\sum_{n_i=1}^{\ell_{m_i}}\Pb\lbrace \hat{r}_{i,c_j} + s_i > \hat{r}_{i*,c_j} + s_{i*} \rbrace.
\end{align*}

But the event $\hat{r}_{i,c_j} + s_i > \hat{r}_{i*,c_j} + s_i$ is possible only when the following three events occur, $\hat{r}_{i*,c_j} \leq r_{i*,c_j} - s_{i*} , \hat{r}_{i,c_j} \geq r_{i,c_j} + s_i$ and $r_{i*}-r_{i} < 2s_i $. But the third event will not happen for $n_i\geq \ell_{m_i}$. Proceeding as before, we can show that the probability of the remaining two events is bounded by,

\begin{align*}
&\sum_{m=0}^{m_i}\sum_{n_i=1}^{\ell_{m_i}}\Pb\lbrace s_i < s_{i*}\rbrace \leq \sum_{m=0}^{m_i}\sum_{n_i=1}^{\ell_{m_i}}\Pb\lbrace n_i > n_{i*}\rbrace \leq \sum_{m=0}^{m_i}\sum_{n_i=1}^{\ell_{m_i}}\Pb\lbrace \hat{r}_{i,c_j} + s_i > \hat{r}_{i*,c_j} + s_{i*} \rbrace \\
%%%%%%%%%%%%%%%%%%%%%%%%%%%%%%%%%%%%%%%%%%%%%%%%5
&\leq \sum_{m=0}^{m_i}\sum_{n_i=1}^{\ell_{m_i}}\left( \Pb\lbrace \hat{r}_{i,c_j} \geq r_{i,c_j} + s_i \rbrace \bigcup \Pb\lbrace \hat{r}_{i*,c_j} \leq r_{i*,c_j} - s_{i*} \rbrace \right)\leq \sum_{m=0}^{m_i}\sum_{n_i=1}^{\ell_{m_i}}\dfrac{\log(\psi T)}{(\psi T)^{\alpha}}\sum_{m=0}^{m_i}\dfrac{1}{\epsilon_{m_i}^{2\alpha + 1}}.
\end{align*}

Finally for the fourth event in equation \ref{eq:arm-pull-opt}, we can bound the probability of its complementary event by following the same steps as above,

\begin{align*}
\sum_{m=0}^{m_i}\sum_{n_i=1}^{\ell_{m_i}}\Pb\lbrace n_i > n_{i*}\rbrace \leq \sum_{m=0}^{m_i}\sum_{n_i=1}^{\ell_{m_i}}\Pb\lbrace r_{i,c_j} + s_i > r_{i*,c_j} + s_{i*} \rbrace \leq \dfrac{\log(\psi T)}{(\psi T)^{\alpha}}\sum_{m=0}^{m_i}\dfrac{1}{\epsilon_{m_i}^{2\alpha +1}}.
\end{align*}


Combining the above four cases we can bound the probability that a sub-optimal arm $i$ will no longer be pulled on or after the $m_i$-th phase by,

\begin{align*}
\dfrac{4\log(\psi T)}{(\psi T)^{\alpha}}\sum_{m=0}^{m_i}\dfrac{1}{\epsilon_{m_i}^{2\alpha +1}}.
\end{align*}

Bounding this trivially by $T\Delta^{opt}_{i,c_j}$ for each arm $i\in\A$ we get the regret suffered for arm $i$ after the $m_i$-th phase  as,

\begin{align*}
\sum_{i\in\A}\left(\dfrac{4T\Delta^{opt}_{i,c_j}\log(\psi T)}{(\psi T)^{\alpha}}\sum_{m=0}^{m_i}\dfrac{1}{\epsilon_{m_i}^{2\alpha +1}}\right) \leq \sum_{i\in\A}\left(\dfrac{4T\Delta^{opt}_{i,c_j}\log(\psi T)}{(\psi T)^{\alpha}}\dfrac{M 2^{2\alpha +1}}{(\Delta^{opt}_{i,c_j})^{4\alpha +2}}\right).
\end{align*}

\textbf{Step 5.(Regret for pulling the sub-optimal arm $i$ on or before $m_i$-th phase):} Either a sub-optimal arm gets pulled $\ell_{m_i}$ number of times till the $m_i$-th phase and after that the probability of it getting pulled is exponentially low (as shown in \textbf{step 4}). Hence, the number of times a sub-optimal arm $i$ is pulled till the $m_i$-th phase is given by,

\begin{align*}
\ell_{m_i} = \left\lceil \dfrac{\log(\psi T\epsilon_{m_i}^2)}{2\epsilon_{m_i}} \right\rceil
\end{align*}

Hence, considering each arm $i\in\A$ the total regret is bounded by,

\begin{align*}
\sum_{i\in\A}\Delta^{opt}_{i,c_j}\left\lceil \dfrac{\log(\psi T\epsilon_{m_i}^2)}{2\epsilon_{m_i}} \right\rceil < \Delta^{opt}_{i,c_j}\left[ 1 + \dfrac{\log(\psi T\epsilon_{m_i}^2)}{2\epsilon_{m_i}} \right] \leq \Delta^{opt}_{i,c_j}\left[ 1 + \dfrac{4\log(\psi T(\Delta^{opt}_{i,c_j})^4)}{2(\Delta^{opt}_{i,c_j})^2}\right].
\end{align*} 

\textbf{Step 6.(Regret for not detecting changepoint $c_{j}$ for arm $i$ after the $p_{i}$-th phase):} Proceeding similarly as in \textbf{step 4} we can show that in the $p_i$-th phase for an arm $i$ if $n_{i}\geq\ell_{p_i}$ then,

\begin{align*}
s_i = \sqrt{\dfrac{\alpha\log(\psi T\epsilon_{p_i}^2)}{2n_{i}}} \leq \sqrt{\dfrac{\alpha\log(\psi T\epsilon_{p_i}^2)}{2\ell_{p_i}}} \leq \sqrt{\alpha\epsilon_{p_i}\dfrac{\log(\psi T\epsilon_{p_i}^2)}{\log(\psi T\epsilon_{p_i}^2)}} \leq \dfrac{\Delta^{chg}_{i,c_j}}{4}.
\end{align*}

Furthermore, we can show that if the following two conditions hold for an arm $i$ then the changepoint will definitely get detected, that is,

\begin{eqnarray}
\hat{r}_{i,c_{j-1}} + s_{i} < \hat{r}_{i,c_{j}} - s_{i},\hspace*{4mm} \hat{r}_{i,c_{j-1}} - s_{i} > \hat{r}_{i,c_{j}} + s_{i} \label{eq:arm-pull-chg}.
\end{eqnarray}

Indeed, we can show that if there is a changepoint at $c_j$ such that $r_{i,c_{j-1}} < r_{i,c_j}$ and if the first conditions in equation \ref{eq:arm-pull-chg} hold  in the $p_i$-th phase then,

\begin{align*}
\hat{r}_{i,c_{j-1}} + s_{i} \leq \hat{r}_{i,c_{j-1}} + 4s_{i} - 2s_{i} \leq \hat{r}_{i,c_{j-1}} + \Delta^{chg}_{i,c_j} -2s_{i} \leq \hat{r}_{i,c_{j}} - 2s_i \leq \hat{r}_{i,c_{j}} - s_i.
\end{align*}

Conversely, for the case $r_{i,c_{j-1}} > r_{i,c_j}$ the second condition will hold for the $p_i$-th phase. Now, to bound the regret we need to bound the probability of the complementary of these two events. For the complementary of the first event in equation \ref{eq:arm-pull-chg} by using Chernoff-Hoeffding bound we can show that,

\begin{align*}
&\sum_{m=0}^{p_i}\sum_{n_i=1}^{\ell_{p_i}}\Pb\lbrace \hat{r}_{i,c_{j-1}} \geq  r_{i,c_{j-1}} + s_i \rbrace \leq \sum_{m=0}^{p_i}\sum_{n_i=1}^{\ell_{p_i}}\exp\left(-2s_i^2n_i \right)
%%%%%%%%%%%%%%%%%%%%%%%%%%%%%%%%%%
\leq \sum_{m=0}^{p_i}\sum_{n_i=1}^{\ell_{p_i}}\exp\left(-2\dfrac{\alpha\log(\psi T\epsilon_{p_i}^2)}{2n_i}n_i \right)\\
%%%%%%%%%%%%%%%%%%%%%%%%%%%%%%%%%%
&\leq \sum_{m=0}^{p_i}\sum_{n_i=1}^{\ell_{p_i}}\dfrac{1}{\psi T\epsilon_{p_i}^2} \leq \sum_{m=0}^{p_i}\dfrac{\log(\psi T\epsilon_{p_i}^2)}{2\epsilon_{p_i}}\dfrac{1}{(\psi T)^{\alpha}\epsilon_{p_i}^2} = \dfrac{\log(\psi T)}{2(\psi T)^{\alpha}}\sum_{m=0}^{p_i}\dfrac{1}{\epsilon_{p_i}^{2\alpha + 1}}.
\end{align*}

Also, again by using Chernoff-Hoeffding bound we can show that,

\begin{align*}
\sum_{m=0}^{p_i}\sum_{n_i=\ell_{p_{i-1}}}^{\ell_{p_i}}\Pb\lbrace \hat{r}_{i,c_j} \leq  r_{i,c_j} - s_i \rbrace &\leq \sum_{m=0}^{p_i}\sum_{n_i=\ell_{p_{i-1}}}^{\ell_{p_i}}\exp\left(-2s_i^2n_i \right) \leq \dfrac{\log(\psi T)}{2(\psi T)^{\alpha}}\sum_{m=0}^{p_i}\dfrac{1}{\epsilon_{p_i}^{2\alpha + 1}}.
\end{align*}

Hence, the probability that the changepoint is not detected by the first event is bounded by,
\begin{align*}
\dfrac{\log(\psi T)}{(\psi T)^{\alpha}}\sum_{m=0}^{p_i}\dfrac{1}{\epsilon_{p_i}^{2\alpha + 1}}.
\end{align*}

Similarly, we can bound the probability of the complementary of the second event in equation \ref{eq:arm-pull-chg} as,

\begin{align*}
&\sum_{m=0}^{p_i}\sum_{n_i=1}^{\ell_{p_i}}\Pb\lbrace \hat{r}_{i,c_{j-1}} \leq  r_{i,c_{j-1}} - s_i \rbrace\leq \dfrac{\log(\psi T)}{2(\psi T)^{\alpha}}\sum_{m=0}^{p_i}\dfrac{1}{\epsilon_{p_i}^{2\alpha + 1}} \textbf{,  and  }\\
%%%%%%%%%%%%%%%%%%%%%%%%%
&\sum_{m=0}^{p_i}\sum_{n_i=\ell_{p_{i-1}}}^{\ell_{p_i}}\Pb\lbrace \hat{r}_{i,c_j} \geq  r_{i,c_j} + s_i \rbrace\leq \dfrac{\log(\psi T)}{(\psi T)^{\alpha}}\sum_{m=0}^{p_i}\dfrac{1}{\epsilon_{p_i}^{2\alpha + 1}}.
\end{align*}

Combining the contribution from these two events, we can show that the probability of not detecting the changepoint $c_j$ for the $i$-th arm after the $p_i$-th phase is upper bounded by,

\begin{align*}
\dfrac{2\log(\psi T)}{(\psi T)^{\alpha}}\sum_{m=0}^{p_i}\dfrac{1}{\epsilon_{p_i}^{2\alpha + 1}}.
\end{align*}

Furthermore, bounding the regret trivially (after the changepoint $c_j$) by $\Delta^{opt}_{i,c_{j+1}}$ for each arm $i\in\A$, we get 
\begin{align*}
\sum_{i\in\A}\left(\dfrac{2T\Delta^{opt}_{i,c_{j+1}}\log(\psi T)}{(\psi T)^{\alpha}}\sum_{m=0}^{p_i}\dfrac{1}{\epsilon_{p_i}^{2\alpha + 1}}\right) \leq \sum_{i\in\A}\left(\dfrac{2T\Delta^{opt}_{i,c_{j+1}}\log(\psi T)}{(\psi T)^{\alpha}}\dfrac{M 2^{2\alpha + 1}}{(\Delta^{chg}_{i,c_j})^{4\alpha + 2}}\right).
\end{align*}

\textbf{Step 7.(Regret for not detecting a changepoint $c_{j}$ for arm $i$ on or before the $p_{i}$-th phase):} The regret for not detecting the changepoint $c_j$ on or before the $p_i$-th phase can be broken into two parts, \textbf{(a)} the worst case events from $t_{c_j}$ to $p_{i}$ and \textbf{(b)} the minimum number of pulls $\ell_{p_i}$ required to detect the changepoint. For the first part (a) we can use Assumption \ref{assm:space-gap}, Assumption ref{assm:chg-gap}, Discussion \ref{dis:gap-delay} and Definition \ref{Def:e-chg-gap} to upper bound the regret as,

\begin{align*}
\sum_{i\in\A}\Delta^{opt}_{\max,c_{j+1}}\left\lceil \dfrac{\log(\psi T\epsilon_{p_i}^2)}{2(\Delta^{chg}_{\epsilon_0,c_j})^2} \right\rceil < \Delta^{opt}_{\max,c_{j+1}}\left[ 1 + \dfrac{\log(\psi T\epsilon_{p_i}^2)}{2(\Delta^{chg}_{\epsilon_0,c_j})^2} \right].
\end{align*}

Again, for the second part (b) the arm $i$ can be pulled no more than $\ell_{p_i}$ number of times. Hence, for each arm $i\in\A$  the regret for this case is bounded by,

\begin{align*}
\sum_{i\in\A}\Delta^{opt}_{i,c_{j+1}}\left\lceil \dfrac{\log(\psi T\epsilon_{p_i}^2)}{2\epsilon_{p_i}} \right\rceil <  \Delta^{opt}_{i,c_{j+1}}\left[ 1 + \dfrac{\log(\psi T\epsilon_{p_i}^2)}{2\epsilon_{p_i}} \right].
\end{align*}

Therefor, combining these two parts (a) and (b) we can show that the total regret for not detecting the changepoint till the $p_i$-th phase is given by,

\begin{align*}
&\sum_{i\in\A}\left\lbrace\Delta^{opt}_{\max,c_{j+1}} + \dfrac{\Delta^{opt}_{\max,c_{j+1}}\log(\psi T\epsilon_{p_i}^2)}{2(\Delta^{chg}_{\epsilon_0,c_j})^2} + \Delta^{opt}_{i,c_{j+1}} + \dfrac{\Delta^{opt}_{i,c_{j+1}}\log(\psi T\epsilon_{p_i}^2)}{2\epsilon_{p_i}}\right\rbrace \\
%%%%%%%%%%%%%%%%%%%%%%%%%
&\leq \sum_{i\in\A}\left\lbrace\Delta^{opt}_{\max,c_{j+1}} + \dfrac{\Delta^{opt}_{\max,c_{j+1}}\log(\psi T(\Delta^{chg}_{i,c_j})^4)}{2\Delta^{chg}_{\epsilon_0,c_j}} + \Delta^{opt}_{i,c_{j+1}} + \dfrac{\Delta^{opt}_{i,c_{j+1}}\log(\psi T(\Delta^{chg}_{i,c_j})^4)}{2(\Delta^{chg}_{i,c_j})^2}\right\rbrace
\end{align*}

\textbf{Step 8.(Final Regret bound):} Combining the assumption in \textbf{Step 0} and from our problem definition we recall that the expected regret till the $T$-th timestep is bounded by,

\begin{align*}
\E[R_{T}]&= \sum_{i = 1}^K\sum_{j=1}^{G}\Delta^{opt}_{i,c_j}\E[n_{i_{t',c_j}\neq i^*_{t',c_j},\forall t'=t_{c_{j-1}}:t_{c_j}}] \\
%%%%%%%%%%%%%%%%%%%%%%%%%%%
& = \sum_{i = 1}^K\sum_{j=1}^{G}\bigg[ 1 
%%%%%%%%%%
+ \underbrace{\dfrac{4T\Delta^{opt}_{i,c_j}\log(\psi T)}{(\psi T)^{\alpha}}\dfrac{M 2^{2\alpha +1}}{(\Delta^{opt}_{i,c_j})^{4\alpha +2}}}_{\textbf{from Step 4}}
%%%%%%%%%%
+ \underbrace{\Delta^{opt}_{i,c_j} + \dfrac{4\log(\psi T(\Delta^{opt}_{i,c_j})^4)}{2(\Delta^{opt}_{i,c_j})}}_{\textbf{from Step 5}}\\
%%%%%%%%%%%%%%%%%%%%%%%%%%%
& + \underbrace{\dfrac{2T\Delta^{opt}_{i,c_{j+1}}\log(\psi T)}{(\psi T)^{\alpha}}\dfrac{M 2^{2\alpha + 1}}{(\Delta^{chg}_{i,c_j})^{4\alpha + 2}}}_{\textbf{from Step 6}}\\
%%%%%%%%%%%%%%%%%%%%%%%%%%%
&+ \underbrace{ \Delta^{opt}_{\max,c_{j+1}} + \dfrac{\Delta^{opt}_{\max,c_{j+1}}\log(\psi T(\Delta^{chg}_{i,c_j})^4)}{2\Delta^{chg}_{\epsilon_0,c_j}} + \Delta^{opt}_{i,c_{j+1}} + \dfrac{\Delta^{opt}_{i,c_{j+1}}\log(\psi T(\Delta^{chg}_{i,c_j})^4)}{2(\Delta^{chg}_{i,c_j})^2}}_{\textbf{from Step 7}}
\bigg]
\end{align*}

Hence, we get the main result of the regret bound.

\end{proof}